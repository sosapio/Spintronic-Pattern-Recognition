\chapter{Micromagnetic Problem File Source: FileSource}\label{sec:filesource}
\index{application!FileSource}

\begin{center}
\includepic{filesource-ss}{FileSource Screen Shot}
\end{center}

\starsechead{Overview}
The application \app{FileSource} provides the same service as
{\hyperrefhtml{\app{mmProbEd}}{\app{mmProbEd}
(Ch.~}{)}{sec:mmprobed}}\index{application!mmProbEd}, supplying
\MIF~1.x\index{file!MIF~1.x} problem descriptions to running \app{mmSolve2D}
micromagnetic solvers.  As the \MIF\ specification evolves,
\app{mmProbEd} may lag behind.  There may be new fields in the \MIF\
specification that \app{mmProbEd} is not capable of editing, or which
\app{mmProbEd} may not pass on to solvers after loading them in from a
file.  To make use of such fields, a \MIF\ file may need to be edited
``by hand'' using a general purpose text editor.  \app{FileSource} may
then be used to supply the \MIF\ problem description contained in a file
to a solver without danger of corrupting its contents.

\starsechead{Launching}
\app{FileSource} must be launched from the command line. You may specify
on the command line the \MIF\ problem description file it should serve
to client applications.  The command line is
\begin{verbatim}
tclsh oommf.tcl FileSource [standard options] [filename]
\end{verbatim}

Although \app{FileSource} does not appear on the list of
{\btn{Programs}} that \app{mmLaunch} offers to launch, running copies do
appear on the list of \btn{Threads} since they do provide a service
registered with the account service directory.

\starsechead{Inputs}
\app{FileSource} takes its \MIF\ problem description from the file named
on the command line, or from a file selected through the
\btn{File\pipe Open} dialog box.  No checking of the file contents
against the \MIF\ specification is performed.  The file contents are
passed uncritically to any client application requesting a problem
description.  Those client applications should raise errors when
presented with invalid problem descriptions.

\starsechead{Outputs}
Each instance of \app{FileSource} provides the contents of exactly one
file at a time.  The file name is displayed in the \app{FileSource}
window to help the user associate each instance of \app{FileSource} with
the data file it provides.  Each instance of \app{FileSource} accepts
and services requests from client applications (typically solvers) for
the contents of the file it exports.

The contents of the file are read at the time of the client request, so
if the contents of a file change between the time of the
\app{FileSource} file selection and the arrival of a request from a
client, the new contents will be served to the client application.

\starsechead{Controls}
The menu selection \btn{File\pipe Exit} terminates the 
\app{FileSource} application.  The \btn{Help} menu provides
the usual help facilities.
