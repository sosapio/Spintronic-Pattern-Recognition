\chapter{Overview of \OOMMF}\label{sec:overview}
Micromagnetics is a continuum model of the magnetization behavior of
ferromagnetic materials. Developed by W. F. Brown Jr.\ in the 1940s and
'50s \cite{brown1940,brown1963}, it allows for a more detailed description
of magnetization structure and behavior \cite{neel1955} than the earlier
domain theory \cite{weiss1906} or Stoner-Wohlfarth models
\cite{stoner1948}.  Computing resources at that time, however, limited
implementations to relatively simple systems \cite{labonte1969}. It
wasn't until the 1990s that fully 2D and later 3D systems of
technological interest could be successfully modeled
\cite{aharoni1996,fidler2000}.

The size of features such as domain walls, vortices, and cross-ties
depends on material parameters (e.g., saturation magnetization $M_s$,
magneto-crystalline anisotopy $K$, exchange coefficient $A$), applied
field ($H_{\rm app}$), and part geometry, but the typical length scale is on
the order of a few nanometers.  Arbitrary shapes and material variations
may be modeled, though part size will be limited by the available
computing power. Micromagnetics can be used to study static
magnetization structures, for example ground states in small particles
\cite{stdprob3mumag1998} and domain walls in film strips
\cite{mcmichael1997}, quasi-static behavior to model M vs.\ H hystersis
loops \cite{stdprob2mumag1998}, and field-driven magnetization dynamics
using the Landau-Lifshitz-Gilbert equation
\cite{gilbert1955,landau1935,stdprob4mumag2000}. More recent extensions
include spin-torque
\cite{stdprob5mumag2014,slonczewski1996,thiaville2005,xiao2004,zhang2004},
thermal \cite{garciapalacios1998}, and DMI \cite{thiaville2012} effects.

The goal of the
\htmladdnormallinkfoot{\OOMMF}{https://math.nist.gov/oommf/} (Object
Oriented MicroMagnetic Framework) project in the
\htmladdnormallinkfoot{Information Technology
  Laboratory}{https://www.nist.gov/itl/} (ITL) at the
\htmladdnormallinkfoot{National Institute of Standards and
  Technology}{https://www.nist.gov/} (NIST) is to develop a portable,
extensible public domain micromagnetic program and associated tools
\cite{aharoni1996,brown1963,fidler2000}.  The first release of \OOMMF, based
on a micromagnetic code previously developed by Robert McMichael and
Michael Donahue, was made in October, 1998. That release included the
\hyperrefhtml{2D micromagnetic solver}{2D micromagnetic solver
  (Ch.~}{)}{sec:mmsolve}, \hyperrefhtml{problem editor}{problem editor
  (Ch.~}{)}{sec:mmprobed}, and several display widgets. Current releases
implement a completely functional, fully 3D micromagnetics package, with
the capability of being extended by other programmers so that people
developing new code can build on the \OOMMF\ foundation.  The primary
developers of \OOMMF\ are \psonly{\htmladdnormallinkfoot{Michael
    Donahue}{https://math.nist.gov/\%7EMDonahue}}
\notpsonly{\htmladdnormallink{Michael
    Donahue}{https://math.nist.gov/\%7EMDonahue}} and
\psonly{\htmladdnormallinkfoot{Donald
    Porter}{https://math.nist.gov/\%7EDPorter}.}
\notpsonly{\htmladdnormallink{Donald
    Porter}{https://math.nist.gov/\%7EDPorter}.}

\OOMMF\ is written in C++, a widely-available, object-oriented language
that can produce programs with good performance as well as
extensibility.  The code uses \Tcl/\Tk\ to create a portable user
interface allowing \OOMMF\ to operate across a wide range of \Unix,
\Windows, and \MacOSX\ platforms.

The code may be modified at three distinct levels.  At the top level,
individual programs interact via well-defined protocols across network
sockets\index{network~socket}.  One may connect these modules together
in various ways from the user interface, and new modules speaking the
same protocol can be transparently added.  The second level of
modification is at the \Tcl/\Tk\ script level.  Some modules allow
\Tcl/\Tk\ scripts to be imported and executed at run time, and the top
level scripts are relatively easy to modify or replace.  At the lowest
level, the C++ source is provided and can be modified (see the
\htmladdnormallinkfoot{\textit{OOMMF Programming
    Manual}}{https://math.nist.gov/oommf/doc/}).  The primary extension
mechanism at this level is through the \hyperrefhtml{\OOMMF\ eXtensible
  Solver}{OOMMF eXtensible Solver (Ch.~}{)}{sec:oxs}, Oxs. The
extensible nature of Oxs allows its capabilities to be varied as
necessary for the problem at hand, and lets \OOMMF\ users
\htmladdnormallinkfoot{extend Oxs with external
  modules.}{https://math.nist.gov/oommf/contrib/oxsext/}

%The first portion of OOMMF released was a magnetization file display
%program called
%\htmladdnormallink{\app{mmDisp}}{https://math.nist.gov/oommf/mmdisp/mmdisp.html}\index{application!mmDisp}.
%A \htmladdnormallinkfoot{working
%release}{https://math.nist.gov/oommf/software.html} of the complete OOMMF
%project was first released in October, 1998.  It included a problem
%editor, a 2D micromagnetic solver\index{simulation~2D}, and several
%display widgets, including an updated version of \app{mmDisp}.  The
%solver can be controlled by an {\hyperrefhtml{interactive
%interface}{interactive interface (Sec.~}{)}{sec:mmsolve2d}}, or through
%a sophisticated {\hyperrefhtml{batch control system}{batch control
%system (Sec.~}{)}{sec:obs}}.  This solver was originally based on a
%micromagnetic code that
%\ifnotpdf{\htmladdnormallink{Mike Donahue}{https://math.nist.gov/\~{}MDonahue}}
%\pdfonly{\htmladdnormallink{Mike Donahue}{https://math.nist.gov/\%7EMDonahue}}
%and
%\htmladdnormallink{Bob McMichael}{mailto:rmcmichael@nist.gov}
%had previously developed.  It utilizes a Landau-Lifshitz
%ODE\index{ODE!Landau-Lifshitz} solver to relax 3D spins on a 2D
%mesh\index{grid} of square cells, using FFT's\index{FFT} to compute the
%self-magnetostatic (demag) field\index{field!demag}.  Anisotropy,
%applied field\index{field!applied}, and initial
%magnetization\index{magnetization!initial} can be varied pointwise, and
%arbitrarily shaped elements\index{boundary} can be modeled.

%The details of programming
%an Oxs extension module are found in the
%\htmladdnormallinkfoot{OOMMF Programming Manual
%}{https://math.nist.gov/oommf/doc/}.

{\samepage
If you want to receive e-mail\index{e-mail}
notification\index{announcements} of updates to this project, register
your e-mail address with the ``\mumag'' mailing list:
\begin{center}
\htmladdnormallink{https://www.ctcms.nist.gov/~rdm/email-list.html}{https://www.ctcms.nist.gov/\%7Erdm/email-list.html}
\end{center}
} % end \samepage

The \OOMMF\ developers are always interested in your comments about
\OOMMF.  See the
\hyperrefhtml{Credits}{Credits (Ch.~}{) }{sec:credits}
for instructions on how to contact them, and for information on
citing \OOMMF.
