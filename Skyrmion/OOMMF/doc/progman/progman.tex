\documentclass[12pt]{report}

% Leave a blank line at top to work around some flakiness in l2h
% on RedHat 5.2/AXP (otherwise the images.tex file can get some
% bad characters dumped into it).
%
% The following interlock is taken from Knuth's  ``The TeXbook'',
% p383 (Appendix D: Dirty Tricks).  It insures this file gets read
% at most once.  (NOTE: Unfortunately, latex2html doesn't do
% \if statements.)
%\ifx\oommfheadread\relax\endinput\else\let\oommfheadread=\relax\fi

% \documentclass[12pt]{article}  % Put this in file inputting this one

\usepackage{makeidx}
\usepackage{multirow}

%begin{latexonly}
% On the one hand, latexml 0.8.6 hiccups on color, but xcolor is OK.
% On the other hand, latex2html 2021 chokes on xcolor. There is no
% "else" equivalent in %begin{latexonly}, so wait until package html
% is loaded and do \html{\usepackage{color}}.
\usepackage{xcolor}
\usepackage{graphics}

\usepackage{ifthen}   % \ifthenelse construct

% Test if a command is defined, by Ulrich Diez in
% tex.stackexchange.com, 3-Nov-2020. Use like
% \checkfor{foo}{\foo is defined}{\foo is undefined}
%begin{latexonly}
\makeatletter
\DeclareRobustCommand\checkfor[1]{%
  \begingroup
  \expandafter\ifx\csname#1\endcsname\relax\expandafter\@firstoftwo\else\expandafter\@secondoftwo\fi
  {%
    \expandafter\endgroup\expandafter\ifx\csname#1\endcsname\relax\expandafter\@firstoftwo\else\expandafter\@secondoftwo\fi
  }{\endgroup\@firstoftwo}%
}%
\makeatother
%end{latexonly}

%begin{latexonly}
% Are we running pdftex with pdf output?
% (This logic is from Heiko Oberdiek's ifpdf package.)
\ifx\pdfoutput\undefined
  % not running PDFTeX
  \def\oommfpdf{0}
\else
  \ifx\pdfoutput\relax
    % not running PDFTeX
    \def\oommfpdf{0}
  \else
    % running PDFTeX, with...
    \ifnum\pdfoutput>0
      % ...PDF output
      \def\oommfpdf{1}
    \else
      % ...DVI output
      \def\oommfpdf{0}
    \fi
  \fi
\fi
%end{latexonly}


%begin{latexonly}
\IfFileExists{latexml.sty}{\usepackage{latexml}}{
  \newif\iflatexml % Assume no latexml.sty => no latexml
  % (\newif creates a command initialized to \iffalse.)
}
%end{latexonly}
% latex2html builds won't load latexml, so won't define \iflatexml.
% But all \if* commands should be wrapped in latexonly blocks so that
% latex2html doesn't see them, since latex2html doesn't process them
% anyway. You can check the tail bit of the latex2html run to see a
% list of commands it didn't recognize. If inputs are coded properly
% then \iflatexml should not be on that list.
%
% NOTE: latexml with unmodified html package bindings chunders
% on \begin/end{rawhtml} and \begin/end{htmlonly} environments if
% the \end part is indented. The html.sty.ltxml in LaTeXML 0.86 has this
% binding for the rawhtml environment (htmlonly is completely
% analogous):
%
%   DefConstructorI(T_CS("\\begin{rawhtml}"), undef, '', reversion => '',
%     afterDigest => [sub {
%         my ($stomach, $whatsit) = @_;
%         my $endmark = "\\end{rawhtml}";
%         my $nlines  = 0;
%         my ($line);
%         my $gullet = $stomach->getGullet;
%         $gullet->readRawLine;    # IGNORE 1st line (after the \begin{$name} !!!
%         while (defined($line = $gullet->readRawLine) && ($line ne $endmark)) {
%           $nlines++; }
%         NoteLog("Skip rawhtml ($nlines lines)"); }]);
%
% The problem is the comparison of $line against $endmark requires an
% exact full-line match. One workaround is to  strip off any leading
% whitespace before making the comparison, e.g.,
%
%     while (defined($line = $gullet->readRawLine)
%            && $line =~ s/^\s*// && ($line ne $endmark)) {
%
% Another more general approach is to use a regexp match, e.g.,
%
%   DefConstructorI(T_CS("\\begin\{htmlonly\}"), undef, '', reversion => '',
%     afterDigest => [sub {
%         my ($stomach, $whatsit) = @_;
%         my $endmark = '^\\s*\\\\end\{rawhtml\}\\s*$';
%         my $nlines  = 0;
%         my ($line);
%         my $gullet = $stomach->getGullet;
%         $gullet->readRawLine;    # IGNORE 1st line (after the \begin{$name} !!!
%         while (defined($line = $gullet->readRawLine) && ($line !~ m/$endmark/)) {
%           $nlines++; }
%         NoteLog("Skip htmlonly ($nlines lines)"); }]);
%
% which allows any amount of leading or trailing whitespace on the \end
% line.
%
% NOTE: LaTeXML will still choke on single-line constructs such as
%   \begin{rawhtml}</BLOCKQUOTE>\end{rawhtml}
% but this is arguably sleazy LaTeX anyway.
%
% To use this, copy the distributed html.sty.ltxml to a local directory,
% edit the file as shown above, and then use the --path option to
% latexml to include the local directory on the latexml search path.

% Is the build targeting PostScript output? The first definition
% of \psonly and \notpsonly holds in place if latex2html is running.
\newcommand{\psonly}[1]{}
\newcommand{\notpsonly}[1]{#1}
%begin{latexonly}
\newif\ifoommfps
\ifnum\oommfpdf=0
  \unless\iflatexml
    \oommfpstrue
  \fi
\fi
\ifoommfps
  \renewcommand{\psonly}[1]{#1}
  \renewcommand{\notpsonly}[1]{}
\fi
%end{latexonly}

%begin{latexonly}
%\ifnum\oommfpdf=0
% pdflatex command not in use
% The html package included below uses \pdfoutput to determine whether
% or not pdf-TeX is being used.  Unfortunately, the code in html.sty
% that determines this is broken, at least v1.39 2001/10/01 as shipped
% with Fedora Core 6 (FC6) when used with the latex in FC6.  This
% results in breakage of some commands defined in html.sty, including at
% least \htmlimage and \htmladdnormallink.  The breakage is such that
% the pdf-versions of these commands are wrongly defined in the case
% where latex or latex2html is running.  One workaround would be to
% redefine these commands after \usepackage{html}, but a more general
% fix would appear to be to just redefine \pdfoutput so that the logic
% in html.sty works.  This latter approach is done here, by unsetting
% \pdfoutput.  I've included this lengthy note because the problematic
% case is when \pdfoutput is defined to value 0.  Is there code
% someplace that functions differently if \pdfoutput is 0 than if it is
% undefined?  I don't know.  If it turns out that this breaks something,
% then one can try redefining \pdfoutput to 0 after \usepackage{html},
% or otherwise leaving \pdfoutput alone and just redefining \htmlimage
% etc. as needed.
% UPDATE 28-May-2022: This breaks \usepackage{html} on latex, or at
%  least on the latex install on my Mac, which is
%   pdfTeX 3.141592653-2.6-1.40.24 (TeX Live 2022/MacPorts 2022.62882_0)
%  YMMV.
%\let\pdfoutput\relax
%\fi
%end{latexonly}

% \ifnum\oommfpdf=0
% pdflatex command not in use
% \renewcommand{\htmlimage}[1]{} % pdf-mode detection code is broken
%    in some versions of html.sty, causing \htmlimage to be re-defined
%    as taking 2 arguments.  This causes some havoc.  Define it back to
%    be safe.
%    NOTE: This code superceded by \let\pdfoutput\relax code above.
%    The code in this current stanza is left in but commented out in
%    case it occurs that the more general fix above breaks stuff.
% \renewcommand{\htmladdnormallink}[2]{#1} % Ditto.
% \fi

\notpsonly{\usepackage[colorlinks=true]{hyperref}}
% Note 1: latex2html reads the .aux files created by latex and dvips,
% but the hyperref package writes material to .aux that latex2html
% hiccups on, with errors like
%
%   *** sub wrap_cmd_HyPL@Entry  failed: Illegal declaration of
%   subroutine main::wrap_cmd_HyPL at (eval 573) line 1.
%
% I don't know if this actually matters or not, but the ps output
% doesn't do anything with hyperref, so just don't load hyperref in
% this case.
%
% Note 2: The html package loads package hyperref if it isn't already
% loaded, and loads with default options.  So to change defaults you
% need to either load hyperref first with desired options, or else use
% the \hypersetup command afterwards.
%
% Note 3: At one time the hyperref package load command we used was
%   \usepackage[pdftex, colorlinks=true, citecolor=blue]{hyperref}
% But on 25-Apr-2022 we dropped the pdftex option because the hyperref
% docs claim the driver (e.g., dvips, pdftex) should be detected
% automatically.
%
% Note 4: The hyperref docs recommend loading hyperref after all other
% packages, because hyperref redefines a lot of commands from other
% packages. So keep this in mind and rearrange package ordering as
% needed.

\usepackage{html}

\html{
\usepackage{l2hbugs}
% latex2html 2021 chokes on xcolor (see xcolor note above)
\usepackage{color}
}

% LINK REFERENCE:
%
% Links to \label
% \htmlonlyref{text}{label}
%   The label argument is ignored in nonHTML mode. In HTML output the
%   text links to the label. In LaTeX2HTML this is a wrapper
%   around \htmlref{}{} from package html. In LaTeXML it's a wrapper
%   about \hyperref from package hyperref, with args reversed.
%
% \hyperrefhtml{htmltext}{nonhtmlpretext}{nonhtmlposttext}{label}
%   HTML output uses htmltext and label, nonHTML output uses
%   pre/posttext + label. For nonHTML this is a wrapper
%   abouht \hyperrefpage from package hyperref.
%
% \pagehyperref{htmltext}{nonhtmlpretext}{nonhtmlposttext}{label}
%   Same as \hyperrefhtml, except that for nonHTML output the page
%   number containing the label is given instead of the label value.
%
%
% Links to arbpts
% \arbtarget{targettext}{label}
% \arbtargetlink{linktext}{prepage}{postpage}{label}
%   The first command sets an anchor at the specified point, the second
%   provides a link in HTML and PDF output, and a page reference in PDF
%   and PS output.
%
% \pttarget[targettext]{label}
% \ptlink{linktext}{label}
%   The first command sets an anchor at the specified point, the second
%   provides a link in HTML and PDF output but no page reference. In PS
%   the linktext is printed alone.
%
% More details on \arb* and \pt* commands are provided further down in
% this file.
%
% Links to URLs
% \htmladdnormallink{text}{url}
%   The url argument is ignored in non-html mode. In html output mode
%   the text links to the url.
%
% \htmladdnormallinkfoot{text}{url}
%   Same as \htmladdnormallink, except that a footnote is added in
%   non-html mode to the url.

% Crutch for latexml issues. The \iflatexml construct is insufficient
% because latex2html doesn't process \if statements.
%begin{latexonly}
\iflatexml
 \newcommand{\latexmlonly}[1]{#1}
 \newcommand{\notlatexmlonly}[1]{}
\else
 \newcommand{\latexmlonly}[1]{}
 \newcommand{\notlatexmlonly}[1]{#1}
\fi
%end{latexonly}
\html{
 \newcommand{\latexmlonly}[1]{}
 \newcommand{\notlatexmlonly}[1]{#1}
}

% latexml defines command \iflatexml true if latexml is running,
% otherwise false. Some one can do
%
%  \iflatexml
%    % latexml version
%  \else
%    % plain latex version
%  \fi
%
% The \else clause is optional if you want just the latexml branch.
% If you want to reorder the branches, or just have the plain latex
% branch, use \unless:
%
% \unless\iflatexml
%    % plain latex version
%  \else
%    % latexml version
% \fi
%
% It would be nice to work latex2html into this processing scheme, but
% according to my notes latex2html doesn't do \if statements.  For
% reference, \newif\iffoo defines a new command \iffoo defined initially
% to be \iffalse, but also commands \footrue and \foofalse which
% redefine \iffoo to be \iftrue and \iffalse, respectively.
%
% The html.sty file provided with latex2html provides five conditional
% environments:
%
%   htmlonly, latexonly, rawhtml, imagesonly, makeimage
%
% and related \html{}, \latex{}, \latexhtml{}{} commands, plus a further
% eight commands:
%
%  htmlref                  61
%  htmladdnormallink        41
%  htmladdnormallinkfoot    26
%  hyperref                311
%  htmlimage                14
%  htmladdimage              0
%  htmlcite                  0
%  htmlrule                  2
%
% The numbers to the right is a count of the lines in .tex files under
% oommf/doc that currently include the command text (excluding this
% appearance but including mention in other comments).
%
% One can also exclude text from processing by latex2html by wrapping it
% inside
%
%  %begin{latexonly}
%  ...
%  %end{latexonly}
%
% Unlike the \begin/\end forms, these aren't processed by LaTeX and
% don't put the contents inside a group. Also, these are processed by
% latex2html directly and only, and so can be used prior to
% \usepackage{html}.
%
% See the LaTeXML document "Bindings" section to see a list of supported
% classes and packages. I note that "html.sty" is included in the
% package list...
%
% BTW, latexml chunders on this construct
%    \begin{htmlonly}
%    ...
%    \end{htmlonly} % My little note
%
% The stock latexml 0.8.6 binding for the htmlonly and rawhtml
% environments only captures the \end{...} statement if it begins in the
% first column and has no trailer. I slapped together a straightforward
% fix that allows whitespace (though not other material) on the \end
% line---see doc/common/xmlextras/html.sty.ltxml. The doc makerules
% files are set up such that this html binding file is loaded instead of
% the stock version, but I'll see if Bruce Miller will change the stock
% version in the next release.

%begin{latexonly}
\renewcommand\htmlimage[1]{}  % Otherwise breaks 'latex' (for .ps
% output) on macOS 25-Apr-2022. No idea why...

%%%%%%%%%%%%%%%%%%%%%%%%%%%%%%%%%% TOCLOFT %%%%%%%%%%%%%%%%%%%%%%%%%%%%%
% If there are more than nine subsections in a section (for example,
% in the ``Command Line Utilities'' section, then in the table of
% contents the subsection numbers run into the subsection titles.  One
% workaround is:
%
%   \usepackage{tocloft}
%   \setlength{\cftsubsecnumwidth}{2.7em}
%
% However this increases the spacing between subsection numbers and
% titles for all subsections in the toc.  A slightly less ugly
% alternative is
%
%begin{latexonly}
\iflatexml\else
\usepackage{tocloft}
\newlength{\oommftocsslen}
\setlength{\oommftocsslen}{0.5em} % need some extra space at end of number
\renewcommand{\cftsecpresnum}{\hfill} % note the double `l'
\renewcommand{\cftsecaftersnum}{\hspace*{\oommftocsslen}}
\addtolength{\cftsecnumwidth}{\oommftocsslen}
\fi
%end{latexonly}
% (Note: latex2html and latexml don't have a bindings for tocloft, so
% exclude this block from latex2html and latexml processing.)
%
% This typesets subsection numbers flushright.  With this the
% subsection numbers after .9 stick out to the left, but otherwise
% everything lines up.
%
% Another approach may be to hack the userguide.toc file directly. Or
% maybe ``Command Line Utilities'' just has too many sections and needs
% to be broken up.
%%%%%%%%%%%%%%%%%%%%%%%%%%%%%%%%%% TOCLOFT %%%%%%%%%%%%%%%%%%%%%%%%%%%%%

%begin{latexonly}
%\ifx\undefined\pdfpagewidth % pdflatex command not in use
\ifnum\oommfpdf=0
% pdflatex command not in use
\newcommand{\pdfonly}[1]{}
\newcommand{\ifnotpdf}[1]{#1}
\else                       % pdflatex command in use
\newcommand{\pdfonly}[1]{#1}
\newcommand{\ifnotpdf}[1]{}
\pdfcompresslevel=9
%\pdfpagesattr={/CropBox [60 290 480 720]}
%\pdfpagewidth=6.0in
%\pdfpageheight=5.5in
%\pdfcatalog{/PageMode /UseOutlines}
\pdfcatalog{            % Catalog dictionary of PDF output.
    /PageMode /UseOutlines
    /URI (https://math.nist.gov/oommf/)
}
% openaction goto page 1 {/Fit}
\fi

\iflatexml
\newcommand{\htmlonlyref}[2]{\hyperref[#2]{#1}}
\newcommand{\pagehyperref}[4]{\htmlonlyref{#1}{#4}}
% Note: In latexml 0.8.6, the "~" in \htmlref{\MIF~1.2}{sec:mif12format}
% gets transferred to HTML as a literal "~" instead of a nbsp. But
% \htmlonlyref{\MIF~1.2}{sec:mif12format} defined as above works OK.
\else
\newcommand{\htmlonlyref}[2]{#1}
\newcommand{\pagehyperref}[4]{\hyperrefpage{#1}{#2}{#3}{#4}}
\fi
%end{latexonly}

\html{
\newcommand{\pdfonly}[1]{}
\newcommand{\ifnotpdf}[1]{#1}
%\let\hyperrefhtml=\hyperref
\newcommand{\hyperrefhtml}[4]{\htmlref{#1}{#4}}
\newcommand{\htmlonlyref}[2]{\htmlref{#1}{#2}}
% Use \htmlonlyref for links to be available in HTML, but not
% in PDF.  In particular, this applies to \label{} commands not
% placed near counter updates, since latex2html drops an anchor
% tag at the right location, but pdflatex just drops the ball
% (well at least pdflatex Version 3.14159-13d (Web2C 7.3.1) does).
\newcommand{\pagehyperref}[4]{\htmlonlyref{#1}{#4}}
}

% Selection by HTML vs non-HTML output
%begin{latexonly}
\iflatexml
  \newcommand{\HTMLoutput}[1]{#1}
  \newcommand{\NONHTMLoutput}[1]{}
\else
  \newcommand{\HTMLoutput}[1]{}
  \newcommand{\NONHTMLoutput}[1]{#1}
\fi
%end{latexonly}
\html{
% NOTE: Do NOT use \newcommand inside \NONHTMLoutput; the \NONHTMLoutput
% wrapper will be ignored and latex2html will define the command as
% given, even if the command is already defined (i.e., should require
% a \renewcommand). If you absolutely positively have to do this, follow
% the \NONHTMLoutput command definition with a \HTMLoutput
% definition. Actually, it is probably more robust to define the command
% as desired for non-HTML output w/o \NONHTMLoutput, and then
% use \HTMLoutput{\renewcommand...}.
%   A working alternative is to use \HTMLoutput and \NONHTMLoutput
% inside \newcommand, rather than the other way around.
  \newcommand{\HTMLoutput}[1]{#1}
  \newcommand{\NONHTMLoutput}[1]{}
}

\setlength{\textwidth}{6.5in}
\setlength{\oddsidemargin}{0in}
\setlength{\textheight}{8.5in}

\begin{htmlonly}
\HTMLset{myTODAY}{\today}
\usepackage{localmods}
\end{htmlonly}

\newcommand{\myimage}[2]{\HTMLcode[#1 #2]{IMG}}

\htmlinfo*
\bodytext{BGCOLOR="#FFFFFF",text="#000000",LINK="#0000FF",
            VLINK="#4498F0",ALINK="00FFFF"}

%\htmladdtonavigation{\htmladdnormallink
%    {\myimage{ALT="OOMMF Home",BORDER="2"}{https://math.nist.gov/oommf/images/oommficon.gif}}{https://math.nist.gov/oommf}}

\htmladdtonavigation{\htmladdnormallink
    {\myimage{ALT="OOMMF Home",BORDER="2"}{oommficon.gif}}{https://math.nist.gov/oommf}}

\usepackage{alltt}  % Verbatim package supporting control sequences.
% Both LaTeXML and LaTeX2HTML have bindings for the alltt environment,
% but the LaTeX2HTML exhibits wonky behavior with alltt. When I first
% starting using alltt, I found that LaTeX2HTML support for the
% inline \textcolor{<color>}{<text>} command was broken inside alltt
% environments. However, it seemed that using \color from the color
% package (see notes above about color vs.  xcolor package conflicts in
% latex2html and latexml), instead of \textcolor worked OK, aside from
% ungrouped \color commands changing hyperref link color in latexml.  So
% I defined \colorit to be a synonym for \textcolor except in
% latex2html, where a grouped \color command was used instead.
%   But then later when I tried to build the documentation on a Rocky 8
% Linux box (9-Aug-2022), LaTeX2HTML threw an error on \color:
%
%  11/23:chapter:..."Debugging OOMMF" for Debugging_OOMMF.html
%  ;........Unescaped left brace in regex is illegal here in regex;
%    marked by <-- HERE in m/^color{ <-- HERE shellcmdcolor}/
%    at /usr/bin/latex2html line 4531.
%
% But \textcolor seems okay. We can probably do away with \colorit,
% but I'm leaving the infrastructure in place for reference.
% begin{latexonly}
\newcommand{\colorit}[2]{\textcolor{#1}{#2}}
%end{latexonly}
\html{
%\newcommand{\colorit}[2]{{\color{#1}{#2}}}
\newcommand{\colorit}[2]{\textcolor{#1}{#2}}
}
% Macros for coloring shell and program commands in alltt environments.
% The colors can be changed using \renewcommand. Constructs like
% "Program commands are colored \pgmcmd{\pgmcmdcolorname}\ for
% visibility" help text-to-color identification for colorblind
% individuals or grayscale printing. The colors are tweaked a little
% to make them either easier to read against a white background (cyan)
% or easier to distinguish from black in grayscale (red).
\definecolor{shellcmdcolor}{RGB}{0,187,238}
\newcommand{\shellcmdcolorname}{cyan}
\newcommand{\shellcmd}[1]{\colorit{shellcmdcolor}{#1}}
\definecolor{pgmcmdcolor}{RGB}{255,51,17}
\newcommand{\pgmcmdcolorname}{red}
\newcommand{\pgmcmd}[1]{\colorit{pgmcmdcolor}{#1}}

%begin{latexonly}
\iflatexml\else
\usepackage{upquote} % Use upright quotes in verbatim environments
\fi
%end{latexonly}
% Note: latexml docs say upquote package is supported, but if used I get
%    Warning:missing_file:upquote Can't find binding for package upquote
% latex2html also doesn't have a binding for upquote.

% I want upright quotes in verbatim environments, but upquote package
% seems to be not well supported. Might try \textquotesingle and
% \textquotedbl in alltt environment.
\usepackage[T1]{fontenc} % for \textquotedbl
\usepackage{textcomp}    % for \textquotesingle (not needed for post-2019 TeX)
\latex{\newcommand{\ssquote}{\textquotesingle}}  % Straight quotes
\latex{\newcommand{\sdquote}{\textquotedbl}}
\html{\newcommand{\ssquote}{\verb+'+}}
\html{\newcommand{\sdquote}{\verb+"+}}

% WARNING: --- Single quote madness ---
% Unicode defines the following single quote characters:
%
% Unicode    UTF-8    Description
% U+0027         27   Apostrophe, neutral (straight) quote
% U+0060         60   Grave accent (backtick)
% U+00B4      C2 B4   Accute accent (mirror image of backtick)
% U+2018   E2 80 98   Left single quotation mark (curved quote)
% U+2019   E2 80 99   Right single quotation mark (curved quote)
%
% If the upquote package is not used, then in verbatim (and alltt)
% environemnts LaTeX renders the keyboard backtick character as U+2018
% and and the keyboard apostrophe U+2019, i.e., as curved quotes. These
% don't look especially similar to their keyboard glyphs, which is
% problematic if you are using backticks to illustrate execution of an
% embedded command in a command shell (in Bourne shell derivatives
% '$(...)'  can be used, but csh only has backticks), or a code sequence
% using paired single quotes.
%    The upquote package handles this fairly well, rendering backticks
% as U+0060 and single quotes as U+0027 inside verbatim (and alltt)
% environments. LaTeX2HTML doesn't have a binding for upquote, but also
% follows this behavior.
%    LaTeXML, OTOH, follows the LaTeX convention and doesn't have a
% binding for upquote (at least as of LaTeXML 0.8.6). Even worse, the
% monospace font used on at least macOS 11.6.7 (Big Sur) has nearly
% indistinguishable glyphs for U+2018 and U+2019.
%    A workaround for the backtick that seems to work in alltt for
% LaTeX, LaTeX2HTML, and LaTeXML, both with and without the upquote
% package, is to request a grave accent on an empty string, as in the
% "\backtick" command defined below. This renders U+0060 in all three
% cases. The "\fronttick" command is defined for completeness, but it
% renders U+00B4, which doesn't have an obvious application to verbatim
% keyboard rendering. To get U+0027, use either the upquote
% package, \textquotesingle, or the "\uptick" command (which assumes tt
% font) defined below.
%
\newcommand{\backtick}{\`{}}
\newcommand{\fronttick}{\'{}}
%begin{latexonly}
\newcommand{\uptick}{\char'15}
%end{latexonly}
\html{\newcommand{\uptick}{{'}}}

% A solution to the ``set marker at arbitrary point on a page'' problem:
%
% Usage: \arbtarget{targettext}{label}
%        \arbtargetlink{linktext}{prepage}{postpage}{label}
%
%        \pttarget[targettext]{label}
%        \ptlink{linktext}{label}
%
% The first pair provide a link in html and pdf output, and a page
% reference in pdf and ps output. The second pair provide a link in html
% and pdf output, but no page reference; \ptlink is transparent in ps
% output, printing only the linktext.
%
% NB: Text like
%   \begin{rawhtml}<BLOCKQUOTE>\end{rawhtml}
% will break LaTeXML parsing and can prevent proper operation of these
% commands. Instead, move the \end{rawhtml} to a separate line. LaTeXML
% also doesn't like text following \end{rawhtml}, e.g.,
%   \end{rawhtml}\index{foobar}
% so break that into two lines too.
%
% NB: If you want to align a pttarget with the top of a document
% section, do not place \pttarget before sectioning command, i.e.,
%
%   \pttarget{label}\section{My Favorite Toys}     %% BAD
%
% because if \section causes a page break then the target anchor will be
% on the preceding page. Putting \pttarget after the sectioning command,
%
%   \section{My Favorite Toys}\pttarget{label}     %% LESS BAD
%
% is a bit better, but the link will tend to point to the line after the
% section header, and if the viewer scrolls to put the anchor at the top
% of the page then you won't see the section title.
%
% A better solution is to use the standard section labels and reference
% with \hyperref, e.g.,
%
%   \section{My Favorite Toys}\label{sec:mft}     %% BEST
%   ...
%   \hyperref[sec:mft]{(Description.)}  % hyperref pkg w/o html pkg
% or
%   \htmlref{(Description.)}{sec:mft}   % w/ html pkg
%
% The html package redefines \hyperref, so if html is loaded you need to
% use the second version if you want active links in the PDF output.
% Also note the change in argument ordering between \hyperref and
% \htmlref, and also the square brackets on the former.
%
% Unfortunately, the \section links are broken in some cases in LaTeXML
% (version 0.8.6). If document splitting is such that sec:mft is the top
% level section in it's .html file, any links to sec:mft from within
% that file aren't placed into the .html file. Since file splitting
% isn't decided until latexmlpost, AFAICT the .xml if fine, it is just a
% bug in the conversion from xml to html.
%
% On a related note, if you use \ptlink in a moving argument like
% \caption, you'll need to \protect it, e.g.,
%
%  \caption{Sheep in a snow storm\protect.\notpsonly{
%    \protect\ptlink{(Description.)}}}
%
% The text to a \section command is also a moving argument, but AFAICT
% \pttarget as used above does not require \protect.
%
%
% Background: In standard LaTeX and PDFLaTeX, the \label command drops a
% pin at the point of the most recent \refstepcounter, so we can define
% links to arbitrary points in the document by creating a special-use
% counter and stepping that at the point of interest. If the hypertex
% package is loaded then in the pdf version the link even points to the
% right position on the page, so this all works nicely.
%
% LaTeXML handles label references differently than (PDF)LaTeX. Rather
% than tying a target to the stepping of a counter, it expects a
% reference point to lie inside some enclosing environment, and links to
% (the start of?) that environment. So we can't use the counter-based
% target method with LaTeXML. Instead, we use the \hyper{target,link}
% commands from the \hyperref package.
%
% LaTeX2HTML Version 2021 does not have a binding for
% \hyper{target,link}, but it does handle \label the same way as
% (PDF)LaTeX, so we use the counter-based method for LaTeX2HTML.
%
% NB: If the label name has the form "html:foobar", then LaTeXML will
% complain about \arbtargetlink calls with the error:
%
%    Error:malformed:document Document fails RelaxNG validation (LaTeXML)
%
% IDK if source of the error is LaTeXML or the hyperref package; the
% docs for the latter indicate special handling for constructs like
%   \href{https://foo.org}{foo home}
%   \href{mailto:bar@foo.org}{bar@foo.org}
%   \href{run:/path/to/my/file.ext}{text displayed}
%
% Regardless, the safest course is to avoid colons in \arbtarget
% labels. By convention, maybe use the label format "PTfoo"?
%
%begin{latexonly}
\iflatexml % LaTeXML version
\newcommand{\arbtarget}[2]{\hypertarget{#2}{#1}}
\newcommand{\arbtargetlink}[4]{\hyperlink{#4}{#1}}
\newcommand{\pttarget}[2][]{\hypertarget{#2}{#1}}
\newcommand{\ptlink}[2]{\hyperlink{#2}{#1}}

\newcommand{\arbtargetlinkonly}[2]{\hyperlink{#2}{#1}}
\else % LaTeX and PDFLaTeX version
  \newcounter{arbtargetcounter}
  \newcommand{\arbtarget}[2]{#1\refstepcounter{arbtargetcounter}\label{#2}}
  \newcommand{\arbtargetlink}[4]{#2\pageref{#4}#3}
  \ifnum\oommfpdf=1 % PDF output (with links)
    \newcommand{\pttarget}[2][]{\hypertarget{#2}{#1}}
    \newcommand{\ptlink}[2]{\hyperlink{#2}{#1}}
  \else % PS output (no links)
    \newcommand{\pttarget}[2][]{#1}
    \newcommand{\ptlink}[2]{#1}
  \fi
\fi
%end{latexonly}
\html{ % LaTeX2HTML version
  \newcounter{arbtargetcounter}
  \newcommand{\arbtarget}[2]{\refstepcounter{arbtargetcounter}\label{#2}}
  \newcommand{\arbtargetlink}[4]{\htmlref{#1}{#4}}
  \newcommand{\pttarget}[2][]{#1\refstepcounter{arbtargetcounter}\label{#2}}
  \newcommand{\ptlink}[2]{\htmlref{#1}{#2}}
}

\newcommand{\MailLink}[2]{%
\HTMLoutput{\htmladdnormallink{#1}{mailto:#2}}%
\NONHTMLoutput{#1 (#2)}}

\newcommand{\blackhole}[1]{}
\newcommand{\Unix}{Unix}
\newcommand{\X}{X}
\newcommand{\Linux}{Linux}
\newcommand{\Windows}{Windows}
\newcommand{\MacOSX}{macOS}
\newcommand{\DOS}{DOS}
\newcommand{\Tcl}{Tcl}  % Tcl Developer Xchange = https://www.tcl-lang.org/
\newcommand{\C}{C}
\newcommand{\Cplusplus}{C++}
\newcommand{\Tk}{Tk}
\newcommand{\OOMMF}{OOMMF}
\newcommand{\MIF}{MIF}
\newcommand{\ODT}{ODT}
\newcommand{\OVF}{OVF}
\newcommand{\SVF}{SVF}
\newcommand{\VIO}{VIO}
\newcommand{\OBS}{OBS}
\newcommand{\eps}{Encapsulated PostScript}
\newcommand{\postscript}{PostScript}
\latex{\newcommand{\mumag}{$\mu$MAG}}
\html{\newcommand{\mumag}{muMAG}}
\newcommand{\micrometer}{\latex{$\mu$m}\html{\begin{rawhtml}&micro;m\end{rawhtml}}}
\newcommand{\munaught}{\latex{$\mu_0$}\html{\begin{rawhtml}&micro;<SUB>0</SUB>\end{rawhtml}}}
\newcommand{\SI}{SI}     % as in SI units
\newcommand{\ASCII}{ASCII}
\newcommand{\emdash}{\latex{---}\html{\begin{rawhtml}&mdash;\end{rawhtml}}}

% --- Font control reference --
% The first set below are cumulative. They come in both command
% (e.g., \textbf{...}) and declarative (e.g., {\bfseries ...}) flavors:
%
% \textrm{...}      {\rmfamily ...}    Roman (default)
% \textsf{...}      {\sffamily ...}    Sans serif
% \texttt{...}      {\ttfamily ...}    Typewriter (monospaced)
%
% \textup{...}      {\upshape ...}     Upright (default)
% \textit{...}      {\itshape ...}     Italics
% \textsl{...}      {\slshape ...}     Slanted
% \textsc{...}      {\scshape ...}     Small caps
%
% \textmd{...}      {\mdseries ...}    Medium weight (default)
% \textbf{...}      {\bfseries ...}    Boldface.
%
% \textnormal{...}  {\normalfont ...}  Main document font
%
% There is also \emph{...} for text to be emphasized. The effect depends
% on the currently active font; typically \emph will switch to italic,
% but will switch e.g. to roman if the active font is italic.
%
%
% The following, older font control method below are "unconditional,"
% meaning non-cumulative. These are all declarative:
%
% {\rm ...}  Roman
% {\sf ...}  Sans serif
% {\tt ...}  Typewriter (monospace, fixed-width)
%
% {\it ...}  Italics
% {\sl ...}  Slanted (oblique)
% {\sc ...}  Small caps
%
% {\bf ...}  Switch to bold face
%
% {\cal ...} Switch to calligraphic letters for math
%
% The \em command is the unconditional form of \emph.
%
% The following, non-cumulative commands are for math mode:
%
% \mathrm{...} Roman
% \mathsf{...} Sans serif
% \mathtt{...} Typewriter
%
% \mathit{...} Italics, aka \mit{...}
%
% \mathbf{...} Boldface
%
% \mathnormal{...} Normal, used inside another type style declaration
%
% \mathcal{...} Calligraphic letters
%
% There are additionally the commands \mathversion{bold}
% and \mathversion{normal} for switching between bold and normal fonts.
%
%
% NB: Latex2html 2021 doesn't handle nested face requests, e.g.,
% \texttt{\textrm{...}} properly --- it drops a <DIV> block that
% effects a newline. So separate \html{...} declarations may be
% needed to get the best achievable results.

\latex{\newcommand{\oxslabel}[1]{\textbf{\textrm{#1}}}}
\latex{\newcommand{\oxsval}[1]{\textit{\textrm{#1}}}}
\html{\newcommand{\oxslabel}[1]{\textbf{#1}}}
\html{\newcommand{\oxsval}[1]{\textit{#1}}}
% Use \oxslabel to refer to Oxs Specify block labels the first time in
% the running text. Use \oxsval for the value portion of label+value
% keys in both the TeX version of the Specify block, and in the running
% text. (BTW, latex2html gives a "cannot wrap" warning on \renewcommand,
% so just use separate \latex and \html versions.)


% Filenames and program code identifiers
\blackhole{
\definecolor{fn}{rgb}{0,0.5,0}
\definecolor{cd}{rgb}{0.5,0,0}
\definecolor{btn}{rgb}{0.5,0,0}
\newcommand{\fn}[1]{\latex{{\tt #1}}\html{\textcolor{fn}{#1}}}   % Files
\newcommand{\cd}[1]{\latex{{\tt #1}}\html{\textcolor{cd}{#1}}}   % Code
\newcommand{\btn}[1]{\latex{{\tt #1}}\html{\textcolor{btn}{#1}}} % Buttons
} % blackhole

%begin{latexonly}
\newcommand{\bftt}[1]{\textsf{\textbf{#1}}}
%% This is meant to be a bold tt, but there is no boldface cmtt (TeX
%% Typewriter font)!
\newcommand{\app}[1]{\textbf{#1}}    % Apps
\newcommand{\key}[1]{\texttt{#1}}    % Keys
\newcommand{\fn}[1]{\texttt{#1}}     % Files
\newcommand{\cd}[1]{\texttt{#1}}     % Code
\newcommand{\btn}[1]{\bftt{#1}}      % Buttons
\newcommand{\wndw}[1]{\textbf{#1}}   % Windows
%end{latexonly}

\begin{htmlonly}
\newcommand{\bftt}[1]{\texttt{\textbf{#1}}}
%% NOTE: There is no boldface cmtt (TeX), but HTML browsers
%%  may render differently?!
\newcommand{\app}[1]{\textbf{#1}}    % Apps
\newcommand{\key}[1]{\bftt{#1}} % Keys
\newcommand{\fn}[1]{\bftt{#1}}  % Files
\newcommand{\cd}[1]{\texttt{#1}}  % Code
\definecolor{btn}{rgb}{0.5,0,0}          % Buttons
\newcommand{\btn}[1]{{\textcolor{btn}{\textbf{#1}}}}
\newcommand{\wndw}[1]{{\bf #1}} % Windows
\end{htmlonly}

% Latex2html inserts unwanted whitespace after \rm, in structures like
%      \newcommand{\vB}{{\rm\bf B}}
% but the following seem to work:
\newcommand{\vB}{\textbf{B}}
\newcommand{\vH}{\textbf{H}}
\newcommand{\vM}{\textbf{M}}
\newcommand{\vm}{\textbf{m}}
\newcommand{\vh}{\textbf{h}}
\newcommand{\vx}{\textbf{x}}
\newcommand{\vu}{\textbf{u}}

\newcommand{\lb}{\texttt{\#}}  % "Pound" symbol
\newcommand{\pipe}{\latex{{\tt|}}\html{|}} % "Pipe" symbol
\newcommand{\bs}{\texttt{\char'134}} % Backslash, tt font
\newcommand{\fs}{\texttt{/}} % Forward slash, tt font

% MIF 2.x Specify block definitions.
\newcommand{\bi}{\hspace*{2em}}
% \bi is bullet indent.
\latex{\newcommand{\ocb}{\textrm{\{}}}
\latex{\newcommand{\ccb}{\textrm{\}}}}
% \ocb is open curly brace, \ccb is close curly brace.
% Latex2html 2021 sometimes mishandles typeface requests, so try
\html{\newcommand{\ocb}{\{}}
\html{\newcommand{\ccb}{\}}}

% O open and close angle brackets (aka less-than and greater-than
% symbols)
\newcommand{\oab}{\latex{{$<$}}\html{\texttt{<}}}
\newcommand{\cab}{\latex{{$>$}}\html{\texttt{>}}}

% Bold open and close angle brackets (aka less-than and greater-than
% symbols)
\newcommand{\boa}{\latex{{\boldmath$<$}}\html{\texttt{\textbf{<}}}}
\newcommand{\bca}{\latex{{\boldmath$>$}}\html{\texttt{\textbf{>}}}}

% ``Launching'' option keyword lists font selection
\newcommand{\optkey}[1]{\latex{\textbf{#1}}\html{\texttt{\textbf{#1}}}}

% Codelisting environment
% \newenvironment{codelisting}[4]{%
%  \def\codelistingtype{#1}     % f for float, p for ``in page''
%  \def\codelistinglabel{#2}    % \label tag
%  \def\codelistingcaption{#3}  % caption
%  \def\codelistingdescription{#4} % Link back to text
%  \def\codelistingdesctype{#5} % Link anchor type: ref or hyperlink
% Parameters #4 and #5 are ignored in postscript output, but for PDF and
% both HTML outputs these provide a link back to a point in the text
% describing the contents. That link can either be a counter-based ref
% accessed via \hyperref or else a \hypertarget type accessed via
% \hyperlink, as indicated by parameter #5.  (In the latex and latexml
% cases the conditional is handled using the ifthenelse command from the
% latex \ifthen package. For latex2html the condition is implemented by
% variable name interpolation.)
%
%begin{latexonly}
\iflatexml %%%%%%%%%%%%%%%%%%%%%%%%%%%%%%%%%%%%%%%%%%%%%%%%%%%%%%%%%%%%%%%%%%%%%%%%
\newenvironment{codelisting}[5]{%
 \def\codelistingtype{#1} % f for float, p for in page (ignored for html)
 \def\codelistinglabel{#2}       % \label tag
 \def\codelistingcaption{#3}     % caption
 \def\codelistingdescription{#4} % Link back to text
 \def\codelistingdesctype{#5}    % Link anchor type: ref or hyperlink
 \ifthenelse{\equal{\codelistingdesctype}{ref}}{
   \newcommand{\cldxyz}{\htmlonlyref{(description)}{\codelistingdescription}}
 }{
   \newcommand{\cldxyz}{\ptlink{(description)}{\codelistingdescription}}
    % Note space between caption and (description)!
 }
 \begin{figure}[h!]
   \centerline{\rule[1ex]{\textwidth}{0.5ex}}
   \caption{\codelistingcaption\label{\codelistinglabel}
     \protect\cldxyz} % Note space between caption and (description)!
}{
  \centerline{\rule[1ex]{\textwidth}{0.5ex}}
  \end{figure}
}
\else % !iflatexml %%%%%%%%%%%%%%%%%%%%%%%%%%%%%%%%%%%%%%%%%%%%%%%%%%%%%%%%%%%%%%%%
\newenvironment{codelisting}[5]{%
 \def\codelistingtype{#1}     % f for float, p for ``in page''
 \def\codelistinglabel{#2}    % \label tag
 \def\codelistingcaption{#3}  % caption
 \def\codelistingdescription{#4} % Link back to text
 \def\codelistingdesctype{#5}    % Link anchor type: ref or hyperlink
 \ifthenelse{\equal{\codelistingdesctype}{ref}}{
   \newcommand{\cldxyz}{\htmlref{(description)}{\codelistingdescription}}
 }{
   \newcommand{\cldxyz}{\ptlink{(description)}{\codelistingdescription}}
 }
 \if\codelistingtype f \begin{figure}
 \fi
}{
 \if\codelistingtype f
   \caption{\codelistingcaption\label{\codelistinglabel}\notpsonly{
       \protect\cldxyz}}\end{figure}
   % Note space between caption and (description)!
 \else
    \nopagebreak\parbox{\textwidth}{
    \begin{center}
    \refstepcounter{figure}
    Figure \thefigure: {\codelistingcaption\label{\codelistinglabel}\notpsonly{
    \cldxyz}}  % Note space between caption and (description)!
    \end{center}
   }\pagebreak[2]
 \fi
}

\fi % iflatexml %%%%%%%%%%%%%%%%%%%%%%%%%%%%%%%%%%%%%%%%%%%%%%%%%%%%%%%%%%%%%%%%%%%%
%end{latexonly}

\html{
\newenvironment{codelisting}[5]{%
  \addtocounter{figure}{1}\label{#2}
  \HTMLsetenv{codelistingcaption}{#3}
  \HTMLsetenv{textlink}{#4}
  \HTMLsetenv{desctype}{#5}   % Link anchor type: ref or hyperlink
  % A cunning plan for conditional processing:
  \HTMLsetenv{Xclref}{\htmlonlyref{(description)}{#4}}
  \HTMLsetenv{Xclhyperlink}{\ptlink{(description)}{#4}}
  \HTMLsetenv{backlink}{\HTMLget{Xcl#5}}
  \htmlrule
}{
  \begin{center}
  Figure \thefigure:
    \HTMLget{codelistingcaption}
    \HTMLget{backlink}
  \end{center}
  \htmlrule
}}

% List structure compatible with latex, pdflatex, latex2html, and
% latexml that can be used to create aligned text like
%
%           tclsh oommf.tcl oxspkg list
%    or
%           tclsh oommf.tcl oxspkg listfiles pkg
%    or
%           tclsh oommf.tcl oxspkg readme pkg
%
% with latex code
%
%    \begin{duplex}
%    \item \verb+tclsh oommf.tcl oxspkg list+
%    \item[\textbf{or}]\html{\\}
%    \item \verb+tclsh oommf.tcl oxspkg listfiles pkg [pkg ...]+
%    \item[\textbf{or}]\html{\\}
%    \item \verb+tclsh oommf.tcl oxspkg readme pkg [pkg ...]+
%    \end{duplex}
%
% Note the interspersed \item commands with labels and no text and text
% with no labels. If you don't include the \html{\\} in the no text
% case then latex2html puts the successive item on the same line.
%
\newenvironment{duplex}%
{\begin{list}%
     {\hspace{\notlatexmlonly{2em}\latexmlonly{4em}}} % labeling
     { \setlength{\leftmargin}{0.5em}
       \setlength{\listparindent}{0pt}
       \setlength{\parindent}{0pt}
       \setlength{\itemsep}{-0.5\baselineskip}
       \setlength{\labelwidth}{0pt}
     } % set spacing
}{\end{list}}


% Ersatz figure environment.  This is a standard figure environment in
% LaTex, but a dummy block in HTML.  This is useful because LaTeX2HTML
% passes figure environments to LaTex, and converts the resulting
% postscript to a graphics bitmap for inclusion.  Sometimes we don't
% want this, for example if the figure data is already in bitmap format.
% Also, we may want to throw in an ALT tag.
% SAMPLE USAGE:
%   \ofig{\includeimage{6in}{!}{oxsclass}{Oxs class diagram}}{OXS
%        top-level class diagram.}{fig:oxsclass}
% Note: The fourth argument to \includeimage is an ALT tag. Beware
%   that newlines in the ALT tag field cause breakage in LaTeX2HTML
%   handling that results in the ALT tag being dropped altogether.
%
%begin{latexonly}
\iflatexml
% \newcommand{\ofig}[3]%
% {\begin{center}
%  \addtocounter{figure}{1}\label{#3}
%  \textbf{Figure \thefigure: #2}\\
%  #1
% \end{center}}
%
%
\newcommand{\ofig}[3]{%
\begin{figure}
 \begin{center}
   #1\\
   \caption{#2\label{#3}}
 \end{center}
\end{figure}}
\else
\newcommand{\ofig}[3]{%
\begin{figure}
 \begin{center}
   #1\\
   \caption{#2\label{#3}}
 \end{center}
\end{figure}}
\fi
%end{latexonly}
\html{
\newcommand{\ofig}[3]%
{\begin{center}
 \addtocounter{figure}{1}\label{#3}
 \textbf{Figure \thefigure: #2}\\
 #1
\end{center}}
}
%% Is \refstepcounter{figure} needed in the \html def?


% Graphics inclusion.
%  Usage: \includepic{basename}{altstring}
%     A fixed scale parameter is used in the LaTeX code;
%   under HTML the graphic is brought directly in without any scaling.
%     Basename is the name of the graphic to include,
%   expanded as psfiles/basename.ps under latex, and
%   giffiles/basename.gif under html.
%     Altstring it a string to be passed to the ALT= tag
%   in HTML.  It is not used in the LaTeX code.
% Note: Previously the scaling for psfiles was set to "0.5", with the
%   note that that setting provided excellent onscreen rendering in
%   ghostview if scaling were set to 4.0 "pixel-based", although this
%   made the PostScript images slightly larger than in the PDF output.
%   To match the sizes the \scalebox setting needed to be 0.462.
%   However, PostScript for screenshots made in July 2021 used
%   different processing, based on the ImageMagick 'convert' tool. The
%   screenshots were collected using the gnome-screenshot command on
%   Linux,
%
%    gnome-screenshot -wbd 5 -e shadow -f mmdisp-ss.png
%
%   and then the pixel density was set like so:
%
%    convert -quality 97 -density 125 -units pixelsperinch mmdisp-ss.png
%
%   The density determines the scaling for the PDF output. The eps
%   files for the PostScript version of the userguide were created
%   with
%
%    convert mmdisp-ss.png pdf:- | pdftops -eps - mmdisp-ss.ps
%
%   With this processing chain the Postscript renders at the same size
%   as the PDF.

%begin{latexonly}
\iflatexml
% includepic for latexml
\newcommand{\includepic}[2]{%
\scalebox{1.0}{\includegraphics{pngfiles/#1.png}} }
\else
 \ifnum\oommfpdf=0
   % includepic for latex
   \newcommand{\includepic}[2]{%
   \scalebox{1.0}{\includegraphics{psfiles/#1.ps}} }
 \else
   % includepic for pdflatex
   \newcommand{\includepic}[2]{%
   \scalebox{1.0}{\includegraphics{pngfiles/#1.png}} }
 \fi
\fi
%end{latexonly}
\begin{htmlonly}
% includepic for latex2html
\newcommand{\includepic}[2]{%
\HTMLcode[../giffiles/#1.gif,ALT="#2"]{IMG}
}
\end{htmlonly}

% Alternate graphics inclusion
%  Usage: \includeimage{width}{height}{basename}{altstring}
%     Width and height are dimensions, e.g., 4in.  One of
%   these may be an exclamation mark '!', in which case
%   the corresponding dimension will be scaled as necessary
%   to keep the original aspect ratio.  Presently these two
%   parameters are used only in the LaTeX code; under HTML
%   (both LaTeX2HTML and LaTeXML) the graphic is brought
%   directly in without any scaling.
%     Basename is the name of the graphic to include,
%   expanded as psfiles/basename.ps under latex, and
%   giffiles/basename.gif or pngfiles/basename.png under html.
%     Altstring it a string to be passed to the ALT= tag
%   in HTML. It works with latex2html, but is ignored by
%   latex and pdflatex. It is also currently ignored by
%   latexml, although there has been some discussion on this
%   on the latexml github issues page, Feb-Dec 2021. Check
%   back later?
%
%   NB: The ALT tag is mostly read verbatim, and newlines cause breakage
%       resulting in the ALT tag being dropped altogether.  Whitespace
%       is retained, but LaTeX non-breaking spaces characters "~" are
%       converted to HTML "&nbsp;". This can be used to protect text
%       against automatic line splitting from text editors.
%
%begin{latexonly}
\iflatexml % latexml
\renewcommand{\includeimage}[4]{%
\includegraphics{pngfiles/#3.png}%
}
\else
\ifnum\oommfpdf=0 % latex
\newcommand{\includeimage}[4]{%
\resizebox{#1}{#2}{\includegraphics{psfiles/#3.ps}}%
}
\else % pdflatex
\newcommand{\includeimage}[4]{%
\resizebox{#1}{#2}{\includegraphics{pngfiles/#3.png}}%
}
\fi % end \oommfpdf=0
\fi % end \iflatexml
%end{latexonly}
\begin{htmlonly}
\newcommand{\includeimage}[4]{%
\HTMLcode[../giffiles/#3.gif,ALT="#4"]{IMG}
}
\end{htmlonly}

% Workaround for some apparently broken LaTeX2HTML Table of Contents
% controls.
% Also, a hackish way to stop LaTeXML file splitting at said sections.
% I think it may be possible to accomplish this via an appropriate XPATH
% option to --splitpath=, but I haven't been able to figure one out.
\latex{
\iflatexml
 \newcommand{\starsechead}[1]{\par\noindent{\Large\textbf{#1}}\\}
 \newcommand{\starssechead}[1]{\par\noindent{\large\textbf{#1}}\\}
 \newcommand{\starsssechead}[1]{\par\noindent{\large\textbf{#1}}\\}
\else
 \def\starsechead{\section*}
 \def\starssechead{\subsection*}
 \def\starsssechead{\subsubsection*}
\fi
}
\html{
\newcommand{\starsechead}[1]{\par\noindent{\Large\bf{#1}}\\}
\newcommand{\starssechead}[1]{\par\noindent{\large\bf{#1}}\\}
\newcommand{\starsssechead}[1]{\par\noindent{\large\bf{#1}}\\}
}


% If an inline formula has positive depth, then LaTeX2HTML handles
% vertical positioning of that formula by adding a vertical rule so
% that the depth and height are equal.  The resulting image is then
% marked in the HTML with the align=middle tag, which aligns the
% vertical center of the image with the current baseline.  This adds
% extra whitespace below the image, sometimes a lot, which can yield
% essentially an extra blank line in the viewed HTML.  The \abovemath
% command raises the math-mode formulae just enough so that the depth
% is zero, in which case the generated image is aligned in the HTML
% with the align=bottom tag.  This also looks bad, so it is a matter
% of choice which is the worse evil.  But it is probably an improvement
% in situations with the formula extends just a *little* below the
% baseline.  WRT the TeX output, this command just renders the formula
% in in-line math mode.
\newcommand{\nodepth}[1]{% Auxiliary command
$\mbox{\renewcommand{\arraystretch}{0}%
$\begin{array}[b]{@{}c@{}}#1\\\rule{1pt}{0pt}\end{array}$}$}
\newcommand{\abovemath}[1]{\latex{$#1$}\html{\nodepth{#1}}}

% Hyphenation
\hyphenation{DataTable}

% Index generation
\makeindex

%% Last line of oommfheadxml is \makeindex

\usepackage{multirow}

%\HTMLset{toppage}{progman.html}
%\htmladdtonavigation{\htmladdnormallink{\htmladdimg{../common/contents.gif}}{progman.html}}

\title{\OOMMF\\Programming Manual}
%This manual documents release 2.0b0.\\[1ex]
%WARNING: In this alpha release, the
%documentation may not be up to date.\\[1ex]
%WARNING: This document is under construction.}
%\author{Michael Donahue and Donald Porter}
%\date{\today}
\author{\today\\[1ex]
This manual documents release 2.0b0.\\[1ex]
WARNING: This document is under construction.}


\begin{document}

%\nocite{*}  % Include all entries from .bib file.
% Putting this at the top retains the .bib file ordering.  Except, as
% far as I can tell, LaTeXML ignores \nocite{*}; you have to manually
% list all references, like so:
\nocite{brown1963,donahue1999}

\pagenumbering{roman}

%begin{latexonly}
\iflatexml
\maketitle
\else
%end{latexonly}
\begin{titlepage}
%\label{page:contents}
\setcounter{page}{0}    % Hack 'destination with the same identifier' warning
\par
%\vspace*{\fill}
\begin{center}
\Large\bf
\OOMMF\\
Programming Manual\\[2ex]
\large
{\today}
{}\\[2ex]
This manual documents release 2.0b0.\\[1ex]
WARNING: This document is under construction.
\end{center}
%\vspace{10\baselineskip}
%\begin{abstract}
This manual provides source code level information on \OOMMF\ (Object Oriented Micromagnetic Framework),
a public domain micromagnetics program developed at the
\htmladdnormallink{National Institute of Standards and Technology}
{https://www.nist.gov/}.  Refer to the \OOMMF\ User's Guide for an
overview of the project and end-user details.
%\end{abstract}
%\vspace*{\fill}
%\par
\end{titlepage}
%begin{latexonly}
\fi
%end{latexonly}

%begin{latexonly}
\iflatexml
% Hack for broken LaTeXML .xslt that typesets table of contents header
% at <h6>, which is ridiculously small. One fix is to replace the "h6"
% setting in the
%
%   <xsl:template match="ltx:TOC">
%      ...
%   </xsl:template>
%
% of oommf/doc/common/xmlextras/xslt/oommf-webpage-xhtml.xsl to
% something larger. But, we don't really need the header in the web
% version of the page, so we can drop it altogether by redefining
% \contentsname to an empty string:
 \renewcommand{\contentsname}{}
\else
 \renewcommand{\contentsname}{Table of Contents}
\fi
\tableofcontents
%end{latexonly}

% Index cross-references; if these are moved to the bottom of this file
% then two LaTeX passes are required to get them in the .idx file.

\chapter*{Disclaimer}
\addcontentsline{toc}{chapter}{Disclaimer}\label{disclaimer}
% Note: LaTeXML uses \label value to set file names.
This software was developed at the National Institute of Standards and
Technology by employees of the Federal Government in the course of their
official duties.  Pursuant to Title 17, United States Code, Section 105,
this software is not subject to copyright protection and is in the
public domain\index{license}.

\OOMMF\ is an experimental system.  NIST assumes no responsibility
whatsoever for its use by other parties, and makes no guarantees,
expressed or implied, about its quality, reliability, or any other
characteristic.

We would appreciate acknowledgement if the software is used.  When
referencing \OOMMF\ software, we recommend citing the NIST technical
report, M. J. Donahue and D. G. Porter, ``OOMMF User's Guide, Version
1.0,'' \textbf{NISTIR 6376}, National Institute of Standards and
Technology, Gaithersburg, MD (Sept 1999).

Commercial equipment and software referred to on these pages are
identified for informational purposes only, and does not imply
recommendation of or endorsement by the National Institute of Standards
and Technology, nor does it imply that the products so identified are
necessarily the best available for the purpose.

\newpage

\pagenumbering{arabic}

\chapter{Programming Overview of \OOMMF}\label{sec:overview}
The
\htmladdnormallinkfoot{\OOMMF}{https://math.nist.gov/oommf/} (Object
Oriented Micromagnetic Framework) project in the
\htmladdnormallinkfoot{Information Technology Laboratory}{https://www.nist.gov/itl/}
(ITL) at the
\htmladdnormallinkfoot{National Institute of Standards and
Technology}{https://www.nist.gov/} (NIST) is intended to develop a
portable, extensible public domain micromagnetic program and associated
tools.  This manual aims to document the programming interfaces to
\OOMMF\ at the source code level.  The main developers of this code are
\psonly{\htmladdnormallinkfoot{Mike Donahue}{https://math.nist.gov/\%7EMDonahue}}
\notpsonly{\htmladdnormallink{Mike Donahue}{https://math.nist.gov/\%7EMDonahue}}
and
\psonly{\htmladdnormallinkfoot{Don Porter}{https://math.nist.gov/\%7EDPorter}.}
\notpsonly{\htmladdnormallink{Don Porter}{https://math.nist.gov/\%7EDPorter}.}

The underlying numerical engine for \OOMMF\ is written in \Cplusplus,
which provides a reasonable compromise with respect to efficiency,
functionality, availability and portability.  The interface and glue
code is written primarily in \Tcl/\Tk, which hides most platform
specific issues. \Tcl\ and \Tk\ are available for free
\htmladdnormallinkfoot{download}{http://purl.org/tcl/home/software/tcltk/choose.html}
from the
\htmladdnormallinkfoot{Tcl Developer Xchange}{http://purl.org/tcl/home/}.

The code may actually be modified at 3 distinct levels.  At the top
level, individual programs interact via well-defined protocols across
network sockets\index{network~socket}.  One may connect these modules
together in various ways from the user interface, and new modules
speaking the same protocol can be transparently added.  The second level
of modification is at the \Tcl/\Tk\ script level.  Some modules allow
\Tcl/\Tk\ scripts to be imported and executed at run time, and the top
level scripts are relatively easy to modify or replace.  The lowest
level is the \Cplusplus\ source code.  The OOMMF extensible solver, OXS,
is designed with modification at this level in mind.

If you want to receive e-mail\index{e-mail}
notification\index{announcements} of updates to this project, register
your e-mail address with the ``{\mumag} Announcement'' mailing list:
% Note: For some reason, the braces above about \mumag discourage line
% breaking between $\mu$ and Mag in the latexml/browser display.
\begin{center}
\htmladdnormallink{\url{https://www.ctcms.nist.gov/~rdm/email-list.html}}{https://www.ctcms.nist.gov/\%7Erdm/email-list.html}.
\end{center}

The \OOMMF\ developers are always interested in your comments about
\OOMMF.  See the \hyperrefhtml{Credits}{Credits (Ch.~}{) }{sec:credits}
for instructions on how to contact them.

\chapter{Platform-Independent Make Operational Details}\label{sec:pimake}

The \OOMMF\ \app{pimake}\index{application!pimake} application compares
file timestamps to determine which libraries and executables are
out-of-date with respect to their source code, and then compiles and
links those files as necessary to make everything up to date. The design
and behavior of \app{pimake} is based on the
\Unix\ \app{make}\index{application!make} program, but \app{pimake} is
written in \Tcl\ and so can run on any platform where \Tcl\ is
installed. Analogous to the \fn{Makefile} or \fn{makefile} of
\app{make}, \app{pimake} uses \fn{makerules.tcl} files that specify
\textit{rules} (actions) for creating or updating \textit{targets} when
the targets are older than their corresponding \textit{dependencies}.
The \fn{makerules.tcl} files are \Tcl\ scripts augmented by a handful of
commands introduced by the \app{pimake} application.

The \fn{makerules.tcl} files in the \app{Oxs} application area include
rules to automatically compile and link all \Cplusplus\ code found under
the \fn{oommf/app/oxs/local/} directory, so programmers who are
developing \cd{Oxs\_Ext} extension modules generally do not need to be
concerned with the intricacies of \app{pimake} beyond the instructions
on running \app{pimake} presented in the
\htmladdnormallinkfoot{\textit{\OOMMF\ User's
    Guide}}{https://math.nist.gov/oommf/doc/}.

This chapter is intended instead for programmers who are debugging,
extending, or creating new \OOMMF\ modules outside of
\fn{oommf/app/oxs/local/}. The following sections provide an overview of
the structure of \fn{makerules.tcl} files and how they control the
behavior of \app{pimake}. Further details may be gleaned from the
\app{pimake} sources in \fn{oommf/app/pimake/}.

\section{Anatomy of \fn{makerules.tcl} files}\label{sec:anatomymakerules}
As may be deduced from the file extension, \fn{makerules.tcl} files are
\Tcl\ scripts and so can make use of the usual \Tcl\ commands. However,
\fn{makerules.tcl} files are run inside a \Tcl\ interpreter that has
been augmented by \app{pimake} with a number of additional commands. We
discuss both types of commands here, beginning with some of the standard
\Tcl\ commands commonly found in \fn{makerules.tcl} files:
\begin{description}
\item[list, llength, lappend, lsort, lindex, lsearch, concat]
  \Tcl\ list formation and access commands.
\item[file]
  Provides platform independent access to the file system, including
  subcommands to split and join file names by path component.
\item[glob] Returns a list of filenames matching a wildcard pattern.
\item[format, subst] Construct strings with variable substitutions.
\end{description}
Refer to the
\htmladdnormallinkfoot{\Tcl\ documentation}{https://www.tcl-lang.org/man}
for full details.

Notice that all the \Tcl\ command names are lowercase.  In contrast,
commands added by \app{pimake} have mixed-case names. The most common
\OOMMF\ commands you'll find in \fn{makerules.tcl} files are
\begin{description}
\item[MakeRule] Defines dependency rules, which is the principle goal
  of \fn{makerules.tcl} files. This command is documented in detail
  \hyperrefhtml{below}{below (Sec.~}{)}{sec:makerule}.
\item[Platform] Platform independent methods for common operations, with
  these subcommands:
\begin{description}
\item[Name] Identifier for current platform, e.g.,
  \cd{windows-x86\_64}, \cd{linux-x86\_64}, \cd{darwin}.
\item[Executables] Given a file stem returns the name for the
  corresponding executable on the current platform by prepending the
  platform directory and appending an execution suffix, if any. For
  example, \cd{Platform Executables varinfo} would return
  \fn{windows-x86\_64/varinfo.exe} on \Windows, and
  \fn{linux-x86\_64/varinfo} on \Linux.
\item[Objects] Similar to Platform Executables, but returns object file
  names; the object file suffix is \cd{.obj} on \Windows\ and \cd{.o} on
  \Linux\ and \MacOSX.
\item[Compile] Uses the compiler specified in the
 \fn{config/platform/<platform>.tcl} to compile the specified
 source code file (\cd{-src} option) into the named object file (\cd{-out}
 option).
\item[Link] Uses the linker specified in
 \fn{config/platform/<platform>.tcl} to link together the specified
 object files (\cd{-obj} option) into the named executable (\cd{-out}
 option).
\end{description}
\item[CSourceFile New] Creates an instance of the \cd{CSourceFile}
  class. The \cd{-inc} option to \cd{New} specifies directories to add
  to the search path for header files. \cd{CSourceFile} instances
  support these subcommands:
  \begin{description}
  \item[Dependencies] Dependency list for specified \Cplusplus\ source
    file consisting of the source file itself, header files included by
    \cd{\#include} statements in the source code files, and also any
    header files found by a recursive tracing of \cd{\#include}
    statements.  The header file search excludes system header files
    requested using angle-brackets, e.g., \cd{\#include <stdio.h>}. A
    source code file can speed the tracing process by placing a \cd{/*
      End includes */} comment following the last \cd{\#include}
    statement, as in this example from
    \fn{oommf/app/mmdisp/mmdispsh.cc}:
\begin{verbatim}
  /* FILE: mmdispsh.cc                 -*-Mode: c++-*-
   *
   * A shell program which includes Tcl commands needed to support a
   * vector display application.
   *
   */

  #include "oc.h"
  #include "vf.h"
  #include "mmdispcmds.h"

  /* End includes */
  ...
\end{verbatim}
   The \cd{/* End includes */} statement terminates the search for
   further \cd{\#include} statements in that file.
  \item[DepPath] List of directories containing files on which
    the specified \Cplusplus\ source file depends.
  \end{description}
\item[Recursive] Given a target, loads the \fn{makerules.tcl} file in
  each child directory of the current directory and executes the rule
  found there for the target. Primarily used with the default targets
  \cd{all}, \cd{configure}, \cd{clean}, \cd{mostlyclean}, \cd{objclean},
  \cd{maintainer-clean}, \cd{distclean}, and \cd{upgrade}. The default
  targets have an implicit rule to do nothing except recurse the action
  into the new child directories. If a \fn{makerules.tcl} file found in
  this manner has an explicit rule defined for the given target, then
  that rule is invoked instead of the implicit rule, and, unless the
  explicit rule makes a \cd{Recursive} call itself, the recursion on
  that directory branch will stop. As an example, the \fn{makerules.tcl}
  file in the \OOMMF\ root directory has the rule
\begin{verbatim}
  MakeRule Define {
    -targets   all
    -script    {Recursive all}
  }
\end{verbatim}
  All of \fn{makerules.tcl} files one level below \fn{oommf/pkg} and
  \fn{oommf/app} have ``\cd{all}'' targets that compile and link their
  corresponding libraries or executables. So
\begin{verbatim}
  tclsh oommf.tcl pimake all
\end{verbatim}
  run in the root \OOMMF\ directory will build all of those libraries
  and applications. In contrast, \fn{makerules.tcl} files under
  \fn{oommf/doc} do \textbf{not} have explicit \cd{all} targets, so the
  \cd{tclsh oommf.tcl pimake all} call has no effect in the \fn{oommf/doc/}
  subtree.

  On the other hand, the \fn{makerules.tcl} in directories under
  \fn{oommf/pkg/}, \fn{oommf/app/}, and \fn{oommf/doc/} \textbf{do} have
  explicit rules for the various \cd{clean} targets, so
\begin{verbatim}
  tclsh oommf.tcl pimake maintainer-clean
\end{verbatim}
  run from the \OOMMF\ root directory will be active throughout all
  three subtrees. The \cd{maintainer-clean} rules delete all files that
  can be regenerated from source, meaning object files, libraries,
  executables, and notably all the documentation files under
  \fn{oommf/doc/}. Building the \OOMMF\ documentation requires a working
  installation of
  \htmladdnormallinkfoot{\LaTeX}{https://www.latex-project.org} and
  either \htmladdnormallinkfoot{\LaTeX2HTML}{https://www.latex2html.org}
  or \htmladdnormallinkfoot{\LaTeXML}{http://dlmf.nist.gov/LaTeXML/}, so
  don't run the \cd{maintainer-clean} target unless you are prepared to
  rebuild the \OOMMF\ documentation!
\end{description}
The \Tcl\ source defining the \cd{MakeRule}, \cd{Platform},
\cd{CSourceFile}, and \cd{Recursive} commands can be found in the
\cd{oommf/app/pimake/} directory. Example use of these commands can be
found in the \htmlonlyref{following section}{sec:makerule}.

\section{The MakeRule command}\label{sec:makerule}
The \fn{makerules.tcl} files consist primarily of a collection of
\cd{MakeRule} commands surrounded by a sprinkling of \Tcl\ support
code. The order of the \cd{MakeRule} commands doesn't matter, except
that the first target in the file, usually \cd{all}, becomes the default
target. (The ``default'' target is the effective target if \app{pimake}
is run without specifying a target.)

The \cd{MakeRule} command supports a number of subcommands, but the
principle subcommand appearing in \fn{makerules.tcl} files is
\cd{Define}. This takes a single argument, which is a list of
option+value pairs, with valid options being \cd{-targets},
\cd{-dependencies}, and \cd{-script}. The value string for the
\cd{-targets} option is a list of one or more build targets. The targets
are usually files, in which case they must lie in the same directory or
a directory below the \fn{makerules.tcl} file. The \cd{-dependencies}
option is a list of one or more files or targets that the target depends
upon. The value to the \cd{-script} option is a \Tcl\ script that is run
if a target does not exist or if any of the file dependencies have a
newer timestamp than any of the targets. The dependency checking is done
recursively, that is, each dependency is checked to see if it up to date
with its own dependencies, and so on.  A target is out of date if it is
older than any of its dependencies, or the dependencies of the
dependencies, etc. If any of the dependencies is out of date with
respect to its own dependencies, then its script will be run during the
dependency resolution. The script associated with the original target is
only run after its dependency resolution is fully completed.

The following examples from \fn{oommf/app/omfsh/makerules.tcl}
should help flesh out the above description:
\begin{verbatim}
  MakeRule Define {
    -targets        [Platform Name]
    -dependencies   {}
    -script         {MakeDirectory [Platform Name]}
  }
\end{verbatim}
Here the target is the platform name, e.g., \fn{windows-x86\_64}, which
is a directory under the current working directory
\fn{oommf/app/omfsh/}. There are no dependencies to check, so the rule
script is run if and only if the directory \fn{windows-x86\_64} does not
exist. In that case the \OOMMF\ \cd{MakeDirectory} routine is called to
create it. This is an important rule because the compilation and linking
commands place their output into this directory, so it must exist before
those commands are run.

Next we look at a more complex rule that is really the bread and
butter of \fn{makerules.tcl}, a rule for compiling a \Cplusplus\ file:
\begin{verbatim}
  MakeRule Define {
    -targets        [Platform Objects omfsh]
    -dependencies   [concat [list [Platform Name]] \
                            [[CSourceFile New _ omfsh.cc] Dependencies]]
    -script         {Platform Compile C++ -opt 1 \
                             -inc [[CSourceFile New _ omfsh.cc] DepPath] \
                             -out omfsh -src omfsh.cc
                    }
  }
\end{verbatim}
In this example the target is the object file associated with the
stem \cd{omfsh}. On \Windows\ this would be
\fn{windows-x86\_64/omfsh.obj}. The dependencies are the platform
directory (e.g., \fn{windows-x86\_64/}), the file \fn{omfsh.cc}, and any
(non-system) files included by \fn{omfsh.cc}. Directory timestamps do
not affect the out-of-date computation, but directories will be
constructed by their \cd{MakeRule} if they don't exist.

Note that part of the \cd{-dependencies} list is
\begin{verbatim}
  [CSourceFile New _ omfsh.cc] Dependencies]
\end{verbatim}
As discussed \hyperrefhtml{earlier}{in Sec.~}{}{sec:anatomymakerules},
this command resolves to a list of all non-system \cd{\#include} header
files from \fn{omfsh.cc}, or header files found recursively from those
header files. The first part of \fn{omfsh.cc} is
\begin{verbatim}
  /* FILE: omfsh.cc                 -*-Mode: c++-*-
   *
   *      A Tcl shell extended by the OOMMF core (Oc) extension
   ...
   */

  /* Header files for system libraries */
  #include <cstring>

  /* Header files for the OOMMF extensions */
  #include "oc.h"
  #include "nb.h"
  #include "if.h"

  /* End includes */
  ...
\end{verbatim}
The header file \fn{cstring} is ignored by the dependency search because
it is specified inside angle brackets rather than double quotes. But the
\fn{oc.h}, \fn{nb.h}, and \fn{if.h} files are all considered. These
files are part of the \cd{Oc}, \cd{Nb}, and \cd{If} package libraries,
respectively, living in subdirectories under \fn{oommf/pkg/}. The file
\fn{oommf/pkg/oc/oc.h}, for example, will be checked for included files
in the same way, and so on. The full dependency tree can be quite
extensive. The \app{pimake} application supports a \cd{-d} option to
print out the dependency tree, e.g.,
\begin{verbatim}
  tclsh oommf.tcl pimake -cwd app/omfsh -d windows-x86_64/omfsh.obj
\end{verbatim}
This output can be helpful is diagnosing dependency issues.

The \verb+/* End includes */+ line terminates the \cd{\#include} file search
inside this file. It is optional but recommended as it will speed-up
dependency resolution.

If \fn{omfsh.obj} is older than any of its dependent files, then the
\Tcl\ script specified by the \cd{-script} option will be triggered. In
this case the script runs \cd{Platform Compile C++}, which is the
\Cplusplus\ compiler as specified by the
\fn{oommf/config/platforms/<platform>.tcl} file. In this command
\cd{-opt} enables compiler optimizations, \cd{-inc} supplements the
include search path for the compiler, \cd{-out omfsh} is the output
object file with name adjusted appropriately for the platform, and
\cd{-src omfsh.cc} specifies the \Cplusplus\ file to be compiled.

The rules for building executables and libraries from collections of
object modules are of a similar nature. See the various
\fn{makerules.tcl} files across the \OOMMF\ directory tree for examples.

In a normal rule, the target is a file and if the script is run it will
create or update the file. Thus, if \app{pimake} is run twice in
succession on the same target, the second run will not trigger the
script because the target will be up to date. In contrast, a
pseudo-target\index{pimake!pseudo-target} does not exist as a file on
the file system, and the associated script does not create the
pseudo-target. Since the pseudo-target never exists as a file, repeated
runs of \app{pimake} on the target will result in repeated runs of the
pseudo-target script.

Common pseudo-targets include \cd{all},
\cd{configure}, \cd{help}, and several \cd{clean} variants.  This last
example illustrates the chaining of \cd{clean} pseudo-targets to remove
constructed files.
\begin{verbatim}
  MakeRule Define {
    -targets         clean
    -dependencies    mostlyclean
  }

  MakeRule Define {
    -targets         mostlyclean
    -dependencies    objclean
    -script          {eval DeleteFiles [Platform Executables omfsh] \
                         [Platform Executables filtersh] \
                         [Platform Specific appindex.tcl]}
  }

  MakeRule Define {
    -targets         objclean
    -dependencies    {}
    -script          {
                      DeleteFiles [Platform Objects omfsh]
                      eval DeleteFiles \
                             [Platform Intermediate {omfsh filtersh}]
                     }
  }
\end{verbatim}
All three of these rules have targets that are non-existent files, so
all three are pseudo-targets. The first rule, for target \cd{clean}, has
no script so the script execution is a no-op. However, the dependencies
are still evaluated, which in this case means the rule for the target
\cd{mostlyclean} is checked. This rule has both a dependency and a
script. The dependencies are evaluated first, so the \cd{objclean}
script is called to delete the \cd{omfsh} object file and also any
intermediate files created as side effects of building the \app{omfsh}
and \app{filtersh} executables. Next the \cd{mostlyclean} script is run,
which deletes the \app{omfsh} and \app{filtersh} executables and also
the platform-specific \fn{appindex.tcl} file. Note that none of the
scripts create their target, so the targets will all remain
pseudo-targets.

\chapter{\OOMMF\ Variable Types and Macros}\label{sec:vartypes}
The following typedefs are defined in the
\fn{oommf/pkg/oc/{\it{platform}}/ocport.h} header file; this file is
created by the \app{pimake} build process (see
\fn{oommf/pkg/oc/procs.tcl}), and contains platform and machine
specific information.
\newcommand{\gbs}{\hspace{0.5em}}
\begin{itemize}
\item{\texttt{OC\_BOOL}} \gbs Boolean type, unspecified width.
\item{\texttt{OC\_BYTE}} \gbs Unsigned integer type exactly one byte wide.
\item{\texttt{OC\_CHAR}} \gbs Character type, may be signed or unsigned.
\item{\texttt{OC\_UCHAR}} \gbs Unsigned character type.
\item{\texttt{OC\_SCHAR}} \gbs Signed character type.  If \texttt{signed char}
  is not supported by a given compiler, then this falls back to a
  plain \texttt{char}, so use with caution.
\item{\texttt{OC\_INT2, OC\_INT4}} \gbs Signed integer with width of
  exactly 2, respectively 4, bytes.
\item{\texttt{OC\_INT2m, OC\_INT4m}} \gbs Signed integer with width of
  at least 2, respectively 4, bytes.  A type wider than the minimum
  may be specified if the wider type is handled faster by the
  particular machine.
\item{\texttt{OC\_UINT2, OC\_UINT4, OC\_UINT2m, OC\_UINT4m}} \gbs Unsigned
  integer versions of the preceding.
\item{\texttt{OC\_REAL4, OC\_REAL8}} \gbs Four byte, respectively eight
  byte, floating point variable.  Typically corresponds to \Cplusplus\
  ``float'' and ``double'' types.
\item{\texttt{OC\_REAL4m, OC\_REAL8m}} \gbs Floating point variable with
  width of at least 4, respectively 8, bytes.  A type wider than the minimum
  may be specified if the wider type is handled faster by the
  particular machine.
\item{\texttt{OC\_REALWIDE}} \gbs Widest type natively supported by the
  underlying hardware.  This is usually the \Cplusplus\ ``long
  double'' type, but may be overridden by the
\begin{center}
  \texttt{program\_compiler\_c++\_typedef\_realwide}
\end{center}
  option in the \fn{oommf/config/platform/{\it{platform}}.tcl} file.
\end{itemize}

The \fn{oommf/pkg/oc/{\it{platform}}/ocport.h} header file also
defines the following macros for use with the floating point variable
types:
\begin{itemize}
\item{\texttt{OC\_REAL8m\_IS\_DOUBLE}} \gbs True if \texttt{OC\_REAL8m} type
  corresponds to the \Cplusplus\ ``double'' type.
\item{\texttt{OC\_REAL8m\_IS\_REAL8}} \gbs True if \texttt{OC\_REAL8m} and
  \texttt{OC\_REAL8} refer to the same type.
\item{\texttt{OC\_REAL4\_EPSILON}} \gbs The smallest value that can be added to
  a \texttt{OC\_REAL4} value of ``1.0'' and yield a value different from
  ``1.0''.  For IEEE 754 compatible floating point, this should be
  \texttt{1.1920929e-007}.
\item{\texttt{OC\_SQRT\_REAL4\_EPSILON}}
    \gbs Square root of the preceding.
\item{\texttt{OC\_REAL8\_EPSILON}} \gbs The smallest value that can be added to
  a \texttt{OC\_REAL8} value of ``1.0'' and yield a value different from
  ``1.0''.  For IEEE 754 compatible floating point, this should be
  \texttt{2.2204460492503131e-016}.
\item{\texttt{OC\_SQRT\_REAL8\_EPSILON, OC\_CUBE\_ROOT\_REAL8\_EPSILON}}
    \gbs Square and cube roots of the preceding.
\item{\texttt{OC\_FP\_REGISTER\_EXTRA\_PRECISION}} \gbs True if
  intermediate floating point operations use a wider precision than
  the floating point variable type; notably, this occurs with some
  compilers on x86 hardware.
\end{itemize}

Note that all of the above macros have a leading ``\texttt{OC\_}''
prefix.  The prefix is intended to protect against possible name
collisions with system header files.  Versions of some of these macros
are also defined without the prefix; these definitions represent
backward support for existing \OOMMF\ extensions.  All new code
should use the versions with the ``\texttt{OC\_}'' prefix, and old code
should be updated where possible.  The complete list of deprecated
macros is:
\begin{quote}
\raggedright
\texttt{BOOL, UINT2m, INT4m, UINT4m,
    REAL4, REAL4m, REAL8, REAL8m, REALWIDE,
    REAL4\_EPSILON, REAL8\_EPSILON,
    SQRT\_REAL8\_EPSILON, CUBE\_ROOT\_REAL8\_EPSILON,
    FP\_REGISTER\_EXTRA\_PRECISION
}
\end{quote}

Macros for system identification:
\begin{itemize}
\item{\texttt{OC\_SYSTEM\_TYPE}} \gbs One of \texttt{OC\_UNIX} or
  \texttt{OC\_WINDOWS}.
\item{\texttt{OC\_SYSTEM\_SUBTYPE}} \gbs For unix systems, this is
    either \texttt{OC\_VANILLA} (general unix) or \texttt{OC\_DARWIN}
    (Mac OS X).  For Windows systems, this is generally
    \texttt{OC\_WINNT}, unless one is running out of a Cygwin shell,
    in which case the value is \texttt{OC\_CYGWIN}.
\end{itemize}

Additional macros and typedefs:
\begin{itemize}
\item{\texttt{OC\_POINTERWIDTH}} \gbs Width of pointer type, in bytes.
\item{\texttt{OC\_INDEX}} \gbs Typedef for signed array index type;
  typically the width of this (integer) type matches the width of the
  pointer type, but is in any event at least four bytes wide and not
  narrower than the pointer type.
\item{\texttt{OC\_UINDEX}} \gbs Typedef for unsigned version of
  OC\_INDEX.  It is intended for special-purpose use only.  In general,
  use OC\_INDEX where possible.
\item{\texttt{OC\_INDEX\_WIDTH}} \gbs Width of \texttt{OC\_INDEX} type.
\item{\texttt{OC\_BYTEORDER}} Either ``4321'' for little endian machines,
  or ``1234'' for big endian.
\item{\texttt{OC\_THROW(x)}} \gbs Throws a \Cplusplus\ exception with
  value ``x''.
\item{\texttt{OOMMF\_THREADS}} \gbs True for threaded (multi-processing) builds.
\item{\texttt{OC\_USE\_NUMA}} \gbs If true, then NUMA (non-uniform memory
  access) libraries are available.
\end{itemize}

% Environment for listing example MIF files. Filenames shouldn't be
% split, so use raggedright. If there is only one filename on the list,
% then set the default parameter to "Example", replacing the default
% value "Examples". Note: The \ignorespaces in the definition is
% necessary to gobble whitespace that otherwise appears when the
% optional argument is specified.
\newenvironment{ExampleMifs}[1][Examples]{
\begin{sloppypar}
\raggedright\textbf{#1:} \ignorespaces}{\end{sloppypar}}

\chapter{OOMMF eXtensible Solver}\label{sec:oxs}%
\index{Oxs}\index{application!Oxs}%
The Oxs (OOMMF eXtensible Solver) is an extensible micromagnetic
computation engine capable of solving problems defined on
three-dimensional grids of rectangular cells holding three-dimensional
spins.  There are two interfaces provided to Oxs: the interactive
interface
\hyperrefhtml{Oxsii}{Oxsii (Sec.~}{)}{sec:oxsii}
intended to be controlled primarily through a graphical
user interface, and the batch mode
\hyperrefhtml{Boxsi}{Boxsi (Sec.~}{)}{sec:boxsi}, which has extended
command line controls making it suitable for use in shell scripts.

Problem definition for Oxs is accomplished using input files in the
\hyperrefhtml{\MIF~2 format}{\MIF~2 format (Sec.~}{)}{sec:mif2format}.
This is an extensible format; the standard OOMMF modules are
\hyperrefhtml{documented below}{documented in Sec.~}{ below}{sec:oxsext}.
Files in the \htmlonlyref{{\MIF} 1.1}{sec:mif1format} and
\htmlonlyref{\MIF~1.2}{sec:mif12format}
formats are also accepted.  They are
passed to \hyperrefhtml{\app{mifconvert}}{\app{mifconvert}
(Sec.~}{)}{sec:mifconvert} for conversion to \MIF~2 format
``on-the-fly.''

Note on \Tk\ dependence: Some \MIF~2 problem descriptions rely on
external image files\index{file!mask}\index{file!bitmap}; examples
include those using the
\htmlonlyref{\cd{Oxs\_ImageAtlas} class}{html:oxsImageAtlas}%
\latex{ (Sec.~\ref{sec:oxsAtlases})}, or those using the \MIF~2
\htmlonlyref{\cd{ReadFile}}{html:ReadFile} command with the \cd{image}
translation specification\latex{ (Sec.~\ref{sec:mif2ExtensionCommands})}.
If the image file is not in the PPM P3 (text) format, then the
\app{any2ppm}\index{application!any2ppm} application may be
launched to read and convert the file.  Since \app{any2ppm}
requires\index{requirement!Tk}\index{requirement!display} \Tk, at the
time the image file is read a valid display must be available.  See the
{\hyperrefhtml{\app{any2ppm} documentation}{\app{any2ppm} documentation
(Sec.~}{)}{sec:any2ppm}} for details.

\section{OOMMF eXtensible Solver Interactive Interface:
Oxsii}\label{sec:oxsii}%
\index{simulation~3D!interactive}\index{application!Oxsii}

\begin{center}
\includepic{oxsii-ss}{Oxsii Screen Shot}
\end{center}

\starssechead{Overview}
The application \app{Oxsii} is the graphical, interactive user interface
to the Oxs micromagnetic computation engine.  Within the
\hyperrefhtml{\OOMMF\ architecture}{\OOMMF\ architecture (see
Ch.~}{)}{sec:arch}, \app{Oxsii} is both a server and a client
application. \app{Oxsii} is a client of data table display and storage
applications, and vector field display and storage applications.
\app{Oxsii} is the server of a solver control service for which the only
client is \hyperrefhtml{\app{mmLaunch}}{\app{mmLaunch}
(Ch.~}{)}{sec:mmlaunch}\index{application!mmLaunch}.  It is through
this service that \app{mmLaunch} provides a user interface window (shown
above) on behalf of \app{Oxsii}.

A micromagnetic problem is communicated to \app{Oxsii} via a
\htmlonlyref{\MIF~2 file}{sec:mif2format},
which defines a collection of \htmlonlyref{Oxs\_Ext objects}{sec:oxsext}
that comprise the problem model.  The problem description includes a
segmentation of the lifetime of the simulation into stages.  Stages mark
discontinuous changes in model attributes, such as applied fields, and
also serve to mark coarse grain simulation progress.  \app{Oxsii}
provides controls to advance the simulation, stopping between
iterations, between stages, or only when the run is complete.
Throughout the simulation, the user may save and display intermediate
results, either interactively or via scheduling based on iteration and
stage counts.

Problem descriptions in the \htmlonlyref{\MIF~1.1}{sec:mif1format} and
\htmlonlyref{\MIF~1.2}{sec:mif12format} formats can also be input.  They are
automatically passed to \hyperrefhtml{\app{mifconvert}}{\app{mifconvert}
(Sec.~}{)}{sec:mifconvert} for implicit conversion to \MIF~2 format.

\starssechead{Launching}
\app{Oxsii} may be started either by selecting the
\btn{Oxsii} button on \htmlonlyref{mmLaunch}{sec:mmlaunch}, or from the
command line via
\begin{verbatim}
tclsh oommf.tcl oxsii [standard options] [-exitondone <0|1>] \
   [-logfile logname] [-loglevel level] [-nice <0|1>] [-nocrccheck <0|1>] \
   [-numanodes nodes] [-outdir dir] [-parameters params] [-pause <0|1>] \
   [-restart <0|1|2>] [-restartfiledir dir] [-threads count] [miffile]
\end{verbatim}
where
\begin{description}
\item[\optkey{-exitondone \boa 0\pipe 1\bca}]
  Whether to exit after solution of the problem is complete.
  Default is to simply await the interactive selection
  of another problem to be solved.
\item[\optkey{-logfile logname}]
  Write log and error messages to file \textit{logname}.  The default log
  file is \fn{oommf/oxsii.errors}.
\item[\optkey{-loglevel level}]
  Controls the detail level of log messages, with larger values of
  \textit{level} producing more output.  Default value is 1.
\item[\optkey{-nice \boa 0\pipe 1\bca}]
  If enabled (i.e., 1), then the program will drop its scheduling
  priority after startup.  The default is 1, i.e., to yield scheduling
  priority to other applications.
\item[\optkey{-nocrccheck \boa 0\pipe 1\bca}]
  On simulation restarts, the CRC
  \hyperrefhtml{CRC}{CRC (Sec.~}{)}{sec:crc32}
  of the \MIF\ file is normally compared against the CRC of the original
  \MIF\ file as recorded in the restart file.  If the CRCs don't match then
  an error is thrown to alert the user that the \MIF\ file has changed.  If
  this option is enabled (i.e., 1) then the check is disabled.
\item[\optkey{-numanodes \boa nodes\bca}]\index{NUMA|(}
  \index{parallelization}
  This option is available on \hyperrefhtml{NUMA-aware}{NUMA-aware
  (Sec.~}{)}{sec:install.parallel} builds of Oxs.  The \textit{nodes}
  parameter must be either a comma separated list of 0-based node
  numbers, the keyword ``auto'', or the keyword ``none''.  In the first
  case, the numbers refer to memory nodes.  These must be passed on the
  command line as a single parameter, so either insure there are no
  spaces in the list, or else protect the spaces with outlying quotes.
  For example, \cd{-numanodes 2,4,6} or \cd{-numanodes "2, 4, 6"}.
  Threads are assigned to the nodes in order, in round-robin fashion.
  The user can either assign all the system nodes to the \app{Oxsii}
  process, or may restrict \app{Oxsii} to run on a subset of the nodes.
  In this way the user may reserve specific processing cores for other
  processes (or other instances of \app{Oxsii}).  Although it varies by
  system, typically there are multiple processing cores associated with
  each memory node.  If the keyword ``auto'' is selected, then the
  threads are assigned to a fixed node sequence that spans the entire
  list of memory nodes.  If the keyword ``none'' is selected, then
  threads are not tied to nodes by \app{Oxsii}, but are instead assigned
  by the operating system.  In this last case, over time the operating
  system is free to move the threads among processors.  In the other two
  cases, each thread is tied to a particular node for the lifetime of
  the \app{Oxsii} instance.  See also the discussion
  on \htmlonlyref{threading considerations}{html:threadconsider} in the
  Boxsi documentation.

  The default value for \textit{nodes} is ``none'', which allows the
  operating system to assign and move threads based on overall system
  usage.  This is also the behavior obtained when the Oxs build is not
  NUMA-aware.  On the other hand, if a machine is dedicated primarily
  to running one instance of \app{Oxsii}, then \app{Oxsii} will
  likely run fastest if the thread count is set to the number of
  processing cores on the machine, and \textit{nodes} is set to
  ``auto''.  If you want to run multiple copies of \app{Oxsii}
  simultaneously, or run \app{Oxsii} in parallel with some other
  application(s), then set the thread count to a number smaller than
  the number of processing cores and restrict \app{Oxsii} to some
  subset of the memory nodes with the \cd{-numanodes} option and an
  explicit nodes list.

  The default behavior is modified (in increasing order of priority) by the
  \cd{numanodes} setting in the active \fn{oommf/config/platform/} platform
  file, by the \cd {numanodes} setting in the \fn{oommf/config/options.tcl}
  or \fn{oommf/config/local/options.tcl} file, or by the environment variable
  \cd{OOMMF\_NUMANODES}\index{environment~variables!OOMMF\_NUMANODES}.  The
  \cd{-numanodes} command line option, if any, overrides all.\index{NUMA|)}
\item[\optkey{-outdir dir}]
  Specifies the directory where output files are written by
  \app{mmArchive}.  This option is useful when the default output
  directory is inaccessible or slow. The environment variable
  \cd{OOMMF\_OUTDIR}\index{environment~variables!OOMMF\_OUTDIR} sets the
  default output directory.  If \cd{OOMMF\_OUTDIR} is set to the empty
  string, or not set at all, then the default is the directory holding
  the \MIF\ file.  If this option is specified on the command line, or
  if \cd{OOMMF\_OUTDIR} is set, then the \app{Oxsii}
  \cd{File\pipe Load\ldots} dialog box includes a control to change
  the output directory.
\item[\optkey{-parameters params}]
  Sets \hyperrefhtml{\MIF~2}{\MIF~2 (Sec.~}{)}{sec:mif2format} file
  parameters.  The \textit{params} argument should be a list with an
  even number of arguments, corresponding to name + value pairs.  Each
  ``name'' must appear in a
  \htmlonlyref{\cd{Parameter}}{html:mif2parameter}
  statement\latex{ (Sec.~\ref{sec:mif2ExtensionCommands})} in the input
  \MIF\ file.  The entire name + value list must be quoted so it is
  presented to \app{Oxsii} as a single item on the command line.  For
  example, if \cd{A} and \cd{Ms} appeared in \cd{Parameter} statements
  in the \MIF\ file, then an option like
\begin{verbatim}
   -parameters "A 13e-12 Ms 800e3"
\end{verbatim}
  could be used to set \cd{A} to 13e-12 and Ms to 800e3.  The quoting
  mechanism is specific to the shell/operating system; refer to your system
  documentation for details.
\item[\optkey{-pause \boa 0\pipe 1\bca}]
  If disabled (i.e., 0), then the program automatically shifts into
  ``Run'' mode after loading the specified \textit{miffile}. The default
  is 1, i.e., to ``Pause'' once the problem is loaded. This switch has
  no effect if miffile is not specified.
\item[\optkey{-restart \boa 0\pipe 1\bca\index{simulation~3D!restarting}}]
  Controls the initial setting of the restart flag, and thereby
  the load restart behavior of any \cd{miffile} specified on the command
  line.  The restart flag is described in the
  \arbtargetlink{Controls section below}{Controls section below
  (page~}{)}{PToxsiirestartflag}. The default value is 0, i.e., no
  restart.
\item[\optkey{-restartfiledir dir}]
  Specifies the directory where restart files are written.
  The default is determined by the environment variable
  \cd{OOMMF\_RESTARTFILEDIR}\index{environment~variables!OOMMF\_RESTARTFILEDIR},
  or if this is not set then by
  \cd{OOMMF\_OUTDIR}\index{environment~variables!OOMMF\_OUTDIR}.  If
  neither environment variable is set then the default is the
  directory holding the \MIF\ file.  Write access is required to the
  restart file directory.  Also, you may want to consider whether the
  restart files should be written to a local temporary directory or a
  network mount.
\item[\optkey{-threads \boa count\bca}]
  \index{parallelization}
  The option is available on \hyperrefhtml{threaded}{threaded
  (Sec.~}{)}{sec:install.parallel} builds of Oxs.  The \textit{count}
  parameter is the number of threads to run.  The default count value is
  set by the \cd{oommf\_thread\_count} value in
  the \fn{config/platforms/} file for your platform, but may be
  overridden by
  the \cd{OOMMF\_THREADS}\index{environment~variables!OOMMF\_THREADS}
  environment variable or this command line option.  In most cases the
  default count value will equal the number of processing cores on the
  system; this can be checked via the command \cd{tclsh oommf.tcl
  +platform}.
\item[\optkey{miffile}]
  Load and solve the problem found in \textit{miffile}, which must be
  either in the \MIF~2 format, or convertible to that format by
  \htmlonlyref{\app{mifconvert}}{sec:mifconvert}.  Optional.
\end{description}
All the above switches are optional.

Since \app{Oxsii}\index{mmLaunch~user~interface} does not present
any user interface window of its own, it depends on
\app{mmLaunch}\index{application!mmLaunch} to provide an interface on
its behalf.  The entry for an instance of \app{Oxsii} in the
\btn{Threads}\index{threads} column of any running copy of
\app{mmLaunch} has a checkbutton next to it.  This button toggles the
presence of a user interface window through which the user may control
that instance of \app{Oxsii}.


\starssechead{Inputs}
Unlike \hyperrefhtml{\app{mmSolve2D}}{\app{mmSolve2D}
(Sec.~}{)}{sec:mmsolve2d},
\app{Oxsii} loads problem specifications
directly from disk (via the \cd{File\pipe Load\ldots} menu selection),
rather than through
\hyperrefhtml{\app{mmProbEd}}{\app{mmProbEd} (Ch.~}{)}{sec:mmprobed} or
\hyperrefhtml{\app{FileSource}}{\app{FileSource}
(Ch.~}{)}{sec:filesource}.  Input files for \app{Oxsii} must be either
in the \hyperrefhtml{\MIF~2}{\MIF~2 (Sec.~}{)}{sec:mif2format}
format, or convertible to that format by the command line tool
\hyperrefhtml{\app{mifconvert}}{\app{mifconvert}
(Sec.~}{)}{sec:mifconvert}.  There are sample \MIF~2 files in the
directory \cd{oommf/app/oxs/examples}.  \MIF\ files may be edited with
any plain text editor.

\starssechead{Outputs\label{html:oxsiioutputs}}
Once a problem has been loaded, the scroll box under the Output
heading will fill with a list of available outputs.  The contents of
this list will depend upon the \cd{Oxs\_Ext} objects specified in the
input \MIF\ file.  Refer to the documentation for those objects for
\hyperrefhtml{specific details}{specific details (Sec.~}{)}{sec:oxsext}.
To send output from \app{Oxsii} to another \OOMMF\ application, highlight the
desired selection under the Output heading, make the corresponding
selection under the Destination heading, and then specify the output
timing under the Schedule heading.  Outputs may be scheduled by the
step or stage, and may be sent out interactively by pressing the
\btn{Send} button.  The initial output configuration is set by
\htmlonlyref{\cd{Destination}}{html:destinationCmd} and
\htmlonlyref{\cd{Schedule}}{html:scheduleCmd} commands in the input
\MIF\ file\latex{ (Sec.~\ref{sec:mif2ExtensionCommands})}.

Outputs fall under two general categories: scalar (single-valued)
outputs and vector field outputs.  The scalar outputs are grouped
together as the \cd{DataTable} entry in the Output scroll box.
Scalar outputs include such items as total and component energies,
average magnetization, stage and iteration counts, max torque values.
When the \cd{DataTable} entry is selected, the Destination box will
list all \OOMMF\ applications accepting datatable-style input, i.e., all
currently running
\hyperrefhtml{\app{mmDataTable}}{\app{mmDataTable}
(Ch.~}{)}{sec:mmdatatable}\index{application!mmDataTable},
\hyperrefhtml{\app{mmGraph}}{\app{mmGraph}
(Ch.~}{)}{sec:mmgraph}\index{application!mmGraph}, and
\hyperrefhtml{\app{mmArchive}}{\app{mmArchive}
(Ch.~}{)}{sec:mmarchive}\index{application!mmArchive} processes.

The vector field outputs include pointwise magnetization, various total
and partial magnetic fields, and torques.  Unlike the scalar
outputs, the vector field outputs are listed individually in the Output
scroll box.  Allowed destinations for vector field output are running
instances of
\hyperrefhtml{\app{mmDisp}}{\app{mmDisp}
(Ch.~}{)}{sec:mmdisp}\index{application!mmDisp} and
\hyperrefhtml{\app{mmArchive}}{\app{mmArchive}
(Ch.~}{)}{sec:mmarchive}\index{application!mmArchive}.  Caution is
advised when scheduling vector field output, especially with large
problems, because the output may run many megabytes.

\starssechead{Controls}
The \btn{File} menu button holds five entries: Load, Show Console, Close
Interface, Clear Schedule and Exit Oxsii.  \btn{File\pipe Load\ldots}
launches a dialog box that allows the user to select an input \MIF\
problem description file.  \btn{File\pipe Show~Console} brings up a
\Tcl\ shell console running off the \app{Oxsii} interface \Tcl\
interpreter.  This console is intended primary for debugging purposes.
In particular, output from
\MIF\ \htmlonlyref{Report}{html:MifReport} commands%
\latex{ (Sec.~\ref{sec:mif2ExtensionCommands})}
may be viewed here.  \btn{File\pipe Close Interface} will remove the
interface window from the display, but leaves the solver running.  This
effect may also be obtained by deselecting the
\app{Oxsii} interface button in the \btn{Threads} list in
\htmlonlyref{\app{mmLaunch}}{sec:mmlaunch}.
\btn{File\pipe Clear Schedule} will disable all currently active
output schedules, exactly as if the user clicked through the interactive
schedule interface one output and destination at a time and disabled
each schedule-enabling checkbutton.
The final entry,
\btn{File\pipe Exit Oxsii}, terminates the \app{Oxsii} solver and closes the
interface window.

\arbtarget{The}{PToxsiirestartflag}
\btn{Options} menu holds two entries: Clear Schedule and Restart
Flag.  The first clears all Step and Stage selections from the active
output schedules, exactly as if the user clicked through the interactive
schedule interface one output and destination at a time and disabled
each schedule-enabling checkbutton.  This control can be used after
loading a problem to override the effect of any \cd{Schedule} commands
in the \MIF\ file.  The restart flag controls problem load behavior.  In
normal usage, the restart flag is not set and the selected problem loads
and runs from the beginning.  Conversely, if the restart flag is set,
then when a problem is loaded a check is made for a restart (checkpoint)
file.  If the checkpoint file is not found, then an error is raised.
Otherwise, the information in the checkpoint file is used to resume the
problem from the state saved in that file.  The restart flag can be set
from the Options menu, the \cd{File\pipe Load} dialog box, or from the
command line.  See the Oxs\_Driver documentation\HTMLoutput{ for
information on }\arbtargetlink{checkpoint files}{,
Sec.~\ref{sec:oxsDrivers} page~}{, for information on checkpoint
files}{PToxsdrivercheckpoint}.

The \btn{Help} menu provides the usual help facilities.

The row of buttons immediately below the menu bar provides simulation
progress control.  These buttons become active once a problem has
been loaded.  The first button, \btn{Reload}, re-reads the most recent
problem \MIF\ input file, re-initializes the solver, and pauses.
\btn{Reset} is similar, except the file is not re-read.  The remaining
four buttons, \btn{Run},
\btn{Relax}, \btn{Step} and \btn{Pause} place the solver into one of
four \textit{run-states}.  In the Pause state, the solver sits idle
awaiting further instructions.  If \btn{Step} is selected, then the
solver will move forward one iteration and then Pause.  In Relax mode,
the solver takes at least one step, and then runs until it reaches a
stage boundary, at which point the solver is paused.  In Run mode, the
solver runs until the end of the problem is reached.  Interactive output
is available in all modes; the scheduled outputs occur appropriately as
the step and stage counts advance.

Directly below the progress control buttons are two display lines,
showing the name of the input \MIF\ file and the current run-state.
Below the run-state \cd{Status} line is the stage display and control
bar.  The simulation stage may be changed at any time by dragging the
scroll bar or by typing the desired stage number into the text display
box to the left of the scroll bar.  Valid stage numbers are integers
from 0 to $N-1$, where $N$ is the number of stages specified by the
\MIF\ input file.

\starssechead{Details}
The simulation model construction is governed by the Specify blocks in
the input \MIF\ file.  Therefore, all aspects of the simulation are
determined by the specified
\hyperrefhtml{Oxs\_Ext classes}{Oxs\_Ext classes (Sec.~}{)}{sec:oxsext}.
Refer to the appropriate Oxs\_Ext class documentation for simulation and
computational details.

%%%%%%%%%%%%%%%%%%%%%%%%%%%%%%%%%%%%%%%%%%%%%%%%%%%%%%%%%%%%%%%%%%%%%%%%

\section{OOMMF eXtensible Solver Batch Interface: boxsi}\label{sec:boxsi}%
\index{simulation~3D!batch}\index{application!boxsi}

\begin{center}
\includepic{boxsi-ss}{boxsi Screen Shot}
\end{center}

\starssechead{Overview}
The application \app{Boxsi} provides a batch mode interface to the Oxs
micromagnetic computation engine.  A restricted graphical interface is
provided, but \app{Boxsi} is primarily intended to be controlled by
command line arguments, and launched by the user either directly from
the shell prompt or from inside a batch file.

Within the \hyperrefhtml{\OOMMF\ architecture}{\OOMMF\ architecture (see
Ch.~}{)}{sec:arch}, \app{Boxsi} is both a server and a client
application. It is a client of data table display and storage
applications, and vector field display and storage applications.
\app{Boxsi} is the server of a solver control service for which the only
client is \hyperrefhtml{\app{mmLaunch}}{\app{mmLaunch}
(Ch.~}{)}{sec:mmlaunch}\index{application!mmLaunch}.  It is through
this service that \app{mmLaunch} provides a user interface window (shown
above) on behalf of \app{Boxsi}.

A micromagnetic problem is communicated to \app{Boxsi} through a
\htmlonlyref{\MIF~2 file}{sec:mif2format} specified on the command line
and loaded from disk.  The \MIF~1.x formats are also accepted; they are
converted to the \MIF~2 format by an automatic call to
\hyperrefhtml{\app{mifconvert}}{\app{mifconvert}
(Sec.~}{)}{sec:mifconvert}.

\starssechead{Launching}
\app{Boxsi} must be started from the command line.  The syntax is
\begin{verbatim}
tclsh oommf.tcl boxsi [standard options] [-exitondone <0|1>] [-kill tags] \
   [-logfile logname] [-loglevel level] [-nice <0|1>] [-nocrccheck <0|1>] \
   [-numanodes nodes] [-outdir dir] [-parameters params] [-pause <0|1>] \
   [-regression_test flag] [-regression_testname basename] \
   [-restart <0|1|2>] [-restartfiledir dir] [-threads count] miffile
\end{verbatim}
where
\begin{description}
\item[\optkey{-exitondone \boa 0\pipe 1\bca}]
  Whether to exit after solution of the problem is complete, or to
  await the interactive selection of the \btn{File\pipe Exit} command.
  The default is 1, i.e., automatically exit when done.
\item[\optkey{-kill tags}]
  On termination, sends requests to other applications to
  shutdown too.  The \textit{tags} argument should be either
  a list of destination tags (which are declared by
  \htmlonlyref{\cd{Destination} commands}{html:destinationCmd}\latex{,
  Sec.~\ref{sec:mif2ExtensionCommands}}) from the input \MIF\
  file, or else the keyword \cd{all}, which is interpreted to mean all
  the destination tags.
\item[\optkey{-logfile logname}]
  Write log and error messages to file \textit{logname}.  The default log
  file is \fn{oommf/boxsi.errors}.
\item[\optkey{-loglevel level}]
  Controls the detail level of log messages, with larger values of
  \textit{level} producing more output.  Default value is 1.
\item[\optkey{-nice \boa 0\pipe 1\bca}]
  If enabled (i.e., 1), then the program will drop its scheduling
  priority after startup.  The default is 0, i.e., to retain its
  original scheduling priority.
\item[\optkey{-nocrccheck \boa 0\pipe 1\bca}]
  On simulation restarts, the CRC
  \hyperrefhtml{CRC}{CRC (Sec.~}{)}{sec:crc32}
  of the \MIF\ file is normally compared against the CRC of the original
  \MIF\ file as recorded in the restart file.  If the CRCs don't match then
  an error is thrown to alert the user that the \MIF\ file has changed.  If
  this option is enabled (i.e., 1) then the check is disabled.
\item[\optkey{-numanodes \boa nodes\bca}]\index{NUMA|(}
  \index{parallelization}
  This option is available on \hyperrefhtml{NUMA-aware}{NUMA-aware
  (Sec.~}{)}{sec:install.parallel} builds of Oxs.  The \textit{nodes}
  parameter must be either a comma separated list of 0-based node
  numbers, the keyword ``auto'', or the keyword ``none''.  In the first
  case, the numbers refer to memory nodes.  These must be passed on the
  command line as a single parameter, so either insure there are no
  spaces in the list, or else protect the spaces with outlying quotes.
  For example, \cd{-numanodes 2,4,6} or \cd{-numanodes "2, 4, 6"}.
  Threads are assigned to the nodes in order, in round-robin fashion.
  The user can either assign all the system nodes to the \app{Boxsi}
  process, or may restrict \app{Boxsi} to run on a subset of the nodes.
  In this way the user may reserve specific processing cores for other
  processes (or other instances of \app{Boxsi}).  Although it varies by
  system, typically there are multiple processing cores associated with
  each memory node.  If the keyword ``auto'' is selected, then the
  threads are assigned to a fixed node sequence that spans the entire
  list of memory nodes.  If the keyword ``none'' is selected, then
  threads are not tied to nodes by \app{Boxsi}, but are instead assigned
  by the operating system.  In this last case, over time the operating
  system is free to move the threads among processors.  In the other two
  cases, each thread is tied to a particular node for the lifetime of
  the \app{Boxsi} instance.  See also the discussion
  on \htmlonlyref{threading considerations}{html:threadconsider} below.

  The default value for \textit{nodes} is ``none'', which allows the
  operating system to assign and move threads based on overall system
  usage.  This is also the behavior obtained when the Oxs build is not
  NUMA-aware.  On the other hand, if a machine is dedicated primarily
  to running one instance of \app{Boxsi}, then \app{Boxsi} will
  likely run fastest if the thread count is set to the number of
  processing cores on the machine, and \textit{nodes} is set to
  ``auto''.  If you want to run multiple copies of \app{Boxsi}
  simultaneously, or run \app{Boxsi} in parallel with some other
  application(s), then set the thread count to a number smaller than
  the number of processing cores and restrict \app{Boxsi} to some
  subset of the memory nodes with the \cd{-numanodes} option and an
  explicit nodes list.

  The default behavior is modified (in increasing order of priority) by the
  \cd{numanodes} setting in the active \fn{oommf/config/platform/} platform
  file, by the \cd {numanodes} setting in the \fn{oommf/config/options.tcl}
  or \fn{oommf/config/local/options.tcl} file, or by the environment variable
  \cd{OOMMF\_NUMANODES}\index{environment~variables!OOMMF\_NUMANODES}.  The
  \cd{-numanodes} command line option, if any, overrides all.\index{NUMA|)}
\item[\optkey{-outdir dir}]
  Specifies the directory where output files are written by
  \app{mmArchive}.  This option is useful when the default output
  directory is inaccessible or slow. The environment variable
  \cd{OOMMF\_OUTDIR}\index{environment~variables!OOMMF\_OUTDIR} sets the
  default output directory.  If \cd{OOMMF\_OUTDIR} is set to the empty
  string, or not set at all, then the default is the directory holding
  the \MIF\ file.
\item[\optkey{-parameters params}]
  Sets \hyperrefhtml{\MIF~2}{\MIF~2 (Sec.~}{)}{sec:mif2format} file
  parameters.  The \textit{params} argument should be a list with an
  even number of arguments, corresponding to name + value pairs.  Each
  ``name'' must appear in a
  \htmlonlyref{\cd{Parameter}}{html:mif2parameter}
  statement\latex{ (Sec.~\ref{sec:mif2ExtensionCommands})} in the input
  \MIF\ file.  The entire name + value list must be quoted so it is
  presented to \app{Boxsi} as a single item on the command line.  For
  example, if \cd{A} and \cd{Ms} appeared in \cd{Parameter} statements
  in the \MIF\ file, then an option like
\begin{verbatim}
   -parameters "A 13e-12 Ms 800e3"
\end{verbatim}
  could be used to set \cd{A} to 13e-12 and Ms to 800e3.  The quoting
  mechanism is specific to the shell/operating system; refer to your system
  documentation for details.
\item[\optkey{-pause \boa 0\pipe 1\bca}]
  If enabled (i.e., 1), then the program automatically pauses after
  loading the specified problem file.  The default is 0, i.e., to
  automatically move into ``Run'' mode once the problem is loaded.
\item[\optkey{-regression\_test flag}]
  This option is used internally by the
  \hyperrefhtml{\app{oxsregression}}{\app{oxsregression} (Sec.~}{)}{sec:oxsregression}
  command line utility to run regression tests.  Default value is 0 (no
  test).
\item[\optkey{-regression\_testname basename}]
  This option is used internally by the
  \hyperrefhtml{\app{oxsregression}}{\app{oxsregression} (Sec.~}{)}{sec:oxsregression}
  command line utility to control temporary file names during regression
  testing.
\item[\optkey{-restart \boa 0\pipe 1\pipe 2\bca\index{simulation~3D!restarting}}]
  If the restart option is 0 (the default), then the problem loads and
  runs from the beginning.  If set to 1, then when loading the problem a
  check is made for a pre-existing restart (checkpoint) file.  If one is
  found, then the problem resumes from the state saved in that file.  If
  no checkpoint file is found, then an error is raised.  If the restart
  option is set to 2, then a checkpoint file is used if one can be
  found, but if not then the problem loads and runs from the beginning
  without raising an error.  See the Oxs\_Driver
  documentation\HTMLoutput{ for information on
  }\arbtargetlink{checkpoint files}{, Sec.~\ref{sec:oxsDrivers} page~}{,
  for information on checkpoint files}{PToxsdrivercheckpoint}.
\item[\optkey{-restartfiledir dir}]
  Specifies the directory where restart files are written.
  The default is determined by the environment variable
  \cd{OOMMF\_RESTARTFILEDIR}\index{environment~variables!OOMMF\_RESTARTFILEDIR},
  or if this is not set then by
  \cd{OOMMF\_OUTDIR}\index{environment~variables!OOMMF\_OUTDIR}.  If
  neither environment variable is set then the default is the
  directory holding the \MIF\ file.  Write access is required to the
  restart file directory.  Also, you may want to consider whether the
  restart files should be written to a local temporary directory or a
  network mount.
\item[\optkey{-threads \boa count\bca}]
  \index{parallelization}
  The option is available on \hyperrefhtml{threaded}{threaded
  (Sec.~}{)}{sec:install.parallel} builds of Oxs.  The \textit{count}
  parameter is the number of threads to run.  The default count value is
  set by the \cd{oommf\_thread\_count} value in
  the \fn{config/platforms/} file for your platform, but may be
  overridden by
  the \cd{OOMMF\_THREADS}\index{environment~variables!OOMMF\_THREADS}
  environment variable or this command line option.  In most cases the
  default count value will equal the number of processing cores on the
  system; this can be checked via the command \cd{tclsh oommf.tcl
  +platform}.
\item[\optkey{miffile}]
  Load and solve the problem found in {\em miffile}, which must be
  either in the \MIF~2 format, or convertible to that format by
  \htmlonlyref{\app{mifconvert}}{sec:mifconvert}.  Required.
\end{description}

Although \app{Boxsi}\index{mmLaunch~user~interface} cannot be
launched by \app{mmLaunch}\index{application!mmLaunch}, nonetheless
a limited graphical interactive interface for \app{Boxsi} is provided
through \app{mmLaunch}, in the same manner as is done for \app{Oxsii}.
Each running instance of \app{Boxsi} is included in the
\btn{Threads}\index{threads} list of \app{mmLaunch}, along with a
checkbutton.  This button toggles the presence of a user interface
window.

\starssechead{Inputs}
\app{Boxsi} loads problem specifications directly from disk as
requested on the command line.  The format for these files is
the \hyperrefhtml{\MIF~2}{\MIF~2 (Sec.~}{)}{sec:mif2format} format,
the same as used by the \app{Oxsii} interactive interface.  The
\htmlonlyref{\MIF~1.1}{sec:mif1format} and
\htmlonlyref{\MIF~1.2}{sec:mif12format}
formats used by the
2D solver \htmlonlyref{\app{mmSolve2D}}{sec:mmsolve2d} can also be input
to \app{Boxsi}, which will automatically call the command line tool
\hyperrefhtml{\app{mifconvert}}{\app{mifconvert}
(Sec.~}{)}{sec:mifconvert} to convert from the \MIF~1.x format to the
\MIF~2 format ``on-the-fly.''  Sample \MIF~2 files can be found in
the directory \cd{oommf/app/oxs/examples}.

\starssechead{Outputs}
The lower panel of the \app{Boxsi} interactive interface presents
Output, Destination, and Schedule sub-windows that display the current
output configuration and allow interactive modification of that
configuration.  These controls are identical to those in the \app{Oxsii}
user interface; refer to the
\htmlonlyref{\app{Oxsii} documentation}{html:oxsiioutputs}\latex{
(Sec.~\ref{sec:oxsii})} for details.
The only difference between \app{Boxsi} and \app{Oxsii} with
respect to outputs is that in practice \app{Boxsi} tends to rely
primarily on
\htmlonlyref{\cd{Destination}}{html:destinationCmd} and
\htmlonlyref{\cd{Schedule}}{html:scheduleCmd} commands in the input
\MIF\ file\latex{ (Sec.~\ref{sec:mif2ExtensionCommands})}
to setup the output configuration.  The interactive output interface is
used for incidental runtime monitoring of the job.

\starssechead{Controls}
The runtime controls provided by the \app{Boxsi} interactive interface
are a restricted subset of those available in the \app{Oxsii} interface.
If the runtime controls provided by \app{Boxsi} are found to be
insufficient for a given task, consider using \app{Oxsii} instead.

The \btn{File} menu holds 4 entries: Show Console, Close
Interface, Clear Schedule, and Exit Oxsii.  \btn{File\pipe Show~Console}
brings up a
\Tcl\ shell console running off the \app{Boxsi} interface \Tcl\
interpreter.  This console is intended primary for debugging purposes.
\btn{File\pipe Close Interface} will remove the interface window from
the display, but leaves the solver running.  This effect may also be
obtained by deselecting the
\app{Boxsi} interface button in the \btn{Threads} list in
\htmlonlyref{\app{mmLaunch}}{sec:mmlaunch}.
\btn{File\pipe Clear Schedule} will disable all currently active
output schedules, exactly as if the user clicked through the interactive
schedule interface one output and destination at a time and disabled
each schedule-enabling checkbutton.
The final entry,
\btn{File\pipe Exit Boxsi}, terminates the \app{Boxsi} solver and closes the
interface window.  Note that there is no \btn{File\pipe Load\ldots}
menu item; the problem specification file must be declared on the
\app{Boxsi} command line.

The \btn{Help} menu provides the usual help facilities.

The row of buttons immediately below the menu bar provides simulation
progress control.  These buttons\emdash\btn{Run}, \btn{Relax},
\btn{Step} and \btn{Pause}\emdash become active once the micromagnetic
problem has been initialized. These buttons allow the user to change the
run state of the solver.  In the Pause state, the solver sits idle
awaiting further instructions.  If \btn{Step} is selected, then the
solver will move forward one iteration and then Pause.  In Relax mode,
the solver takes at least one step, and then runs until it reaches a
stage boundary, at which point the solver is paused.  In Run mode, the
solver runs until the end of the problem is reached.  When the problem
end is reached, the solver will either pause or exit, depending upon the
setting of the \cd{-exitondone} command line option.

Normally the solver progresses automatically from problem initialization
into Run mode, but this can be changed by the \cd{-pause} command line
switch.  Interactive output is available in all modes; the scheduled
outputs occur appropriately as the step and stage counts advance.

Directly below the run state control buttons are three display lines,
showing the name of the input \MIF\ file, the current run-state, and the
current stage number/maximum stage number.  Both stage numbers are
0-indexed.

\starssechead{Details}
As with \app{Oxsii}, the simulation model construction is governed by
the Specify blocks in the input \MIF\ file, and all aspects of the
simulation are determined by the specified
\hyperrefhtml{Oxs\_Ext classes}{Oxs\_Ext classes (Sec.~}{)}{sec:oxsext}.
Refer to the appropriate Oxs\_Ext class documentation for simulation and
computational details.

\starssechead{Threading considerations\label{html:threadconsider}}
\index{parallelization|(}\index{NUMA|(}
As an example, suppose you are running on a four dual-core processor
box, where each of the four processors is connected to a separate memory
node.  In other words, there are eight cores in total, and each pair of
cores shares a memory node.  Further assume that the processors are
connected via point-to-point links such as AMD's HyperTransport or
Intel's QuickPath Interconnect.

If you want to run a single instance of \app{Boxsi} as quickly as
possible, you might use the \cd{-threads 8} option, which, assuming the
default value of \cd{-numanodes none} is in effect, would allow the
operating system to schedule the eight threads among the system's eight
cores as it sees fit.  Or, you might reduce the thread count to reserve
one or more cores for other applications.  If the job is long running,
however, you may find that the operating system tries to run multiple
threads on a single core\emdash perhaps in order to leave other cores idle so
that they can be shut down to save energy.  Or, the operating system may
move threads away from the memory node where they have allocated memory,
which effectively reduces memory bandwidth.  In such cases you might want
to launch \app{Boxsi} with the \cd{-numanodes auto} option.  This
overrides the operating systems preferences, and ties threads to
particular memory nodes for the lifetime of the process.  (On Linux
boxes, you should also check the ``cpu frequency governor'' and ``huge page
support'' selection and settings.)

If you want to run two instances of \app{Boxsi} concurrently, you might
launch each with the \cd{-threads 4} option, so that each job has four
threads for the operating system to schedule.  If you don't like the
default scheduling by the operating system, you can use the
\cd{-numanodes} option, but what you \textbf{don't} want to do is launch
two jobs with \cd{-numanodes auto}, because the ``auto'' option assigns
threads to memory nodes from a fixed sequence list, so both jobs will be
assigned to the same nodes.  Instead, you should manually assign the
nodes, with a different set to each job.  For example, you may launch
the first job with \cd{-numanodes 0,1} and the second job with
\cd{-numanodes 2,3}.  One point to keep in mind when assigning nodes is
that some node pairs are ``closer'' (with respect to memory latency and
bandwidth) than others.  For example, memory node 0 and memory node 1
may be directly connected via a point-to-point link, so that data can be
transferred in a single ``hop.''  But sending data from node 0 to node 2
may require two hops (from node 0 to node 1, and then from node 1 to
node 2).  In this case \cd{-numanodes 0,1} will probably run faster than
\cd{-numanodes 0,2}.

The \cd{-numanodes} option is only available on Linux boxes if the
``numactl'' and ``numactl-devel'' packages are installed.  The
\cd{numactl} command itself can be used to tie jobs to particular memory
nodes, similar to the \cd{boxsi -numanodes} option, except that
\cd{-numanodes} ties threads whereas \cd{numactl} ties jobs.  The
\cd{numactl --hardware} command will tell you how many memory nodes are
in the system, and also reports a measure of the (memory latency and
bandwidth) distance between nodes.  This information can be used in
selecting nodes for the \cd{boxsi -numanodes} option, but in
practice the distance information reported by \cd{numactl} is often not
reliable.  For best results one should experiment with different
settings, or run memory bandwidth tests with different node
pairs.\index{NUMA|)}\index{parallelization|)}

\starssechead{Batch Scheduling Systems}
\OOMMF\ jobs submitted to a batch queuing system (e.g., Condor, PBS,
NQS) can experience sporadic failures caused by interactions between
separate \OOMMF\ jobs running simultaneously on the same compute
node.  These problems can be prevented by using the \OOMMF\ command
line utility\index{application!launchhost}
\hyperrefhtml{\app{launchhost}}{\app{launchhost} (Sec.~}{)}{sec:launchhost}
to isolate each job.

%%%%%%%%%%%%%%%%%%%%%%%%%%%%%%%%%%%%%%%%%%%%%%%%%%%%%%%%%%%%%%%%%%%%%%%%

\section{Standard Oxs\_Ext Child Classes}\label{sec:oxsext}%
\index{Oxs\_Ext~child~classes}
An Oxs simulation is built as a collection of \cd{Oxs\_Ext} (Oxs
Extension) objects.  These are defined via Specify blocks in the input
\hyperrefhtml{\MIF~2 file.}{\MIF~2 file (Sec.~}{).}{sec:mif2format}
The reader will find the information and
\hyperrefhtml{sample \MIF\ file}{sample \MIF\ file,
Fig.~}{,}{fig:mif2sample} provided in that section to be a helpful
adjunct to the material presented below.  Addition example \MIF~2
files can be found in the directory \fn{oommf/app/oxs/examples}.

This section describes the \cd{Oxs\_Ext} classes available in the
standard \OOMMF\ distribution, including documentation of their Specify
block initialization strings, and a list of some sample \MIF\ files from
the \fn{oommf/app/oxs/examples} directory that use the class.  The
standard \cd{Oxs\_Ext} objects, i.e., those that are distributed with
\OOMMF, can be identified by the \cd{Oxs\_} prefix in their names.
Additional \cd{Oxs\_Ext} classes may be available on your system.  Check
local documentation for details.

% NOTE: The html links hardcoded into the <PRE> sections below assume
% the anchors are on the same html page.  This will have to be reworked
% if (when?) we decide to break this section into multiple pages.
In the following presentation, the \cd{Oxs\_Ext} classes are organized
into 8 categories: atlases, meshes, energies, evolvers, drivers, scalar
field objects, vector field objects, and \MIF\ support classes.  The
following \cd{Oxs\_Ext} classes are currently available:
\begin{itemize}
%begin{latexonly}
\newlength{\leftcolwidth}
\iflatexml
% LaTeXML 0.8.6 does not set tab stops properly (tabbing environment),
% and \settowidth also doesn't work right. So instead estimate the max
% first column width in em and set \leftcolwidth directly. This may be
% inaccurate depending on the display font, but it seems the best
% workaround at this time.
\setlength{\leftcolwidth}{18em}
%\settowidth{\leftcolwidth}{\texttt{8}}
%\setlength{\leftcolwidth}{23\leftcolwidth}
\else
\settowidth{\leftcolwidth}{\tt Oxs\_AffineTransformVectorField}
\addtolength{\leftcolwidth}{3em}
\fi
\setlength{\parskip}{0pt}
\setlength{\topsep}{0pt}
\setlength{\itemsep}{\baselineskip}
%end{latexonly}
\item {\bf Atlases}
   {\newline\tt\begin{tabular}{@{}p{\leftcolwidth}@{}l@{}}
      \ptlink{Oxs\_BoxAtlas}{PTBA}        & \ptlink{Oxs\_EllipseAtlas}{PTEA} \\
      \ptlink{Oxs\_EllipsoidAtlas}{PTESA} & \ptlink{Oxs\_ImageAtlas}{PTIA} \\
      \ptlink{Oxs\_MultiAtlas}{PTMA}      & \ptlink{Oxs\_ScriptAtlas}{PTSA} \\
     \end{tabular}}
% Note: Earlier versions of this file had raw html blocks like
%
%   \begin{rawhtml}
%   <PRE>
%      <A HREF="#BA">Oxs_BoxAtlas</A>                    <A HREF="#EA">Oxs_EllipseAtlas</A>
%      <A HREF="#ESA">Oxs_EllipsoidAtlas</A>              <A HREF="#IA">Oxs_ImageAtlas</A>
%      <A HREF="#MA">Oxs_MultiAtlas</A>                  <A HREF="#SA">Oxs_ScriptAtlas</A>
%   </PRE>
%   \end{rawhtml}
%
% for latex2html output, with associated
%
%   \begin{rawhtml}
%   <A NAME="BA"></A>
%   \end{rawhtml}
%
% targets placed appropriately later in the file. This rendered pretty
% nicely, but the links break if the html file sectioning is set so that
% the targets end up in different files. This feels fragile, and
% considering the maintenance required for parallel blocks, it is hard
% to justify keeping the preformatted html blocks.
\item {\bf Meshes}
   {\newline\tt\begin{tabular}{@{}p{\leftcolwidth}@{}l@{}}
      \ptlink{Oxs\_RectangularMesh}{PTRM} & \ptlink{Oxs\_PeriodicRectangularMesh}{PTPRM}
    \end{tabular}}
\item {\bf Energies}
   {\newline\tt\begin{tabular}{@{}p{\leftcolwidth}@{}l@{}}
      \ptlink{Oxs\_CubicAnisotropy}{PTCA}     & \ptlink{Oxs\_Demag}{PTDE}              \\
      \ptlink{Oxs\_Exchange6Ngbr}{PTE6}       & \ptlink{Oxs\_ExchangePtwise}{PTEP}     \\
      \ptlink{Oxs\_FixedZeeman}{PTFZ}         & \ptlink{Oxs\_RandomSiteExchange}{PTSE} \\
      \ptlink{Oxs\_ScriptUZeeman}{PTSU}       & \ptlink{Oxs\_SimpleDemag}{PTSD}        \\
      \ptlink{Oxs\_StageZeeman}{PTSZ}         & \ptlink{Oxs\_TransformZeeman}{PTTZ}    \\
      \ptlink{Oxs\_TwoSurfaceExchange}{PTTS}  & \ptlink{Oxs\_UniaxialAnisotropy}{PTUA} \\
      \ptlink{Oxs\_UniformExchange}{PTUE}     & \ptlink{Oxs\_UZeeman}{PTUZ}
     \end{tabular}}
\item {\bf Evolvers}
   {\newline\tt\begin{tabular}{@{}p{\leftcolwidth}@{}l@{}}
      \ptlink{Oxs\_CGEvolve}{PTCG}            &    \ptlink{Oxs\_EulerEvolve}{PTEE}  \\
      \ptlink{Oxs\_RungeKuttaEvolve}{PTRK}    &    \ptlink{Oxs\_SpinXferEvolve}{PTSX}
     \end{tabular}}
\item {\bf Drivers}
   {\newline\tt\begin{tabular}{@{}p{\leftcolwidth}@{}l@{}}
     \ptlink{Oxs\_MinDriver}{PTMD}      &    \ptlink{Oxs\_TimeDriver}{PTTD}
     \end{tabular}}
\item {\bf Scalar Field Objects}
   {\newline\tt\begin{tabular}{@{}p{\leftcolwidth}@{}l@{}}
     \ptlink{Oxs\_AtlasScalarField}{PTASF}
     &  \ptlink{Oxs\_LinearScalarField}{PTLSF}        \\
     \ptlink{Oxs\_RandomScalarField}{PTRSF}
     &  \ptlink{Oxs\_ScriptScalarField}{PTSSF}        \\
     \ptlink{Oxs\_UniformScalarField}{PTUSF}
     &  \ptlink{Oxs\_VecMagScalarField}{PTVMSF}       \\
     \ptlink{Oxs\_ScriptOrientScalarField}{PTSOSF}
     &  \ptlink{Oxs\_AffineOrientScalarField}{PTAOSF} \\
     \ptlink{Oxs\_AffineTransformScalarField}{PTATSF}
     & \ptlink{Oxs\_ImageScalarField}{PTISF}
   \end{tabular}}
\item {\bf Vector Field Objects}
  {\newline\tt\begin{tabular}{@{}p{\leftcolwidth}@{}l@{}}
    \ptlink{Oxs\_AtlasVectorField}{PTAVF}
    &  \ptlink{Oxs\_FileVectorField}{PTFVF}          \\
    \ptlink{Oxs\_PlaneRandomVectorField}{PTPRVF}
    &  \ptlink{Oxs\_RandomVectorField}{PTRVF}        \\
    \ptlink{Oxs\_ScriptVectorField}{PTSVF}
    &  \ptlink{Oxs\_UniformVectorField}{PTUVF}       \\
    \ptlink{Oxs\_ScriptOrientVectorField}{PTSOVF}
    &  \ptlink{Oxs\_AffineOrientVectorField}{PTAOVF} \\
    \ptlink{Oxs\_AffineTransformVectorField}{PTATVF}
    &  \ptlink{Oxs\_MaskVectorField}{PTMVF}          \\
    \ptlink{Oxs\_ImageVectorField}{PTIVF}
   \end{tabular}}
\item {\bf \MIF\ Support Classes}
  {\newline\tt\begin{tabular}{@{}p{\leftcolwidth}@{}l@{}}
      \ptlink{Oxs\_LabelValue}{PTLV}
     \end{tabular}}
\end{itemize}

\subsection{Atlases}\label{sec:oxsAtlases}
Geometric volumes of spaces are specified in Oxs via \textit{atlases},
which divide their domain into one or more disjoint subsets called
\textit{regions}.  Included in each atlas definition is the atlas
\textit{bounding box}, which is an axes parallel rectangular
parallelepiped containing all the regions.  There is also the special
\textit{universe} region, which consists of all points outside the
regions specified in the atlas.  The universe region is not considered
to be part of any atlas, and the \cd{universe} keyword should not be
used to label any of the atlas regions.

The most commonly used atlas is the simple \cd{Oxs\_BoxAtlas}.  For
combining multiple atlases, use \cd{Oxs\_MultiAtlas}.

\begin{description}
\index{Oxs\_Ext~child~classes!Oxs\_BoxAtlas}%
\pttarget{PTBA}\item[Oxs\_BoxAtlas:]
An axes parallel rectangular parallelepiped,
containing a single region that is coterminous with the atlas itself.
The specify block has the form
\begin{latexonly}
\begin{quote}
\tt Specify Oxs\_BoxAtlas:\oxsval{atlasname} \ocb\\
\bi xrange \ocb\oxsval{ xmin xmax }\ccb\\
\bi yrange \ocb\oxsval{ ymin ymax }\ccb\\
\bi zrange \ocb\oxsval{ zmin zmax }\ccb\\
\bi name \oxsval{regionname}\\
\ccb
\end{quote}
\end{latexonly}
\begin{rawhtml}
<BLOCKQUOTE><DL><DT>
<TT>Specify Oxs_BoxAtlas:</TT><I>atlasname</I> <TT>{</TT>
<DD><TT>xrange {</TT> <I>xmin</I><TT>&nbsp;</TT><I>xmax</I> <TT>}</TT>
<DD><TT>yrange {</TT> <I>ymin</I><TT>&nbsp;</TT><I>ymax</I> <TT>}</TT>
<DD><TT>zrange {</TT> <I>zmin</I><TT>&nbsp;</TT><I>zmax</I> <TT>}</TT>
<DD><TT>name </TT> <I>regionname</I>
<DT><TT>}</TT></DL></BLOCKQUOTE><P>
\end{rawhtml}
where \oxsval{xmin, xmax, \ldots} are coordinates in meters, specifying
the extents of the volume being defined.  The \oxsval{regionname} label
specifies the name assigned to the region contained in the atlas.  The
\oxslabel{name} entry is optional; if not specified then the
region name is taken from the object instance name, i.e.,
\oxsval{atlasname}.

\begin{ExampleMifs}
 \fn{sample.mif}, \fn{cgtest.mif}.
\end{ExampleMifs}

\pttarget{PTEA}\index{Oxs\_Ext~child~classes!Oxs\_EllipseAtlas}%
\item[Oxs\_EllipseAtlas:]
Defines a volume in the shape of a right elliptical cylinder with axes
parallel to the coordinate axes. This functionality can be obtained
using appropriate \Tcl\ scripts with the \cd{Oxs\_ScriptAtlas} class,
but this class is somewhat easier to use and runs faster.  The Specify
block has the form
\begin{latexonly}
\begin{quote}\tt
Specify Oxs\_EllipseAtlas:\oxsval{atlasname} \ocb\\
\bi xrange \ocb\oxsval{ xmin xmax }\ccb\\
\bi yrange \ocb\oxsval{ ymin ymax }\ccb\\
\bi zrange \ocb\oxsval{ zmin zmax }\ccb\\
\bi margin \ocb\oxsval{ margins }\ccb\\
\bi axis \oxsval{axisdir}\\
\bi name \ocb\oxsval{ regions }\ccb\\
\ccb
\end{quote}
\end{latexonly}
\begin{rawhtml}
<BLOCKQUOTE><DL><DT>
<TT>Specify Oxs_EllipseAtlas:</TT><I>atlasname</I> <TT>{</TT>
<DD><TT>xrange {</TT> <I>xmin</I><TT>&nbsp;</TT><I>xmax</I> <TT>}</TT>
<DD><TT>yrange {</TT> <I>ymin</I><TT>&nbsp;</TT><I>ymax</I> <TT>}</TT>
<DD><TT>zrange {</TT> <I>zmin</I><TT>&nbsp;</TT><I>zmax</I> <TT>}</TT>
<DD><TT>margin {</TT> <I>margins</I> <TT>}</TT>
<DD><TT>axis </TT> <I>axisdir</I>
<DD><TT>name {</TT> <I>regions</I> <TT>}</TT>
<DT><TT>}</TT></DL></BLOCKQUOTE><P>
\end{rawhtml}
Here \oxsval{xmin, xmax, \ldots} are coordinates in meters, specifying
the bounding box for the atlas, similar to the layout of the Specify
block for the \cd{Oxs\_BoxAtlas} class. The \oxslabel{margin} setting
combines with the bounding box to determine the extent of the elliptical
cylinder. The \oxsval{margins} value is a list consisting of one, three,
or six values, in units of meters. If the full six values \ocb $m_0$,
$m_1$, \ldots, $m_5$\ccb\ are specified they determine the bounding box
for the elliptical cylinder as
$[xmin+m_0,xmax-m_1]\times[ymin+m_2,ymax-m_3]\times[zmin+m_4,zmax-m_5]$.
If three values are given then they are interpreted as margins for the
x-coordinates, y-coordinates, and z-coordinates, respectively. If a
single margin value is listed then that value is applied along all six
faces. If the two margin values for a given coordinate are not equal,
then the center of the cylinder will be shifted from the center of the
atlas. If a margin value is negative then part of the cylinder will be
clipped at the atlas boundary. If \oxslabel{margin} is not given then
the default is 0.

The \oxsval{axisdir} should be one of x, y, or z, specifying the axis of
symmetry for the cylinder. If not given the default is z.

The \oxslabel{name} setting is a list of one or two elements. A single
value specifies the region name for the interior of the elliptical
cylinder.  In this case the exterior is automatically assigned to the
global ``universe'' region. In the case of a two element list, the first
element is the name assigned to the interior of the cylinder, the second
element is the name assigned to the exterior of the cylinder. If
desired, either one may be specified as ``universe'' to assign the
corresponding volume to the global universe region. If \oxslabel{name}
is not specified then it is treated by default as a one element list
using the atlas object instance name, i.e., \oxsval{atlasname}, as the
interior region name.

\begin{ExampleMifs}
 \fn{ellipse.mif},  \fn{ellipsea.mif}.
\end{ExampleMifs}

\pttarget{PTESA}\index{Oxs\_Ext~child~classes!Oxs\_EllipsoidAtlas}%
\item[Oxs\_EllipsoidAtlas:]
Conceptually analogous to \cd{Oxs\_EllipseAtlas}, this class defines an
ellipsoidal region with axes parallel to the coordinate axes. With
appropriate \Tcl\ scripts, the \cd{Oxs\_ScriptAtlas} class can provide
the same functionality, but this class is somewhat easier to use
and runs faster.  The Specify block has the form
\begin{latexonly}
\begin{quote}\tt
Specify Oxs\_EllipsoidAtlas:\oxsval{atlasname} \ocb\\
\bi xrange \ocb\oxsval{ xmin xmax }\ccb\\
\bi yrange \ocb\oxsval{ ymin ymax }\ccb\\
\bi zrange \ocb\oxsval{ zmin zmax }\ccb\\
\bi margin \ocb\oxsval{ margins }\ccb\\
\bi name \ocb\oxsval{ regions }\ccb\\
\ccb
\end{quote}
\end{latexonly}
\begin{rawhtml}
<BLOCKQUOTE><DL><DT>
<TT>Specify Oxs_EllipsoidAtlas:</TT><I>atlasname</I> <TT>{</TT>
<DD><TT>xrange {</TT> <I>xmin</I><TT>&nbsp;</TT><I>xmax</I> <TT>}</TT>
<DD><TT>yrange {</TT> <I>ymin</I><TT>&nbsp;</TT><I>ymax</I> <TT>}</TT>
<DD><TT>zrange {</TT> <I>zmin</I><TT>&nbsp;</TT><I>zmax</I> <TT>}</TT>
<DD><TT>margin {</TT> <I>margins</I> <TT>}</TT>
<DD><TT>name {</TT> <I>regions</I> <TT>}</TT>
<DT><TT>}</TT></DL></BLOCKQUOTE><P>
\end{rawhtml}
All entries are interpreted in the same manner as for the
\cd{Oxs\_EllipseAtlas} class.

\begin{ExampleMifs}
  \fn{ellipsoid.mif} and \fn{ellipsoid.mif}.  See
  \fn{ellipsoid-atlasproc.mif} and \fn{ellipsoid-fieldproc.mif} for
  examples equivalent to \fn{ellipsoid.mif} using \Tcl\ scripts.
\end{ExampleMifs}

\pttarget{PTIA}\index{Oxs\_Ext~child~classes!Oxs\_ImageAtlas}%
\item[Oxs\_ImageAtlas:\label{html:oxsImageAtlas}]%
\index{file!mask|(}\index{file!bitmap|(}
This class is designed to allow an image file
to be used to define regions in terms of colors in the image.  It is
intended for use in conjunction with the \cd{Oxs\_AtlasScalarField} and
\cd{Oxs\_AtlasVectorField} classes in circumstances where a small
number of distinct species (materials) are being modeled.  This provides
a generalization of the \htmlonlyref{mask file}{sec:partgeometry}
functionality of the 2D solver\latexonly{
(Sec.~\ref{sec:partgeometry})}.

For situations requiring continuous variation in material parameters,
the script field classes should be used in conjunction with the
\cd{ReadFile} \MIF\ extension command.  See the
\cd{ColorField} sample proc in the \htmlonlyref{\cd{ReadFile}
documentation}{html:ReadFile} \latex{in
Sec.~\ref{sec:mif2ExtensionCommands}} for an example of this
technique.

The \cd{Oxs\_ImageAtlas} Specify block has the following form:
\begin{latexonly}
\begin{quote}\tt
Specify Oxs\_ImageAtlas:\oxsval{name} \ocb\\
\bi xrange \ocb\oxsval{ xmin xmax }\ccb\\
\bi yrange \ocb\oxsval{ ymin ymax }\ccb\\
\bi zrange \ocb\oxsval{ zmin zmax }\ccb\\
\bi viewplane \oxsval{view}\\
\bi image \oxsval{pic}\\
\bi colormap \ocb\oxsval{\\
\bi \bi color-1 region\_name\\
\bi \bi color-2 region\_name\\
\bi \bi \ldots\\
\bi \bi color-n region\_name}\\
\bi\ccb\\
\bi matcherror \oxsval{max\_color\_distance}\\
\ccb
\end{quote}
\end{latexonly}
\begin{rawhtml}
<BLOCKQUOTE><DL><DT>
<TT>Specify Oxs_ImageAtlas:</TT><I>name</I> <TT>{</TT>
<DD><TT>xrange {</TT> <I>xmin</I><TT>&nbsp;</TT><I>xmax</I> <TT>}</TT>
<DD><TT>yrange {</TT> <I>ymin</I><TT>&nbsp;</TT><I>ymax</I> <TT>}</TT>
<DD><TT>zrange {</TT> <I>zmin</I><TT>&nbsp;</TT><I>zmax</I> <TT>}</TT>
<DD><TT>viewplane </TT> <I>view</I>
<DD><TT>image </TT> <I>pic</I>
<DD><TT>colormap {</TT><DL>
   <DD><I>color-1</I><TT>&nbsp;</TT><I>region_name</I>
   <DD><I>color-2</I><TT>&nbsp;</TT><I>region_name</I>
   <DD> ...
   <DD><I>color-n</I><TT>&nbsp;</TT><I>region_name</I>
</DL><TT>}</TT>
<DD><TT>matcherror </TT> <I>max_color_distance</I>
<DT><TT>}</TT></DL></BLOCKQUOTE><P>
\end{rawhtml}
The \oxslabel{xrange}, \oxslabel{yrange}, \oxslabel{zrange} entries
specify the extent of the atlas, in meters.  The \oxslabel{viewplane}
\oxsval{view} value should be one of the three two-letter codes \cd{xy},
\cd{zx} or \cd{yz}, which specify the mapping of the horizontal and
vertical axes of the image respectively to axes in the simulation.  The
image is scaled as necessary along each dimension to match the atlas
extents along the corresponding axes. The image is overlaid through the
entire depth of the perpendicular dimension, i.e., along the axis absent
from the \cd{viewplane} specification.  The \cd{Oxs\_ImageAtlas} class
can be used inside a \cd{Oxs\_MultiAtlas} object to specify regions in a
multilayer structure, as in example file \fn{imagelayers.mif}.  Note
that if the image aspect ratio doesn't match the ratio of the viewplane
ranges, then the scaling will stretch or contract the image along one
axis. One workaround for this is to set the extents in the
\cd{Oxs\_ImageAtlas} to match the image aspect ratio, and use a separate
atlas (perhaps an \cd{Oxs\_BoxAtlas}) to define the mesh and simulation
extents. This approach can also be used to translate the image relative
to the simulation extents. For an example see \fn{imageatlas2.mif}.

The \oxslabel{image} entry specifies the name of the image file to use.
If the file path is relative, then it will be taken with respect to the
directory containing the \MIF\ file.  The image format may be any of
those recognized by \hyperrefhtml{\cd{any2ppm}}{\cd{any2ppm}
  (Sec.~}{)}{sec:any2ppm}.  The file will be read directly by Oxs if it
is in one of the PPM or Microsoft BMP (uncompressed) formats, otherwise
\cd{any2ppm} will be automatically launched to perform the conversion.

The \oxslabel{colormap} value is an even length list of color + region
name pairs.  The colors may be specified in any of several ways.  The
most explicit is to use one of the \Tk\ numeric formats,
\lb rgb, \lb rrggbb, \lb rrrgggbbb or \lb rrrrggggbbbb, where
each r, g, and b is one hex digit (i.e., 0-9 or A-F) representing the
red, green and blue components of the color, respectively.  For example,
\lb F00 is bright (full-scale) red, \lb 800 would be a darker red, while
\lb FF0 and \lb FFFF00 would both be bright yellow.  Refer to the
\cd{Tk\_GetColor} documentation for details.  For shades of gray the
special notation \cd{grayD} or \cd{greyD} is available, where D is a
decimal value between 0 and 100, e.g., \cd{grey0} is black and
\cd{grey100} is white.  Alternatively, one may use any of the symbolic
names defined in the \fn{oommf/config/colors.config} file, such as
\cd{red}, \cd{white} and \cd{skyblue}.  When comparing symbolic names,
spaces and capitalization are ignored.  The list of symbolic names can
be extended by adding additional files to the \cd{Color filename} option
in the \fn{options.tcl} \hyperrefhtml{customization file}{customization
file (Sec.~}{)}{sec:install.custom}.  Finally, one \oxsval{color} in the
\cd{colormap} list may optionally be the special keyword ``default''.
All pixels that don't match any of the other specified colors (as
determined by the \cd{matcherror} option) are assigned to region
paired with \cd{default}.

Each of the specified colors should be distinct, but the region names
are allowed to be repeated as desired.  The region names may be chosen
arbitrarily, except the special keyword ``universe'' is reserved for
points not in any of the regions.  This includes all points outside the
atlas bounding box defined by the \cd{xrange}, \cd{yrange}, \cd{zrange}
entries, but may also include points inside that boundary.

Pixels in the image are assigned to regions by comparing the color of
the pixel to the list of colors specified in \cd{colormap}.  If the
pixel color is closer to a \cd{colormap} color than
\oxsval{max\_color\_distance}, then the colors are considered matched.
If a pixel color matches exactly one \cd{colormap} color, then the pixel
is assigned to the corresponding region.  If a pixel color matches more
than one \cd{colormap} color, the pixel is assigned to the region
corresponding to the closest match.  If a pixel color doesn't match any
of the \cd{colormap} colors, then it is assigned to the \textit{default
region}, which is the region paired with the ``default'' keyword.  If
\cd{default} does not explicitly appear in the \cd{colormap} colors
list, then \cd{universe} is made the default region.

To calculate the distance between two colors, each color is first
converted to a scaled triplet of floating point red, green, and blue
values, $(r,g,b)$, where each component lies in the interval $[0,1]$,
with $(0,0,0)$ representing black and $(1,1,1)$ representing white.  For
example, $(0,0,1)$ is bright blue.  Given two colors in this
representation, the distance is computed using the standard Euclidean
norm with uniform weights, i.e., the distance between $(r_1,g_1,b_1)$
and $(r_2,g_2,b_2)$ and is
\begin{displaymath}
\sqrt{(r_1-r_2)^2 + (g_1-g_2)^2 + (b_1-b_2)^2}.
\end{displaymath}
Since the difference in any one component is at most 1, the distance
between any two colors is at most \abovemath{\sqrt{3}}.

As explained above, two colors are considered to match if the distance
between them is less than the specified \oxslabel{matcherror} value.  If
\oxsval{max\_color\_distance} is sufficiently small, then it may easily
happen that a pixel's color does not match any of the specified region
colors, so the pixel would be assigned to the default region.  On the
other hand, if \oxsval{max\_color\_distance} is larger than
\abovemath{\sqrt{3}}, then all colors will match, and no pixels will be
assigned to the default region.  If \cd{matcherror} is not specified,
then the default value for \oxsval{max\_color\_distance} is 3, which
means all colors match.

The following example should help clarify these matters.
\begin{rawhtml}
<BLOCKQUOTE>
\end{rawhtml}
%begin<latexonly>
\begin{quote}
%end<latexonly>
\begin{verbatim}
Specify Oxs_ImageAtlas:atlas {
    xrange { 0 400e-9 }
    yrange { 0 200e-9 }
    zrange { 0  20e-9 }
    image  mypic.gif
    viewplane "xy"
    colormap {
        blue   cobalt
        red    permalloy
        green  universe
        default cobalt
    }
    matcherror .1
}
\end{verbatim}
%begin<latexonly>
\end{quote}
%end<latexonly>
\begin{rawhtml}
</BLOCKQUOTE>
\end{rawhtml}
Blue pixels get mapped to the ``cobalt'' region and red pixels
to the ``permalloy'' region.  Green pixels are mapped to the
``universe'' non-region, which means they are considered to be outside
the atlas entirely.  This is a fine point, but comes into
play when atlases with overlapping bounding boxes are brought together
inside an \cd{Oxs\_MultiAtlas}.  To which region would an orange pixel
be assigned?  The scaled triplet representation for orange is
$(1,0.647,0)$, so the distance to blue is $1.191$, the distance to red
is $0.647$, and the distance to green is $1.06$.  Thus the closest color
is red, but $0.647$ is outside the \cd{matcherror} setting of $0.1$, so
orange doesn't match any of the colors and is hence assigned to the
default region, which in this case is cobalt.  On the other hand, if
\cd{matcherror} had been set to say 1, then orange and red would match
and orange would be assigned to the permalloy region.

Pixels with colors that are equidistant to and match more than one color
in the colormap will be assigned to one of the closest color regions.
The user should not rely on any particular selection, that is to say,
the explicit matching procedure in this case is not defined.
\index{file!mask|)}\index{file!bitmap|)}

\begin{ExampleMifs}
 \fn{imageatlas.mif}, \fn{imageatlas2.mif}, \fn{imagelayers.mif}, \fn{grill.mif}.
\end{ExampleMifs}

\pttarget{PTMA}\index{Oxs\_Ext~child~classes!Oxs\_MultiAtlas}%
\item[Oxs\_MultiAtlas:]
This atlas is built up as an ordered list of
other atlases.  The set of regions defined by the \cd{Oxs\_MultiAtlas}
is the union of the regions of all the atlases contained therein.  The
sub-atlases need not be disjoint, however each point is assigned to the
region in the first sub-atlas in the list that contains it, so the
regions defined by the \cd{Oxs\_MultiAtlas} are effectively disjoint.

The \cd{Oxs\_MultiAtlas} specify block has the form
\begin{latexonly}
{\samepage
\begin{quote}\tt
Specify Oxs\_MultiAtlas:\oxsval{name} \ocb\\
\bi atlas \ \ \oxsval{atlas\_1\_spec}\\
\bi atlas \ \ \oxsval{atlas\_2\_spec}\\
\bi\ldots\\
\bi xrange \ocb\oxsval{ xmin xmax }\ccb\\
\bi yrange \ocb\oxsval{ ymin ymax }\ccb\\
\bi zrange \ocb\oxsval{ zmin zmax }\ccb\\
\ccb
\end{quote}}
\end{latexonly}
\begin{rawhtml}
<BLOCKQUOTE><DL><DT>
<TT>Specify Oxs_MultiAtlas:</TT><I>name</I> <TT>{</TT>
<DD> atlas &nbsp;&nbsp; <I>atlas_1_spec</I>
<DD> atlas &nbsp;&nbsp; <I>atlas_2_spec</I>
<DD> ...
<DD><TT>xrange {</TT> <I>xmin<TT>&nbsp;</TT>xmax</I> <TT>}</TT>
<DD><TT>yrange {</TT> <I>ymin<TT>&nbsp;</TT>ymax</I> <TT>}</TT>
<DD><TT>zrange {</TT> <I>zmin<TT>&nbsp;</TT>zmax</I> <TT>}</TT>
<DT><TT>}</TT></DL></BLOCKQUOTE><P>
\end{rawhtml}
Each \oxsval{atlas\_spec} may be either a reference to an atlas defined
earlier and outside the current Specify block, or else an inline,
embedded atlas definition.  The bounding box \oxslabel{xrange},
\oxslabel{yrange} and \oxslabel{zrange} specifications are each
optional.  If not specified the corresponding range for the atlas
bounding box is taken from the minimal bounding box containing all the
sub-atlases.

If the atlases are not disjoint, then the regions as defined by an
\cd{Oxs\_MultiAtlas} can be somewhat different from those of the
individual component atlases.  For example, suppose \cd{regionA} is a
rectangular region in \cd{atlasA} with corner points (5,5,0) and
(10,10,10), and \cd{regionB} is a rectangular region in \cd{atlasB} with
corner points (0,0,0) and (10,10,10).  When composed in the order
\cd{atlasA}, \cd{atlasB} inside an \cd{Oxs\_MultiAtlas}, \cd{regionA}
reported by the \cd{Oxs\_MultiAtlas} will be the same as \cd{regionA}
reported by \cd{atlasA}, but \cd{regionB} as reported by the
\cd{Oxs\_MultiAtlas} will be the ``L'' shaped volume of those points in
\cd{atlasB}'s \cd{regionB} not inside \cd{regionA}.  If the
\cd{Oxs\_MultiAtlas} is constructed with \cd{atlasB} first and
\cd{atlasA} second, then \cd{regionB} as reported by the
\cd{Oxs\_MultiAtlas} would agree with that reported by \cd{atlasB}, but
\cd{regionA} would be empty.

NOTE: The \htmlonlyref{\cd{attributes}}{par:specifyAttributes} key label
\latex{(cf.\ Sec.~\ref{par:specifyAttributes})} is not supported by this
class.

\begin{ExampleMifs}
  \fn{manyregions-multiatlas.mif}, \fn{spinvalve.mif},
  \fn{spinvalve-af.mif}, \fn{yoyo.mif}.
\end{ExampleMifs}

\pttarget{PTSA}\index{Oxs\_Ext~child~classes!Oxs\_ScriptAtlas}%
\item[Oxs\_ScriptAtlas:]
An atlas where the regions are defined via a \Tcl\ script.  The specify
block has the form
\begin{latexonly}
\begin{quote}\tt
Specify Oxs\_ScriptAtlas:\oxsval{name} \ocb\\
\bi xrange \ocb\oxsval{ xmin xmax }\ccb\\
\bi yrange \ocb\oxsval{ ymin ymax }\ccb\\
\bi zrange \ocb\oxsval{ zmin zmax }\ccb\\
\bi regions \ocb\oxsval{ rname\_1 rname\_2 \ldots\ rname\_n }\ccb\\
\bi script\_args \ocb\oxsval{ args\_request }\ccb\\
\bi script \oxsval{\Tcl\_script}\\
\ccb
\end{quote}
\end{latexonly}
\begin{rawhtml}
<BLOCKQUOTE><DL><DT>
<TT>Specify Oxs_ScriptAtlas:</TT><I>name</I> <TT>{</TT>
<DD><TT>xrange {</TT> <I>xmin<TT>&nbsp;</TT>xmax</I> <TT>}</TT>
<DD><TT>yrange {</TT> <I>ymin<TT>&nbsp;</TT>ymax</I> <TT>}</TT>
<DD><TT>zrange {</TT> <I>zmin<TT>&nbsp;</TT>zmax</I> <TT>}</TT>
<DD><TT>regions {</TT>
 <I>rname_1<TT>&nbsp;</TT>rname_2<TT>&nbsp;</TT>...<TT>&nbsp;</TT>rname_n</I>
 <TT>}</TT>
<DD><TT>script_args {</TT> <I>args_request</I> <TT>}</TT>
<DD><TT>script </TT> <I>Tcl_script</I>
<DT><TT>}</TT></DL></BLOCKQUOTE><P>
\end{rawhtml}
Here \oxsval{xmin, xmax, \ldots} are coordinates in meters, specifying
the extents of the axes-parallel rectangular parallelepiped enclosing
the total volume being identified.  This volume is subdivided
into \oxsval{n} sub-regions, using the names as given in the
\oxslabel{regions} list.  The \oxslabel{script} is used to assign
points to the various regions.  Appended to the script are the arguments
requested by \oxslabel{script\_args}, in the manner explained in the
\hyperrefhtml{User Defined Support Procedures}{User Defined Support
Procedures section (Sec.~}{)}{par:supportProcs}\html{ section} of the
\MIF~2 file format documentation.  The value \oxsval{args\_request}
should be a subset of \cd{\ocb relpt rawpt minpt maxpt span\ccb}.  If
\cd{script\_args} is not specified, the default value \cd{relpt} is
used.  When executed, the return value from the script should be an
integer in the range $1$ to $n$, indicating the user-defined region in
which the point lies, or else $0$ if the point is not in any
of the $n$ regions.  Region index $0$ is reserved for the implicit
``universe'' region, which is all-encompassing.  The following example
may help clarify the discussion:
% The extra BLOCKQUOTE's here are a workaround for an apparent
% latex2html bug
\begin{rawhtml}
<BLOCKQUOTE>
\end{rawhtml}
%begin{latexonly}
\begin{quote}
%end{latexonly}
\begin{verbatim}
proc Octs { cellsize x y z xmin ymin zmin xmax ymax zmax } {
    set xindex [expr {int(floor(($x-$xmin)/$cellsize))}]
    set yindex [expr {int(floor(($y-$ymin)/$cellsize))}]
    set zindex [expr {int(floor(($z-$zmin)/$cellsize))}]
    set octant [expr {1+$xindex+2*$yindex+4*$zindex}]
    if {$octant<1 || $octant>8} {
       return 0
    }
    return $octant
}

Specify Oxs_ScriptAtlas:octant {
    xrange {-20e-9 20e-9}
    yrange {-20e-9 20e-9}
    zrange {-20e-9 20e-9}
    regions { VIII V VII VI IV I III II }
    script_args { rawpt minpt maxpt }
    script { Octs 20e-9 }
}
\end{verbatim}
%begin{latexonly}
\end{quote}
%end{latexonly}
\begin{rawhtml}
</BLOCKQUOTE>
\end{rawhtml}
This atlas divides the rectangular volume between $(-20,-20,-20)$ and
$(20,20,20)$ (nm) into eight regions, corresponding to the standard
octants, I through VIII.  The \texttt{Octs} \Tcl\ procedure returns a
value between 1 and 8, with 1 corresponding to octant VIII and 8 to
octant II.  The canonical octant ordering starts with I as the
$+x,+y,+z$ space, proceeds counterclockwise in the $+z$ half-space, and
concludes in the $-z$ half-space with V directly beneath I, VI beneath
II, etc.  The ordering computed algorithmically in \texttt{Octs}
starts with 1 for the $-x,-y,-z$ space, 2 for the $+x,-y,-z$ space, 3
for the $-x,+y,-z$ space, etc.  The conversion between the two systems
is accomplished by the ordering of the \texttt{regions} list.

\begin{ExampleMifs}
  \fn{manyregions-scriptatlas.mif}, \fn{octant.mif}, \fn{pattern.mif},
  \fn{tclshapes.mif}, \fn{diskarray.mif}, \fn{ellipsoid-atlasproc.mif}.
\end{ExampleMifs}
\end{description}

\subsection{Meshes}\label{sec:Meshes}
Meshes define the discretization impressed on the simulation.  There
should be exactly one mesh declared in a \MIF~2 file.  The usual
(finite) mesh type is \cd{Oxs\_RectangularMesh}.  For simulations that
are periodic along one or more axes, use the
\cd{Oxs\_PeriodicRectangularMesh} type.

\begin{description}
\pttarget{PTRM}\label{html:oxsrectangularmesh}%
\index{Oxs\_Ext~child~classes!Oxs\_RectangularMesh}
\item[Oxs\_RectangularMesh:]
This mesh is comprised of a lattice of rectangular prisms.
The specify block has the form
\begin{latexonly}
\begin{quote}\tt
Specify Oxs\_RectangularMesh:\oxsval{name} \ocb \\
\bi cellsize \ocb\oxsval{ xstep ystep zstep }\ccb\\
\bi atlas \oxsval{atlas\_spec}\\
\ccb
\end{quote}
\end{latexonly}
\begin{rawhtml}
<BLOCKQUOTE><DL><DT>
<TT>Specify Oxs_RectangularMesh:</TT><I>name</I> <TT>{</TT>
<DD><TT>cellsize {</TT>
  <I>xstep<TT>&nbsp;</TT>ystep<TT>&nbsp;</TT>zstep</I>
  <TT>}</TT>
<DD><TT>atlas </TT> <I>atlas_spec</I>
<DT><TT>}</TT></DL></BLOCKQUOTE><P>
\end{rawhtml}
This creates an axes parallel rectangular mesh across the entire space
covered by \oxslabel{atlas}.  The mesh sample rates along each axis are
specified by \oxslabel{cellsize} (in meters).  The mesh is
cell-based, with the center of the first cell one half step in from the
minimal extremal point (xmin,ymin,ymax) for \oxsval{atlas\_spec}.
The \oxsval{name} is commonly set to ``mesh'', in which case the mesh
object may be referred to by other \cd{Oxs\_Ext} objects by the short
name \cd{:mesh}.

\begin{ExampleMifs}
 \fn{sample.mif}, \fn{stdprob3.mif}, \fn{stdprob4.mif}.
\end{ExampleMifs}

\pttarget{PTPRM}\label{html:oxsperiodicrectangularmesh}%
\index{Oxs\_Ext~child~classes!Oxs\_PeriodicRectangularMesh}
\item[Oxs\_PeriodicRectangularMesh:]
Like the \cd{Oxs\_RectangularMesh}, this mesh is also comprised of a
lattice of rectangular prisms.  However, in this case the
mesh is declared to be periodic along one or more of the axis
directions.  The specify block has the form
\begin{latexonly}
\begin{quote}\tt
Specify Oxs\_PeriodicRectangularMesh:\oxsval{name} \ocb \\
\bi cellsize \ocb\oxsval{ xstep ystep zstep }\ccb\\
\bi atlas \oxsval{atlas\_spec}\\
\bi periodic \oxsval{periodic\_axes}\\
\ccb
\end{quote}
\end{latexonly}
\begin{rawhtml}
<BLOCKQUOTE><DL><DT>
<TT>Specify Oxs_PeriodicRectangularMesh:</TT><I>name</I> <TT>{</TT>
<DD><TT>cellsize {</TT>
  <I>xstep<TT>&nbsp;</TT>ystep<TT>&nbsp;</TT>zstep</I>
  <TT>}</TT>
<DD><TT>atlas </TT> <I>atlas_spec</I>
<DD><TT>periodic </TT> <I>periodic_axes</I>
<DT><TT>}</TT></DL></BLOCKQUOTE><P>
\end{rawhtml}
The \oxslabel{atlas} and \oxslabel{cellsize} values are the same as
for the \cd{Oxs\_RectangularMesh} class.  The \oxsval{periodic\_axis}
value should be a string consisting of one or more of the letters
``x'', ``y'', or ``z'', denoting the periodic direction(s).
\cd{Oxs\_Ext} objects that are incompatible with
\cd{Oxs\_PeriodicRectangularMesh} will issue an error message at
runtime.  In particular, the
\ptlink{\cd{Oxs\_Demag}}{PTDE} class supports
periodicity in none or one direction, but not more.  Also, some
third-party extensions provide independent periodicity support using
the older \cd{Oxs\_RectangularMesh} class rather than
\cd{Oxs\_PeriodicRectangularMesh}.

\begin{ExampleMifs}
 \fn{pbcbrick.mif}, \fn{pbcstripes.mif}.
\end{ExampleMifs}

\end{description}

\subsection{Energies}\label{sec:oxsEnergies}
The following subsections describe the available energy terms.  In
order to be included in the simulation energy and field calculations,
each energy term must be declared in its own, top-level Specify block,
i.e., energy terms should not be declared inline inside other
\cd{Oxs\_Ext} objects.  There is no limitation on the number of energy
terms that may be specified in the input \MIF\ file.  Many of these
terms have spatially varying parameters that are initialized via
\hyperrefhtml{\oxsval{field\_object\_spec}}{\oxsval{field\_object\_spec}
  entries (Sec.~}{)}{sec:oxsFieldObjects}\HTMLoutput{ entries} in their
\hyperrefhtml{Specify initialization block}{Specify initialization block
(see Sec.~}{)}{par:oxsExtReferencing}.

\textbf{Outputs:} For each magnetization configuration, three standard
outputs are provided by all energy terms: the scalar output
``Energy,'' which is the total energy in joules contributed by this
energy term, the scalar field output ``Energy density,'' which is a
cell-by-cell map of the energy density in
\latexhtml{J/m${}^3$}{J/m\begin{rawhtml}<SUP>3</SUP>\end{rawhtml}},
and the three-component vector field output ``Field,'' which is the
pointwise field in A/m.  If the code was compiled with the macro
\cd{NDEBUG} not defined, then there will be an additional scalar
output, ``Calc count,'' which counts the number of times the term has
been calculated in the current simulation.  This is intended for
debugging purposes only; this number should agree with the ``Energy
calc count'' value provided by the evolver.

\starsssechead{Anisotropy Energy}
\begin{description}
\pttarget{PTUA}\index{Oxs\_Ext~child~classes!Oxs\_UniaxialAnisotropy}%
\item[Oxs\_UniaxialAnisotropy:] Uniaxial magneto-crystalline
  anisotropy.  The Specify block has the form
   \begin{latexonly}
      \begin{quote}\tt
      Specify Oxs\_UniaxialAnisotropy:\oxsval{name} \ocb\\
        \bi K1 \oxsval{K} \\
        \bi Ha \oxsval{H} \\
        \bi axis \oxsval{u} \\
      \ccb
      \end{quote}
   \end{latexonly}
   \begin{rawhtml}
   <BLOCKQUOTE><DL><DT>
   <TT>Specify Oxs_UniaxialAnisotropy:</TT><I>name</I> <TT>{</TT>
       <DD> <TT>K1 </TT><I>K</I>
       <DD> <TT>Ha </TT><I>H</I>
       <DD> <TT>axis </TT><I>u</I>
   <DT><TT>}</TT></DL></BLOCKQUOTE><P>
   \end{rawhtml}
  Exactly one of either \oxslabel{K1} or \oxslabel{Ha} should be
  specified, where \oxslabel{K1} is the crystalline anisotropy constant
  (in
  \latexhtml{J/m${}^3$}{J/m\begin{rawhtml}<sup>3</sup>\end{rawhtml}}),
  and \oxslabel{Ha} is the anistropy field (in A/m).  In either case,
  \oxslabel{axis} is the anisotropy direction.  \oxslabel{K1},
  \oxslabel{Ha}, and \oxslabel{axis} may each be varied cellwise across
  the mesh: \oxslabel{K1} and \oxslabel{Ha} are initialized with scalar
  field objects, while \oxslabel{axis} takes a vector field object.  (A
  constant value will be interpreted as a uniform field object having
  the stated value, as usual.)  The axis direction must be non-zero at
  each point, and will be normalized to unit magnitude before being
  used.

  The axis direction is an easy axis if \oxslabel{K1} (or \oxslabel{Ha})
  is $>$0, in which case the cellwise anisotropy energy density (in
  J/m${}^3$) is given by
   \begin{displaymath}
         E_i = K_i(1 - \vm_i\cdot\vu_i)^2 \qquad \mbox{or} \qquad
               \frac{1}{2}\, \mu_0 M_s H_i (1 - \vm_i\cdot\vu_i)^2,
   \end{displaymath}
  respectively.  (Here $m_i$ is the unit magnetization and $M_s$ the
  saturation magnetization in cell $i$.)  Otherwise, if \oxslabel{K1}
  (or \oxslabel{Ha}) is $<0$, the axis direction is the normal to the
  easy plane and the cellwise anisotropy energy density is given by
   \begin{displaymath}
         E_i = -K_i(\vm_i\cdot\vu_i)^2 \qquad \mbox{or} \qquad
               -\frac{1}{2}\, \mu_0 M_s H_i (\vm_i\cdot\vu_i)^2.
   \end{displaymath}
  The formulae in the two cases (easy axis vs.\ easy plane) differ by a
  constant offset, and in each case the energy is non-negative.

\begin{ExampleMifs}
 \fn{diskarray.mif}, \fn{stdprob3.mif}, \fn{grill.mif}.
\end{ExampleMifs}

\pttarget{PTCA}\index{Oxs\_Ext~child~classes!Oxs\_CubicAnisotropy}%
\item[Oxs\_CubicAnisotropy:] Cubic magneto-crystalline anisotropy.
  The Specify block has the form
   \begin{latexonly}
      \begin{quote}\tt
      Specify Oxs\_CubicAnisotropy:\oxsval{name} \ocb\\
        \bi K1 \oxsval{K} \\
        \bi Ha \oxsval{H} \\
        \bi axis1 \oxsval{$u_1$} \\
        \bi axis2 \oxsval{$u_2$} \\
      \ccb
      \end{quote}
   \end{latexonly}
   \begin{rawhtml}
   <BLOCKQUOTE><DL><DT>
   <TT>Specify Oxs_CubicAnisotropy:</TT><I>name</I> <TT>{</TT>
       <DD> <TT>K1 </TT><I>K</I>
       <DD> <TT>Ha </TT><I>H</I>
       <DD> <TT>axis1 </TT><I>u<sub>1</sub></I>
       <DD> <TT>axis2 </TT><I>u<sub>2</sub></I>
   <DT><TT>}</TT></DL></BLOCKQUOTE><P>
   \end{rawhtml}
  Exactly one of either \oxslabel{K1} or \oxslabel{Ha} should be
  specified, where \oxslabel{K1} is the crystalline anisotropy constant
  (in
  \latexhtml{J/m${}^3$}{J/m\begin{rawhtml}<sup>3</sup>\end{rawhtml}}),
  and \oxslabel{Ha} is the anistropy field (in A/m).  In either case,
  \oxslabel{axis1} and \oxslabel{axis2} are two anisotropy directions;
  the third anisotropy axis $u_3$ is computed as the vector product,
  $u_1\times u_2$.  For each cell, the axis directions are easy axes if
  \oxslabel{K1} (or \oxslabel{Ha}) is $>$0, or hard axes if
  \oxslabel{K1} (or \oxslabel{Ha}) is $<$0.  All may be varied cellwise
  across the mesh.  \oxslabel{K1} or \oxslabel{Ha} is initialized with a
  scalar field object, and the axis directions are initialized with
  vector field objects.  (Constant values will be interpreted as uniform
  fields with the indicated value, as usual.)  The \oxslabel{axis1} and
  \oxslabel{axis2} directions must be mutually orthogonal and non-zero
  at each point ($u_1$ and $u_2$ are automatically scaled to unit
  magnitude before use).

  The anisotropy energy density (in J/m${}^3$) for cell $i$ is given by
   \begin{displaymath}
         E_i = K_i\left(a_1^2a_2^2 + a_2^2a_3^2 + a_3^2a_1^2\right),
   \end{displaymath}
   or
   \begin{displaymath}
         E_i = \frac{1}{2}\, \mu_0 M_s H_i
         \left(a_1^2a_2^2 + a_2^2a_3^2 + a_3^2a_1^2\right),
   \end{displaymath}
  where $a_1 = \vm\cdot\vu_1$, $a_2 = \vm\cdot\vu_2$, $a_3 =
  \vm\cdot\vu_3$, for reduced (normalized) magnetization $m$ and
  orthonormal anisotropy axes $\vu_1$, $\vu_2$, and $\vu_3$ at cell $i$.
  In the second form, $M_s$ is the saturation magnetization in cell $i$.
  For each cell, if \oxslabel{K1} (resp.\ \oxslabel{Ha}) is $>$0 then
  the computed energy will be non-negative, otherwise for \oxslabel{K1}
  (resp.\ \oxslabel{Ha}) $<$0 the computed energy will be non-positive.

\begin{ExampleMifs}
 \fn{cgtest.mif}, \fn{sample2.mif}, \fn{grill.mif}.
\end{ExampleMifs}

\end{description}

\starsssechead{Exchange Energy}
\begin{description}
\pttarget{PTE6}\index{Oxs\_Ext~child~classes!Oxs\_Exchange6Ngbr}%
\item[Oxs\_Exchange6Ngbr:]
   Standard 6-neighbor exchange energy.  The
   exchange energy density contribution from cell $i$ is given by
   \begin{equation}
        E_i =  \sum_{j\in N_i} A_{ij}
         \frac{\vm_i\cdot\left(\vm_i - \vm_j\right)}{\Delta_{ij}^2}
   \label{eq:ExchangeEnergy}
   \end{equation}
   where $N_i$ is the set consisting of the 6 cells nearest to cell $i$,
   $A_{ij}$ is the exchange coefficient between cells $i$ and $j$ in J/m,
   and $\Delta_{ij}$ is the discretization step size between cell $i$ and
   cell $j$ (in meters).

   The Specify block for this term has the form
   \begin{latexonly}
      \begin{quote}\tt
      Specify Oxs\_Exchange6Ngbr:\oxsval{name} \ocb\\
         \bi default\_A \oxsval{value}\\
         \bi atlas \oxsval{atlas\_spec}\\
         \bi A \ocb\\
         \bi  \bi \oxsval{ region-1 region-1 A${}_{11}$ }\\
         \bi  \bi \oxsval{ region-1 region-2 A${}_{12}$ }\\
         \bi  \bi \ldots\\
         \bi  \bi \oxsval{ region-m region-n A${}_{mn}$ }\\
         \bi \ccb\\
      \ccb
      \end{quote}
   \end{latexonly}
   \begin{rawhtml}
   <BLOCKQUOTE><DL><DT>
   <TT>Specify Oxs_Exchange6Ngbr:</TT><I>name</I> <TT>{</TT>
   <DD><TT>default_A </TT><I>value</I>
   <DD><TT>atlas </TT><I>atlas_spec</I>
   <DD><TT>A {</TT><DL>
       <DD>
        <I>region-1</I><TT>&nbsp;</TT>
          <I>region-1</I><TT>&nbsp;</TT><I>A<SUB>11</SUB></I>
       <DD>
        <I>region-1</I><TT>&nbsp;</TT>
          <I>region-2</I><TT>&nbsp;</TT><I>A<SUB>12</SUB></I>
       <DD> ...
       <DD>
        <I>region-m</I><TT>&nbsp;</TT>
          <I>region-n</I><TT>&nbsp;</TT><I>A<SUB>mn</SUB></I>
   </DL><TT>}</TT>
   <DT><TT>}</TT></DL></BLOCKQUOTE><P>
   \end{rawhtml}
   or
   \begin{latexonly}
      \begin{quote}\tt
      Specify Oxs\_Exchange6Ngbr:\oxsval{name} \ocb\\
         \bi default\_lex \oxsval{value}\\
         \bi atlas \oxsval{atlas\_spec}\\
         \bi lex \ocb\\
         \bi  \bi \oxsval{ region-1 region-1 lex${}_{11}$ }\\
         \bi  \bi \oxsval{ region-1 region-2 lex${}_{12}$ }\\
         \bi  \bi \ldots\\
         \bi  \bi \oxsval{ region-m region-n lex${}_{mn}$ }\\
         \bi \ccb\\
      \ccb
      \end{quote}
   \end{latexonly}
   \begin{rawhtml}
   <BLOCKQUOTE><DL><DT>
   <TT>Specify Oxs_Exchange6Ngbr:</TT><I>name</I> <TT>{</TT>
   <DD><TT>default_lex </TT><I>value</I>
   <DD><TT>atlas </TT><I>atlas_spec</I>
   <DD><TT>lex {</TT><DL>
       <DD>
        <I>region-1</I><TT>&nbsp;</TT>
          <I>region-1</I><TT>&nbsp;</TT><I>lex<SUB>11</SUB></I>
       <DD>
        <I>region-1</I><TT>&nbsp;</TT>
          <I>region-2</I><TT>&nbsp;</TT><I>lex<SUB>12</SUB></I>
       <DD> ...
       <DD>
        <I>region-m</I><TT>&nbsp;</TT>
          <I>region-n</I><TT>&nbsp;</TT><I>lex<SUB>mn</SUB></I>
   </DL><TT>}</TT>
   <DT><TT>}</TT></DL></BLOCKQUOTE><P>
   \end{rawhtml}
   where \oxslabel{lex} specifies the magnetostatic-exchange length, in
   meters, defined by ${\rm lex} = \sqrt{2A/(\mu_0 M_s^2)}$.

   In the first case, the \oxslabel{A} block specifies $A_{ij}$ values
   on a region by region basis, where the regions are labels declared by
   \oxsval{atlas\_spec}.  This allows for specification of $A$ both
   inside a given region (e.g., $A_{ii}$) and along interfaces between
   regions (e.g., $A_{ij}$).  By symmetry, if $A_{ij}$ is specified,
   then the same value is automatically assigned to $A_{ji}$ as well.
   The \oxslabel{default\_A} value is applied to any otherwise
   unassigned $A_{ij}$.

   In the second case, one specifies the magnetostatic-exchange length
   instead of $A$, but the interpretation is otherwise analogous.

   Although one may specify $A_{ij}$ (resp.\ ${\rm lex}_{ij}$) for any
   pair of regions $i$ and $j$, it is only required and only active if
   the region pair are in contact.  If long-range exchange interaction
   is required, use \cd{Oxs\_TwoSurfaceExchange}.

   In addition to the standard energy and field outputs,
   \cd{Oxs\_Exchange6Ngbr} provides three scalar outputs involving the
   angle between spins at neighboring cells:
\begin{itemize}
\item \textbf{Max Spin Ang:} maximum angle, in degrees, between
  neigboring spins for the current magnetization state.
\item \textbf{Stage Max Spin Ang:} Maximum value of \cd{Max Spin Ang}
  for the current stage.
\item \textbf{Run Max Spin Ang:} Maximum value obtained by
  \cd{Max Spin Ang} during the simulation.
\end{itemize}

   \begin{ExampleMifs}
     \fn{grill.mif}, \fn{spinvalve.mif}, \fn{tclshapes.mif}.
   \end{ExampleMifs}

\pttarget{PTUE}\index{Oxs\_Ext~child~classes!Oxs\_UniformExchange}%
\item[Oxs\_UniformExchange:]
   Similar to \cd{Oxs\_Exchange6Ngbr}, except the exchange constant $A$
   (or exchange length ${\rm lex}$) is uniform across all space.  The
   Specify block is very simple, consisting of either the label
   \oxslabel{A} with the desired exchange coefficient value in J/m, or
   the label \oxslabel{lex} with the desired magnetostatic-exchange
   length in meters.  Since \cd{A} (resp.\ \cd{lex}) is not spatially
   varying, it is initialized with a simple constant (as opposed to a
   scalar field object).

   In addition to the standard energy and field outputs,
   \cd{Oxs\_UniformExchange} provides the three scalar outputs
   \cd{Max Spin Ang}, \cd{Stage Max Spin Ang}, and
   \cd{Run Max Spin Ang} as described for \cd{Oxs\_Exchange6Ngbr}.
   These values are also accessible through the \MIF\
   \htmlonlyref{\cd{GetStateData}}{html:GetStateData} command.

   \begin{ExampleMifs}
     \fn{sample.mif}, \fn{cgtest.mif}, \fn{stdprob3.mif}.
   \end{ExampleMifs}

\pttarget{PTEP}\index{Oxs\_Ext~child~classes!Oxs\_ExchangePtwise}%
\item[Oxs\_ExchangePtwise:]
   The exchange coefficient $A_i$ is specified on
   a point-by-point (or cell-by-cell) basis, as opposed to the pairwise
   specification model used by \cd{Oxs\_Exchange6Ngbr}.  The exchange
   energy density at a cell $i$ is computed across its nearest 6 neighbors,
   $N_i$, using the formula
   \begin{displaymath}
        E_i =  \sum_{j\in N_i} A_{ij,{\rm eff}}
         \frac{\vm_i\cdot\left(\vm_i - \vm_j\right)}{\Delta_{ij}^2}
   \end{displaymath}
   where $\Delta_{ij}$ is the discretization step size from cell $i$ to
   cell $j$ in meters, and
   \begin{displaymath}
         A_{ij,{\rm eff}} = \frac{2A_iA_j}{A_i+A_j},
   \end{displaymath}
   with $A_{ij,{\rm eff}} = 0$ if $A_i$ and $A_j$ are 0.

   Note that $A_{ij,{\rm eff}}$ satisfies
   the following properties:
   \begin{eqnarray*}
        A_{ij,{\rm eff}} & = & A_{ji,{\rm eff}} \\
        A_{ij,{\rm eff}} & = & A_i \qquad \mbox{if $A_i=A_j$} \\
        \lim_{A_i\downarrow 0} A_{ij,{\rm eff}} & = & 0.
   \end{eqnarray*}
   Additionally, if $A_i$ and $A_j$ are non-negative,
   \begin{displaymath}
        \min(A_i,A_j) \leq  A_{ij,{\rm eff}}  \leq \max(A_i,A_j).
   \end{displaymath}
   Evaluating the exchange energy with this formulation of $A_{ij,{\rm
   eff}}$ is equivalent to finding the minimum possible exchange energy
   between cells $i$ and $j$ under the assumption that $A_i$ and $A_j$
   are constant in each of the two cells.  Similar considerations are
   made in computing the exchange energy for a \htmlonlyref{2D variable
   thickness model}{html:mifvariablethickness} \cite{porter2001}.

   The Specify block for \cd{Oxs\_ExchangePtwise} has the form
   \begin{latexonly}
      \begin{quote}\tt
      Specify Oxs\_ExchangePtwise:\oxsval{name} \ocb\\
        \bi A \oxsval{scalarfield\_spec}\\
      \ccb
      \end{quote}
   \end{latexonly}
   \begin{rawhtml}
   <BLOCKQUOTE><DL><DT>
   <TT>Specify Oxs_ExchangePtwise:</TT><I>name</I> <TT>{</TT>
       <DD> <TT>A </TT><I>scalarfield_spec</I>
   <DT><TT>}</TT></DL></BLOCKQUOTE><P>
   \end{rawhtml}
   where \oxsval{scalarfield\_spec} is an arbitrary
   \hyperrefhtml{scalar field object}{scalar field object
   (Sec.~}{)}{sec:oxsFieldObjects} returning the desired exchange
   coefficient in J/m.

   In addition to the standard energy and field outputs,
   \cd{Oxs\_ExchangePtwise} provides the three scalar outputs
   \cd{Max Spin Ang}, \cd{Stage Max Spin Ang}, and
   \cd{Run Max Spin Ang} as described for \cd{Oxs\_Exchange6Ngbr}.

   \begin{ExampleMifs}[Example]
     \fn{antidots-filled.mif}.
   \end{ExampleMifs}

\pttarget{PTTS}\index{Oxs\_Ext~child~classes!Oxs\_TwoSurfaceExchange}%
\item[Oxs\_TwoSurfaceExchange:]
   Provides long-range bilinear and biquadratic exchange.  Typically
   used to simulate RKKY-style coupling across non-magnetic spacers in
   spinvalves.  The specify block has the form
   \begin{latexonly}
      \begin{quote}\tt
      Specify Oxs\_TwoSurfaceExchange:\oxsval{name} \ocb\\
        \bi sigma \oxsval{value}\\
        \bi sigma2 \oxsval{value}\\
        \bi surface1 \ocb\\
        \bi \bi atlas  \oxsval{atlas\_spec}\\
        \bi \bi region \oxsval{region\_label}\\
        \bi \bi scalarfield \oxsval{scalarfield\_spec}\\
        \bi \bi scalarvalue \oxsval{fieldvalue}\\
        \bi \bi scalarside  \oxsval{side}\\
        \bi \ccb\\
        \bi surface2 \ocb\\
        \bi \bi atlas  \oxsval{atlas\_spec}\\
        \bi \bi region \oxsval{region\_label}\\
        \bi \bi scalarfield \oxsval{scalarfield\_spec}\\
        \bi \bi scalarvalue \oxsval{fieldvalue}\\
        \bi \bi scalarside  \oxsval{side}\\
        \bi \ccb\\
      \ccb
      \end{quote}
   \end{latexonly}
   \begin{rawhtml}
   <BLOCKQUOTE><DL><DT>
   <TT>Specify Oxs_TwoSurfaceExchange:</TT><I>name</I> <TT>{</TT>
   <DD><TT>sigma  </TT><I>value</I>
   <DD><TT>sigma2 </TT><I>value</I>
   <DD><TT>surface1 {</TT><DL>
       <DD> <TT>atlas </TT><I>atlas_spec</I>
       <DD> <TT>region </TT><I>region_label</I>
       <DD> <TT>scalarfield </TT><I>scalarfield_spec</I>
       <DD> <TT>scalarvalue </TT><I>fieldvalue</I>
       <DD> <TT>scalarside </TT><I>side</I>
   </DL><TT>}</TT>
   <DD><TT>surface2 {</TT><DL>
       <DD> <TT>atlas </TT><I>atlas_spec</I>
       <DD> <TT>region </TT><I>region_label</I>
       <DD> <TT>scalarfield </TT><I>scalarfield_spec</I>
       <DD> <TT>scalarvalue </TT><I>fieldvalue</I>
       <DD> <TT>scalarside </TT><I>side</I>
   </DL><TT>}</TT>
   <DT><TT>}</TT></DL></BLOCKQUOTE><P>
   \end{rawhtml}
   Here \textbf{sigma} and \textbf{sigma2} are the bilinear and
   biquadratic surface (interfacial) exchange energies, in
   J/m${}^2$.  Either is optional, with default value 0.

   The \oxslabel{surface1} and \oxslabel{surface2} sub-blocks describe
   the two interacting surfaces.  Each description consists of 5
   name-values pairs, which must be listed in the order shown.  In each
   sub-block, \oxsval{atlas\_spec} specifies an atlas, and
   \oxsval{region\_label} specifies a region in that atlas.  These bound
   the extent of the desired surface.  The following
   \oxslabel{scalarfield}, \oxslabel{scalarvalue} and
   \oxslabel{scalarside} entries define a discretized surface inside the
   bounding region.  Here \oxsval{scalarfield\_spec} references a scalar
   field object, \oxsval{fieldvalue} should be a floating point value,
   and \oxsval{side} should be one of \cd{<}, \cd{<=}, \cd{>=}, or
   \cd{>}.  Any point for which the scalar field object takes a value
   less than, less than or equal, greater than or equal, or greater
   than, respectively, the \cd{scalarvalue} value is considered to be
   ``inside'' the surface. (Values \cd{-} and \cd{+} for \oxsval{side}
   are deprecated synonyms for \cd{<=} and \cd{>=}.)  The discretized
   surface determined is the set of all points on the problem mesh that
   are in the bounding region, are ``inside'' the surface, and have a
   (nearest-) neighbor that is ``outside'' (i.e., not inside) the
   surface.  A neighbor is determined by the mesh; in a typical
   rectangular mesh each cell has six neighbors.

   In this way, 2 discrete lists of cells representing the two
   surfaces are obtained.  Each cell from the first list (representing
   \cd{surface1}) is then matched with the closest cell from the
   second list (i.e., from \cd{surface2}).  Note the asymmetry in
   this matching process: each cell from the first list is included in
   exactly one match, but there may be cells in the second list that
   are included in many match pairs, or in none.  If the two surfaces
   are of different sizes, then in practice typically the smaller will
   be made the first surface, because this will usually lead to fewer
   multiply-matched cells, but this designation is not required.

   The resulting exchange energy density at cell $i$ on one surface
   from matching cell $j$ on the other is given by
   \begin{displaymath}
        E_{ij} =  \frac{\sigma\left[1-\vm_i\cdot\vm_j\right]
         +\sigma_2\left[1-\left(\vm_i\cdot\vm_j\right)^2\right]
        }{\Delta_{ij}}
   \end{displaymath}
   where $\sigma$ and $\sigma_2$, respectively, are the bilinear and
   biquadratic surface exchange coefficients between the two surfaces,
   in J/m${}^2$, $\vm_i$ and $\vm_j$ are the normalized, unit spins
   (i.e., magnetization directions) at cells $i$ and $j$, and
   $\Delta_{ij}$ is the discretization cell size in the direction from
   cell $i$ towards cell $j$, in meters.  Note that if $\sigma$ is
   negative, then the surfaces will be anti-ferromagnetically coupled.
   Likewise, if $\sigma_2$ is negative, then the biquadratic term will
   favor orthogonal alignment.

   The following example produces an antiferromagnetic exchange coupling
   between the lower surface of the ``top'' layer and the upper surface
   of the ``bottom'' layer, across a middle ``spacer'' layer.  The
   simple \cd{Oxs\_LinearScalarField} object is used here to provide
   level surfaces that are planes orthogonal to the $z$-axis.  In
   practice this example might represent a spinvalve, where the top and
   bottom layers would be composed of ferromagnetic material and the
   middle layer could be a copper spacer.
% The extra BLOCKQUOTE's here are a workaround for an apparent
% latex2html bug
\begin{rawhtml}
<BLOCKQUOTE>
\end{rawhtml}
%begin{latexonly}
\begin{quote}
%end{latexonly}
\begin{verbatim}
Specify Oxs_MultiAtlas:atlas {
    atlas { Oxs_BoxAtlas {
        name top
        xrange {0 500e-9}
        yrange {0 250e-9}
        zrange {6e-9 9e-9}
    } }
    atlas { Oxs_BoxAtlas {
        name spacer
        xrange {0 500e-9}
        yrange {0 250e-9}
        zrange {3e-9 6e-9}
    } }
    atlas { Oxs_BoxAtlas {
        name bottom
        xrange {0 500e-9}
        yrange {0 250e-9}
        zrange {0 3e-9}
    } }
}

Specify Oxs_LinearScalarField:zheight {
    vector {0 0 1}
    norm   1.0
}

Specify Oxs_TwoSurfaceExchange:AF {
    sigma -1e-4
    surface1 {
               atlas  :atlas
              region  bottom
         scalarfield  :zheight
         scalarvalue  3e-9
          scalarside  <=
    }
    surface2 {
               atlas  :atlas
              region  top
         scalarfield  :zheight
         scalarvalue  6e-9
          scalarside  >=
    }
}
\end{verbatim}
%begin{latexonly}
\end{quote}
%end{latexonly}
\begin{rawhtml}
</BLOCKQUOTE>
\end{rawhtml}

   In addition to the standard energy and field outputs,
   \cd{Oxs\_TwoSurfaceExchange} provides the three scalar outputs
   \cd{Max Spin Ang}, \cd{Stage Max Spin Ang}, and
   \cd{Run Max Spin Ang} as described for \cd{Oxs\_Exchange6Ngbr}.

\begin{ExampleMifs}[Example]
  \fn{spinvalve-af.mif}.
\end{ExampleMifs}


\pttarget{PTSE}\index{Oxs\_Ext~child~classes!Oxs\_RandomSiteExchange}%
\item[Oxs\_RandomSiteExchange:]
   A randomized exchange energy.  The Specify block has the form
   \begin{latexonly}
      \begin{quote}\tt
      Specify Oxs\_RandomSiteExchange:\oxsval{name} \ocb\\
        \bi linkprob \oxsval{probability} \\
        \bi Amin \oxsval{A\_lower\_bound} \\
        \bi Amax \oxsval{A\_upper\_bound} \\
      \ccb
      \end{quote}
   \end{latexonly}
   \begin{rawhtml}
   <BLOCKQUOTE><DL><DT>
   <TT>Specify Oxs_RandomSiteExchange:</TT><I>name</I> <TT>{</TT>
       <DD> <TT>linkprob </TT><I>probability</I>
       <DD> <TT>Amin </TT><I>A_lower_bound</I>
       <DD> <TT>Amax </TT><I>A_upper_bound</I>
   <DT><TT>}</TT></DL></BLOCKQUOTE><P>
   \end{rawhtml}
   Each adjacent pair of cells $i$, $j$, is given \oxslabel{linkprob}
   probability of having a non-zero exchange coefficient $A_{ij}$.  Here
   two cells are adjacent if they lie in each other's 6-neighborhood.
   If a pair is found to have a non-zero exchange coefficient, then
   $A_{ij}$ is drawn uniformly from the range $[\cd{Amin},\cd{Amax}]$.
   The exchange energy is computed using (\ref{eq:ExchangeEnergy}), the
   formula used by the \cd{Oxs\_Exchange6Ngbr} energy object.  The
   value $A_{ij}$ for each pair of cells is determined during problem
   initialization, and is held fixed thereafter.  The limits
   \oxsval{A\_lower\_bound} and \oxsval{A\_upper\_bound} may be any real
   numbers; negative values may be used to weaken the exchange
   interaction arising from other exchange energy terms.  The only
   restriction is that \oxsval{A\_lower\_bound} must not be greater than
   \oxsval{A\_upper\_bound}.  The \cd{linkprob} value \oxsval{probability}
   must lie in the range $[0,1]$.

   In addition to the standard energy and field outputs,
   \cd{Oxs\_RandomSiteExchange} provides the three scalar outputs
   \cd{Max Spin Ang}, \cd{Stage Max Spin Ang}, and
   \cd{Run Max Spin Ang} as described for \cd{Oxs\_Exchange6Ngbr}.

   \begin{ExampleMifs}[Example]
     \fn{randexch.mif}.
   \end{ExampleMifs}

\end{description}

\starsssechead{Self-Magnetostatic Energy}
\begin{description}
\pttarget{PTDE}\index{Oxs\_Ext~child~classes!Oxs\_Demag}%
\item[Oxs\_Demag:]
   Standard demagnetization energy term, built upon
   the assumption that the magnetization is constant in each cell.
   It computes the average demagnetization field in each cell using
   formulae from \cite{aharoni1998,newell1993} and convolution
   via the Fast Fourier Transform.  This class supports non-periodic
   simulations if the mesh object in the \MIF\ file is of the
   \htmlonlyref{\cd{Oxs\_RectangularMesh}}{html:oxsrectangularmesh}
   type; for simulations periodic along one axis direction use the
   \htmlonlyref{\cd{Oxs\_PeriodicRectangularMesh}}{html:oxsperiodicrectangularmesh}
   class.  Periodicity in more than one direction is not supported at
   this time.  The specify block has the form
   \begin{latexonly}
      \begin{quote}\tt
        Specify Oxs\_Demag:\oxsval{name} \ocb\\
        \bi asymptotic\_order \oxsval{error\_order}\\
        \bi demag\_tensor\_error \oxsval{relerror}\\
      \ccb
      \end{quote}
   \end{latexonly}
   \begin{rawhtml}
   <BLOCKQUOTE><DL><DT>
   <TT>Specify Oxs_Demag:</TT><I>name</I> <TT>{</TT>
       <DD> <TT>asymptotic_order </TT><I>error_order</I>
       <DD> <TT>demag_tensor_error </TT><I>relerror</I>
   <DT><TT>}</TT></DL></BLOCKQUOTE><P>
   \end{rawhtml}
   The demag kernel is computed using a combination of analytic formulae
   for near field terms, high-order asymptotic formulae for far field
   terms, and summed subdivided cell asymptotic formulae for midrange
   terms, where the offset $R$ between cell pairs determines the field
   range (based on extensions of earlier work\cite{lebecki2008}).
   The transition $R$ values are selected to give the best computation
   speed while meeting the error requested by \oxsval{relerror}. The
   demag kernel computation is a one-time operation performed during
   problem initialization, so the kernel computation time is generally
   of relatively minor concern, and accordingly the default value for
   \oxsval{relerror} is 1e-15, i.e., nearly full double-precision
   accuracy.

   Asymptotic formulae are used to compute the demag kernel for larger
   cell offset pair distances $R$. By default an expansion with error
   $\mathcal{O}\left(1/R^{11}\right)$ is used, but lower orders can be
   requested through the \oxsval{error\_order} option. Valid values for
   \oxsval{error\_order} are 5, 7, 9, and 11, where
   \oxsval{error\_order}=5 is the dipole approximation.

   There is also backward support for the now deprecated option
   \oxslabel{asymptotic\_radius}, which set the cutoff between the
   analytic and asymptotic computation forms in units of cells. If
   \oxslabel{asymptotic\_radius} is specified then it is converted to a
   more-or-less equivalent value for \oxsval{relerror}, with
   the special values of 0 and -1 mapping to \oxsval{relerror}=1 and
   1e-16, respectively.

   The example file \fn{demagtensor.mif} can be used to extract the
   computed demagnetization tensor coefficients for a specified cell
   geometry; see the description at the top of that file for usage
   details.

   \begin{ExampleMifs}
     \fn{sample.mif}, \fn{cgtest.mif}, \fn{pbcbrick.mif}, \fn{demagtensor.mif}.
   \end{ExampleMifs}

\pttarget{PTSD}\index{Oxs\_Ext~child~classes!Oxs\_SimpleDemag}%
\item[Oxs\_SimpleDemag:]
   This is the same as the \cd{Oxs\_Demag} object, except that
   periodicity is not supported and asymptotic formulae are not used.
   The implementation does not use any of the symmetries
   inherent in the demagnetization kernel, or special properties of the
   Fourier Transform when applied to a real (non-complex) function.
   As a result, the source code is
   considerably simpler than for \cd{Oxs\_Demag}, but the run time
   performance and memory usage are poorer.  \cd{Oxs\_SimpleDemag} is
   included for validation checks, and as a base for user-defined
   demagnetization implementations.  The Specify initialization string
   for \cd{Oxs\_SimpleDemag} is an empty string, i.e., \ocb\ccb.

   \begin{ExampleMifs}[Example]
     \fn{squarecubic.mif}.
   \end{ExampleMifs}
\end{description}

\starsssechead{Zeeman Energy}
\begin{description}
\pttarget{PTUZ}\index{Oxs\_Ext~child~classes!Oxs\_UZeeman}%
\item[Oxs\_UZeeman:\label{html:UZeeman}]
   Uniform (homogeneous) applied field energy.  This class is frequently
   used for simulating hysteresis loops.  The specify block takes an
   optional \textbf{multiplier} entry, and a required field range list
   \textbf{Hrange}.  The field range list should be a compound list,
   with each sublist consisting of 7 elements: the first 3 denote the
   $x$, $y$, and $z$ components of the start field for the range, the
   next 3 denote the $x$, $y$, and $z$ components of the end field for
   the  range, and the last element specifies the number of (linear) steps
   through the range.  If the step count is 0, then the range consists
   of the start field only.  If the step count is bigger than 0, then
   the start field is skipped over if and only if it is the same field
   that ended the previous range (if any).

   The fields specified in the range entry are nominally in A/m, but
   these values are multiplied by \cd{multiplier}, which may be used to
   effectively change the units.  For example,
   \begin{latexonly}
      \begin{quote}\tt
      Specify Oxs\_UZeeman \ocb \\
         \bi multiplier 795.77472\\
         \bi Hrange \ocb\\
         \bi\bi \ocb\ 0 0 0 10 0 0 2 \ccb\\
         \bi\bi \ocb\ 10 0 0 0 0 0 1 \ccb\\
         \bi\ccb\\
      \ccb
      \end{quote}
   \end{latexonly}
   \begin{rawhtml}
   <BLOCKQUOTE><DL><DT>
      <TT>Specify Oxs_UZeeman {</TT>
         <DD><TT> multiplier 795.77472</TT>
         <DD><TT> Hrange {</TT><DL>
              <DD><TT> { 0 0 0 10 0 0 2 }</TT>
              <DD><TT> { 10 0 0 0 0 0 1 }</TT>
         </DL><TT>}</TT>
   <DT><TT>}</TT></DL></BLOCKQUOTE><P>
   \end{rawhtml}
   The applied field steps between 0~mT, 5~mT, 10~mT and back to 0~mT,
   for four stages in total.  If the first field in the second range
   sublist was different from the second field in the first range
   sublist, then a step would have been added between those field
   values, so five stages would have resulted.  In this example, note
   that 795.77472=0.001/\munaught.

   In addition to the standard energy and field outputs, the
   \cd{Oxs\_UZeeman} class provides these four scalar outputs:
   \begin{itemize}
   \item \textbf{B:} Magnitude of the applied field, in
   mT.  This is a non-negative quantity.
   \item \textbf{Bx:} Signed amplitude of the $x$-component
   of the applied field, in mT.
   \item \textbf{By:} Signed amplitude of the $y$-component
   of the applied field, in mT.
   \item \textbf{Bz:} Signed amplitude of the $z$-component
   of the applied field, in mT.
   \end{itemize}

   \begin{ExampleMifs}
     \fn{sample.mif}, \fn{cgtest.mif}, \fn{marble.mif}.
   \end{ExampleMifs}

\pttarget{PTFZ}\index{Oxs\_Ext~child~classes!Oxs\_FixedZeeman}%
\item[Oxs\_FixedZeeman:]
   Non-uniform, non-time varying applied field.
   This can be used to simulate a biasing field.  The specify block
   takes one required parameter, which defines the field, and one
   optional parameter, which specifies a multiplication factor.
      \begin{latexonly}
      \begin{quote}\tt
      Specify Oxs\_FixedZeeman:\oxsval{name} \ocb\\
       \bi field \oxsval{vector\_field\_spec}\\
       \bi multiplier \oxsval{multiplier}\\
      \ccb
      \end{quote}
      \end{latexonly}
      \begin{rawhtml}
      <BLOCKQUOTE><DL><DT>
      <TT>Specify Oxs_FixedZeeman:</TT><I>name</I> <TT>{</TT>
      <DD><TT> field </TT> <I>vector_field_spec</I>
      <DD><TT> multiplier </TT> <I>multiplier</I>
      <DT><TT>}</TT></DL></BLOCKQUOTE><P>
      \end{rawhtml}
   The default value for \oxsval{multiplier} is 1.  The field units,
   after scaling by \oxsval{multiplier}, should be A/m.

   \begin{ExampleMifs}
     \fn{spinvalve.mif}, \fn{spinvalve-af.mif}, \fn{yoyo.mif}.
   \end{ExampleMifs}

\pttarget{PTSU}\index{Oxs\_Ext~child~classes!Oxs\_ScriptUZeeman}%
\item[Oxs\_ScriptUZeeman:]
   Spatially uniform applied field,
   potentially varying as a function of time and stage, determined by a
   \Tcl\ script.  The Specify block has the form
      \begin{latexonly}
      \begin{quote}\tt
      Specify Oxs\_ScriptUZeeman:\oxsval{name} \ocb\\
       \bi script\_args \ocb\oxsval{ args\_request }\ccb\\
       \bi script \oxsval{\Tcl\_script}\\
       \bi multiplier \oxsval{multiplier}\\
       \bi stage\_count \oxsval{number\_of\_stages} \\
      \ccb
      \end{quote}
      \end{latexonly}
      \begin{rawhtml}
      <BLOCKQUOTE><DL><DT>
      <TT>Specify Oxs_ScriptUZeeman:</TT><I>name</I> <TT>{</TT>
      <DD><TT>script_args {</TT> <I>args_request</I> <TT>}</TT>
      <DD><TT>script </TT> <I>Tcl_script</I>
      <DD><TT>multiplier </TT> <I>multiplier</I>
      <DD><TT>stage_count </TT> <I>number_of_stages</I>
      <DT><TT>}</TT></DL></BLOCKQUOTE><P>
      \end{rawhtml}
   Here \oxslabel{script} indicates the \Tcl\ script to use.  The script
   is called once each iteration.  Appended to the script are the
   arguments requested by \oxslabel{script\_args}, in the manner
   explained in the \hyperrefhtml{User Defined Support Procedures}{User
   Defined Support Procedures section (Sec.~}{)}{par:supportProcs}
   \html{section} of the \MIF~2 file format documentation.  The value
   \oxsval{args\_request} should be a subset of \cd{\ocb stage
   stage\_time total\_time base\_state\_id current\_state\_id\ccb}.
   The units for the time options are seconds.  The two
   \cd{state\_id} options are intended for use with the
   \MIF\ \htmlonlyref{\cd{GetStateData}}{html:GetStateData} command;
   refer to the documentation on that command in the \MIF~2.1 section
   for details.  If \cd{script\_args} is not specified, the default
   argument list is \cd{\ocb stage stage\_time total\_time\ccb}.

   The return value from the script should be a 6-tuple of numbers,
   \ocb\cd{Hx}, \cd{Hy}, \cd{Hz}, \cd{dHx}, \cd{dHy}, \cd{dHz}\ccb,
   representing the applied field and the time derivative of the applied
   field.  The field as a function of time must be differentiable for
   the duration of each stage.  Discontinuities are permitted between
   stages.  If a time evolver is being used, then it is very important
   that the time derivative values are correct; otherwise the evolver
   will not function properly.  This usual symptom of this problem is a
   collapse in the time evolution step size.

   The field and its time derivative are multiplied by the
   \oxslabel{multiplier} value before use.  The final field value should
   be in A/m; if the \Tcl\ script returns the field in T, then a
   \cd{multiplier} value of 1/\munaught\ (approx.\ 795774.72) should be
   applied to convert the \Tcl\ result into A/m.  The default value for
   \cd{multiplier} is 1.

   The \oxslabel{stage\_count} parameter informs the
   \hyperrefhtml{\cd{Oxs\_Driver}}{\cd{Oxs\_Driver}
   (Sec.~}{)}{sec:oxsDrivers} as to how many stages the
   \cd{Oxs\_ScriptUZeeman} object wants.  A value of 0 (the default)
   indicates that the object is prepared for any range of stages.  The
   \cd{stage\_count} value given here must be compatible with the
   \arbtargetlink{\cd{stage\_count} setting in the driver Specify
   block}{\cd{stage\_count} setting in the driver Specify block
   (page~}{)}{PToxsdriverstagecount}.


   The following example produces a sinusoidally varying field of
   frequency 1 GHz and amplitude 800 A/m, directed along the $x$-axis.
% The extra BLOCKQUOTE's here are a workaround for an apparent
% latex2html bug
\begin{rawhtml}
<BLOCKQUOTE>
\end{rawhtml}
%begin<latexonly>
\begin{quote}
%end<latexonly>
\begin{verbatim}
proc SineField { total_time } {
    set PI [expr {4*atan(1.)}]
    set Amp 800.0
    set Freq [expr {1e9*(2*$PI)}]
    set Hx [expr {$Amp*sin($Freq*$total_time)}]
    set dHx [expr {$Amp*$Freq*cos($Freq*$total_time)}]
    return [list $Hx 0 0 $dHx 0 0]
}

Specify Oxs_ScriptUZeeman {
   script_args total_time
   script SineField
}
\end{verbatim}
%begin<latexonly>
\end{quote}
%end<latexonly>
\begin{rawhtml}
</BLOCKQUOTE>
\end{rawhtml}

   In addition to the standard energy and field outputs, the
   \cd{Oxs\_ScriptUZeeman} class provides these four scalar outputs:
   \begin{itemize}
   \item \textbf{B:} Magnitude of the applied field, in
   mT.  This is a non-negative quantity.
   \item \textbf{Bx:} Signed amplitude of the $x$-component
   of the applied field, in mT.
   \item \textbf{By:} Signed amplitude of the $y$-component
   of the applied field, in mT.
   \item \textbf{Bz:} Signed amplitude of the $z$-component
   of the applied field, in mT.
   \end{itemize}

   \begin{ExampleMifs}
     \fn{acsample.mif}, \fn{pulse.mif}, \fn{rotate.mif},
     \fn{varalpha.mif}, \fn{yoyo.mif}.
   \end{ExampleMifs}

\pttarget{PTTZ}\index{Oxs\_Ext~child~classes!Oxs\_TransformZeeman}%
\item[Oxs\_TransformZeeman:]
   Essentially a combination of the \cd{Oxs\_FixedZeeman} and
   \cd{Oxs\_ScriptUZeeman} classes, where an applied field is produced
   by applying a spatially uniform, but time and stage varying linear
   transform to a spatially varying but temporally static field.  The
   transform is specified by a \Tcl\ script.

   The Specify block has the form
      \begin{latexonly}
      \begin{quote}\tt
      Specify Oxs\_TransformZeeman:\oxsval{name} \ocb\\
       \bi field \oxsval{vector\_field\_spec}\\
       \bi type \oxsval{transform\_type}\\
       \bi script \oxsval{\Tcl\_script}\\
       \bi script\_args \ocb\oxsval{ args\_request }\ccb\\
       \bi multiplier \oxsval{multiplier}\\
       \bi stage\_count \oxsval{number\_of\_stages} \\
      \ccb
      \end{quote}
      \end{latexonly}
      \begin{rawhtml}
      <BLOCKQUOTE><DL><DT>
      <TT>Specify Oxs_TransformZeeman:</TT><I>name</I> <TT>{</TT>
      <DD><TT>field </TT> <I>vector_field_spec</I>
      <DD><TT>type </TT> <I>transform_type</I>
      <DD><TT>script </TT> <I>Tcl_script</I>
      <DD><TT>script_args {</TT> <I>args_request</I> <TT>}</TT>
      <DD><TT>multiplier </TT> <I>multiplier</I>
      <DD><TT>stage_count </TT> <I>number_of_stages</I>
      <DT><TT>}</TT></DL></BLOCKQUOTE><P>
      \end{rawhtml}
   The \oxslabel{field} specified by \oxsval{vector\_field\_spec} is
   evaluated during problem initialization and held throughout the life
   of the problem.  On each iteration, the specified \Tcl\
   \oxslabel{script} is called once.  Appended to the script are the
   arguments requested by \oxslabel{script\_args}, as explained in the
   \hyperrefhtml{User Defined Support Procedures}{User Defined Support
   Procedures section (Sec.~}{)}{par:supportProcs}\html{ section} of the
   \MIF~2 file format documentation.  The value for \cd{script\_args}
   should be a subset of \cd{\ocb stage stage\_time total\_time\ccb}.
   The default value for \cd{script\_args} is the complete list in the
   aforementioned order.  The time arguments are specified in seconds.

   The script return value should define a 3x3 linear transform and its
   time derivative.  The transform must be differentiable with respect
   to time throughout each stage, but is allowed to be discontinuous
   between stages.  As noted in the \cd{Oxs\_ScriptUZeeman}
   documentation, it is important that the derivative information be
   correct.  The transform is applied pointwise to the fixed
   field obtained from \oxsval{vector\_field\_spec}, which is
   additionally scaled by \oxsval{multiplier}.  The
   \oxslabel{multiplier} entry is optional, with default value 1.0.

   The \oxslabel{type} \oxsval{transform\_type} value declares the
   format of the result returned from the \Tcl\ script.  Recognized
   formats are \cd{identity}, \cd{diagonal}, \cd{symmetric} and
   \cd{general}.  The most flexible is \cd{general}, which indicates
   that the return from the \Tcl\ script is a list of 18 numbers,
   defining a general 3x3 matrix and its 3x3 matrix of time derivatives.
   The matrices are specified in row-major order, i.e., $M_{1,1}$,
   $M_{1,2}$, $M_{1,3}$, $M_{2,1}$, $M_{2,2}$, \ldots.  Of course, this
   is a long list to construct; if the desired transform is symmetric or
   diagonal, then the \cd{type} may be set accordingly to reduce the
   size of the \Tcl\ result string.  Scripts of the \cd{symmetric} type
   return 12 numbers, the 6 upper diagonal entries in row-major order,
   i.e., $M_{1,1}$, $M_{1,2}$, $M_{1,3}$, $M_{2,2}$, $M_{2,3}$,
   $M_{3,3}$, for both the transformation matrix and its time
   derivative.  Use the \cd{diagonal} type for diagonal matrices, in
   which case the \Tcl\ script result should be a list of 6 numbers.

   The simplest \oxsval{transform\_type} is \cd{identity}, which is the
   default.  This identifies the transform as the identity matrix, which
   means that effectively no transform is applied, aside from the
   \cd{multiplier} option which is still active.  For the \cd{identity}
   transform type, \cd{script} and \cd{script\_args} should not be
   specified, and \cd{Oxs\_TransformZeeman} becomes a clone of the
   \cd{Oxs\_FixedZeeman} class.

   The following example produces a 1000 A/m field that rotates in the
   $xy$-plane at a frequency of 1 GHz:
% The extra BLOCKQUOTE's here are a workaround for an apparent
% latex2html bug
\begin{rawhtml}
<BLOCKQUOTE>
\end{rawhtml}
%begin<latexonly>
\begin{quote}
%end<latexonly>
\begin{verbatim}
proc Rotate { freq stage stagetime totaltime } {
   global PI
   set w [expr {$freq*2*$PI}]
   set ct [expr {cos($w*$totaltime)}]
   set mct [expr {-1*$ct}]      ;# "mct" is "minus cosine (w)t"
   set st [expr {sin($w*$totaltime)}]
   set mst [expr {-1*$st}]      ;# "mst" is "minus sine (w)t"
   return [list  $ct $mst  0 \
                 $st $ct   0 \
                   0   0   1 \
                 [expr {$w*$mst}] [expr {$w*$mct}] 0 \
                 [expr {$w*$ct}]  [expr {$w*$mst}] 0 \
                        0                0         0]
}

Specify Oxs_TransformZeeman {
  type general
  script {Rotate 1e9}
  field {0 1000. 0}
}
\end{verbatim}
%begin<latexonly>
\end{quote}
%end<latexonly>
\begin{rawhtml}
</BLOCKQUOTE>
\end{rawhtml}
This particular effect could be obtained using the
\cd{Oxs\_ScriptUZeeman} class, because the \cd{field} is uniform.
But the field was taken uniform only to simplify the example.  The
\oxsval{vector\_field\_spec} may be any \hyperrefhtml{Oxs vector field
object}{Oxs vector field object (Sec.~}{)}{sec:oxsFieldObjects}.  For
example, the base field could be large in the center of the sample, and
decay towards the edges.  In that case, the above example would generate
an applied rotating field that is concentrated in the center of the
sample.

The \oxslabel{stage\_count} parameter informs the
\hyperrefhtml{\cd{Oxs\_Driver}}{\cd{Oxs\_Driver}
(Sec.~}{)}{sec:oxsDrivers} as to how many stages the
\cd{Oxs\_TransformZeeman} object wants.  A value of 0 (the default)
indicates that the object is prepared for any range of stages.  The
\cd{stage\_count} value given here must be compatible with the
\arbtargetlink{\cd{stage\_count} setting in the driver Specify
block}{\cd{stage\_count} setting in the driver Specify block
(page~}{)}{PToxsdriverstagecount}.

\begin{ExampleMifs}
  \fn{sample2.mif}, \fn{tickle.mif}, \fn{rotatecenter.mif}.
\end{ExampleMifs}

\pttarget{PTSZ}\index{Oxs\_Ext~child~classes!Oxs\_StageZeeman}%
\item[Oxs\_StageZeeman:]
   The \cd{Oxs\_StageZeeman} class provides spatially varying applied
   fields that are updated once per stage.  In its general form, the
   field at each stage is provided by an \hyperrefhtml{Oxs vector field
   object}{Oxs vector field object (Sec.~}{)}{sec:oxsFieldObjects}
   determined by a user supplied \Tcl\ script.  There is also a
   simplified interface that accepts a list of \hyperrefhtml{vector
   field files}{vector field files (Ch.~}{)}{sec:vfformats}, one per
   stage, that are used to specify the applied field.

   The Specify block takes the form
      \begin{latexonly}
      \begin{quote}\tt
      Specify Oxs\_StageZeeman:\oxsval{name} \ocb\\
       \bi script \oxsval{Tcl\_script}\\
       \bi files \ocb\oxsval{ list\_of\_files }\ccb\\
       \bi stage\_count \oxsval{number\_of\_stages} \\
       \bi multiplier \oxsval{multiplier}\\
      \ccb
      \end{quote}
      \end{latexonly}
      \begin{rawhtml}
      <BLOCKQUOTE><DL><DT>
      <TT>Specify Oxs_StageZeeman:</TT><I>name</I> <TT>{</TT>
      <DD><TT>script </TT> <I>Tcl_script</I>
      <DD><TT>files </TT> <TT>{</TT> <I>list_of_files</I> <TT>}</TT>
      <DD><TT>stage_count </TT> <I>number_of_stages</I>
      <DD><TT>multiplier </TT> <I>multiplier</I>
      <DT><TT>}</TT></DL></BLOCKQUOTE><P>
      \end{rawhtml}
   The initialization string should specify either \cd{script} or
   \cd{files}, but not both.  If a \oxslabel{script} is specified,
   then each time a new stage is started in the simulation, a \Tcl\
   command is formed by appending to \oxsval{Tcl\_script} the 0-based
   integer stage number.  This command should return a reference to an
   \cd{Oxs\_VectorField} object, as either the instance name of an
   object defined via a top-level Specify block elsewhere in the
   \MIF\ file, or as a two item list consisting of the name of an
   \cd{Oxs\_VectorField} class and an appropriate initialization string.
   In the latter case the \cd{Oxs\_VectorField} object will be created
   as a temporary object via an inlined Specify call.

   The following example should help clarify the use of the \cd{script}
   parameter.
% The extra BLOCKQUOTE's here are a workaround for an apparent
% latex2html bug
\begin{rawhtml}
<BLOCKQUOTE>
\end{rawhtml}
%begin<latexonly>
\begin{quote}
%end<latexonly>
\begin{verbatim}
proc SlidingField { xcutoff xrel yrel zrel } {
   if {$xrel>$xcutoff} { return [list 0. 0. 0.] }
   return [list 2e4 0. 0.]
}

proc SlidingFieldSpec { stage } {
  set xcutoff [expr {double($stage)/10.}]
  set spec Oxs_ScriptVectorField
  lappend spec [subst {
      atlas :atlas
      script {SlidingField $xcutoff}
   }]
   return $spec
}

Specify Oxs_StageZeeman {
  script SlidingFieldSpec
  stage_count 11
}
\end{verbatim}
%begin<latexonly>
\end{quote}
%end<latexonly>
\begin{rawhtml}
</BLOCKQUOTE>
\end{rawhtml}

   The \cd{SlidingFieldSpec} proc is used to generate the initialization
   string for an \cd{Oxs\_ScriptVectorField} vector field object, which
   in turn uses the \cd{SlidingField} proc to specify the applied field
   on a position-by-position basis.  The resulting field will be
   \latex{$2\times 10^4$}\html{2e4} A/m in the positive x-direction at
   all points with relative x-coordinate larger than \cd{\$stage/10.},
   and 0 otherwise.  \cd{\$stage} is the stage index, which here is
   one of 0, 1, \ldots, 10.  For example, if \cd{\$stage} is 5, then the
   left half of the sample will see a \latex{$2\times 10^4$}\html{2e4}
   A/m field directed to the right, and the right half of the sample
   will see none.  The return value from \cd{SlidingFieldSpec} in this
   case will be
% The extra BLOCKQUOTE's here are a workaround for an apparent
% latex2html bug
\begin{rawhtml}
<BLOCKQUOTE>
\end{rawhtml}
%begin<latexonly>
\begin{quote}
%end<latexonly>
\begin{verbatim}
Oxs_ScriptVectorField {
   atlas :atlas
   script {SlidingField 0.5}
}
\end{verbatim}
%begin<latexonly>
\end{quote}
%end<latexonly>
\begin{rawhtml}
</BLOCKQUOTE>
\end{rawhtml}
   The \verb+:atlas+ reference is to an \cd{Oxs\_Atlas} object defined
   elsewhere in the \MIF\ file.

   The \oxslabel{stage\_count} parameter lets the
   \hyperrefhtml{\cd{Oxs\_Driver}}{\cd{Oxs\_Driver}
   (Sec.~}{)}{sec:oxsDrivers} know how many stages the
   \cd{Oxs\_StageZeeman} object wants.  A value of 0 indicates that the
   object is prepared for any range of stages.  Zero is the default
   value for \cd{stage\_count} when using the \oxsval{Tcl\_script}
   interface.  The \cd{stage\_count} value given here must be compatible
   with the \arbtargetlink{\cd{stage\_count} setting in the driver
   Specify block}{\cd{stage\_count} setting in the driver Specify block
   (page~}{)}{PToxsdriverstagecount}.


   The example above made use of two scripts, one to specify the
   \cd{Oxs\_VectorField} object, and one used internally by the
   \cd{Oxs\_ScriptVectorField} object.  But any \cd{Oxs\_VectorField}
   class may be used, as in the next example.
% The extra BLOCKQUOTE's here are a workaround for an apparent
% latex2html bug
\begin{rawhtml}
<BLOCKQUOTE>
\end{rawhtml}
%begin<latexonly>
\begin{quote}
%end<latexonly>
\begin{verbatim}
proc FileField { stage } {
  set filelist { field-a.ohf field-b.ohf field-c.ohf }
  set spec Oxs_FileVectorField
  lappend spec [subst {
      atlas :atlas
      file [lindex $filelist $stage]
   }]
   return $spec
}

Specify Oxs_StageZeeman {
  script FileField
  stage_count 3
}
\end{verbatim}
%begin<latexonly>
\end{quote}
%end<latexonly>
\begin{rawhtml}
</BLOCKQUOTE>
\end{rawhtml}
   The \cd{FileField} proc yields a specification for an
   \cd{Oxs\_FileVectorField} object that loads one of three files,
   \cd{field-a.ohf}, \cd{field-b.ohf}, or \cd{field-c.ohf}, depending on
   the stage number.

   Specifying applied fields from a sequence of files is common enough
   to warrant a simplified interface.  This is the purpose of the
   \oxslabel{files} parameter:
% The extra BLOCKQUOTE's here are a workaround for an apparent
% latex2html bug
\begin{rawhtml}
<BLOCKQUOTE>
\end{rawhtml}
%begin<latexonly>
\begin{quote}
%end<latexonly>
\begin{verbatim}
Specify Oxs_StageZeeman {
  files { field-a.ohf field-b.ohf field-c.ohf }
}
\end{verbatim}
%begin<latexonly>
\end{quote}
%end<latexonly>
\begin{rawhtml}
</BLOCKQUOTE>
\end{rawhtml}
   This is essentially equivalent to the preceding example, with two
   differences.  First, \cd{stage\_count} is not needed because
   \cd{Oxs\_StageZeeman} knows the length of the list of files.  You may
   specify \cd{stage\_count}, but the default value is the length of the
   \cd{files} list.  This is in contrast to the default value
   of 0 when using the \cd{script} interface.  If \cd{stage\_count} is
   set larger than the file list, then the last file is repeated as
   necessary to reach the specified size.

   The second difference is that no \cd{Oxs\_Atlas} is specified when
   using the \cd{files} interface.  The \cd{Oxs\_FileVectorField} object
   spatially scales the field read from the file to match a specified
   volume.  Typically a volume is specified by explicit reference to an
   atlas, but with the \cd{files} interface to \cd{Oxs\_StageZeeman} the
   file fields are implicitly scaled to match the whole of the meshed
   simulation volume.  This is the most common case; to obtain a
   different spatial scaling use the \cd{script} interface as
   illustrated above with a different atlas or an explicit x/y/z-range
   specification.

   The \oxsval{list\_of\_files} value is interpreted as a
   \htmlonlyref{\textit{grouped list}}{par:groupedLists}.  \latex{See
   the notes in Sec.~\ref{par:groupedLists} for details on grouped
   lists.}

   The remaining \cd{Oxs\_StageZeeman} parameter is
   \oxslabel{multiplier}.  The value of this parameter is applied as a
   scale factor to the field magnitude on a point-by-point basis.  For
   example, if the field returned by the \cd{Oxs\_VectorField} object
   were in Oe, instead of the required A/m, then \cd{multiplier} could
   be set to 79.5775 to perform the conversion.  The direction of the
   applied field can be reversed by supplying a negative \cd{multiplier}
   value.

   In addition to the standard energy and field outputs, the
   \cd{Oxs\_StageZeeman} class provides these four scalar outputs:
   \begin{itemize}
   \item \textbf{B max:} Pointwise maximum magnitude of the applied
     field, in mT.  This is a non-negative quantity;
     \begin{latexonly}
       $\textrm{B max} = \sqrt{(\textrm{Bx max})^2
                               +(\textrm{By max})^2+(\textrm{Bz max})^2}.$
     \end{latexonly}
     \begin{rawhtml}
B&nbsp;max&nbsp;=&nbsp;[(Bx&nbsp;max)<SUP>2</SUP>+(By&nbsp;max)<SUP>2</SUP>+(Bz&nbsp;max)<SUP>2</SUP>]<SUP>1/2</SUP>.
     \end{rawhtml}
   \item \textbf{Bx max:} Signed value of the $x$-component of the applied
     field at the point of maximum applied field magnitude, in mT.
   \item \textbf{By max:} Signed value of the $y$-component of the applied
     field at the point of maximum applied field magnitude, in mT.
   \item \textbf{Bz max:} Signed value of the $z$-component of the applied
     field at the point of maximum applied field magnitude, in mT.
   \end{itemize}

   \begin{ExampleMifs}
     \fn{sliding.mif}, \fn{slidingproc.mif}, \fn{rotatestage.mif},
     \fn{rotatecenterstage.mif}.
   \end{ExampleMifs}

\end{description}

\subsection{Evolvers\label{sec:oxsEvolvers}}
Evolvers are responsible for updating the magnetization configuration
from one step to the next.  There are two types of evolvers,
\textit{time evolvers}, which track Landau-Lifshitz-Gilbert dynamics,
and \textit{minimization evolvers}, which locate local minima in the
energy surface through direct minimization techniques.  Evolvers are
controlled by \hyperrefhtml{drivers}{\textit{drivers}
(Sec.~}{)}{sec:oxsDrivers}, and must be matched with the appropriate
driver type, i.e., time evolvers must be paired with
\htmlonlyref{time drivers}{item:TimeDriver}, and
minimization evolvers must be paired with \htmlonlyref{minimization
drivers}{html:MinDriver}.  The drivers hand a magnetization
configuration to the evolvers with a request to advance the
configuration by one \textit{step} (also called an \textit{iteration}).
It is the role of the drivers, not the evolvers, to determine when a
simulation stage or run is complete.  Specify blocks for evolvers
contain parameters to control all aspects of individual stepwise
evolution, but stopping criteria are communicated in the Specify block
of the driver, not the evolver.

There are currently three time evolvers and one minimization evolver in the
standard \OOMMF\ distribution.  The time evolvers are
\htmlonlyref{\cd{Oxs\_EulerEvolve}}{html:EulerEvolve},
\htmlonlyref{\cd{Oxs\_RungeKuttaEvolve}}{html:RungeKuttaEvolve}, and
\htmlonlyref{\cd{Oxs\_SpinXferEvolve}}{html:SpinXferEvolve}.
The minimization evolver is
\htmlonlyref{\cd{Oxs\_CGEvolve}}{html:CGEvolve}.
\begin{description}
\pttarget{PTEE}\index{Oxs\_Ext~child~classes!Oxs\_EulerEvolve}%
\item[Oxs\_EulerEvolve:\label{html:EulerEvolve}]
Time evolver implementing a simple first order forward Euler method with
step size control on the Landau-Lifshitz
ODE\index{ODE!Landau-Lifshitz}~\cite{gilbert1955,landau1935}:\\
\begin{equation}
\htmlimage{antialias}
  \frac{d\vM}{dt} = -|\bar{\gamma}|\,\vM\times\vH_{\rm eff}
   - \frac{|\bar{\gamma}|\alpha}{M_s}\,
     \vM\times\left(\vM\times\vH_{\rm eff}\right),
\label{eq:oxsllode}
\end{equation}
where $\vM$ is the magnetization, $\vH_{\rm eff}$ is the effective
field, \abovemath{\bar{\gamma}} is the Landau-Lifshitz gyromagnetic ratio, and
\abovemath{\alpha} is the damping constant. The Gilbert form
\begin{equation}
\htmlimage{antialias}
  \frac{d\vM}{dt} = -|\gamma|\,\vM\times\vH_{\rm eff}
   + \frac{\alpha}{M_s}
     \left(\vM\times\frac{d\vM}{dt}\right),
\label{eq:oxsllgode}
\end{equation}
where \abovemath{\gamma} is the Gilbert gyromagnetic ratio, is
mathematically equivalent to the Landau-Lifshitz form under the
relation \abovemath{\gamma = (1+\alpha^2)\,\bar{\gamma}}.

The Specify block has the form
   \begin{latexonly}
   \begin{quote}\tt
   Specify Oxs\_EulerEvolve:\oxsval{name} \ocb\\
    \bi alpha                  \oxsval{$\alpha$}\\
    \bi gamma\_LL              \oxsval{$\bar{\gamma}$}\\
    \bi gamma\_G               \oxsval{$\gamma$}\\
    \bi do\_precess            \oxsval{precess}\\
    \bi min\_timestep          \oxsval{minimum\_stepsize}\\
    \bi max\_timestep          \oxsval{maximum\_stepsize}\\
    \bi fixed\_spins \ocb\\
    \bi\bi \oxsval{atlas\_spec}\\
    \bi\bi \oxsval{region1 region2 \ldots}\\
    \bi\ccb\\
    \bi start\_dm              \oxsval{$\Delta \vm$}\\
    \bi error\_rate            \oxsval{rate}\\
    \bi absolute\_step\_error  \oxsval{abs\_error}\\
    \bi relative\_step\_error  \oxsval{rel\_error}\\
    \bi step\_headroom         \oxsval{headroom}\\
   \ccb
   \end{quote}
   \end{latexonly}%
   \begin{htmlonly}
   \begin{rawhtml}
   <BLOCKQUOTE><DL><DT>
   <TT>Specify Oxs_EulerEvolve:</TT><I>name</I> <TT>{</TT>
   <DD><TT> alpha </TT>
   \end{rawhtml}
   \abovemath{\alpha}
   \begin{rawhtml}
   <DD><TT> gamma_LL </TT>
   \end{rawhtml}
   \abovemath{\bar{\gamma}}
   \begin{rawhtml}
   <DD><TT> gamma_G </TT>
   \end{rawhtml}
   \abovemath{\gamma}
   \begin{rawhtml}
   <DD><TT> do_precess </TT> <I>precess</I>
   <DD><TT> min_timestep </TT> <I>minimum_stepsize</I>
   <DD><TT> max_timestep </TT> <I>maximum_stepsize</I>
   <DD><TT> fixed_spins {</TT><DL>
       <DD><I>atlas_spec</I>
       <DD><I>region1</I><TT>&nbsp;</TT><I>region2</I><TT> ...</TT>
       <DT><TT>}</TT></DL>
   <DD><TT> start_dm </TT>
   \end{rawhtml}
   $\Delta \vm$
   \begin{rawhtml}
   <DD><TT> error_rate </TT> <I>rate</I>
   <DD><TT> absolute_step_error </TT> <I>abs_error</I>
   <DD><TT> relative_step_error </TT> <I>rel_error</I>
   <DD><TT> step_headroom </TT> <I>headroom</I>
   <DT><TT>}</TT></DL></BLOCKQUOTE><P>
   \end{rawhtml}
   \end{htmlonly}
All the entries have default values, but the ones most commonly adjusted
are listed first.

The options \oxslabel{alpha}, \oxslabel{gamma\_LL} and
\oxslabel{gamma\_G} are as in the Landau-Lifshitz-Gilbert ODE
(\ref{eq:oxsllode}), (\ref{eq:oxsllgode}), where the units on
\abovemath{\bar{\gamma}} and \abovemath{\gamma} are m/A$\cdot$s and
\abovemath{\alpha} is dimensionless.  At most one of
\abovemath{\bar{\gamma}} and \abovemath{\gamma} should be specified.  If
neither is specified, then the default is
\latex{$\gamma=2.211\times 10^5$.}%
\html{\abovemath{\gamma}$=2.211\times 10^5$.}
(Because of the absolute value convention adopted on
\abovemath{\bar{\gamma}} and \abovemath{\gamma} in
(\ref{eq:oxsllode}), (\ref{eq:oxsllgode}), the sign given to the value of
\texttt{gamma\_LL} or \texttt{gamma\_G} in the Specify block is
irrelevant.)  The default value for \abovemath{\alpha} is 0.5, which is
large compared to experimental values, but allows simulations to
converge to equilibria in a reasonable time.  However, for accurate
dynamic studies it is important to assign an appropriate value to
\abovemath{\alpha}.

The \oxslabel{do\_precess} value should be either 1 or 0, and determines
whether or not the precession term in the Landau-Lifshitz ODE (i.e., the
first term on the righthand side in (\ref{eq:oxsllode})) is used.  If
\oxsval{precess} is 0, then precession is disabled and the simulation
evolves towards equilibrium along a steepest descent path.  The default
value is 1.

The \oxslabel{min\_timestep} and \oxslabel{max\_timestep} parameters provide
soft limits on the size of steps taken by the evolver.  The minimum
value may be overridden by the driver if a smaller step is needed to
meet time based stopping criteria.  The maximum value will be ignored if
a step of that size would produce a magnetization state numerically
indistinguishable from the preceding state.  The units for
\texttt{min\_timestep} and \texttt{max\_timestep} are seconds.  Default
values are 0 and $10^{-10}$ respectively.

The optional \oxslabel{fixed\_spins} entry allows the magnetization in
selected regions of the simulation to be frozen in its initial
configuration.  The value portion of the entry should be a list, with
the first element of the list being either an inline atlas definition
(grouped as a single item), or else the name of a previously defined
atlas.  The remainder of the list are names of regions in that atlas for
which the magnetization is to be be fixed, i.e., $\vM(t)=\vM(0)$ for all
time $t$ for all points in the named regions.  Fields and energies are
computed and reported normally across these regions.  Although any atlas
may be used, it is frequently convenient to set up an atlas with special
regions defined expressly for this purpose.

The stepsize for the first candidate iteration in the problem run is
selected so that the maximum change in the normalized (i.e., unit)
magnetization $\vm$ is the value specified by \oxslabel{start\_dm}.  The
units are degrees, with default value 0.01.

The four remaining entries, \oxslabel{error\_rate},
\oxslabel{absolute\_step\_error}, \oxslabel{relative\_step\_error}, and
\oxslabel{step\_headroom}, control fine points of stepsize selection,
and are intended for advance use only.  Given normalized magnetization
$\vm_i(t)$ at time $t$ and position $i$, and candidate magnetization
$\vm_i(t+\Delta t)$ at time $t+\Delta t$, the error at position $i$ is
estimated to be
\begin{displaymath}
\htmlimage{antialias}
\mbox{Error}_i =
  \left|\dot{\vm}_i(t+\Delta t) - \dot{\vm}_i(t)\right|\Delta t
      \,/\,2,
\end{displaymath}
where the derivative with respect to time, $\dot{\vm}$, is computed
using the Landau-Lifshitz ODE (\ref{eq:oxsllode}).  First order methods
essentially assume that $\dot{\vm}$ is constant on the interval
$[t,t+\Delta t]$; the above formula uses the difference in $\dot{\vm}$
at the endpoints of the interval to estimate (guess) how untrue that
assumption is.

A candidate step is accepted if the maximum error across all positions
$i$ is smaller than \texttt{absolute\_step\_error},
\texttt{error\_rate}$\,\times\,\Delta t$, and
\texttt{relative\_step\_error}$\,\times\,|\dot{\vm}_{\rm
max}|\Delta t$, where $|\dot{\vm}_{\rm max}|$ is the maximum value of
$|\dot{\vm}_i|$ across all $i$ at time $t$.  If the step is rejected,
then a smaller stepsize is computed that appears to pass the above
tests, and a new candidate step is proposed using that smaller stepsize
times \texttt{step\_headroom}.  Alternatively, if the step is accepted,
then the error information is used to determine the stepsize for the
next step, modified in the same manner by \texttt{step\_headroom}.

The error calculated above is in terms of unit magnetizations, so the
natural units are radians or radians/second.  Inside the Specify block,
however, the \texttt{error\_rate} and \texttt{absolute\_step\_error} are
specified in degrees/nanosecond and degrees, respectively; they are
converted appropriately inside the code before use.  The
\texttt{relative\_step\_error} is a dimensionless quantity, representing a
proportion between 0 and 1.  The error check controlled by each of these
three quantities may be disabled by setting the quantity value to -1.
They are all optional, with default values of -1 for \texttt{error\_rate},
0.2 for \texttt{absolute\_step\_error}, and 0.2 for
\texttt{relative\_step\_error}.

The \cd{headroom} quantity should lie in the range $(0,1)$, and controls
how conservative the code will be in stepsize selection.  If \cd{headroom}
is too large, then much computation time will be lost computing
candidate steps that fail the error control tests.  If \cd{headroom} is
small, then most candidate steps will pass the error control tests, but
computation time may be wasted calculating more steps than are
necessary.  The default value for \cd{headroom} is 0.85.

In addition to the above error control tests, a candidate step will also
be rejected if the total energy, after adjusting for effects due to any
time varying external field, is found to increase.  In this case the
next candidate stepsize is set to one half the rejected stepsize.

The \cd{Oxs\_EulerEvolve} module provides five scalar, one scalar
field, and three vector field outputs.  The scalar outputs are
\begin{itemize}
\item \textbf{Max dm/dt:} maximum $|d\vm/dt|$, in degrees per
   nanosecond; $\vm$ is the unit magnetization direction.
\item \textbf{Total energy:} in joules.
\item \textbf{Delta E:} change in energy between last step and current
   step, in joules.
\item \textbf{dE/dt:} derivative of energy with respect to time, in
   joules per second.
\item \textbf{Energy calc count:} number of times total energy has been
   calculated.
\end{itemize}

The scalar field output is
\begin{itemize}
\item \textbf{Total energy density:} cellwise total energy density, in
\latexhtml{J/m${}^3$}{J/m\begin{rawhtml}<SUP>3</SUP>\end{rawhtml}}.
\end{itemize}

The vector field outputs are
\begin{itemize}
\item \textbf{Total field:} total effective field $\vH$ in A/m.
\item \textbf{mxH:} torque in A/m; $\vm$ is the unit magnetization
   direction, $\vH$ is the total effective field.
\item \textbf{dm/dt:} derivative of spin $\vm$ with respect to time, in
   radians per second.
\end{itemize}

\begin{ExampleMifs}[Example]
  \fn{octant.mif}.
\end{ExampleMifs}

\pttarget{PTRK}\index{Oxs\_Ext~child~classes!Oxs\_RungeKuttaEvolve}%
\item[Oxs\_RungeKuttaEvolve:\label{html:RungeKuttaEvolve}]
Time evolver implementing several Runge-Kutta methods for integrating
the Landau-Lifshitz-Gilbert\index{ODE!Landau-Lifshitz} ODE
(\ref{eq:oxsllode}), (\ref{eq:oxsllgode}), with step size control.  In
most cases it will greatly outperform the \cd{Oxs\_EulerEvolve} class.
The Specify block has the form
   \begin{latexonly}
   \begin{quote}\tt
   Specify Oxs\_RungeKuttaEvolve:\oxsval{name} \ocb\\
    \bi alpha                  \oxsval{$\alpha$}\\
    \bi gamma\_LL              \oxsval{$\bar{\gamma}$}\\
    \bi gamma\_G               \oxsval{$\gamma$}\\
    \bi do\_precess            \oxsval{precess}\\
    \bi allow\_signed\_gamma \oxsval{signed\_gamma}\\
    \bi min\_timestep          \oxsval{minimum\_stepsize}\\
    \bi max\_timestep          \oxsval{maximum\_stepsize}\\
    \bi fixed\_spins \ocb\\
    \bi\bi \oxsval{atlas\_spec}\\
    \bi\bi  \oxsval{region1 region2 \ldots}\\
    \bi\ccb\\
    \bi start\_dm              \oxsval{$\Delta \vm$}\\
    \bi start\_dt              \oxsval{start\_timestep}\\
    \bi stage\_start           \oxsval{scontinuity}\\
    \bi error\_rate            \oxsval{rate}\\
    \bi absolute\_step\_error  \oxsval{abs\_error}\\
    \bi relative\_step\_error  \oxsval{rel\_error}\\
    \bi energy\_precision      \oxsval{eprecision}\\
    \bi min\_step\_headroom    \oxsval{min\_headroom}\\
    \bi max\_step\_headroom    \oxsval{max\_headroom}\\
    \bi reject\_goal           \oxsval{reject\_proportion}\\
    \bi method                 \oxsval{subtype}\\
   \ccb
   \end{quote}
   \end{latexonly}%
   \begin{htmlonly}
   \begin{rawhtml}
   <BLOCKQUOTE><DL><DT>
   <TT>Specify Oxs_RungeKuttaEvolve:</TT><I>name</I> <TT>{</TT>
   <DD><TT> alpha </TT>
   \end{rawhtml}
   \abovemath{\alpha}
   \begin{rawhtml}
   <DD><TT> gamma_LL </TT>
   \end{rawhtml}
   \abovemath{\bar{\gamma}}
   \begin{rawhtml}
   <DD><TT> gamma_G </TT>
   \end{rawhtml}
   \abovemath{\gamma}
   \begin{rawhtml}
   <DD><TT> do_precess </TT> <I>precess</I>
   <DD><TT> allow_signed_gamma </TT> <I>signed_gamma</I>
   <DD><TT> min_timestep </TT> <I>minimum_stepsize</I>
   <DD><TT> max_timestep </TT> <I>maximum_stepsize</I>
   <DD><TT> fixed_spins {</TT><DL>
       <DD><I>atlas_spec</I>
       <DD><I>region1</I><TT>&nbsp;</TT><I>region2</I><TT> ...</TT>
       <DT><TT>}</TT></DL>
   <DD><TT> start_dm </TT>
   \end{rawhtml}
   $\Delta \vm$
   \begin{rawhtml}
   <DD><TT> start_dt </TT> <I>start_timestep</I>
   <DD><TT> stage_start </TT> <I>scontinuity</I>
   <DD><TT> error_rate </TT> <I>rate</I>
   <DD><TT> absolute_step_error </TT> <I>abs_error</I>
   <DD><TT> relative_step_error </TT> <I>rel_error</I>
   <DD><TT> energy_precision </TT> <I>eprecision</I>
   <DD><TT> min_step_headroom </TT> <I>min_headroom</I>
   <DD><TT> max_step_headroom </TT> <I>max_headroom</I>
   <DD><TT> reject_goal </TT> <I>reject_proportion</I>
   <DD><TT> method </TT> <I>subtype</I>
   <DT><TT>}</TT></DL></BLOCKQUOTE><P>
   \end{rawhtml}
   \end{htmlonly}
Most of these options appear also in the
\htmlonlyref{\cd{Oxs\_EulerEvolve}}{html:EulerEvolve} class.
The repeats have the same meaning as in that class, and the same
default values except for \texttt{relative\_step\_error} and
\texttt{error\_rate}, which for \cd{Oxs\_RungeKuttaEvolve} have the
default values of 0.01 and 1.0, respectively.  Additionally, the
\oxslabel{alpha}, \oxslabel{gamma\_LL} and \oxslabel{gamma\_G} options
may be initialized using scalar field objects, to allow these material
parameters to vary spatially.

The \oxslabel{allow\_signed\_gamma} parameter is for simulation testing
purposes, and is intended for advanced use only.  There is some lack of
consistency in the literature with respect to the sign of
\abovemath{\gamma}.  For this reason the Landau-Lifshitz-Gilbert
equations are presented above (\ref{eq:oxsllode}, \ref{eq:oxsllgode})
using the absolute value of \abovemath{\gamma}.  This is the
interpretation used if \cd{allow\_signed\_gamma} is 0 (the default).  If
instead \cd{allow\_signed\_gamma} is set to 1, then the
Landau-Lifshitz-Gilbert equations are interpreted without the absolute
values and with a sign change on the \abovemath{\gamma} terms, i.e., the
default value for \abovemath{\gamma} in this case is $-2.211 \times
10^5$ (units are m/A$\cdot$s).  In this setting, if \abovemath{\gamma}
is set positive then the spins will precess backwards about the
effective field, and the damping term will force the spins \textbf{away}
from the effective field and increase the total energy.  If you are
experimenting with \abovemath{\gamma>0}, you should either set
\abovemath{\alpha<=0} to force spins back towards the effective field,
or disable the energy precision control (discussed
\htmlonlyref{below}{html:oxsrkeprecision}).

The two controls \oxslabel{min\_step\_headroom} (default value 0.33) and
\oxslabel{max\_step\_headroom} (default value 0.95) replace the single
\cd{step\_headroom} option in \cd{Oxs\_EulerEvolve}.  The effective
\cd{step\_headroom} is automatically adjusted by the evolver between the
\oxsval{min\_headroom} and \oxsval{max\_headroom} limits to make the
observed reject proportion approach the \oxslabel{reject\_goal} (default
value 0.05).

The \oxslabel{method} entry selects a particular Runge-Kutta
implementation.  It should be set to one of \oxsval{rk2},
\oxsval{rk4}, \oxsval{rkf54}, \oxsval{rkf54m}, or \oxsval{rkf54s};
the default value is \oxsval{rkf54}.  The \oxsval{rk2} and
\oxsval{rk4} methods implement canonical second and fourth global order
Runge-Kutta methods\cite{stoer1993}, respectively.  For \oxsval{rk2},
stepsize control is managed by comparing $\dot{\vm}$ at the middle and
final points of the interval, similar to what is done for stepsize
control for the \cd{Oxs\_EulerEvolve} class.  One step of the
\oxsval{rk2} method involves 2 evaluations of $\dot{\vm}$.

In the \oxsval{rk4} method, two successive steps are taken at half the
nominal step size, and the difference between that end point and that
obtained with one full size step are compared.  The error is estimated at
1/15th the maximum difference between these two states.  One step of the
\oxsval{rk4} method involves 11 evaluations of $\dot{\vm}$, but the
end result is that of the 2 half-sized steps.

The remaining methods, \oxsval{rkf54},  \oxsval{rkf54m},
and \oxsval{rkf54s}, are closely related Runge-Kutta-Fehlberg methods
derived by Dormand and Prince\cite{dormand1980,dormand1986}.  In the
nomenclature of these papers,
\oxsval{rkf54} implements RK5(4)7FC,
\oxsval{rkf54m} implements RK5(4)7FM, and
\oxsval{rkf54s} implements RK5(4)7FS.
All are 5th global order with an embedded 4th order method for stepsize
control.  Each step of these methods requires 6 evaluations of
$\dot{\vm}$ if the step is accepted, 7 if rejected.  The difference
between the methods involves tradeoffs between stability and error
minimization.  The RK5(4)7FS method has the best stability, RK5(4)7FM
the smallest error, and RK5(4)7FC represents a compromise between the
two.  The default method used by \cd{Oxs\_RungeKuttaEvolve} is
RK5(4)7FC.

\label{html:oxsrkeprecision}
The remaining undiscussed entry in the \cd{Oxs\_RungeKuttaEvolve}
Specify block is \oxslabel{energy\_precision}.  This should be set to an
estimate of the expected relative accuracy of the energy calculation.
After accounting for any change in the total energy arising from
time-varying applied fields, the energy remainder should decrease from
one step of the LLG ODE to the next.  \cd{Oxs\_RungeKuttaEvolve} will
reject a step if the energy remainder is found to increase by more than
that allowed by \oxsval{eprecision}.  The default value for
\oxsval{eprecision} is \latex{$10^{-10}$.}\html{1e-10.}  This control
may be disabled by setting \oxsval{eprecision} to -1.

The \cd{Oxs\_RungeKuttaEvolve} module provides the same scalar, scalar
field, and vector field outputs as \cd{Oxs\_EulerEvolve}.  It also
provides the following state data accessible through the \MIF\
\htmlonlyref{\cd{GetStateData}}{html:GetStateData} command:
\begin{rawhtml}
<BLOCKQUOTE>
\end{rawhtml}
%begin{latexonly}
\begin{quote}
%end{latexonly}
\begin{tabular}{l@{\hskip 3em}l@{\hskip 3em}l}
    \cd{Mx} & \cd{My} & \cd{Mz} \\
    \cd{dMx/dt} & \cd{dMy/dt} & \cd{dMz/dt} \\
    \cd{Total E} & \cd{dE/dt} & \cd{pE/pt} \\
    \multicolumn{3}{l}{\cd{\sdquote Timestep lower bound\sdquote}}
\end{tabular}
%begin{latexonly}
\end{quote}
%end{latexonly}
\begin{rawhtml}
</BLOCKQUOTE>
\end{rawhtml}
The full name for each of these items is
\begin{rawhtml}
<BLOCKQUOTE>
\end{rawhtml}
%begin{latexonly}
\begin{quote}
%end{latexonly}
\cd{Oxs\_RungeKuttaEvolve:{\oab}instance\_name{\cab}:{\oab}item\_name{\cab}},
%begin{latexonly}
\end{quote}
%end{latexonly}
\begin{rawhtml}
</BLOCKQUOTE>
\end{rawhtml}
where \cd{{\oab}instance\_name{\cab}} is the instance name of the object
(typically an empty string or something like ``evolver'').  For stage
start states only the \cd{Mx}, \cd{My}, and \cd{Mz} items are available.
These terms, and the corresponding \cd{dMx/dt}, \cd{dMy/dt}, and
\cd{dMz/dt}, are component values averaged across the full simulation.

\begin{ExampleMifs}
  \fn{sample.mif}, \fn{acsample.mif}, \fn{spinmag.mif},
  \fn{spinmag2.mif}, \fn{varalpha.mif}, \fn{yoyo.mif}.
\end{ExampleMifs}

\pttarget{PTSX}\index{Oxs\_Ext~child~classes!Oxs\_SpinXferEvolve}%
\item[Oxs\_SpinXferEvolve:\label{html:SpinXferEvolve}]
Time evolver that integrates an
Landau-Lifshitz-Gilbert\index{ODE!Landau-Lifshitz} ODE augmented with a
spin momentum term \cite{xiao2004},
\begin{equation}
\htmlimage{antialias}
  \frac{d\vm}{dt} = -|\gamma|\,\vm\times\vH_{\rm eff}
   + \alpha
     \left(\vm\times\frac{d\vm}{dt}\right)
   + |\gamma|\beta\epsilon
     \left(\vm\times\vm_p\times\vm\right)
   - |\gamma|\beta\epsilon^\prime\,\vm\times\vm_p
\label{eq:oxsllgspinxfer}
\end{equation}
(compare to (\ref{eq:oxsllgode})), where
\begin{eqnarray*}
\vm & = & \mbox{reduced magnetization, $\vM/M_s$} \\
\gamma & = & \mbox{Gilbert gyromagnetic ratio} \\
\beta & = & \left|\frac{\hbar}{\mu_0 e}\right|\frac{J}{t M_s} \\
\vm_p & = & \mbox{(unit) electron polarization direction} \\
\epsilon & = &
\frac{P\Lambda^2}{(\Lambda^2+1)+(\Lambda^2-1)(\vm\cdot\vm_p)} \\
\epsilon^\prime & = & \mbox{secondary spin tranfer term}.
\end{eqnarray*}
In the definition of $\beta$, $e$ is the electron charge in C, $J$ is
current density in A/m${}^2$, $t$ is the free layer thickness in meters,
and $M_s$ is the saturation magnetization in A/m.

The various parameters are defined in the Specify block, which is an
extension of that for the
\htmlonlyref{\cd{Oxs\_RungeKuttaEvolve}}{html:RungeKuttaEvolve} class:
   \begin{latexonly}
   \begin{quote}\tt
   Specify Oxs\_SpinXferEvolve:\oxsval{name} \ocb\\
    \bi alpha                  \oxsval{$\alpha$}\\
    \bi gamma\_LL              \oxsval{$\bar{\gamma}$}\\
    \bi gamma\_G               \oxsval{$\gamma$}\\
    \bi do\_precess            \oxsval{precess}\\
    \bi allow\_signed\_gamma \oxsval{signed\_gamma}\\
    \bi min\_timestep          \oxsval{minimum\_stepsize}\\
    \bi max\_timestep          \oxsval{maximum\_stepsize}\\
    \bi fixed\_spins \ocb\\
    \bi\bi \oxsval{atlas\_spec}\\
    \bi\bi  \oxsval{region1 region2 \ldots}\\
    \bi\ccb\\
    \bi start\_dm              \oxsval{$\Delta \vm$}\\
    \bi stage\_start           \oxsval{scontinuity}\\
    \bi error\_rate            \oxsval{rate}\\
    \bi absolute\_step\_error  \oxsval{abs\_error}\\
    \bi relative\_step\_error  \oxsval{rel\_error}\\
    \bi energy\_precision      \oxsval{eprecision}\\
    \bi min\_step\_headroom    \oxsval{min\_headroom}\\
    \bi max\_step\_headroom    \oxsval{max\_headroom}\\
    \bi reject\_goal           \oxsval{reject\_proportion}\\
    \bi method                 \oxsval{subtype}\\
    \bi P                      \oxsval{polarization}\\
    \bi P\_fixed               \oxsval{p\_fixed\_layer}\\
    \bi P\_free                \oxsval{p\_free\_layer}\\
    \bi Lambda                 \oxsval{$\Lambda$}\\
    \bi Lambda\_fixed          \oxsval{$\Lambda$\_fixed\_layer}\\
    \bi Lambda\_free           \oxsval{$\Lambda$\_free\_layer}\\
    \bi eps\_prime             \oxsval{ep}\\
    \bi J                      \oxsval{current\_density}\\
    \bi J\_profile             \oxsval{Jprofile\_script}\\
    \bi J\_profile\_args       \oxsval{Jprofile\_script\_args}\\
    \bi mp                     \oxsval{p\_direction}\\
    \bi energy\_slack          \oxsval{eslack}\\
   \ccb
   \end{quote}
   \end{latexonly}%
   \begin{htmlonly}
   \begin{rawhtml}
   <BLOCKQUOTE><DL><DT>
   <TT>Specify Oxs_SpinXferEvolve:</TT><I>name</I> <TT>{</TT>
   <DD><TT> alpha </TT>
   \end{rawhtml}
   \abovemath{\alpha}
   \begin{rawhtml}
   <DD><TT> gamma_LL </TT>
   \end{rawhtml}
   \abovemath{\bar{\gamma}}
   \begin{rawhtml}
   <DD><TT> gamma_G </TT>
   \end{rawhtml}
   \abovemath{\gamma}
   \begin{rawhtml}
   <DD><TT> do_precess </TT> <I>precess</I>
   <DD><TT> allow_signed_gamma </TT> <I>signed_gamma</I>
   <DD><TT> min_timestep </TT> <I>minimum_stepsize</I>
   <DD><TT> max_timestep </TT> <I>maximum_stepsize</I>
   <DD><TT> fixed_spins {</TT><DL>
       <DD><I>atlas_spec</I>
       <DD><I>region1</I><TT>&nbsp;</TT><I>region2</I><TT> ...</TT>
       <DT><TT>}</TT></DL>
   <DD><TT> start_dm </TT>
   \end{rawhtml}
   $\Delta \vm$
   \begin{rawhtml}
   <DD><TT> stage_start </TT> <I>scontinuity</I>
   <DD><TT> error_rate </TT> <I>rate</I>
   <DD><TT> absolute_step_error </TT> <I>abs_error</I>
   <DD><TT> relative_step_error </TT> <I>rel_error</I>
   <DD><TT> energy_precision </TT> <I>eprecision</I>
   <DD><TT> min_step_headroom </TT> <I>min_headroom</I>
   <DD><TT> max_step_headroom </TT> <I>max_headroom</I>
   <DD><TT> reject_goal </TT> <I>reject_proportion</I>
   <DD><TT> method </TT> <I>subtype</I>
   <DD><TT> P </TT> <I>polarization</I>
   <DD><TT> P_fixed </TT> <I>p_fixed_layer</I>
   <DD><TT> P_free </TT> <I>p_free_layer</I>
   <DD><TT> Lambda </TT>
   \end{rawhtml}
   $\Lambda$
   \begin{rawhtml}
   <DD><TT> Lambda_fixed </TT>
   \end{rawhtml}
   $\Lambda$\_fixed\_layer
   \begin{rawhtml}
   <DD><TT> Lambda_free </TT>
   \end{rawhtml}
   $\Lambda$\_free\_layer
   \begin{rawhtml}
   <DD><TT> eps_prime </TT> <I>ep</I>
   <DD><TT> J </TT> <I>current_density</I>
   <DD><TT> J_profile </TT> <I>Jprofile_script</I>
   <DD><TT> J_profile_args </TT> <I>Jprofile_script_args</I>
   <DD><TT> mp </TT> <I>p_direction</I>
   <DD><TT> energy_slack </TT> <I>eslack</I>
   <DT><TT>}</TT></DL></BLOCKQUOTE><P>
   \end{rawhtml}
   \end{htmlonly}
The options duplicated in the
\htmlonlyref{\cd{Oxs\_RungeKuttaEvolve}}{html:RungeKuttaEvolve} class
Specify block have the same meaning and default values here, with the
exception of \texttt{error\_rate}, which for
\cd{Oxs\_SpinXferEvolve} has the default value of -1 (i.e., disabled).

The default values for \oxslabel{P} and \oxslabel{Lambda} are 0.4 and 2,
respectively.  If preferred, values for the fixed and free layers may be
instead specified separately, through \oxslabel{P\_fixed},
\oxslabel{P\_free}, \oxslabel{Lambda\_fixed}, and
\oxslabel{Lambda\_free}.  Otherwise P\_fixed = P\_free = P and
Lambda\_fixed = Lambda\_free = Lambda.  Lambda must be larger than or
equal to 1; set Lambda=1 to remove the dependence of
\abovemath{\epsilon} on $\vm\cdot\vm_p$.  If you want non-zero
\abovemath{\epsilon^\prime}, it is set directly as
\oxslabel{eps\_prime}.

Current density \oxslabel{J} and unit polarization direction
\oxslabel{mp} are required.  The units on J are A/m${}^2$.  Positive J
produces torque that tends to align \abovemath{\vm} towards
\abovemath{\vm_p}.

Parameters J, mp, P, Lambda, and eps\_prime may all be varied pointwise,
but are fixed with respect to time.  However, J can be multiplied by a
time varying ``profile,'' to model current rise times, pulses, etc.  Use
the \oxslabel{J\_profile} and \oxslabel{J\_profile\_args} options to
enable this feature.  The \oxsval{Jprofile\_script} should be a \Tcl\
script that returns a single scalar.  \oxsval{Jprofile\_script\_args}
should be a subset of \cd{\ocb stage stage\_time total\_time\ccb}, to
specify arguments appended to \oxsval{Jprofile\_script} on each time
step.  Default is the entire set, in the order as listed.

The \cd{Oxs\_SpinXferEvolve} module provides the same five scalar
outputs and three vector outputs as \cd{Oxs\_RungeKutta}, plus the
scalar output ``average J,'' and the vector field outputs ``Spin
torque'' (which is
$|\gamma|\beta\epsilon\left(\vm\times\vm_p\times\vm\right)$) and
``J*mp.''  (Development note: In the case \texttt{propagate\_mp} is
enabled, \texttt{mp} is actually $\Delta_x\partial\vm/\partial\vx$,
where $\vx$ is the flow direction and $\Delta_x$ is the cell dimension
in that direction.)

The \cd{Oxs\_SpinXferEvolve} class does \textbf{not} include any oersted
field arising from the current.  Of course, arbitrary fields simulating
the oersted field may be added separately as Zeeman energy terms.  An
example of this is contained in the \fn{spinxfer.mif} sample file.

There are no temperature effects in this evolver, i.e., it is a T = 0 K
code.

Note also that $\vm_p$ is fixed.

For basic usage, the Specify block can be as simple as
% The extra BLOCKQUOTE's here are a workaround for an apparent
% latex2html bug
\begin{rawhtml}
<BLOCKQUOTE>
\end{rawhtml}
%begin<latexonly>
\begin{quote}
%end<latexonly>
\begin{verbatim}
Specify Oxs_SpinXferEvolve:evolve {
  alpha 0.014
  J 7.5e12
  mp {1 0 0}
  P 0.4
  Lambda 2
}
\end{verbatim}
%begin<latexonly>
\end{quote}
%end<latexonly>
\begin{rawhtml}
</BLOCKQUOTE>
\end{rawhtml}

This class is still in early development; at this time the example files
are located in \fn{oommf/app/oxs/local} instead of
\fn{oommf/app/oxs/examples}.

\begin{ExampleMifs}
  \fn{spinxfer.mif}, \fn{spinxfer-miltat.mif}, \fn{spinxfer-onespin.mif}.
\end{ExampleMifs}

\pttarget{PTCG}\index{Oxs\_Ext~child~classes!Oxs\_CGEvolve}%
\item[Oxs\_CGEvolve:\label{html:CGEvolve}]
The minimization evolver is \cd{Oxs\_CGEvolve}, which is an
in-development conjugate gradient minimizer with no preconditioning. The
Specify block has the form
   \begin{latexonly}
   \begin{quote}\tt
   Specify Oxs\_CGEvolve:\oxsval{name} \ocb\\
    \bi gradient\_reset\_angle   \oxsval{reset\_angle}\\
    \bi gradient\_reset\_count   \oxsval{count}\\
    \bi minimum\_bracket\_step   \oxsval{minbrack}\\
    \bi maximum\_bracket\_step   \oxsval{maxbrack}\\
    \bi line\_minimum\_angle\_precision \oxsval{min\_prec\_angle}\\
    \bi line\_minimum\_relwidth  \oxsval{relwidth}\\
    \bi energy\_precision \oxsval{eprecision}\\
    \bi method \oxsval{cgmethod}\\
    \bi fixed\_spins \ocb\\
    \bi\bi \oxsval{atlas\_spec}\\
    \bi\bi  \oxsval{region1 region2 \ldots}\\
    \bi\ccb\\
   \ccb
   \end{quote}
   \end{latexonly}%
   \begin{rawhtml}
   <BLOCKQUOTE><DL><DT>
   <TT>Specify Oxs_CGEvolve:</TT><I>name</I> <TT>{</TT>
   <DD><TT> gradient_reset_angle </TT> <I>reset_angle</I>
   <DD><TT> gradient_reset_count </TT> <I>count</I>
   <DD><TT> minimum_bracket_step </TT> <I>minbrack</I>
   <DD><TT> maximum_bracket_step </TT> <I>maxbrack</I>
   <DD><TT> line_minimum_angle_precision </TT> <I>min_prec_angle</I>
   <DD><TT> line_minimum_relwidth </TT> <I>relwidth</I>
   <DD><TT> energy_precision </TT> <I>eprecision</I>
   <DD><TT> method </TT> <I>cgmethod</I>
   <DD><TT> fixed_spins {</TT><DL>
       <DD><I>atlas_spec</I>
       <DD><I>region1</I><TT>&nbsp;</TT><I>region2</I><TT> ...</TT>
       <DT><TT>}</TT></DL>
   <DT><TT>}</TT></DL></BLOCKQUOTE><P>
   \end{rawhtml}
All entries have default values.

The evolution to an energy minimum precedes by a sequence of line
minimizations.  Each line represents a one dimensional affine subspace
in the $3N$ dimensional space of possible magnetization configurations,
where $N$ is the number of spins in the simulation.  Once a minimum has
been found along a line, a new direction is chosen that is ideally
orthogonal to all preceding directions, but related to the gradient of
the energy taken with respect to the magnetization.  In practice the
line direction sequence cannot be extended indefinitely; the parameters
\oxslabel{gradient\_reset\_angle} and \oxslabel{gradient\_reset\_count}
control the gradient resetting process.  The first checks the angle
between the new direction and the gradient.  If that angle is larger
than \oxsval{reset\_angle} (expressed in degrees), then the selected
direction is thrown away, and the conjugate-gradient process is
re-initialized with the gradient direction as the new first direction.
In a similar vein, \oxsval{count} specifies the maximum number of line
directions selected before resetting the process.  Because the first
line in the sequence is selected along the gradient direction, setting
\oxsval{count} to 1 effectively turns the algorithm into a steepest
descent minimization method.  The default values for
\oxsval{reset\_angle} and \oxsval{count} are 80 degrees and 50,
respectively.

Once a minimization direction has been selected, the first stage of the
line minimization is to bracket the minimum energy on that line, i.e.,
given a start point on the line\emdash the location of the minimum from the
previous line minimization\emdash find another point on the line such that
the energy minimum lies between those two points.  As one moves along
the line, the spins in the simulation rotate, with one spin rotating
faster than (or at least as fast as) all the others.  If the start point
was not the result of a successful line minimization from the previous
stage, then the first bracket attempt step is sized so that the fastest
moving spin rotates through the angle specified by
\oxslabel{minimum\_bracket\_step}.  In the more usual case that the
start point is a minimum from the previous line minimization stage, the
initial bracket attempt step size is set to the distance between the
current start point and the start point of the previous line
minimization stage.

The energy and gradient of the energy are examined at the candidate
bracket point to test if an energy minimum lies in the interval.  If
not, the interval is extended, based on the size of the first bracket
attempt interval and the derivatives of the energy at the interval
endpoints.  This process is continued until either a minimum is
bracketed or the fastest moving spin rotates through the angle specified
by \oxslabel{maximum\_bracket\_step}.

If the bracketing process is successful, then a one dimensional
minimization is carried out in the interval, using both energy and
energy derivative information.  Each step in this process reduces the
width of the bracketing interval.  This process is continued until
the angle between the line direction and the computed energy
gradient is within \oxslabel{line\_minimum\_angle\_precision} degrees of
orthogonal, and the width of the interval relative to the distance of the
interval from the start point (i.e., the stop point from the previous
line minimization process) is less than
\oxslabel{line\_minimum\_relwidth}.  The stop point,
i.e., the effective minimum, is taken to be the endpoint of the final
interval having smaller energy.  The default value for
\oxsval{min\_prec\_angle} is 1 degree, and the default value for
\oxsval{relwidth} is 1.  This latter setting effectively disables the
\texttt{line\_minimum\_relwidth} control, which should generally be used
only as a secondary control.

If the bracketing process is unsuccessful, i.e., the check for bracketed
energy minimum failed at the maximum bracket interval size allowed by
\texttt{maximum\_bracket\_step}, then the maximum bracket endpoint is
accepted as the next point in the minimization iteration.

Once the line minimum stop point has been selected, the next iteration
begins with selection of a new line direction, as described above,
except in the case where the stop point was not obtained as an actual
minimum, but rather by virtue of satisfying the
\texttt{maximum\_bracket\_step} constraint.  In that case the orthogonal
line sequence is reset, in the same manner as when the
\texttt{gradient\_reset\_angle} or \texttt{gradient\_reset\_count}
controls are triggered, and the next line direction is taken directly
from the energy gradient.

There are several factors to bear in mind when selecting values for
the parameters \texttt{minimum\_bracket\_step},
\texttt{maximum\_bracket\_step}, and \texttt{line\_minimum\_relwidth}.
If \texttt{minimum\_bracket\_step} is too small, then it may take a
great many steps to obtain an interval large enough to bracket the
minimum.  If \texttt{minimum\_bracket\_step} is too large, then the
bracket interval will be unnecessarily generous, and many steps may be
required to locate the minimum inside the bracketing interval.  However,
this value only comes into play when resetting the line minimization
direction sequence, so the setting is seldom critical.  It is specified
in degrees, with default value 0.05.

If \texttt{maximum\_bracket\_step} is too small, then the minima will be
mostly not bracketed, and the minimization will degenerate into a type
of steepest descent method.  On the other hand, if
\texttt{maximum\_bracket\_step} is too large, then the line
minimizations may draw the magnetization far away from a local energy
minimum (i.e., one on the full $3N$ dimensional magnetization space),
eventually ending up in a different, more distant minimum.  The value
for \texttt{maximum\_bracket\_step} is specified in degrees, with
default value 10.

The \texttt{line\_minimum\_angle\_precision} and
\texttt{line\_minimum\_relwidth} values determine the precision of the
individual line minimizations, not the total minimization procedure,
which is governed by the stopping criteria specified in the driver's
Specify block.  However, these values are important because the
precision of the line minimizations affects the the line direction
sequence orthogonality.  If both are too coarse, then the selected line
directions will quickly drift away from mutual orthogonality.
Conversely, setting either too fine will
produce additional line minimization steps that do nothing to improve
convergence towards the energy minimum in the full $3N$ dimensional
magnetization space.

The \oxslabel{energy\_precision} parameter estimates the relative
precision of the energy computations.  This is used to introduce a slack
factor into the energy comparisons during the bracketing and line
minimization stages, that is, if the computed energy values at two
points have relative error difference smaller than
\oxsval{eprecision}, they are treated as having the same energy.  The
default value for \oxsval{eprecision} is 1e-10.  The true precision
will depend primarily on the number of spins in the simulation.  It may
be necessary for very large simulations to increase the
\oxsval{eprecision} value.

The \oxslabel{method} parameter can be set to either
\texttt{Fletcher-Reeves} or \texttt{Polak-Ribiere} to specify the
conjugate gradient direction selection algorithm.  The default is
Fletcher-Reeves, which has somewhat smaller memory requirements.

The last parameter, \oxslabel{fixed\_spins}, performs the same function
as for the \htmlonlyref{\cd{Oxs\_EulerEvolve}}{html:EulerEvolve} class.

The \cd{Oxs\_CGEvolve} module provides nine scalar, one scalar
 field, and two vector field outputs.  The scalar outputs are
\begin{itemize}
\item \textbf{Max mxHxm:} maximum $|\vm\times\vH\times\vm|$, in A/m;
   $\vm$ is the unit magnetization direction.
\item \textbf{Total energy:} in joules.
\item \textbf{Delta E:} change in energy between last step and current
   step, in joules.
\item \textbf{Energy calc count:} number of times total energy has been
   calculated.
\item \textbf{Bracket count:} total number of attempts required to
   bracket energy minimum during first phase of line minimization
   procedures.
\item \textbf{Line min count:} total number of minimization steps during
   second phase of line minimization procedures (i.e., steps after
   minimum has been bracketed).
\item \textbf{Cycle count:} number of line direction selections.
\item \textbf{Cycle sub count:} number of line direction selections
  since the last gradient direction reset.
\item \textbf{Conjugate cycle count:} number of times the conjugate gradient
   process has been reset to the gradient direction.
\end{itemize}

The scalar field output is
\begin{itemize}
\item \textbf{Total energy density:} cellwise total energy density, in
\latexhtml{J/m${}^3$}{J/m\begin{rawhtml}<SUP>3</SUP>\end{rawhtml}}.
\end{itemize}

The vector field outputs are
\begin{itemize}
\item \textbf{H:} total effective field in A/m.
\item \textbf{mxHxm:} in A/m; $\vm$ is the unit magnetization
   direction.
\end{itemize}

\begin{ExampleMifs}
  \fn{cgtest.mif}, \fn{stdprob3.mif}, \fn{yoyo.mif}.
\end{ExampleMifs}

\end{description}

\subsection{Drivers\label{sec:oxsDrivers}}
While \hyperrefhtml{evolvers}{evolvers (Sec.~}{)}{sec:oxsEvolvers} are
responsible for moving the simulation forward in individual steps,
\textit{drivers} coordinate the action of the evolver on the
simulation as a whole, by grouping steps into tasks, stages and runs.

Tasks are small groups of steps that can be completed without adversely
affecting user interface responsiveness.  Stages are larger units
specified by the \MIF\ problem description; in particular, problem
parameters are not expected to change in a discontinuous manner inside a
stage.  The run is the complete sequence of stages, from problem start to
finish.  The driver detects when stages and runs are finished, using
criteria specified in the \MIF\ problem description, and can enforce
constraints, such as making sure stage boundaries respect time stopping
criteria.

There are two drivers in Oxs,
\htmlonlyref{\cd{Oxs\_TimeDriver}}{item:TimeDriver}
for controlling time evolvers such as
\htmlonlyref{\cd{Oxs\_RungeKuttaEvolve}}{html:RungeKuttaEvolve},
and
\htmlonlyref{\cd{Oxs\_MinDriver}}{html:MinDriver}
for controlling minimization evolvers like
\htmlonlyref{\cd{Oxs\_CGEvolve}}{html:CGEvolve}.

\begin{description}
\pttarget{PTTD}\index{Oxs\_Ext~child~classes!Oxs\_TimeDriver}%
\item[Oxs\_TimeDriver:\label{item:TimeDriver}]
The Oxs time driver is \textbf{Oxs\_TimeDriver}.  The specify block has
the form
\begin{latexonly}
\begin{quote}\tt
Specify Oxs\_TimeDriver:\oxsval{name} \ocb\\
 \bi evolver \oxsval{evolver\_spec}\\
 \bi mesh \oxsval{mesh\_spec}\\
 \bi Ms \oxsval{scalar\_field\_spec}\\
 \bi m0 \oxsval{vector\_field\_spec}\\
 \bi stopping\_dm\_dt \oxsval{torque\_criteria}\\
 \bi stopping\_time \oxsval{time\_criteria}\\
 \bi stage\_iteration\_limit \oxsval{stage\_iteration\_count}\\
 \bi total\_iteration\_limit \oxsval{total\_iteration\_count}\\
 \bi stage\_count \oxsval{number\_of\_stages}\\
 \bi stage\_count\_check \oxsval{test}\\
 \bi checkpoint\_file \oxsval{restart\_file\_name}\\
 \bi checkpoint\_interval \oxsval{checkpoint\_minutes}\\
 \bi checkpoint\_disposal \oxsval{cleanup\_behavior}\\
 \bi start\_iteration \oxsval{iteration}\\
 \bi start\_stage \oxsval{stage}\\
 \bi start\_stage\_iteration \oxsval{stage\_iteration}\\
 \bi start\_stage\_start\_time \oxsval{stage\_time}\\
 \bi start\_stage\_elapsed\_time \oxsval{stage\_elapsed\_time}\\
 \bi start\_last\_timestep \oxsval{timestep}\\
 \bi normalize\_aveM\_output \oxsval{aveMflag}\\
 \bi report\_max\_spin\_angle \oxsval{report\_angle}\\
 \bi report\_wall\_time \oxsval{report\_time}\\
\ccb
\end{quote}
\end{latexonly}
\begin{rawhtml}
<BLOCKQUOTE><DL><DT>
<TT>Specify Oxs_TimeDriver:</TT><I>name</I> <TT>{</TT>
<DD><TT> evolver </TT><I>evolver_spec</I>
<DD><TT> mesh </TT><I>mesh_spec</I>
<DD><TT> Ms </TT> <I>scalar_field_spec</I>
<DD><TT> m0 </TT> <I>vector_field_spec</I>
<DD><TT> stopping_dm_dt </TT><I>torque_criteria</I>
<DD><TT> stopping_time </TT><I>time_criteria</I>
<DD><TT> stage_iteration_limit </TT><I>stage_iteration_count</I>
<DD><TT> total_iteration_limit </TT><I>total_iteration_count</I>
<DD><TT> stage_count </TT><I>number_of_stages</I>
<DD><TT> stage_count_check </TT><I>test</I>
<DD><TT> checkpoint_file </TT> <I>restart_file_name</I>
<DD><TT> checkpoint_interval </TT> <I>checkpoint_minutes</I>
<DD><TT> checkpoint_disposal </TT> <I>cleanup_behavior</I>
<DD><TT> start_iteration </TT> <I>iteration</I>
<DD><TT> start_stage </TT> <I>stage</I>
<DD><TT> start_stage_iteration </TT> <I>stage_iteration</I>
<DD><TT> start_stage_start_time </TT> <I>stage_time</I>
<DD><TT> start_stage_elapsed_time </TT> <I>stage_elapsed_time</I>
<DD><TT> start_last_timestep </TT> <I>timestep</I>
<DD><TT> normalize_aveM_output </TT> <I>aveMflag</I>
<DD><TT> report_max_spin_angle </TT> <I>report_angle</I>
<DD><TT> report_wall_time </TT> <I>report_time</I>
<DT><TT>}</TT></DL></BLOCKQUOTE><P>
\end{rawhtml}
The first four parameters, \oxslabel{evolver}, \oxslabel{mesh},
\oxslabel{Ms} and \oxslabel{m0} provide references to a time evolver, a
mesh, a scalar field and a vector field, respectively.  Here \cd{Ms} is
the pointwise saturation magnetization in A/m, and \cd{m0} is the
initial configuration for the magnetization unit spins, i.e., $|\vm|=1$
at each point.  These four parameters are required.

The next group of 3 parameters control stage stopping criteria.  The
\oxslabel{stopping\_dm\_dt} value, in degrees per nanosecond, specifies that a
stage should be considered complete when the maximum $|d\vm/dt|$ across
all spins drops below this value.  Similarly, the
\oxslabel{stopping\_time} value specifies the maximum ``Simulation
time,'' i.e., the Landau-Lifshitz-Gilbert ODE (\ref{eq:oxsllode}),
(\ref{eq:oxsllgode}) time, allowed per stage.  For example, if
\oxsval{time\_criteria} is
\latex{$10^{-9}$}\html{1e-9}, then no stage will evolve for more than
1~ns.  If there were a total of 5 stages in the simulation, then the
total simulation time would be not more than 5~ns.  The third way to
terminate a stage is with a \oxslabel{stage\_iteration\_limit}.  This is
a limit on the number of successful evolver steps allowed per stage.  A
stage is considered complete when any one of these three criteria are
met.  Each of the criteria may be either a single value, which is
applied to every stage, or else a
\htmlonlyref{\textit{grouped list}}{par:groupedLists}
\latex{(Sec.~\ref{par:groupedLists})} of values.  If the
simulation has more stages than a criteria list has entries, then the
last criteria value is applied to all additional stages.  These stopping
criteria all provide a default value of 0, meaning no constraint, but
usually at least one is specified since otherwise there is no automatic
stage termination control.  For quasi-static simulations, a
\cd{stopping\_dm\_dt} value in the range of 1.0 to 0.01 is reasonable;
the numerical precision of the energy calculations usually makes in not
possible to obtain $|d\vm/dt|$ much below 0.001 degree per nanosecond.


\arbtarget{The}{PToxsdriverstagecount}
\oxslabel{total\_iteration\_limit}, \oxslabel{stage\_count} and
\oxslabel{stage\_count\_check} parameters involve simulation run
completion conditions.  The default value for the first is 0,
interpreted as no limit, but one may limit the total number of steps
performed in a simulation by specifying a positive integer value here.
The more usual run completion condition is based on the stage count.  If
a positive integer value is specified for \cd{stage\_count}, then the
run will be considered complete when the stage count reaches that value.
If \cd{stage\_count} is not specified, or is given the value 0, then the
effective \oxsval{number\_of\_stages} value is computed by examining the
length of the stopping criteria lists, and also any other \cd{Oxs\_Ext}
object that has stage length expectations, such as
\htmlonlyref{\cd{Oxs\_UZeeman}}{html:UZeeman}.  The longest of these is
taken to be the stage limit value.  Typically these lengths, along with
\cd{stage\_count} if specified, will all be the same, and any
differences indicate an error in the \MIF\ file.  Oxs will automatically
test this condition, provided \cd{stage\_count\_check} is set to 1,
which is the default value.  Stage length requests of 0 or 1 are ignored
in this test, since those lengths are commonly used to represent
sequences of arbitrary length.  At times a short sequence is
intentionally specified that is meant to be implicitly extended to match
the full simulation stage length.  In this case, the stage count check
can be disabled by setting \oxsval{test} to 0.

The \arbtarget{checkpoint}{PToxsdrivercheckpoint}%
\index{simulation~3D!restarting}\index{file!checkpoint}
options are used to control the saving of solver state to disk; these
saves are used by the \app{oxsii} and \app{boxsi} restart feature.
The value of the \oxslabel{checkpoint\_file} option is the name to
use for the solver state file.  The default is
\textit{base\_file\_name}.restart.

Cleanup of the checkpoint file is determined by the setting of
\oxslabel{checkpoint\_disposal}, which should be one of
\textit{standard} (the default), \textit{done\_only}, or \textit{never}.
Under the standard setting, the checkpoint file is automatically deleted
upon normal program termination, either because the solver reached the end
of the problem, or because the user interactively terminated the problem
prematurely.  If \oxsval{cleanup\_behavior} is set to
\texttt{done\_only}, then the checkpoint file is only deleted if the
problem endpoint is reached.  If \oxsval{cleanup\_behavior} is
\texttt{never}, then OOMMF does not delete checkpoint file; the
user is responsible for deleting this file as she desires.

The \oxslabel{checkpoint\_interval} value is the time in minutes between
overwrites of the checkpoint file.  No checkpoint file is written until
\oxsval{checkpoint\_minutes} have elapsed.  Checkpoint writes occur
between solver iterations, so the actual interval time may be somewhat
longer than the specified time.  If \oxsval{checkpoint\_minutes} is 0,
then each step is saved.  Setting \oxsval{checkpoint\_minutes} to -1
disables checkpointing.  The default checkpoint interval is 15 minutes.

The six \oxslabel{start\_*} options control the problem run start point.
These are intended primarily for automatic use by the restart feature.
The default value for each is 0.

The \oxslabel{normalize\_aveM\_output} option is used to control the
scaling and units on the average magnetization components $M_x$, $M_y$
and $M_z$ sent as DataTable output (this includes output sent to
\hyperrefhtml{\app{mmDataTable}}{\app{mmDataTable}
(Ch.~}{)}{sec:mmdatatable}\index{application!mmDataTable},
\hyperrefhtml{\app{mmGraph}}{\app{mmGraph}
(Ch.~}{)}{sec:mmgraph}\index{application!mmGraph}, and
\hyperrefhtml{\app{mmArchive}}{\app{mmArchive}
(Ch.~}{)}{sec:mmarchive}\index{application!mmArchive}).  If
\oxsval{aveMflag} is true (1), then the output values are scaled to lie
in the range $[-1,1]$, where the extreme values are obtained only at
saturation (i.e., all the spins are aligned).  If \oxsval{aveMflag} is
false (0), then the output is in A/m.  The default setting is 1.

In the older \MIF~2.1 format, the driver Specify block supports three
additional values: \oxslabel{basename},
\oxslabel{scalar\_output\_format}, and
\oxslabel{vector\_field\_output\_format}.  In the \MIF~2.2 format
these output controls have been moved into the \cd{SetOptions} block.
See the \hyperrefhtml{\cd{SetOptions}}{\cd{SetOptions}
(Sec.~}{)}{html:mif2SetOptions}\index{SetOptions~command~(MIF)}
documentation for details.

\cd{Oxs\_TimeDriver} provides 12 scalar outputs and 2 vector field
outputs.  The scalar outputs are
\begin{itemize}
\item \textbf{Stage:} current stage number, counting from 0.
\item \textbf{Stage iteration:} number of successful evolver steps
in the current stage.
\item \textbf{Iteration:} number of successful evolver steps in the
current simulation.
\item \textbf{Simulation time:} Landau-Lifshitz-Gilbert evolution
time, in seconds.
\item \textbf{Last time step:} The size of the preceding time step, in
seconds.
\item \textbf{Mx/mx:} magnetization component in the $x$ direction,
averaged across the entire simulation, in A/m (Mx) or normalized units
(mx), depending on the setting of the \cd{normalize\_aveM\_output}
option.
\item \textbf{My/my:} magnetization component in the $y$ direction,
averaged across the entire simulation, in A/m (My) or normalized units
(my), depending on the setting of the \cd{normalize\_aveM\_output}
option.
\item \textbf{Mz/mz:} magnetization component in the $z$ direction,
averaged across the entire simulation, in A/m (Mz) or normalized units
(mz), depending on the setting of the \cd{normalize\_aveM\_output}
option.
\item \textbf{Max Spin Ang:} maximum angle between neighboring spins
having non-zero magnetization $M_s$, measured in degrees.  The definition
of ``neighbor'' depends on the mesh, but for \cd{Oxs\_RectangularMesh}
the neighborhood of a point consists of 6 points, those nearest
forward and backward along each of the 3 coordinate axis directions.
\item \textbf{Stage Max Spin Ang:} the largest value of ``Max Spin
Ang'' obtained across the current stage, in degrees.
\item \textbf{Run Max Spin Ang:} the largest value of ``Max Spin
Ang'' obtained across the current run, in degrees.
\item \textbf{Wall time:} Wall clock time, in seconds.
\end{itemize}
The three ``Max Spin Ang'' outputs are disabled by default.  In general
one should refer instead to the neighboring spin angle outputs provided
by the exchange energies.  However, for backward compatibility, or for
simulations without any exchange energy terms, the driver spin angle
outputs can be enabled by setting the
\oxslabel{report\_max\_spin\_angle} option to to 1.

The ``Wall time'' output is also disabled by default.  It can be enabled
by setting the \oxslabel{report\_wall\_time} option to to 1.  It reports
the wall clock time, in seconds, since a system-dependent zero-time.
This output may be useful for performance comparisions and
debugging. (Note: The timestamp for a magnetization state is recorded
when output is first requested for that state; the timestamp is not
directly tied to the processing of the state.)

The vector field outputs are
\begin{itemize}
\item \textbf{Magnetization:} magnetization vector $\vM$, in A/m.
\item \textbf{Spin:} unit magnetization $\vm$.  This output ignores the
\cd{vector\_field\_output\_format} \oxsval{precision} setting, instead
always exporting at full precision.
\end{itemize}

\begin{ExampleMifs}
  \fn{sample.mif}, \fn{pulse.mif}.
\end{ExampleMifs}

\pttarget{PTMD}\index{Oxs\_Ext~child~classes!Oxs\_MinDriver}%
\item[Oxs\_MinDriver:\label{html:MinDriver}]
The Oxs driver for controlling minimization evolvers is
\textbf{Oxs\_MinDriver}.  The specify block has the form
\begin{latexonly}
\begin{quote}\tt
Specify Oxs\_MinDriver:\oxsval{name} \ocb\\
 \bi evolver \oxsval{evolver\_spec}\\
 \bi mesh \oxsval{mesh\_spec}\\
 \bi Ms \oxsval{scalar\_field\_spec}\\
 \bi m0 \oxsval{vector\_field\_spec}\\
 \bi stopping\_mxHxm \oxsval{torque\_criteria}\\
 \bi stage\_iteration\_limit \oxsval{stage\_iteration\_count}\\
 \bi total\_iteration\_limit \oxsval{total\_iteration\_count}\\
 \bi stage\_count \oxsval{number\_of\_stages}\\
 \bi stage\_count\_check \oxsval{test}\\
 \bi checkpoint\_file \oxsval{restart\_file\_name}\\
 \bi checkpoint\_interval \oxsval{checkpoint\_minutes}\\
 \bi checkpoint\_disposal \oxsval{cleanup\_behavior}\\
 \bi start\_iteration \oxsval{iteration}\\
 \bi start\_stage \oxsval{stage}\\
 \bi start\_stage\_iteration \oxsval{stage\_iteration}\\
 \bi start\_stage\_start\_time \oxsval{stage\_time}\\
 \bi start\_stage\_elapsed\_time \oxsval{stage\_elapsed\_time}\\
 \bi start\_last\_timestep \oxsval{timestep}\\
 \bi normalize\_aveM\_output \oxsval{aveMflag}\\
 \bi report\_max\_spin\_angle \oxsval{report\_angle}\\
 \bi report\_wall\_time \oxsval{report\_time}\\
\ccb
\end{quote}
\end{latexonly}
\begin{rawhtml}
<BLOCKQUOTE><DL><DT>
<TT>Specify Oxs_MinDriver:</TT><I>name</I> <TT>{</TT>
<DD><TT> evolver </TT><I>evolver_spec</I>
<DD><TT> mesh </TT><I>mesh_spec</I>
<DD><TT> Ms </TT> <I>scalar_field_spec</I>
<DD><TT> m0 </TT> <I>vector_field_spec</I>
<DD><TT> stopping_mxHxm </TT><I>torque_criteria</I>
<DD><TT> stage_iteration_limit </TT><I>stage_iteration_count</I>
<DD><TT> total_iteration_limit </TT><I>total_iteration_count</I>
<DD><TT> stage_count </TT><I>number_of_stages</I>
<DD><TT> stage_count_check </TT><I>test</I>
<DD><TT> checkpoint_file </TT> <I>restart_file_name</I>
<DD><TT> checkpoint_interval </TT> <I>checkpoint_minutes</I>
<DD><TT> checkpoint_disposal </TT> <I>cleanup_behavior</I>
<DD><TT> start_iteration </TT> <I>iteration</I>
<DD><TT> start_stage </TT> <I>stage</I>
<DD><TT> start_stage_iteration </TT> <I>stage_iteration</I>
<DD><TT> start_stage_start_time </TT> <I>stage_time</I>
<DD><TT> start_stage_elapsed_time </TT> <I>stage_elapsed_time</I>
<DD><TT> start_last_timestep </TT> <I>timestep</I>
<DD><TT> normalize_aveM_output </TT> <I>aveMflag</I>
<DD><TT> report_max_spin_angle </TT> <I>report_angle</I>
<DD><TT> report_wall_time </TT> <I>report_time</I>
<DT><TT>}</TT></DL></BLOCKQUOTE><P>
\end{rawhtml}
These parameters are the same as those described for the
\htmlonlyref{\cd{Oxs\_TimeDriver}}{item:TimeDriver}
class\latex{ (page~\pageref{item:TimeDriver})}, except that
\oxslabel{stopping\_mxHxm} replaces \cd{stopping\_dm\_dt}, and there is no
analogue to \cd{stopping\_time}.  The value for \cd{stopping\_mxHxm} is
in A/m, and may be a
\htmlonlyref{\textit{grouped list}}{par:groupedLists}
\latex{(Sec.~\ref{par:groupedLists})}.
Choice depends on the particulars of the simulation, but typical values
are in the range 10 to 0.1.  Limits in the numerical precision of the
energy calculations usually makes it not possible to obtain
$|\vm\times\vH\times\vm|$ below about 0.01 A/m.  This control can be
disabled by setting it to 0.0.

As with \cd{Oxs\_TimeDriver}, in the older \MIF~2.1 format this Specify
block supports three additional values: \oxslabel{basename} to control
output filenames, and output format controls
\oxslabel{scalar\_output\_format} and
\oxslabel{vector\_field\_output\_format}.  In the \MIF~2.2 format these
output controls have been moved into the \cd{SetOptions} block.  See the
\hyperrefhtml{\cd{SetOptions}}{\cd{SetOptions}
(Sec.~}{)}{html:mif2SetOptions}\index{SetOptions~command~(MIF)}
documentation for details.

\cd{Oxs\_MinDriver} provides 10 scalar outputs and 2 vector
field outputs.  The scalar outputs are
\begin{itemize}
\item \textbf{Stage:} current stage number, counting from 0.
\item \textbf{Stage iteration:} number of successful evolver steps
in the current stage.
\item \textbf{Iteration:} number of successful evolver steps in the
current simulation.
\item \textbf{Mx/mx:} magnetization component in the $x$ direction,
averaged across the entire simulation, in A/m (Mx) or normalized units
(mx), depending on the setting of the \cd{normalize\_aveM\_output}
option.
\item \textbf{My/my:} magnetization component in the $y$ direction,
averaged across the entire simulation, in A/m (My) or normalized units
(my), depending on the setting of the \cd{normalize\_aveM\_output}
option.
\item \textbf{Mz/mz:} magnetization component in the $z$ direction,
averaged across the entire simulation, in A/m (Mz) or normalized units
(mz), depending on the setting of the \cd{normalize\_aveM\_output}
option.
\item \textbf{Max Spin Ang:} maximum angle between neighboring spins
having non-zero magnetization $M_s$, measured in degrees.  The definition
of ``neighbor'' depends on the mesh, but for \cd{Oxs\_RectangularMesh}
the neighborhood of a point consists of 6 points, those nearest
forward and backward along each of the 3 coordinate axis directions.
\item \textbf{Stage Max Spin Ang:} the largest value of ``Max Spin
Ang'' obtained across the current stage, in degrees.
\item \textbf{Run Max Spin Ang:} the largest value of ``Max Spin
Ang'' obtained across the current run, in degrees.
\item \textbf{Wall time:} Wall clock time, in seconds.
\end{itemize}
As is the case for the \cd{Oxs\_TimeDriver}, the three ``Max Spin Ang''
outputs and ``Wall time'' are disabled by default.  They angle outputs
are enabled by setting the \oxslabel{report\_max\_spin\_angle} option to
to 1, and the wall time output is enabled by setting the
\oxslabel{report\_wall\_time} option to to 1.

The vector field outputs are
\begin{itemize}
\item \textbf{Magnetization:} magnetization vector $\vM$, in A/m.
\item \textbf{Spin:} unit magnetization $\vm$.  This output ignores the
\cd{vector\_field\_output\_format} \oxsval{precision} setting, instead
always exporting at full precision.
\end{itemize}

\begin{ExampleMifs}
  \fn{cgtest.mif}, \fn{stdprob3.mif}.
\end{ExampleMifs}

\end{description}

\subsection{Field Objects\label{sec:oxsFieldObjects}}
Field objects return values (either scalar or vector) as a function of
position.  These are frequently used as embedded objects inside Specify
blocks of other \cd{Oxs\_Ext} objects to initialize spatially varying
quantities, such as material parameters or initial magnetization spin
configurations.  Units on the returned values will be dependent upon the
context in which they are used.

Scalar field objects are documented first.  Vector field objects are
considered farther below.
\begin{description}
\pttarget{PTUSF}\index{Oxs\_Ext~child~classes!Oxs\_UniformScalarField}%
\item[Oxs\_UniformScalarField:\label{item:UniformScalarField}]
   Returns the same constant value regardless of the import position.
   The Specify block takes one parameter, \textbf{value}, which is the
   returned constant value.  This class is frequently embedded inline to
   specify homogeneous material parameters.  For example, inside a driver
   Specify block we may have
\begin{rawhtml}
<BLOCKQUOTE>
\end{rawhtml}
%begin<latexonly>
\begin{quote}
%end<latexonly>
\begin{verbatim}
Specify Oxs_TimeDriver {
    ...
    Ms { Oxs_UniformScalarField {
       value 8e5
    }}
    ...
}
\end{verbatim}
%begin<latexonly>
\end{quote}
%end<latexonly>
\begin{rawhtml}
</BLOCKQUOTE>
\end{rawhtml}
As discussed in
\html{the section on \htmlonlyref{Oxs\_Ext
referencing}{par:oxsExtReferencing} in the \htmlonlyref{MIF
2}{sec:mif2format} documentation,}
\latex{the \MIF\ 2 documentation (Sec.~\ref{par:oxsExtReferencing},
page~\pageref{par:oxsExtReferencing}),}
when embedding \cd{Oxs\_UniformScalarField}
or \htmlonlyref{\cd{Oxs\_UniformVectorField}}{item:UniformVectorField}
objects, a notational shorthand is allowed that lists only the value.
The previous example is exactly equivalent to
\begin{rawhtml}
<BLOCKQUOTE>
\end{rawhtml}
%begin<latexonly>
\begin{quote}
%end<latexonly>
\begin{verbatim}
Specify Oxs_TimeDriver {
    ...
    Ms 8e5
    ...
}
\end{verbatim}
%begin<latexonly>
\end{quote}
%end<latexonly>
\begin{rawhtml}
</BLOCKQUOTE>
\end{rawhtml}
where an implicit \cd{Oxs\_UniformScalarField} object is
created with \cd{value} set to \cd{8e5}.

\begin{ExampleMifs}
  \fn{sample.mif}, \fn{cgtest.mif}.
\end{ExampleMifs}


\pttarget{PTASF}\index{Oxs\_Ext~child~classes!Oxs\_AtlasScalarField}%
\item[Oxs\_AtlasScalarField:\label{item:AtlasScalarField}]
   Declares values that are defined across individual regions of an
   \cd{Oxs\_Atlas}.  The Specify block looks like
      \begin{latexonly}
      \begin{quote}\tt
      Specify Oxs\_AtlasScalarField:\oxsval{value} \ocb\\
       \bi atlas \oxsval{atlas\_spec}\\
       \bi multiplier \oxsval{mult}\\
       \bi default\_value \oxsval{scalar\_field\_spec}\\
       \bi values \ocb\\
         \bi\bi\oxsval{ region1\_label scalar\_field\_spec1 }\\
         \bi\bi\oxsval{ region2\_label scalar\_field\_spec2 }\\
         \bi\bi \ldots\\
       \bi\ccb\\
      \ccb
      \end{quote}
      \end{latexonly}
      \begin{rawhtml}
      <BLOCKQUOTE><DL><DT>
      <TT>Specify Oxs_AtlasScalarField {</TT>
      <DD><TT> atlas </TT><I>atlas_spec</I>
      <DD><TT> multiplier </TT><I>mult</I>
      <DD><TT> default_value </TT><I>scalar_field_spec</I>
      <DD><TT> values {</TT><DL>
          <DD><I>region1_label</I><TT>&nbsp;</TT><I>scalar_field_spec1</I>
          <DD><I>region2_label</I><TT>&nbsp;</TT><I>scalar_field_spec2</I>
          <DD> ...
      </DL><TT>}</TT>
      <DT><TT>}</TT></DL></BLOCKQUOTE><P>
      \end{rawhtml}

   The specified \oxslabel{atlas} is used to map cell locations to
   regions; the value at the cell location of the scalar field from the
   corresponding \oxslabel{values} sub-block is assigned to that cell.
   The \oxslabel{default\_value} entry is optional; if specified, and if
   a cell's region is not included in the \cd{values} sub-block, then
   the \cd{default\_value} scalar field is used.  If \cd{default\_value}
   is not specified, then missing regions will raise an error.

   The scalar field entries may specify any of the scalar field types
   described in this (Field Objects) section.  As usual, one may provide
   a single numeric value in any of the \cd{scalar\_field\_spec}
   positions; this will be interpreted as requesting a uniform (spatially
   homogeneous) field with the indicated value.

   If the optional \oxslabel{multiplier} value is provided, then each
   field value is scaled (multiplied) by the value \oxsval{mult}.

   The vector field analogue to this class is
   \htmlonlyref{\cd{Oxs\_AtlasVectorField}}{item:AtlasVectorField},
   which is described below in the vector fields portion of this
   section.

   \begin{ExampleMifs}
     \fn{diskarray.mif}, \fn{ellipsoid.mif},
     \fn{grill.mif}, \fn{spinvalve.mif}, \fn{tclshapes.mif}.
   \end{ExampleMifs}

\pttarget{PTLSF}\index{Oxs\_Ext~child~classes!Oxs\_LinearScalarField}%
\item[Oxs\_LinearScalarField:]
   Returns a value that varies linearly with position.  The Specify
   block has the form:
      \begin{latexonly}
      \begin{quote}\tt
      Specify Oxs\_LinearScalarField:\oxsval{name} \ocb\\
       \bi norm \oxsval{value}\\
       \bi vector \ocb\oxsval{ $v_x$ $v_y$ $v_z$ }\ccb\\
       \bi offset \oxsval{off}\\
      \ccb
      \end{quote}
      \end{latexonly}
      \begin{rawhtml}
      <BLOCKQUOTE><DL><DT>
      <TT>Specify Oxs_LinearScalarField {</TT>
      <DD><TT> norm </TT><I>value</I>
      <DD><TT> vector {</TT>
         <I>v<sub>x</sub></I><TT>&nbsp;</TT>
         <I>v<sub>y</sub></I><TT>&nbsp;</TT>
         <I>v<sub>z</sub></I> <TT>}</TT>
      <DD><TT> offset </TT><I>off</I>
      <DT><TT>}</TT></DL></BLOCKQUOTE><P>
      \end{rawhtml}
   If optional value \oxslabel{norm} is specified, then the given
   \oxslabel{vector} is first scaled to the requested size.  The
   \oxslabel{offset} entry is optional, with default value 0. For any
   given point $(x,y,z)$, the scalar value returned by this
   object will be $xv_x+yv_y+zv_z + off$.

\begin{ExampleMifs}[Example]
  \fn{spinvalve-af.mif}.
\end{ExampleMifs}

\pttarget{PTRSF}\index{Oxs\_Ext~child~classes!Oxs\_RandomScalarField}%
\item[Oxs\_RandomScalarField:]\label{item:RandomScalarField}
Defines a scalar field that varies spatially in a random fashion.
The Specify block has the form:
      \begin{latexonly}
      \begin{quote}\tt
      Specify Oxs\_RandomScalarField:\oxsval{name} \ocb\\
       \bi range\_min \oxsval{minvalue}\\
       \bi range\_max \oxsval{maxvalue}\\
       \bi cache\_grid \oxsval{mesh\_spec}\\
      \ccb
      \end{quote}
      \end{latexonly}
      \begin{rawhtml}
      <BLOCKQUOTE><DL><DT>
      <TT>Specify Oxs_RandomScalarField {</TT>
      <DD><TT> range_min </TT><I>minvalue</I>
      <DD><TT> range_max </TT><I>maxvalue</I>
      <DD><TT> cache_grid </TT><I>mesh_spec</I>
      <DT><TT>}</TT></DL></BLOCKQUOTE><P>
      \end{rawhtml}
The value at each position is drawn uniformly from the range declared by
the two required parameters, \oxslabel{range\_min} and
\oxslabel{range\_max}.  There is also an optional parameter,
\oxslabel{cache\_grid}, which takes a mesh specification that describes
the grid used for spatial discretization.  If
\oxslabel{cache\_grid} is not specified, then each call to
\cd{Oxs\_RandomScalarField} generates a different field.  If you want to
use the same random scalar field in two places (as a base for setting,
say anisotropy coefficients and saturation magnetization), then specify
\oxslabel{cache\_grid} with the appropriate (usually the base problem)
mesh.

\begin{ExampleMifs}
  \fn{randomshape.mif}, \fn{stdprob1.mif}.
\end{ExampleMifs}

\pttarget{PTSSF}\index{Oxs\_Ext~child~classes!Oxs\_ScriptScalarField}%
\item[Oxs\_ScriptScalarField:\label{item:ScriptScalarField}]
Analogous to the parallel
\htmlonlyref{\cd{Oxs\_ScriptVectorField}}{item:ScriptVectorField}
class, this class produces a scalar field dependent on a \Tcl\ script
and optionally other scalar and vector fields.  The Specify block has
the form
\begin{latexonly}
\begin{quote}\tt
Specify Oxs\_ScriptScalarField:\oxsval{name} \ocb\\
\bi script \oxsval{\Tcl\_script}\\
\bi script\_args \ocb\oxsval{ args\_request }\ccb\\
\bi scalar\_fields \ocb\oxsval{ scalar\_field\_spec \ldots }\ccb\\
\bi vector\_fields \ocb\oxsval{ vector\_field\_spec \ldots }\ccb\\
\bi atlas \oxsval{atlas\_spec}\\
\bi xrange \ocb\oxsval{ xmin xmax }\ccb\\
\bi yrange \ocb\oxsval{ ymin ymax }\ccb\\
\bi zrange \ocb\oxsval{ zmin zmax }\ccb\\
\ccb
\end{quote}
\end{latexonly}
\begin{rawhtml}
<BLOCKQUOTE><DL><DT>
<TT>Specify Oxs_ScriptScalarField:</TT><I>name</I> <TT>{</TT>
<DD><TT>script </TT> <I>Tcl_script</I>
<DD><TT>script_args {</TT> <I>args_request</I> <TT>}</TT>
<DD><TT>scalar_fields {</TT> <I>scalar_field_spec</I><TT> ...}</TT>
<DD><TT>vector_fields {</TT> <I>vector_field_spec</I><TT> ...}</TT>
<DD><TT>atlas </TT> <I>atlas_spec</I>
<DD><TT>xrange {</TT> <I>xmin<TT>&nbsp;</TT>xmax</I> <TT>}</TT>
<DD><TT>yrange {</TT> <I>ymin<TT>&nbsp;</TT>ymax</I> <TT>}</TT>
<DD><TT>zrange {</TT> <I>zmin<TT>&nbsp;</TT>zmax</I> <TT>}</TT>
<DT><TT>}</TT></DL></BLOCKQUOTE><P>
\end{rawhtml}
   For each point of interest, the specified \oxslabel{script} is
   called with the arguments requested by \oxslabel{script\_args}
   appended to the command, as explained in the \hyperrefhtml{User
   Defined Support Procedures}{User Defined Support Procedures section
   (Sec.~}{)}{par:supportProcs} \html{section} of the \MIF~2 file
   format documentation.  The value for \cd{script\_args} should be a
   subset of \cd{\ocb rawpt relpt minpt maxpt span scalars vectors\ccb}.

   If \cd{rawpt} is requested, then when the \Tcl\ proc is called, at
   the corresponding spot in the argument list the \cd{x}, \cd{y},
   \cd{z} values of point will be placed, in problem coordinates (in
   meters).  The points so passed will usually be node points in the
   simulation discretization (the \htmlonlyref{mesh}{sec:Meshes}), but
   this does not have to be the case in general.  The \cd{relpt},
   \cd{minpt}, \cd{maxpt}, and \cd{span} rely on a definition of a
   \textit{bounding box}, which is an axes parallel parallelepiped.  The
   bounding box must be specified by either referencing an
   \oxslabel{atlas}, or by explicitly stating the range via the three
   entries \oxslabel{xrange}, \oxslabel{yrange}, \oxslabel{zrange} (in
   meters).  The \cd{minpt} and \cd{maxpt} arguments list the minimum
   and maximum values of the bounding box (coordinate by coordinate),
   while \cd{span} provides the 3-vector resulting from $(\cd{maxpt} -
   \cd{minpt})$.  The \cd{relpt} selection provides \cd{x\_rel},
   \cd{y\_rel}, \cd{z\_rel}, where each element lies in the range
   $[0,1]$, indicating a relative position between \cd{minpt} and
   \cd{maxpt}, coordinate-wise.

   Each of the \cd{script\_args} discussed so far places exactly 3
   arguments onto the \Tcl\ proc argument list.  The last two,
   \cd{scalars} and \cd{vectors}, place arguments depending on the size
   of the \oxslabel{scalar\_fields} and \oxslabel{vector\_fields} lists.
   The \cd{scalar\_fields} value is a list of other scalar field
   objects.  Each scalar field is evaluated at the point in question,
   and the resulting scalar value is placed on the \Tcl\ proc argument
   list, in order.  The \cd{vector\_fields} option works similarly,
   except each vector field generates three points for the \Tcl\ proc
   argument list, since the output from vector field objects is a three
   vector.  Although the use of these entries appears complicated, this
   is a quite powerful facility that allows nearly unlimited control for
   the modification and combination of other field objects.  Both
   \cd{scalar\_fields} and \cd{vector\_fields} entries are optional.

   If \cd{script\_args} is not specified, the default value \cd{relpt}
   is used.

   Note that if \cd{script\_args} includes \cd{relpt}, \cd{minpt},
   \cd{maxpt}, or \cd{span}, then a bounding box must be specified, as
   discussed above.  The following example uses the explicit range
   method.  See the \ptlink{\cd{Oxs\_ScriptVectorField}}{PTSVF}
   documentation\latex{ (page~\pageref{item:ScriptVectorField})}
   for an example using an atlas specification.
% The extra BLOCKQUOTE's here are a workaround for an apparent
% latex2html bug
\begin{rawhtml}
<BLOCKQUOTE>
\end{rawhtml}
%begin<latexonly>
\begin{quote}
%end<latexonly>
\begin{verbatim}
proc Ellipsoid { xrel yrel zrel } {
    set xrad [expr {$xrel - 0.5}]
    set yrad [expr {$yrel - 0.5}]
    set zrad [expr {$zrel - 0.5}]
    set test [expr {$xrad*$xrad+$yrad*$yrad+$zrad*$zrad}]
    if {$test>0.25} {return 0}
    return 8.6e5
}

Specify Oxs_ScriptScalarField {
    script Ellipsoid
    xrange { 0   1e-6 }
    yrange { 0 250e-9 }
    zrange { 0  50e-9 }
}
\end{verbatim}
%begin<latexonly>
\end{quote}
%end<latexonly>
\begin{rawhtml}
</BLOCKQUOTE>
\end{rawhtml}
   This \cd{Oxs\_ScriptScalarField} object returns \latex{$8.6\times
   10^5$}\html{8.6e5} if the import (x,y,z) lies within the ellipsoid
   inscribed inside the axes parallel parallelepiped defined by (xmin=0,
   ymin=0, zmin=0) and (xmax=1e-6, ymax=250e-9, zmax=50e-9), and 0
   otherwise.
   See also the discussion of the
   \htmlonlyref{\cd{ReadFile}}{html:ReadFile} \MIF\ extension command
   \latex{in Sec.~\ref{sec:mif2ExtensionCommands}} for an example using
   an imported image file for similar purposes.

   Below is one more example, illustrating the use of the
   \cd{vector\_fields} option.
% The extra BLOCKQUOTE's here are a workaround for an apparent
% latex2html bug
\begin{rawhtml}
<BLOCKQUOTE>
\end{rawhtml}
%begin<latexonly>
\begin{quote}
%end<latexonly>
\begin{verbatim}
proc DotProduct { x1 y1 z1 x2 y2 z2 } {
    return [expr {$x1*$x2+$y1*$y2+$z1*$z2}]
}

Specify Oxs_FileVectorField:file1 {
    atlas :atlas
    file  file1.omf
}

Specify Oxs_UniformVectorField:dir111 {
    norm 1
    vector {1 1 1}
}

Specify Oxs_ScriptScalarField:project {
    script DotProduct
    script_args vectors
    vector_fields {:file1 :dir111}
}
\end{verbatim}
%begin<latexonly>
\end{quote}
%end<latexonly>
\begin{rawhtml}
</BLOCKQUOTE>
\end{rawhtml}
The scalar field \cd{:project} yields at each point in space the
projection of the vector field \cd{:file1} onto the [1,1,1] direction.

\begin{ExampleMifs}
  \fn{antidots-filled.mif}, \fn{ellipsoid-fieldproc.mif},
  \fn{manyregions-scriptfields.mif}, \fn{manyspheres.mif},
  \fn{varalpha.mif}.
\end{ExampleMifs}

\pttarget{PTVMSF}\index{Oxs\_Ext~child~classes!Oxs\_VecMagScalarField}%
\item[Oxs\_VecMagScalarField:]
The \cd{Oxs\_VecMagScalarField} class produces a scalar field
from a vector field by taking the norm of the vector field on a
point-by-point basis, i.e.,
\begin{displaymath}
   \|\mbox{\boldmath $v$}\| = \sqrt{v_x^2+v_y^2+v_z^2}.
\end{displaymath}
The Specify block has the form:
\begin{latexonly}
\begin{quote}\tt
Specify Oxs\_VecMagScalarField:\oxsval{name} \ocb\\
 \bi field \oxsval{vector\_field\_spec}\\
 \bi multiplier \oxsval{mult}\\
 \bi offset \oxsval{off}\\
\ccb
\end{quote}
\end{latexonly}
\begin{rawhtml}
<BLOCKQUOTE><DL><DT>
<TT>Specify Oxs_VecMagScalarField {</TT>
<DD><TT> field </TT><I>vector_field_spec</I>
<DD><TT> multiplier </TT><I>mult</I>
<DD><TT> offset </TT><I>off</I>
<DT><TT>}</TT></DL></BLOCKQUOTE><P>
\end{rawhtml}
The \oxslabel{multiplier} and \oxslabel{offset} entries are applied
after the vector norm, i.e., the resulting scalar field is
$\cd{mult}\ast\|\mbox{\boldmath $v$}\|+\cd{off}$.  The default values
for \cd{mult} and \cd{off} are 1 and 0, respectively.

The functionality of the \cd{Oxs\_VecMagScalarField} class may be achieved
with the \cd{Oxs\_ScriptScalarField} class by using the
\cd{vector\_fields} option and a \Tcl\ script to compute the vector
norm.  However, this particular functionality is needed frequently
enough that a specialized class is useful.  For example, this class can
be used in conjunction with a vector field object to set
both the saturation magnetization distribution ($M_s$) and the initial
magnetization:
% The extra BLOCKQUOTE's here are a workaround for an apparent
% latex2html bug
\begin{rawhtml}
<BLOCKQUOTE>
\end{rawhtml}
%begin<latexonly>
\begin{quote}
%end<latexonly>
\begin{verbatim}
Specify Oxs_FileVectorField:file1 {
    atlas :atlas
    file  file1.omf
}

Specify Oxs_TimeDriver {
    basename test
    evolver :evolve
    stopping_dm_dt 0.01
    mesh :mesh
    m0 :file1
    Ms { Oxs_VecMagScalarField {
       field :file1
    }}
}
\end{verbatim}
%begin<latexonly>
\end{quote}
%end<latexonly>
\begin{rawhtml}
</BLOCKQUOTE>
\end{rawhtml}

\begin{ExampleMifs}[Example]
  \fn{sample-vecrotate.mif}.
\end{ExampleMifs}

\pttarget{PTSOSF}\index{Oxs\_Ext~child~classes!Oxs\_ScriptOrientScalarField}%
\item[Oxs\_ScriptOrientScalarField:\label{item:ScriptOrientScalarField}]
Scalar fields provide scalar values as a function of position across
three-space.  The \cd{Oxs\_ScriptOrientScalarField} class is used to
compose a transformation on the input position before evaluation by
a scalar field.  The Specify block has the form:
\begin{latexonly}
\begin{quote}\tt
Specify Oxs\_ScriptOrientScalarField:\oxsval{name} \ocb\\
\bi field \oxsval{scalar\_field\_spec}\\
\bi script \oxsval{\Tcl\_script}\\
\bi script\_args \ocb\oxsval{ args\_request }\ccb\\
\bi atlas \oxsval{atlas\_spec}\\
\bi xrange \ocb\oxsval{ xmin xmax }\ccb\\
\bi yrange \ocb\oxsval{ ymin ymax }\ccb\\
\bi zrange \ocb\oxsval{ zmin zmax }\ccb\\
\ccb
\end{quote}
\end{latexonly}
\begin{rawhtml}
<BLOCKQUOTE><DL><DT>
<TT>Specify Oxs_ScriptOrientScalarField:</TT><I>name</I> <TT>{</TT>
<DD><TT>field </TT> <I>scalar_field_spec</I>
<DD><TT>script </TT> <I>Tcl_script</I>
<DD><TT>script_args {</TT> <I>args_request</I> <TT>}</TT>
<DD><TT>atlas </TT> <I>atlas_spec</I>
<DD><TT>xrange {</TT> <I>xmin</I><TT>&nbsp;</TT><I>xmax</I> <TT>}</TT>
<DD><TT>yrange {</TT> <I>ymin</I><TT>&nbsp;</TT><I>ymax</I> <TT>}</TT>
<DD><TT>zrange {</TT> <I>zmin</I><TT>&nbsp;</TT><I>zmax</I> <TT>}</TT>
<DT><TT>}</TT></DL></BLOCKQUOTE><P>
\end{rawhtml}
The \oxslabel{field} argument should refer to a scalar field object.
The \oxslabel{script} is a \Tcl\ script that should return a position
vector that will be sent on the \cd{field} object to ultimately
produce a scalar value.  The arguments to the \cd{\Tcl\_script} are
determined by \oxslabel{script\_args}, which should be a subset of
\cd{\ocb relpt rawpt minpt maxpt span\ccb}.  If any arguments other than
\cd{rawpt} are requested, then the bounding box must be specified by
either the \oxslabel{atlas} option, or else through the three \oxslabel{xrange},
\oxslabel{yrange}, \oxslabel{zrange} entries.  The default value for
\cd{script\_args} is \cd{relpt}.

The \cd{Oxs\_ScriptOrientScalarField} class can be used to change the
``orientation'' of a scalar field, as in the following simple example,
which reflects the \cd{:file1mag} scalar field across the yz-plane:
% The extra BLOCKQUOTE's here are a workaround for an apparent
% latex2html bug
\begin{rawhtml}
<BLOCKQUOTE>
\end{rawhtml}
%begin<latexonly>
\begin{quote}
%end<latexonly>
\begin{verbatim}
Specify Oxs_FileVectorField:file1 {
    atlas :atlas
    file  file1.omf
}

Specify Oxs_VecMagScalarField:file1mag {
   field :file1
}

proc Reflect { x y z xmin ymin zmin xmax ymax zmax} {
   return [list [expr {($xmax+$xmin-$x)}] $y $z]
}

Specify Oxs_ScriptOrientScalarField:reflect {
   field :file1mag
   script Reflect
   script_args {rawpt minpt maxpt}
   atlas :atlas
}
\end{verbatim}
%begin<latexonly>
\end{quote}
%end<latexonly>
\begin{rawhtml}
</BLOCKQUOTE>
\end{rawhtml}
See also the
\htmlonlyref{\cd{Oxs\_ScriptOrientVectorField}}{item:ScriptOrientVectorField}
class\latex{ (page~\pageref{item:ScriptOrientVectorField})} for
analogous operations on vector fields.

\begin{ExampleMifs}[Example]
  \fn{sample-reflect.mif}.
\end{ExampleMifs}

\pttarget{PTAOSF}\index{Oxs\_Ext~child~classes!Oxs\_AffineOrientScalarField}%
\item[Oxs\_AffineOrientScalarField:\label{item:AffineOrientScalarField}]
The \cd{Oxs\_AffineOrientScalarField} class is similar to the
\cd{Oxs\_ScriptOrientScalarField} class, except that the transformation
on the import position is by an affine transformation defined in terms
of a \latex{$3\times 3$}\html{3x3} matrix and an offset instead of a
\Tcl\ script.  Although this functionality can be obtained by an
appropriate \Tcl\ script, the \cd{Oxs\_AffineOrientScalarField} is
easier to use and will run faster, as the underlying transformation is
performed by compiled C++ instead of \Tcl\ script.

The Specify block has the form:
\begin{latexonly}
\begin{quote}\tt
Specify Oxs\_AffineOrientScalarField:\oxsval{name} \ocb\\
 \bi field \oxsval{scalar\_field\_spec}\\
 \bi M \ocb\oxsval{ matrix\_entries \ldots }\ccb\\
 \bi offset \ocb\oxsval{ off${}_x$ off${}_y$ off${}_z$ }\ccb\\
 \bi inverse \oxsval{invert\_flag}\\
 \bi inverse\_slack \oxsval{slack}\\
\ccb
\end{quote}
\end{latexonly}
\begin{rawhtml}
<BLOCKQUOTE><DL><DT>
<TT>Specify Oxs_AffineOrientScalarField {</TT>
<DD><TT> field </TT><I>scalar_field_spec</I>
<DD><TT> M
  {</TT> <I>matrix_entries</I><TT>&nbsp;</TT><I>...</I> <TT>}</TT>
<DD><TT> offset {</TT>
   <I>off<sub>x</sub></I><TT>&nbsp;</TT>
   <I>off<sub>y</sub></I><TT>&nbsp;</TT>
   <I>off<sub>z</sub></I> <TT>}</TT>
<DD><TT> inverse </TT><I>invert_flag</I>
<DD><TT> inverse_slack </TT><I>slack</I>
<DT><TT>}</TT></DL></BLOCKQUOTE><P>
\end{rawhtml}
If $F(\mbox{\boldmath$x$})$ represents the scalar field specified by the
\oxslabel{field} value, then the resulting transformed scalar field is
$F(M\mbox{\boldmath$x$}+\textbf{off})$.  Here \oxslabel{M} is a
\latex{$3\times 3$}\html{3x3} matrix, which may be specified by a list
of 1, 3, 6 or 9 entries.  If the \cd{matrix\_entries} list consists of a
single value, then $M$ is taken to be that value times the identity
matrix, i.e., $M$ is a homogeneous scaling transformation.  If
\cd{matrix\_entries} consists of 3 values, then $M$ is taken to be the
diagonal matrix with those three values along the diagonal.
If \cd{matrix\_entries} is 6 elements long, then $M$ is assumed to be a
symmetric matrix, where the 6 elements specified correspond to $M_{11}$,
$M_{12}$, $M_{13}$, $M_{22}$, $M_{23}$, and $M_{33}$.  Finally, if
\cd{matrix\_entries} is 9 elements long, then the elements specify the
entire matrix, in the order $M_{11}$, $M_{12}$, $M_{13}$, $M_{21}$,
\ldots, $M_{33}$.  If $M$ is not specified, then it is taken to be
the identity matrix.

The \oxslabel{offset} entry is simply a 3-vector that is added to
$M\mbox{\boldmath$x$}$.  If \cd{offset} is not specified, then
it is set to the zero vector.

It is frequently the case that the transformation that one wants to
apply is not $M\mbox{\boldmath$x$}+\textbf{off}$, but rather the
inverse, i.e., $M^{-1}(\mbox{\boldmath$x$}-\textbf{off})$.  Provided $M$
is nonsingular, this can be accomplished by setting the
\oxslabel{inverse} option to 1.  In this case the matrix $M.M^{-1}$ is
compared to the identity matrix, to check the accuracy of the matrix
inversion.  If any entry in $M.M^{-1}$ differs from $I$ by more than the
8-byte float machine precision (typically
\latex{$2\times 10^{-16}$}\html{2e-16}) times the value of
\oxslabel{inverse\_slack}, then an error is raised. The default setting
for \cd{invert\_flag} is 0, meaning don't invert,
and the default setting for \cd{slack} is 128.

Here is an example using \cd{Oxs\_AffineOrientScalarField} to rotate a
field by \latex{$90^{\circ}$}\html{90 degrees} counterclockwise about the
$z$-axis.  Note that the specified atlas is square in $x$ and $y$, with
the origin of the atlas coordinates in the center of the atlas volume.
\begin{rawhtml}
<BLOCKQUOTE>
\end{rawhtml}
%begin<latexonly>
\begin{quote}
%end<latexonly>
\begin{verbatim}
Specify Oxs_BoxAtlas:atlas {
  xrange {-250e-9 250e-9}
  yrange {-250e-9 250e-9}
  zrange { -15e-9  15e-9}
}

Specify Oxs_FileVectorField:file1 {
    atlas :atlas
    file  file1.omf
}

Specify Oxs_VecMagScalarField:file1mag {
   field :file1
}

Specify Oxs_AffineOrientScalarField:reflect {
   field :file1mag
   M { 0 1 0
      -1 0 0
       0 0 1 }
}
\end{verbatim}
%begin<latexonly>
\end{quote}
%end<latexonly>
\begin{rawhtml}
</BLOCKQUOTE>
\end{rawhtml}

See also the
\htmlonlyref{\cd{Oxs\_AffineOrientVectorField}}{item:AffineOrientVectorField}
class\latex{ (page~\pageref{item:AffineOrientVectorField})} for
analogous operations on vector fields.

\begin{ExampleMifs}[Example]
  \fn{sample-rotate.mif}.
\end{ExampleMifs}

\pttarget{PTATSF}\index{Oxs\_Ext~child~classes!Oxs\_AffineTransformScalarField}%
\item[Oxs\_AffineTransformScalarField:\label{item:AffineTransformScalarField}]
Like the \cd{Oxs\_AffineOrientScalarField} class, this class composes
an affine transform with a separate scalar field, but in this case the
affine transform is applied \textit{after} the field evaluation.
The Specify block has the form:
\begin{latexonly}
\begin{quote}\tt
Specify Oxs\_AffineTransformScalarField:\oxsval{name} \ocb\\
 \bi field \oxsval{scalar\_field\_spec}\\
 \bi multiplier \oxsval{mult}\\
 \bi offset \oxsval{off}\\
 \bi inverse \oxsval{invert\_flag}\\
\ccb
\end{quote}
\end{latexonly}
\begin{rawhtml}
<BLOCKQUOTE><DL><DT>
<TT>Specify Oxs_AffineTransformScalarField {</TT>
<DD><TT> field </TT><I>scalar_field_spec</I>
<DD><TT> multiplier </TT><I>mult</I>
<DD><TT> offset </TT><I>off</I>
<DD><TT> inverse </TT><I>invert_flag</I>
<DT><TT>}</TT></DL></BLOCKQUOTE><P>
\end{rawhtml}
If $F(\mbox{\boldmath$x$})$ represents the scalar field specified by the
\oxslabel{field} value, then the resulting scalar field is
$\textrm{mult}*F(\mbox{\boldmath$x$})+\textrm{off}$.  Since the output
from $F$ is a scalar, both \oxslabel{multiplier} and \oxslabel{offset}
are scalars.  If \oxslabel{inverse} is 1, then the transform is changed
to $\left(F(\mbox{\boldmath$x$})-\textrm{off}\right)/\textrm{mult}$,
provided \cd{mult} is non-zero.

The default values for \oxsval{mult}, \oxsval{off}, and
\oxsval{invert\_flag} are 1, 0, and 0, respectively.  The \cd{field}
value is the only required entry.

The functionality provided by \cd{Oxs\_AffineTransformScalarField} can
also be produced by the
\htmlonlyref{\cd{Oxs\_ScriptScalarField}}{item:ScriptScalarField}
class\latex{ (page~\pageref{item:ScriptScalarField})} with the
\cd{scalar\_fields} entry, but the \cd{Oxs\_AffineTransformScalarField}
class is faster and has a simpler interface.  See also the
\htmlonlyref{\cd{Oxs\_AffineTransformVectorField}}{item:AffineTransformVectorField}
class\latex{ (page~\pageref{item:AffineTransformVectorField})} for analogous
operations on vector fields.

\begin{ExampleMifs}[Example]
  \fn{sample-rotate.mif}.
\end{ExampleMifs}

\pttarget{PTISF}\index{Oxs\_Ext~child~classes!Oxs\_ImageScalarField}%
\item[Oxs\_ImageScalarField:\label{item:ImageScalarField}]
This class creates a scalar field using an image.  The Specify block has
the form
\begin{latexonly}
\begin{quote}\tt
Specify Oxs\_ImageScalarField:\oxsval{name} \ocb\\
 \bi image \oxsval{pic}\\
 \bi invert \oxsval{invert\_flag}\\
 \bi multiplier \oxsval{mult}\\
 \bi offset \oxsval{off}\\
 \bi viewplane \oxsval{view}\\
 \bi atlas \oxsval{atlas\_spec}\\
 \bi xrange \ocb\oxsval{ xmin xmax }\ccb\\
 \bi yrange \ocb\oxsval{ ymin ymax }\ccb\\
 \bi zrange \ocb\oxsval{ zmin zmax }\ccb\\
 \bi exterior \oxsval{ext\_flag}\\
\ccb
\end{quote}
\end{latexonly}
\begin{rawhtml}
<BLOCKQUOTE><DL><DT>
<TT>Specify Oxs_ImageScalarField:</TT><I>name</I> <TT>{</TT>
<DD><TT>image </TT> <I>pic</I>
<DD><TT>invert </TT> <I>invert_flag</I>
<DD><TT>multiplier </TT><I>mult</I>
<DD><TT>offset </TT><I>off</I>
<DD><TT>viewplane </TT> <I>view</I>
<DD><TT>atlas </TT> <I>atlas_spec</I>
<DD><TT>xrange {</TT> <I>xmin</I><TT>&nbsp;</TT><I>xmax</I> <TT>}</TT>
<DD><TT>yrange {</TT> <I>ymin</I><TT>&nbsp;</TT><I>ymax</I> <TT>}</TT>
<DD><TT>zrange {</TT> <I>zmin</I><TT>&nbsp;</TT><I>zmax</I> <TT>}</TT>
<DD><TT>exterior </TT><I>ext_flag</I>
<DT><TT>}</TT></DL></BLOCKQUOTE><P>
\end{rawhtml}
The \oxslabel{image} is interpreted as a monochromatic map, yielding a
scalar field with black corresponding to zero and white to one if
\oxslabel{invert} is 0 (the default), or with black corresponding to 1
and white to 0 if \cd{invert} is 1.  Color images are converted to
grayscale by simply summing the red, green, and blue components.  A
\oxslabel{multiplier} option is available to change the range of values
from $[0,1]$ to $[0,\mbox{\texttt{mult}}]$, after which the
\oxslabel{offset} value, if any, is added.

The \oxslabel{viewplane} is treated in the same manner as the
\ptlink{\cd{Oxs\_ImageAtlas}}{PTIA} viewplane option, and should
likewise take one of the three two-letter codes \cd{xy} (default),
\cd{zx} or \cd{yz}.  The spatial scale is adjusted to fit the volume
specified by either the \oxslabel{atlas} or
\oxslabel{xrange/yrange/zrange} selections.  If the specified volume
does not fill the entire simulation volume, then points outside the
specified volume are handled as determined by the \oxslabel{exterior}
setting, which should be either a floating point value, or one of the
keywords \cd{boundary} or \cd{error}.  In the first case, the floating
point value is treated as a default value for points outside the image,
and should have a value in the range $[0,1]$.  The multiplier and offset
adjustments are made to this value in the same way as to points inside
the image.  If \oxsval{ext\_flag} is \cd{boundary}, then points outside the
image are filled with the value of the closest point on the boundary of
the image.  If \oxsval{ext} is \cd{error} (the default), then an error
is raised if a value is needed for any point outside the image.

\begin{ExampleMifs}
  \fn{rotatecenterstage.mif}, \fn{sample-reflect.mif}.
\end{ExampleMifs}

\end{description}

\noindent
The available vector field objects are:
\begin{description}
\pttarget{PTUVF}\index{Oxs\_Ext~child~classes!Oxs\_UniformVectorField}%
\item[Oxs\_UniformVectorField:\label{item:UniformVectorField}]
   Returns the same constant value regardless of the import position.
   The Specify block takes one required parameter, \oxslabel{vector},
   which is a 3-element list of the vector to return, and one optional
   parameter, \oxslabel{norm}, which if specified adjusts the size of
   export vector to the specified magnitude.  For example,
   \begin{latexonly}
   \begin{quote}\tt
      Specify Oxs\_UniformVectorField \ocb\\
      \bi norm 1\\
      \bi vector \ocb 1 1 1\ccb\\
      \ccb
   \end{quote}
   \end{latexonly}
   \begin{rawhtml}
   <BLOCKQUOTE><DL><DT>
      <TT>Specify Oxs_UniformVectorField {</TT>
      <DD><TT> norm 1</TT>
      <DD><TT> vector {1 1 1}</TT>
   <DT><TT>}</TT></DL></BLOCKQUOTE><P>
   \end{rawhtml}
   This object returns the unit vector $(a,a,a)$, where
   \latexhtml{$a=1/\sqrt{3}$}{a=1/sqrt(3)}, regardless of the import
   position.

   This class is frequently embedded inline to specify spatially uniform
   quantities.  For example, inside a driver Specify block we may have

\begin{rawhtml}
<BLOCKQUOTE>
\end{rawhtml}
%begin<latexonly>
\begin{quote}
%end<latexonly>
\begin{verbatim}
Specify Oxs_TimeDriver {
    ...
    m0 { Oxs_UniformVectorField {
       vector {1 0 0}
    }}
    ...
}
\end{verbatim}
%begin<latexonly>
\end{quote}
%end<latexonly>
\begin{rawhtml}
</BLOCKQUOTE>
\end{rawhtml}
As discussed in
\html{the section on \htmlonlyref{Oxs\_Ext
referencing}{par:oxsExtReferencing} in the \htmlonlyref{MIF
2}{sec:mif2format} documentation,}
\latex{the \MIF\ 2 documentation (Sec.~\ref{par:oxsExtReferencing},
page~\pageref{par:oxsExtReferencing}),}
when embedding \cd{Oxs\_UniformVectorField}
or \htmlonlyref{\cd{Oxs\_UniformScalarField}}{item:UniformScalarField}
objects, a notational shorthand is allowed that lists only the required
value.  The previous example is exactly equivalent to
\begin{rawhtml}
<BLOCKQUOTE>
\end{rawhtml}
%begin<latexonly>
\begin{quote}
%end<latexonly>
\begin{verbatim}
Specify Oxs_TimeDriver {
    ...
    m0 {1 0 0}
    ...
}
\end{verbatim}
%begin<latexonly>
\end{quote}
%end<latexonly>
\begin{rawhtml}
</BLOCKQUOTE>
\end{rawhtml}
where an implicit \cd{Oxs\_UniformVectorField} object is
created with the value of \cd{vector} set to \cd{\ocb 1 0 0\ccb}.

\begin{ExampleMifs}
  \fn{sample.mif}, \fn{cgtest.mif}.
\end{ExampleMifs}

\pttarget{PTAVF}\index{Oxs\_Ext~child~classes!Oxs\_AtlasVectorField}%
\item[Oxs\_AtlasVectorField:\label{item:AtlasVectorField}]
   Declares vector values that are defined across individual regions of
   an \cd{Oxs\_Atlas}.  The Specify block has the form
      \begin{latexonly}
      \begin{quote}\tt
      Specify Oxs\_AtlasVectorField:\oxsval{name} \ocb\\
       \bi atlas \oxsval{atlas\_spec}\\
       \bi norm \oxsval{magval}\\
       \bi multiplier \oxsval{mult}\\
       \bi default\_value \oxsval{vector\_field\_spec}\\
       \bi values \ocb\\
       \bi\bi\oxsval{ region1\_label vector\_field\_spec1 }\\
       \bi\bi\oxsval{ region2\_label vector\_field\_spec2 }\\
       \bi\bi \ldots\\
       \bi\ccb\\
      \ccb
      \end{quote}
      \end{latexonly}
      \begin{rawhtml}
      <BLOCKQUOTE><DL><DT>
      <TT>Specify Oxs_AtlasVectorField {</TT>
      <DD><TT> atlas </TT><I>atlas_spec</I>
      <DD><TT> norm </TT><I>magval</I>
      <DD><TT> multiplier </TT><I>mult</I>
      <DD><TT> default_value </TT><I>vector_field_spec</I>
      <DD><TT> values {</TT><DL>
          <DD><I>region1_label</I><TT>&nbsp;</TT><I>vector_field_spec1</I>
          <DD><I>region2_label</I><TT>&nbsp;</TT><I>vector_field_spec2</I>
          <DD> ...
      </DL><TT>}</TT>
      <DT><TT>}</TT></DL></BLOCKQUOTE><P>
      \end{rawhtml}
   Interpretation is analogous to the
   \htmlonlyref{\cd{Oxs\_AtlasScalarField}}{item:AtlasScalarField}
   specify block, except here the output values are 3 dimensional
   vectors rather than scalars.  Thus the values associated with each
   region are vector fields rather than scalar fields.  Any of the
   vector field types described in this (Field Objects) section may be
   used.  As usual, one may provided a braced list of three numeric
   values to request a uniform (spatially homogeneous) vector field with
   the indicated value.

   The optional \oxslabel{norm} parameter causes each vector value to be
   scaled to have magnitude \oxsval{magval}.  The optional
   \oxslabel{multiplier} value scales the field values.  If both
   \cd{norm} and \cd{multiplier} are specified, then the field vectors
   are first normalized before being scaled by the multiplier value.

   \begin{ExampleMifs}
     \fn{diskarray.mif}, \fn{exchspring.mif},
     \fn{imageatlas.mif}, \fn{spinvalve.mif}.
   \end{ExampleMifs}

\pttarget{PTSVF}\index{Oxs\_Ext~child~classes!Oxs\_ScriptVectorField}%
\item[Oxs\_ScriptVectorField:\label{item:ScriptVectorField}]
Conceptually similar to the
\htmlonlyref{\cd{Oxs\_ScriptScalarField}}{item:ScriptScalarField} scalar
field object\latex{ (page~\pageref{item:ScriptScalarField})},
except that the script should return a vector (as a 3 element list)
rather than a scalar.  In addition to the parameters accepted by
\cd{Oxs\_ScriptScalarField}, \cd{Oxs\_ScriptVectorField} also accepts
an optional parameter \oxslabel{norm}.  If specified, the return
values from the script are size adjusted to the specified magnitude.
If both \cd{norm} and \cd{multiplier} are specified, then
the field vectors are first normalized before being scaled by the
multiplier value.

The following example produces a vortex-like unit vector field, with
an interior core region pointing parallel to the $z$-axis.  Here the
scaling region is specified using an \cd{atlas} reference to an
object named ``:atlas'', which is presumed to be defined earlier in
the \MIF\ file.  See the \cd{Oxs\_ScriptScalarField} sample Specify
block for an example using the explicit range option.
% The extra BLOCKQUOTE's here are a workaround for an apparent
% latex2html bug
\begin{rawhtml}
<BLOCKQUOTE>
\end{rawhtml}
%begin{latexonly}
\begin{quote}
%end{latexonly}
\begin{verbatim}
proc Vortex { xrel yrel zrel } {
    set xrad [expr {$xrel-0.5}]
    set yrad [expr {$yrel-0.5}]
    set normsq [expr {$xrad*$xrad+$yrad*$yrad}]
    if {$normsq <= 0.025} {return "0 0 1"}
    return [list [expr {-1*$yrad}] $xrad 0]
}

Specify Oxs_ScriptVectorField {
    script Vortex
    norm  1
    atlas :atlas
}
\end{verbatim}
%begin{latexonly}
\end{quote}
%end{latexonly}
\begin{rawhtml}
</BLOCKQUOTE>
\end{rawhtml}
See also the
\htmlonlyref{\cd{Oxs\_MaskVectorField}}{item:MaskVectorField}
documentation and the discussion of the
\htmlonlyref{\cd{ReadFile}}{html:ReadFile} \MIF\ extension command
\latex{in Sec.~\ref{sec:mif2ExtensionCommands}}
for other example uses of the \cd{Oxs\_ScriptVectorField} class.

\begin{ExampleMifs}
  \fn{cgtest.mif}, \fn{ellipsoid.mif},
  \fn{manyregions-scriptfields.mif}, \fn{sample-vecreflect.mif},
  \fn{stdprob3.mif}, \fn{yoyo.mif}.
\end{ExampleMifs}

\pttarget{PTFVF}\index{Oxs\_Ext~child~classes!Oxs\_FileVectorField}%
\item[Oxs\_FileVectorField:\label{item:FileVectorField}]
   Provides a file-specified vector field.  The Specify block has the form
      \begin{latexonly}
      \begin{quote}\tt
      Specify Oxs\_FileVectorField:\oxsval{name} \ocb\\
       \bi file  \oxsval{filename}\\
       \bi atlas \oxsval{atlas\_spec}\\
       \bi xrange \ocb\oxsval{ xmin xmax }\ccb\\
       \bi yrange \ocb\oxsval{ ymin ymax }\ccb\\
       \bi zrange \ocb\oxsval{ zmin zmax }\ccb\\
       \bi spatial\_scaling \ocb\oxsval{ xscale yscale zscale }\ccb\\
       \bi spatial\_offset \ocb\oxsval{ xoff yoff zoff }\ccb\\
       \bi exterior \oxsval{ext\_flag}\\
       \bi norm  \oxsval{magnitude}\\
       \bi multiplier \oxsval{mult}\\
      \ccb
      \end{quote}
      \end{latexonly}
      \begin{rawhtml}
      <BLOCKQUOTE><DL><DT>
      <TT>Specify Oxs_FileVectorField {</TT>
      <DD><TT>file </TT><I>filename</I>

      <DD><TT>atlas </TT> <I>atlas_spec</I>
      <DD><TT>xrange {</TT> <I>xmin</I><TT>&nbsp;</TT><I>xmax</I> <TT>}</TT>
      <DD><TT>yrange {</TT> <I>ymin</I><TT>&nbsp;</TT><I>ymax</I> <TT>}</TT>
      <DD><TT>zrange {</TT> <I>zmin</I><TT>&nbsp;</TT><I>zmax</I> <TT>}</TT>
      <DD><TT>spatial_scaling {</TT>
        <I>xscale</I><TT>&nbsp;</TT><I>yscale</I><TT>&nbsp;</TT><I>zscale</I>
      <TT>}</TT>
      <DD><TT>spatial_offset {</TT>
        <I>xoff</I><TT>&nbsp;</TT><I>yoff</I><TT>&nbsp;</TT><I>zoff</I>
      <TT>}</TT>
      <DD><TT>exterior </TT><I>ext_flag</I>
      <DD><TT>norm </TT><I>magnitude</I>
      <DD><TT>multiplier </TT><I>mult</I>
      <DT><TT>}</TT></DL></BLOCKQUOTE><P>
      \end{rawhtml}
   Required values in the Specify block are the name of the input vector
   field file and the desired scaling parameters.  The filename is
   specified via the \oxslabel{file} entry, which names a file
   containing a vector field in one of the formats recognized by
   \hyperrefhtml{\app{avf2ovf}}{\app{avf2ovf}
   (Sec.~}{)}{sec:avf2ovf}\index{application!avf2ovf}.  If
   \oxslabel{atlas} or \oxslabel{xrange/yrange/zrange} are specified,
   then the file will be scaled and translated as necessary to fit that
   scaling region, in the same manner as done, for example, by
   the \ptlink{\cd{Oxs\_ScriptScalarField}}{PTSSF} and
   \ptlink{\cd{Oxs\_ScriptVectorField}}{PTSVF} classes.
   Alternatively, one may specify \oxslabel{spatial\_scaling} and
   \oxslabel{spatial\_offset} directly.  In this case the vector spatial
   positions are taken as specified in the file, multiplied
   component-wise by \cd{(xscale,yscale,zscale)}, and then translated by
   \cd{(xoff,yoff,zoff)}.  If you want to use the spatial coordinates as
   directly specified in the file, use \cd{(1,1,1)} for spatial\_scaling
   and \cd{(0,0,0)} for spatial\_offset.

   In all cases, once the input field has been scaled and translated, it
   is then sub-sampled (zeroth-order fit) as necessary to match the
   simulation mesh.

   The \oxslabel{exterior} flag determines the behavior at ``exterior
   points'', i.e., locations (if any) in the simulation mesh that lie
   outside the extent of the scaled and translated vector field.  The
   \oxsval{ext\_flag} should be either a three-vector, or one of the
   keywords \cd{boundary} or \cd{error}.  If a three-vector is given,
   then that value is supplied at all exterior points.  If
   \oxsval{ext\_flag} is set to \cd{boundary}, then the value used is
   the point on the boundary of the input vector field that is closest
   to the exterior point.  The default setting for \oxsval{ext\_flag} is
   \cd{error}, which raises an error if there are any exterior points.

   The magnitude of the field can be modified by the optional
   \oxslabel{norm} and \oxslabel{multiplier} attributes.  If the norm
   parameter is given, then each vector in the field will be
   renormalized to the specified magnitude.  If the multiplier parameter
   is given, then each vector in the field will be multiplied by the
   given scalar value.  If the multiplier value is negative, the field
   direction will be reversed.  If both \cd{norm} and \cd{multiplier}
   are given, then the field vectors are renormalized before being
   scaled by the multiplier value.

\begin{ExampleMifs}
  \fn{stdprob3.mif}, \fn{yoyo.mif}.
\end{ExampleMifs}

\pttarget{PTRVF}\index{Oxs\_Ext~child~classes!Oxs\_RandomVectorField}%
\item[Oxs\_RandomVectorField:]
Similar to
\htmlonlyref{\cd{Oxs\_RandomScalarField}}{item:RandomScalarField}%
\latex{ (q.v.)}, but defines a vector field rather than a scalar field that
varies spatially in a random fashion.  The Specify block has the form:
      \begin{latexonly}
      \begin{quote}\tt
      Specify Oxs\_RandomVectorField:\oxsval{name} \ocb\\
       \bi min\_norm \oxsval{minvalue}\\
       \bi max\_norm \oxsval{maxvalue}\\
       \bi cache\_grid \oxsval{mesh\_spec}\\
      \ccb
      \end{quote}
      \end{latexonly}
      \begin{rawhtml}
      <BLOCKQUOTE><DL><DT>
      <TT>Specify Oxs_RandomVectorField {</TT>
      <DD><TT> min_norm </TT><I>minvalue</I>
      <DD><TT> max_norm </TT><I>maxvalue</I>
      <DD><TT> cache_grid </TT><I>mesh_spec</I>
      <DT><TT>}</TT></DL></BLOCKQUOTE><P>
      \end{rawhtml}
The Specify block takes two required parameters, \oxslabel{min\_norm}
and \oxslabel{max\_norm}.  The vectors produced will have magnitude
between these two specified values.  If \cd{min\_norm} = \cd{max\_norm},
then the samples are uniformly distributed on the sphere of that radius.
Otherwise, the samples are uniformly distributed in the hollow spherical
volume with inner radius \cd{min\_norm} and outer radius \cd{max\_norm}.
There is also an optional parameter, \oxslabel{cache\_grid}, which takes
a mesh specification that describes the grid used for cache spatial
discretization.  If \oxslabel{cache\_grid} is not specified, then each
call to \cd{Oxs\_RandomVectorField} generates a different field.  If you
want to use the same random vector field in two places (as a base for
setting, say anisotropy axes and initial magnetization), then specify
\oxslabel{cache\_grid} with the appropriate (usually the base problem)
mesh.

\begin{ExampleMifs}
  \fn{diskarray.mif}, \fn{sample2.mif}, \fn{randomshape.mif}
  \fn{stdprob1.mif}.
\end{ExampleMifs}

\pttarget{PTPRVF}\index{Oxs\_Ext~child~classes!Oxs\_PlaneRandomVectorField}%
\item[Oxs\_PlaneRandomVectorField:]
   Similar to \cd{Oxs\_RandomVectorField}, except that samples are
   drawn from 2D planes rather than 3-space.  The Specify block has the
   form
      \begin{latexonly}
      \begin{quote}\tt
      Specify Oxs\_RandomVectorField:\oxsval{name} \ocb\\
       \bi plane\_normal \oxsval{vector\_field\_spec}\\
       \bi min\_norm \oxsval{minvalue}\\
       \bi max\_norm \oxsval{maxvalue}\\
       \bi cache\_grid \oxsval{mesh\_spec}\\
      \ccb
      \end{quote}
      \end{latexonly}
      \begin{rawhtml}
      <BLOCKQUOTE><DL><DT>
      <TT>Specify Oxs_RandomVectorField {</TT>
      <DD><TT> plane_normal </TT><I>vector_field_spec</I>
      <DD><TT> min_norm </TT><I>minvalue</I>
      <DD><TT> max_norm </TT><I>maxvalue</I>
      <DD><TT> cache_grid </TT><I>mesh_spec</I>
      <DT><TT>}</TT></DL></BLOCKQUOTE><P>
      \end{rawhtml}
  The \oxslabel{min\_norm}, \oxslabel{max\_norm}, and
  \oxslabel{cache\_grid} parameters have the same meaning as for the
  \cd{Oxs\_RandomVectorField} class.  The additional parameter,
  \oxslabel{plane\_normal}, specifies a vector field that at each point
  provides a vector that is orthogonal to the plane from which the
  random vector at that point is to be drawn.  If the vector field is
  specified explicitly as three real values, then a spatially uniform
  vector field is produced and all the random vectors will lie in the
  same plane.  More generally, however, the normal vectors (and
  associated planes) may vary from point to point.  As a special case,
  if a normal vector at a point is the zero vector, then no planar
  restriction is made and the resulting random vector is drawn uniformly
  from a hollow ball in three space satisfying the minimum/maximum norm
  constraints.

\begin{ExampleMifs}[Example]
  \fn{sample2.mif}.
\end{ExampleMifs}

\pttarget{PTSOVF}\index{Oxs\_Ext~child~classes!Oxs\_ScriptOrientVectorField}%
\item[Oxs\_ScriptOrientVectorField:\label{item:ScriptOrientVectorField}]
This class is analogous to the
\htmlonlyref{\cd{Oxs\_ScriptOrientScalarField}}{item:ScriptOrientScalarField}
class\latex{ (page~\pageref{item:ScriptOrientScalarField})}.
The Specify block has the form:
\begin{latexonly}
\begin{quote}\tt
Specify Oxs\_ScriptOrientVectorField:\oxsval{name} \ocb\\
\bi field \oxsval{vector\_field\_spec}\\
\bi script \oxsval{\Tcl\_script}\\
\bi script\_args \ocb\oxsval{ args\_request }\ccb\\
\bi atlas \oxsval{atlas\_spec}\\
\bi xrange \ocb\oxsval{ xmin xmax }\ccb\\
\bi yrange \ocb\oxsval{ ymin ymax }\ccb\\
\bi zrange \ocb\oxsval{ zmin zmax }\ccb\\
\ccb
\end{quote}
\end{latexonly}
\begin{rawhtml}
<BLOCKQUOTE><DL><DT>
<TT>Specify Oxs_ScriptOrientVectorField:</TT><I>name</I> <TT>{</TT>
<DD><TT>field </TT> <I>vector_field_spec</I>
<DD><TT>script </TT> <I>Tcl_script</I>
<DD><TT>script_args {</TT> <I>args_request</I> <TT>}</TT>
<DD><TT>atlas </TT> <I>atlas_spec</I>
<DD><TT>xrange {</TT> <I>xmin</I><TT>&nbsp;</TT><I>xmax</I> <TT>}</TT>
<DD><TT>yrange {</TT> <I>ymin</I><TT>&nbsp;</TT><I>ymax</I> <TT>}</TT>
<DD><TT>zrange {</TT> <I>zmin</I><TT>&nbsp;</TT><I>zmax</I> <TT>}</TT>
<DT><TT>}</TT></DL></BLOCKQUOTE><P>
\end{rawhtml}
The interpretation of the specify block and the operation of the \Tcl\
script is exactly the same as for the \cd{Oxs\_ScriptOrientScalarField}
class, except the input \oxslabel{field} and the resulting field are
vector fields instead of scalar fields.

Note that the ``orientation'' transformation is applied to the import
spatial coordinates only, not the output vector.  For example, if the
\cd{field} value represents a shaped vector field, and the \cd{script}
proc is a rotation transformation, then the resulting vector field shape
will be rotated as compared to the original vector field, but the output
vectors themselves will still point in their original directions.  In such
cases one may wish to compose the \cd{Oxs\_ScriptOrientVectorField} with
a \htmlonlyref{\cd{Oxs\_ScriptVectorField}}{item:ScriptVectorField}
object\latex{ (page~\pageref{item:ScriptVectorField})} to rotate the
output vectors as well.  This situation occurs also with the
\cd{Oxs\_AffineOrientVectorField} class.  See the
\htmlonlyref{\cd{Oxs\_AffineTransformVectorField}}{item:AffineTransformVectorField}
class documentation\latex{ (page~\pageref{item:AffineTransformVectorField})} for an
example illustrating the composition of an object of that class with a
\htmlonlyref{\cd{Oxs\_AffineOrientVectorField}}{item:AffineOrientVectorField}
object.

\begin{ExampleMifs}[Example]
  \fn{sample-vecreflect.mif}.
\end{ExampleMifs}

\pttarget{PTAOVF}\index{Oxs\_Ext~child~classes!Oxs\_AffineOrientVectorField}%
\item[Oxs\_AffineOrientVectorField:\label{item:AffineOrientVectorField}]
This class is analogous to the
\htmlonlyref{\cd{Oxs\_AffineOrientScalarField}}{item:AffineOrientScalarField}
class\latex{ (page~\pageref{item:AffineOrientScalarField})}.
The Specify block has the form:
\begin{latexonly}
\begin{quote}\tt
Specify Oxs\_AffineOrientVectorField:\oxsval{name} \ocb\\
 \bi field \oxsval{vector\_field\_spec}\\
 \bi M \ocb\oxsval{ matrix\_entries \ldots }\ccb\\
 \bi offset \ocb\oxsval{ off${}_x$ off${}_y$ off${}_z$ }\ccb\\
 \bi inverse \oxsval{invert\_flag}\\
 \bi inverse\_slack \oxsval{slack}\\
\ccb
\end{quote}
\end{latexonly}
\begin{rawhtml}
<BLOCKQUOTE><DL><DT>
<TT>Specify Oxs_AffineOrientVectorField {</TT>
<DD><TT> field </TT><I>vector_field_spec</I>
<DD><TT> M
  {</TT> <I>matrix_entries</I><TT>&nbsp;</TT><I>...</I> <TT>}</TT>
<DD><TT> offset {</TT>
   <I>off<sub>x</sub></I><TT>&nbsp;</TT>
   <I>off<sub>y</sub></I><TT>&nbsp;</TT>
   <I>off<sub>z</sub></I> <TT>}</TT>
<DD><TT> inverse </TT><I>invert_flag</I>
<DD><TT> inverse_slack </TT><I>slack</I>
<DT><TT>}</TT></DL></BLOCKQUOTE><P>
\end{rawhtml}
The interpretation of the specify block and the affine transformation
is exactly the same as for the \cd{Oxs\_AffineOrientScalarField}
class, except the input \oxslabel{field} and the resulting field are
vector fields instead of scalar fields.

As explained in the
\htmlonlyref{\cd{Oxs\_ScriptOrientVectorField}}{item:ScriptOrientVectorField}
documentation, the ``orientation'' transformation is applied to the
import spatial coordinates only, not the output vector.  If one wishes
to rotate the output vectors, then a
\htmlonlyref{\cd{Oxs\_AffineTransformVectorField}}{item:AffineTransformVectorField}
object may be applied with the opposite rotation.  See that section for
an example.

\begin{ExampleMifs}
  \fn{yoyo.mif}, \fn{sample-vecrotate.mif}.
\end{ExampleMifs}

\pttarget{PTATVF}\index{Oxs\_Ext~child~classes!Oxs\_AffineTransformVectorField}%
\item[Oxs\_AffineTransformVectorField:\label{item:AffineTransformVectorField}]
This class applies an affine transform to the output of a vector field.
It is similar to the
\htmlonlyref{\cd{Oxs\_AffineTransformScalarField}}{item:AffineTransformScalarField}
class\latex{ (page~\pageref{item:AffineTransformScalarField})}, except
that in this case the affine transform is applied to a vector instead of
a scalar.  The Specify block has the form:
\begin{latexonly}
\begin{quote}\tt
Specify Oxs\_AffineTransformVectorField:\oxsval{name} \ocb\\
 \bi field \oxsval{vector\_field\_spec}\\
 \bi M \ocb\oxsval{ matrix\_entries \ldots }\ccb\\
 \bi offset \ocb\oxsval{ off${}_x$ off${}_y$ off${}_z$ }\ccb\\
 \bi inverse \oxsval{invert\_flag}\\
 \bi inverse\_slack \oxsval{slack}\\
\ccb
\end{quote}
\end{latexonly}
\begin{rawhtml}
<BLOCKQUOTE><DL><DT>
<TT>Specify Oxs_AffineTransformVectorField {</TT>
<DD><TT> field </TT><I>vector_field_spec</I>
<DD><TT> M
  {</TT> <I>matrix_entries</I><TT>&nbsp;</TT><I>...</I> <TT>}</TT>
<DD><TT> offset {</TT>
   <I>off<sub>x</sub></I><TT>&nbsp;</TT>
   <I>off<sub>y</sub></I><TT>&nbsp;</TT>
   <I>off<sub>z</sub></I> <TT>}</TT>
<DD><TT> inverse </TT><I>invert_flag</I>
<DD><TT> inverse_slack </TT><I>slack</I>
<DT><TT>}</TT></DL></BLOCKQUOTE><P>
\end{rawhtml}
Because the output from \oxslabel{field} is a 3-vector, the transform
defined by \oxslabel{M} and \oxslabel{offset} requires \cd{M} to be a
\latex{$3\times 3$}\html{3x3} matrix and \cd{offset} to be a 3-vector.
Thus, if $\mbox{\boldmath$v$}(\mbox{\boldmath$x$})$ represents the
vector field specified by the \oxslabel{field} value, then the resulting
vector field is
$M.\mbox{\boldmath$v$}(\mbox{\boldmath$x$})+\textbf{off}$.

\cd{M} is described by a list of from one to nine entries, in exactly
the same manner as for the
\htmlonlyref{\cd{Oxs\_AffineOrientVectorField}}{item:AffineOrientVectorField}
and
\htmlonlyref{\cd{Oxs\_AffineOrientScalarField}}{item:AffineOrientScalarField}
classes\latex{ (page~\pageref{item:AffineTransformScalarField})}.  The
interpretation of \oxslabel{offset}, \oxslabel{inverse}, and
\oxslabel{inverse\_slack} is also the same.  In particular, if
\oxsval{invert\_flag} is 1, then the resulting vector field is
$M^{-1}.\left(\mbox{\boldmath$v$}(\mbox{\boldmath$x$})-\textbf{off}\right)$.

The following example illustrates combining a
\cd{Oxs\_AffineTransformVectorField} with a
\htmlonlyref{\cd{Oxs\_AffineOrientVectorField}}{item:AffineOrientVectorField}
to completely rotate a vector field.
% The extra BLOCKQUOTE's here are a workaround for an apparent
% latex2html bug
\begin{rawhtml}
<BLOCKQUOTE>
\end{rawhtml}
%begin<latexonly>
\begin{quote}
%end<latexonly>
\begin{verbatim}
Specify Oxs_BoxAtlas:atlas {
  xrange {-80e-9 80e-9}
  yrange {-80e-9 80e-9}
  zrange {0  40e-9}
}

proc Trap { x y z } {
     if {$y<=$x && $y<=0.5} {return [list 0 1 0]}
     return [list 0 0 0]
}

Specify Oxs_ScriptVectorField:trap {
   script Trap
   atlas :atlas
}

Specify Oxs_AffineOrientVectorField:orient {
   field :trap
   M { 0 -1 0
       1  0 0
       0  0 1 }
   offset { -20e-9 0 0 }
   inverse 1
}

Specify Oxs_AffineTransformVectorField:rot {
   field :orient
   M { 0 -1 0
       1  0 0
       0  0 1 }
}

proc Threshold { vx vy vz } {
   set magsq [expr {$vx*$vx+$vy*$vy+$vz*$vz}]
   if {$magsq>0} {return 8e5}
   return 0.0
}

Specify Oxs_ScriptScalarField:Ms {
  vector_fields :rot
  script Threshold
  script_args vectors
}

Specify Oxs_TimeDriver {
 m0 :rot
 Ms :Ms
 stopping_dm_dt 0.01
 evolver :evolve
 mesh :mesh
}
\end{verbatim}
%begin<latexonly>
\end{quote}
%end<latexonly>
\begin{rawhtml}
</BLOCKQUOTE>
\end{rawhtml}
The base field here is given by the \cd{Oxs\_ScriptVectorField:trap}
object, which produces a vector field having a trapezoidal shape with
the non-zero vectors pointing parallel to the $y$-axis.  The
\cd{:orient} and \cd{:rot} transformations rotate the shape and the
vectors counterclockwise \latex{$90^{\circ}$}\html{90 degrees}.
Additionally, the \cd{offset} option in \cd{:orient} translates the
shape 20~nm towards the left.  The original and transformed fields are
illustrated below.
\latex{
%  Usage: \includeimage{width}{height}{basename}{altstring}
\parbox{0.45\textwidth}{
\centerline{\includeimage{0.4\textwidth}{!}{trap-orig}{Original field}}
\centerline{Original field}
}
\parbox{0.45\textwidth}{
\centerline{\includeimage{0.4\textwidth}{!}{trap-rot}{Rotated field}}
\centerline{Rotated field}
}}%
\html{
% Latex2html doesn't handle above construction properly.
% So, instead just use figures with captions already
% pasted in.
\includeimage{1pt}{1pt}{trap-orig}{Original field}
\includeimage{1pt}{1pt}{trap-rot}{Rotate field}
}

\begin{ExampleMifs}[Example]
  \fn{sample-vecrotate.mif}.
\end{ExampleMifs}

\pttarget{PTMVF}\index{Oxs\_Ext~child~classes!Oxs\_MaskVectorField}%
\item[Oxs\_MaskVectorField:\label{item:MaskVectorField}]
Multiplies a vector field pointwise by a scalar vector field (the mask)
to produce a new vector field.  The Specify block has the form:
\begin{latexonly}
\begin{quote}\tt
Specify Oxs\_MaskVectorField:\oxsval{name} \ocb\\
 \bi mask  \oxsval{scalar\_field\_spec}\\
 \bi field \oxsval{vector\_field\_spec}\\
\ccb
\end{quote}
\end{latexonly}
\begin{rawhtml}
<BLOCKQUOTE><DL><DT>
<TT>Specify Oxs_MaskVectorField {</TT>
<DD><TT> mask </TT><I>scalar_field_spec</I>
<DD><TT> field </TT><I>vector_field_spec</I>
<DT><TT>}</TT></DL></BLOCKQUOTE><P>
\end{rawhtml}
This functionality can be achieved, if in a somewhat more complicated
fashion, with the
\htmlonlyref{\cd{Oxs\_ScriptVectorField}}{item:ScriptVectorField}
class.  For example, given a scalar field \cd{:mask} and a vector field
\cd{:vfield}, this example using the \cd{Oxs\_MaskVectorField} class
% The extra BLOCKQUOTE's here are a workaround for an apparent
% latex2html bug
\begin{rawhtml}
<BLOCKQUOTE>
\end{rawhtml}
%begin<latexonly>
\begin{quote}
%end<latexonly>
\begin{verbatim}
Specify Oxs_MaskVectorField {
   mask :mask
   field :vfield
}
\end{verbatim}
%begin<latexonly>
\end{quote}
%end<latexonly>
\begin{rawhtml}
</BLOCKQUOTE>
\end{rawhtml}
is equivalent to this example using the \cd{Oxs\_ScriptVectorField}
class
% The extra BLOCKQUOTE's here are a workaround for an apparent
% latex2html bug
\begin{rawhtml}
<BLOCKQUOTE>
\end{rawhtml}
%begin<latexonly>
\begin{quote}
%end<latexonly>
\begin{verbatim}
proc MaskField { m vx vy vz } {
   return [list [expr {$m*$vx}] [expr {$m*$vy}] [expr {$m*$vz}]]
}

Specify Oxs_ScriptVectorField {
  script MaskField
  script_args {scalars vectors}
  scalar_fields { :mask }
  vector_fields { :vfield }
}
\end{verbatim}
%begin<latexonly>
\end{quote}
%end<latexonly>
\begin{rawhtml}
</BLOCKQUOTE>
\end{rawhtml}
Of course, the \cd{Oxs\_ScriptVectorField} approach is easily
generalized to much more complicated and arbitrary combinations of
scalar and vector fields.

\begin{ExampleMifs}[Example]
  \fn{rotatecenterstage.mif}.
\end{ExampleMifs}

\pttarget{PTIVF}\index{Oxs\_Ext~child~classes!Oxs\_ImageVectorField}%
\item[Oxs\_ImageVectorField:\label{item:ImageVectorField}]
This class creates a vector field using an image.  The Specify block has
the form
\begin{latexonly}
\begin{quote}\tt
Specify Oxs\_ImageVectorField:\oxsval{name} \ocb\\
 \bi image \oxsval{pic}\\
 \bi multiplier \oxsval{mult}\\
 \bi vx\_multiplier \oxsval{xmult}\\
 \bi vy\_multiplier \oxsval{ymult}\\
 \bi vz\_multiplier \oxsval{zmult}\\
 \bi vx\_offset \oxsval{xoff}\\
 \bi vy\_offset \oxsval{yoff}\\
 \bi vz\_offset \oxsval{zoff}\\
 \bi norm \oxsval{norm\_magnitude}\\
 \bi viewplane \oxsval{view}\\
 \bi atlas \oxsval{atlas\_spec}\\
 \bi xrange \ocb\oxsval{ xmin xmax }\ccb\\
 \bi yrange \ocb\oxsval{ ymin ymax }\ccb\\
 \bi zrange \ocb\oxsval{ zmin zmax }\ccb\\
 \bi exterior \oxsval{ext\_flag}\\
\ccb
\end{quote}
\end{latexonly}
\begin{rawhtml}
<BLOCKQUOTE><DL><DT>
<TT>Specify Oxs_ImageVectorField:</TT><I>name</I> <TT>{</TT>
<DD><TT>image </TT> <I>pic</I>
<DD><TT>multiplier </TT><I>mult</I>
<DD><TT>vx_multiplier </TT><I>xmult</I>
<DD><TT>vy_multiplier </TT><I>ymult</I>
<DD><TT>vz_multiplier </TT><I>zmult</I>
<DD><TT>vx_offset </TT><I>xoff</I>
<DD><TT>vy_offset </TT><I>yoff</I>
<DD><TT>vz_offset </TT><I>zoff</I>
<DD><TT>norm </TT> <I>norm_magnitude</I>
<DD><TT>viewplane </TT> <I>view</I>
<DD><TT>atlas </TT> <I>atlas_spec</I>
<DD><TT>xrange {</TT> <I>xmin</I><TT>&nbsp;</TT><I>xmax</I> <TT>}</TT>
<DD><TT>yrange {</TT> <I>ymin</I><TT>&nbsp;</TT><I>ymax</I> <TT>}</TT>
<DD><TT>zrange {</TT> <I>zmin</I><TT>&nbsp;</TT><I>zmax</I> <TT>}</TT>
<DD><TT>exterior </TT><I>ext_flag</I>
<DT><TT>}</TT></DL></BLOCKQUOTE><P>
\end{rawhtml}
The \oxslabel{image} is interpreted as a three-color map, yielding a
vector field where each (x,y,z) component is determined by the red,
green, and blue color components, respectively.\ldots

The \oxslabel{viewplane}, \oxslabel{atlas},
\oxslabel{xrange/yrange/zrange}, and \oxslabel{exterior} are treated
the same as for the
\htmlonlyref{\cd{Oxs\_ImageScalarField} class}{item:ImageScalarField}%
\latex{ (q.v.)}

\begin{ExampleMifs}
  \fn{NONE}.
\end{ExampleMifs}

\end{description}

\subsection{\MIF\ Support Classes}\label{oxsMIF}
\pttarget{PTLV}\begin{description}
\index{Oxs\_Ext~child~classes!Oxs\_LabelValue}%
\item[Oxs\_LabelValue:]
   A convenience object that holds label + value
   pairs.  \cd{Oxs\_LabelValue} objects may be referenced via the
   standard \cd{attributes} field in other Specify blocks, as in
   this example:
\begin{rawhtml}
<BLOCKQUOTE>
\end{rawhtml}
%begin{latexonly}
\begin{quote}
%end{latexonly}
\begin{verbatim}
Specify Oxs_LabelValue:probdata {
  alpha 0.5
  start_dm 0.01
}

Specify Oxs_EulerEvolve {
  attributes :probdata
}
\end{verbatim}
%begin{latexonly}
\end{quote}
%end{latexonly}
\begin{rawhtml}
</BLOCKQUOTE>
\end{rawhtml}
   The Specify block string for \cd{Oxs\_LabelValue} objects is an
   arbitrary \Tcl\ list with an even number of elements.  The first
   element in each pair is interpreted as a label, the second as the
   value.  The \cd{attribute} option causes this list to be dropped
   verbatim into the surrounding object.  This technique is most useful
   if the label + value pairs in the \cd{Oxs\_LabelValue} object are
   used in multiple Specify blocks, either inside the same \MIF\
   file, or across several \MIF\ files into which the
   \cd{Oxs\_LabelValue} block is imported using the \cd{ReadFile} \MIF\
   extension command.

   \begin{ExampleMifs}
     The \MIF\ files \fn{sample-rotate.mif} and
     \fn{sample-reflect.mif} use the \cd{Oxs\_LabelValue} object
     stored in the \fn{sample-attributes.tcl} file.
   \end{ExampleMifs}

\end{description}

%begin{latexonly}
Refer to Sec.~\ref{sec:mif2format} for details on the base \MIF~2
format specification.
%end{latexonly}
\begin{htmlonly}
Refer to the \MIF~2 documentation for details on the base format
specification.
\end{htmlonly}

\section{Contributed Oxs\_Ext Child Classes}\label{sec:contriboxsext}%
\index{Oxs\_Ext~child~classes!contributed}
The \cd{Oxs\_Ext} classes discussed in the \htmlonlyref{previous section}{sec:oxsext}
were written and are maintained by the \OOMMF\ core development
team. Additional \cd{Oxs\_Ext} classes are available from third-parties
or may be written by the end user (see
the \htmladdnormallinkfoot{\textit{\OOMMF\ Programming Manual}}{https://math.nist.gov/oommf/doc/}). Available extensions known to the \OOMMF\
developers are listed on the
\htmladdnormallinkfoot{Oxs Extensions Modules}{https://math.nist.gov/oommf/contrib/oxsext/}
page. These extensions are included in \OOMMF\ distributions, current as
of the time of release.

Contributed extensions can be found under \fn{oommf/app/oxs/contrib/} in
individual directories. These are ``installed'' by (1) using
the \NONHTMLoutput{\app{oxspkg install}
command}\hyperrefhtml{\app{oxspkg install}}{
(Sec.~}{)}{sec:oxspkg}\HTMLoutput{ command} to copy the source code from
the distribution area into \fn{oommf/app/oxs/local/}, and then (2)
running \hyperrefhtml{\app{pimake}}{\app{pimake} (Sec.~}{)}{sec:pimake}
to compile and link the extension into the Oxs executable.

All of the extensions distributed in an \OOMMF\ release that do not
require libraries beyond those needed to run \OOMMF\ are distributed in
the installed state. It is up to the end user to install additional
libraries in the needed if they wish to activate other extensions. See
the \cd{requires} subcommand in
the \htmlonlyref{\app{oxspkg}}{sec:oxspkg} documentation for details.

Note that extensions are distributed as source code; at this time there
are no provisions for binary distribution of \cd{Oxs\_Ext} modules. This
means that to add extensions to your \OOMMF\ installation you need to
have a supported
\hyperrefhtml{\Cplusplus\ compiler}{\Cplusplus\ compiler installed
(Sec.~}{)}{sec:install.requirements}\HTMLoutput{ installed}, and you
must run \htmlonlyref{\app{pimake}}{sec:pimake} to build the
extensions. If you are running \OOMMF\ on \Windows\ with pre-built
binaries, then you should first \cd{cd} to the \OOMMF\ root directory and
run
\begin{verbatim}
$ tclsh oommf.tcl pimake distclean
\end{verbatim}
to completely delete all pre-built binary files. Afterwards run
\begin{verbatim}
$ tclsh oommf.tcl pimake
\end{verbatim}
to create a fresh build of \OOMMF. This ensures that the new extension
binaries are compatible with the rest of the \OOMMF\ infrastructure.

Third party extensions are not maintained or documented by the \OOMMF\
development team. Such extensions are provided ``as-is.'' Some extension
authors may provide limited support for their extensions, but be aware
that extensions are generally provided as a public service to the
community and support is not compulsory.

% There is a strange bug in LaTeX2HTML where using multiple instances of
% the alltt environment causes LaTeX2HTML to stop interpreting a single
% blank line as a paragraph break. (Two blank lines seem to work, but
% maybe not in some cases, and anyway, yuck.) The bug appears to be
% related in some manner with \include, so one hack-around is to \input
% the file instead. A web search reports that \include differs from
% \input in that \include places \clearpage command before and after the
% included text, and also opens up a separate .aux file. I don't want to
% pander to LaTeX2HTML more than necessary (LaTeXML seems to work pretty
% well, so if the annoyance level of LaTeX2HTML gets too great we'll
% drop it), so we apply this hack only in the LaTeX2HTML processing
% case. There don't appear to be any obvious problems with this route,
% but if something turns up then checking .aux file usage might be a
% good place to start.
%begin{latexonly}
\chapter{Debugging \OOMMF}\label{sec:debug}
\newlength{\sswidth} % Width for screen shot figures
\setlength{\sswidth}{\textwidth}
\addtolength{\sswidth}{-1em}
This chapter provides an introduction to debugging \OOMMF\ and
\OOMMF\ extension source code, providing background to the \OOMMF\ build
architecture and detailing some tools and techniques for uncovering
programming errors. It begins with a look at the \OOMMF\ \app{pimake}
application used for compiling and linking \OOMMF\ programs, followed by
some considerations involving the \app{oommf.tcl} bootstrap
wrapper. Then configuration files governing build and runtime behavior
are detailed.  After this methods for identifying and locating runtime
errors are presented, including a brief introduction on using debugger
applications with \OOMMF. Although the primary focus of this chapter is
on errors in \C++\ code, the interface and glue code linking the various
\OOMMF\ applications rely on \Tcl\ script.  An example of working with
\Tcl\ in \OOMMF\ is provided in Fig.~\ref{fig:oommftclintrospection}

Throughout this chapter, unless otherwise stated, commands are
implicitly assumed to be run from the \OOMMF\ root directory (i.e. the
directory containing the file \fn{oommf.tcl}), and directory paths are
taken relative to this directory (e.g., \fn{app/oxs/} refers
to the directory \fn{<oommf\_root>/app/oxs/}).

In text blocks containing command statements and program output,
command statements are indicated with a leading character
representing the shell command prompt. On \Windows\ this character is
typically ``\verb+>+'', whereas the \Unix\ and \MacOSX\ shells more commonly
use ``\verb+$+'' with \cd{bash} shells or ``\verb+%+'' with
\cd{zsh}. All three are used below, but ``\verb+%+'' is limited to
\MacOSX\ specific examples to minimize confusion with the \Tcl\ command
prompt, which is also ``\verb+%+''. For additional visibility shell
commands are colored \shellcmd{\shellcmdcolorname}\ and program commands
(\Tcl\ and debugger) are colored \pgmcmd{\pgmcmdcolorname}. (Computer
responses remain in black text.)

Some details in what follows may vary depending on the particular
operating system and application version, but hopefully the differences
are sufficiently small that this description remains a useful guide.

\section{Configuration Files}\label{sec:debug:configfiles}
There are several \OOMMF\ configuration settings that impact debug
operations. The controlling files are \fn{config/options.tcl} and
\fn{config/platforms/<platform>.tcl}, where the \texttt{<platform>} is
\texttt{windows-x86\_64}, \texttt{linux-x86\_64}, or \texttt{darwin} for
\Windows, \Linux, or \MacOSX\ operating systems respectively. In practice,
rather than modifying the default distribution files directly, you should
place your modifications in local files
\fn{config/local/options.tcl} and
\fn{config/platforms/local/<platform>.tcl}.
The \fn{local/} directories and files are not part of the
\OOMMF\ distribution; you will need to create them manually. The files
can be empty initially, and then populated as desired.

The \fn{options.tcl} file contains platform-agnostic settings that are
stored in the \cd{Oc\_Option} database. Some of these settings affect
the build process, while others control post-build runtime behavior.
All are set using the \cd{Oc\_Option} command, which takes
name\,+\,value pairs.  The \cd{cflags} and \cd{optlevel} settings
control compiler options. The default setting for \cd{cflags} is
\begin{verbatim}
Oc_Option Add * Platform cflags {-def NDEBUG}
\end{verbatim}
which causes the C macro ``\texttt{NDEBUG}'' to be defined. If this is
not set then various run-time checks such as \cd{assert} statements and
some array index checks are activated. These checks slow execution but
may be helpful in diagnosing errors. Other \cd{cflag} options include
\cd{-warn}, which enables compiler warning messages, and \cd{-debug},
which tells the compiler to generate debugging symbols. A good
\cd{cflags} setting for debugging is
\begin{verbatim}
Oc_Option Add * Platform cflags {-warn 1 -debug 1}
\end{verbatim}
There is also an \cd{lflags} option, similar to \cd{cflags}, that
controls options to the linker. The default is an empty string (no
options), and you generally don't need to change this.

The \cd{optlevel} option sets the compiler optimization level, with an
integer value between 0 and 3. The default value is 2, which selects for
a high but reliable level of optimizations. Some optimizations may
reorder and combine source code statements, making it harder to debug
code, so you may want to use
\begin{verbatim}
Oc_Option Add * Platform optlevel 0
\end{verbatim}
to disable all optimizations.

The \fn{config/platforms/<platform>.tcl} files set default platform and
compiler specific options. For example,
\fn{config/platforms/windows-x86\_64.tcl} is the base platform file for
64-bit \Windows. There are separate sections inside this file for the
various supported compilers. You can make local changes to the default
settings by creating a subdirectory of \fn{config/platforms/} named
\cd{local/}, and creating there an initially empty file with the
same name as the base platform file. Inside the base platform file is a
code block labeled \cd{LOCAL CONFIGURATION}, which lists all the
available local modifications. You can copy some or all of this
\Tcl\ code block to your new \cd{config/platforms/local/} file, and then
uncomment and modify options as desired. For example, if you are using
the Visual \Cplusplus\ compiler on \Windows, you may want to include the
\cd{/RTCus} compiler flag to enable some run-time error checks. You can
do that with these lines in your
\fn{local/windows-x86\_64.tcl} file:
\begin{verbatim}
$config SetValue program_compiler_c++_remove_flags {/O2}
$config SetValue program_compiler_c++_remove_valuesafeflags {/O2}
$config SetValue program_compiler_c++_add_flags {/RTCus}
$config SetValue program_compiler_c++_add_valuesafeflags {/RTCus}
\end{verbatim}
The \cd{*\_valuesafeflags} options are for code with sensitive
floating-point operations that must be evaluated exactly as
specified. This pertains primarily to the double-double routines in
\fn{pkg/xp/}. The \cd{*\_flags} options are for everything else. The
\cd{*\_remove\_*} controls remove options from the default compile
command. This can be a (\Tcl) list, with each element matching as a
regular expression. (Refer to the
\htmladdnormallinkfoot{\Tcl\ documentation}{https://www.tcl-lang.org/man/}
on the \cd{regexp} command for details.) The \cd{*\_add\_*} controls
append options. \OOMMF\ sets \cd{/O2} optimization by default, but
\cd{/O2} is incompatible with \cd{/RTCus}, so in this example \cd{/O2}
is removed to allow \cd{/RTCus} to be added. (Setting \cd{optlevel 0} in
the \fn{config/local/options.tcl} file, as explained above, replaces
\cd{/O2} with \cd{/Od}. So strictly speaking it is not necessary to
remove \cd{/O2} in that case, but it doesn't hurt either.)

You can run the command ``\cd{oommf.tcl +platform +v}'' to see the
effects of your current \fn{options.tcl} and \fn{<platform>.tcl}
settings. For example,
\begin{alltt}
$ \shellcmd{tclsh oommf.tcl +platform +v}
[...]
--- Local config options ---
[...]
   Oc_Option Add * Platform cflags {-debug 1 -warn 1}
   Oc_Option Add * Platform optlevel 0
[...]
--- Local platform options ---
   $config SetValue program_compiler_c++_remove_flags {/O2}
   $config SetValue program_compiler_c++_remove_valuesafeflags {/O2}
   $config SetValue program_compiler_c++_add_flags {/RTCus}
   $config SetValue program_compiler_c++_add_valuesafeflags {/RTCus}

--- Compiler options ---
     Standard options: /Od /D_CRT_SECURE_NO_DEPRECATE /RTCus
   Value-safe options: /Od /fp:precise /D_CRT_SECURE_NO_DEPRECATE /RTCus
\end{alltt}

To see the exact, full platform-specific compile and link commands, you
can delete and rebuild individual executables in the
\OOMMF\ package. Two examples, one using the standard build options
(\fn{pkg/oc/<platform>/varinfo}) and one using the value-safe options
(\fn{pkg/xp/<platform>/build\_port}) are presented below. (The response
lines have been edited for clarity.)
\begin{alltt}
% \shellcmd{cd pkg/oc}
% \shellcmd{tclsh ../../oommf.tcl pimake clean}
% \shellcmd{tclsh ../../oommf.tcl pimake darwin/varinfo}
clang++ -c -DNDEBUG -m64 -std=c++11 -Ofast -o darwin/varinfo.o varinfo.cc
clang++ -m64 darwin/varinfo.o -o darwin/varinfo

% \shellcmd{cd ../..}
% \shellcmd{cd pkg/xp}
% \shellcmd{tclsh ../../oommf.tcl pimake clean}
% \shellcmd{tclsh ../../oommf.tcl pimake darwin/build_port}
clang++ -c -DNDEBUG -m64 -std=c++11 -O3 -DXP_USE_MPFR=0
   -o darwin/build_port.o build_port.cc
clang++ -m64 darwin/build_port.o -o darwin/build_port
\end{alltt}\html{\newline}
The above is for \MacOSX. Adjust the \cd{<platform>} field as appropriate,
and on \Windows\ append \fn{.exe} to the executable targets (\fn{varinfo}
and \fn{build\_port}).

You can also use this method to manually compile and/or link individual
files: (1) Change to the relevant build directory (always one level below
either \cd{pkg} or \cd{app}), (2) delete the file you want to rebuild from
the \cd{<platform>} directory, (3) run \cd{pimake} as above to build the
file, (4) copy and paste the compile/link command to the shell prompt,
edit as desired, and rerun.

The \fn{varinfo} and \fn{build\_port} executables are
used to construct the platform-specific header files
\fn{pkg/oc/<platform>/ocport.h} and
\fn{pkg/xp/<platform>/xpport.h}. These files contain
\Cplusplus\ macro definitions, typedefs, and function wrappers,
and are an important adjunct when reading the \OOMMF\ source code.

\pttarget{PTtclintrospection}
For in-depth investigations \Tcl\ can be used to directly query and debug
\OOMMF\ initialization scripts. Start a \Tcl\ shell, and from inside the
shell append the \OOMMF\ \fn{pkg/oc} directory to the \Tcl\ global
\cd{auto\_path} variable. Next run \cd{package require Oc} to load the
\Tcl-only portion of the \OOMMF\ \cd{Oc} library into the shell. Then
you can check any and all \cd{Oc\_Option} values from
\fn{config/options.tcl}, platform configuration settings from
\fn{config/platforms/<platform>.tcl}, and perform various other types of
introspection from the \Tcl\ shell. See
Fig.~\ref{fig:oommftclintrospection} for a sample session.

\begin{codelisting}{f}{fig:oommftclintrospection}{Sample \Tcl-level
    \OOMMF\ introspection session. Shell commands are colored
    \shellcmd{\shellcmdcolorname}\ (with \texttt{\$} prompt) and
    \Tcl\ commands are colored \pgmcmd{\pgmcmdcolorname}\ (with
    \texttt{\%} prompt).}{PTtclintrospection}{hyperlink}
\begin{alltt}
$ \shellcmd{pwd}
/Users/barney/oommf
$ \shellcmd{tclsh}
% \pgmcmd{set env(OOMMF_BUILD_ENVIRONMENT_NEEDED) 1}
% \pgmcmd{lappend auto_path [file join [pwd] pkg oc]}
% \pgmcmd{package require Oc}

% # Miscellaneous utilities from Oc_Main (oommf/pkg/oc/main.tcl)
% \pgmcmd{Oc_Main GetOOMMFRootDir}    ;# OOMMF root directory
/Users/barney/oommf
% \pgmcmd{Oc_Main GetPid}             ;# Process id
17423

% # Oc_Option database (oommf/config/options.tcl)
% # Code details in oommf/pkg/oc/option.tcl
% \pgmcmd{Oc_Option Get *}            ;# Registered Option classes (glob-match)
Net_Link Oc_Url Platform Menu Nb_InputFilter Net_Server Oc_Class Color
Net_Host MIFinterp OxsLogs {}
% \pgmcmd{Oc_Option Get Platform *}   ;# All options for class Platform (glob-match)
cflags lflags optlevel
% \pgmcmd{Oc_Option GetValue Platform cflags}  ;# Platform,cflags value
-def NDEBUG

% # Configuration values (oommf/config/platforms/<platform>.tcl)
% # Code details in oommf/pkg/oc/config.tcl
% \pgmcmd{set config [Oc_Config RunPlatform]}
% \pgmcmd{$config GetValue platform_name}                          ;# Platform name
darwin
% \pgmcmd{$config GetValue program_compiler_c++_name}              ;# C++ compiler
clang++
% \pgmcmd{$config GetValue program_compiler_c++_typedef_realwide}  ;# realwide typedef
long double
% \pgmcmd{$config Features program_linker*}             ;# GetValue names (glob-match)
program_linker_option_lib program_linker program_linker_rpath
program_linker_uses_-L-l program_linker_option_out program_linker_option_obj

% \pgmcmd{exit}                                ;# Exit Tcl shell
\end{alltt}\html{\newline}
\end{codelisting}


\section{Understanding \app{pimake}}\label{sec:debug:pimake}
The \OOMMF\ \app{pimake} application controls the compiling and linking
of \OOMMF's \Cplusplus\ components. Based broadly on the \Unix\ make
utility, \app{pimake} is a platform independent tool written in
\Tcl. Each of the source code directories in the \OOMMF\ distribution
tree has a \fn{makerules.tcl} file that specifies build targets and
dependencies. A dependency tree is build from this information augmented
with recursive tracking of \cd{\#include} statements inside the
referenced source code files.  Each time \app{pimake} is run it compares
file timestamps against the dependency tree, and compiles and links any
object and executable files that are older than any of their
dependencies.

After editing \fn{*.h} or \fn{*.cc} files in \OOMMF, you should run
\app{pimake} to propagate your changes to the associated
\OOMMF\ executable(s).  If you run \cd{tclsh oommf.tcl pimake} in a
directory below the \OOMMF\ root directory, then only changes at that
directory and lower are affected. You can use the \cd{-cwd} option to
\app{pimake} to change the effective starting directory. Changes to the
\OOMMF\ \hyperrefhtml{configuration files}{configuration files
  (Sec.~}{)}{sec:debug:configfiles} do \textbf{not} trigger dependency
updates, so if you make changes affecting the build process in these
files you should manually run
\begin{alltt}
$ \shellcmd{tclsh oommf.tcl pimake distclean}
$ \shellcmd{tclsh oommf.tcl pimake}
\end{alltt}\html{\newline}
from the \OOMMF\ root directory to delete and then rebuild the full
\OOMMF\ project.

\section{Bypassing the \cd{oommf.tcl} bootstrap}\label{sec:debug:bootstrap}
When an application is launched by clicking a button in \app{mmLaunch} or from
the command shell like
\begin{alltt}
> \shellcmd{tclsh oommf.tcl mmdisp}
\end{alltt}\html{\newline}
the application (here \app{mmDisp}) is not executed directly but rather
through the ``bootstrap'' program \cd{oommf.tcl}. The bootstrap
constructs a list linking application names to commands
using the \fn{appindex.tcl} files in the various application (\fn{oommf/app/})
directories, and then runs the command associated with the given
name. This is convenient for normal use, but the additional execution
layer can obfuscate the debugging process. You can obtain the
direct command from the bootstrap program itself with the \cd{+command}
option
\begin{alltt}
> \shellcmd{tclsh oommf.tcl mmdisp +command}
app/mmdisp/windows-x86_64/mmdispsh.exe app/mmdisp/scripts/mmdisp.tcl &
\end{alltt}\html{\newline}
The response is the command as used inside a \Tcl\ shell to launch the
application. You may need to make minor edits to run the application at
your shell command prompt. For example, the trailing ampersand runs the
program in the background, which is not what one usually wants when
debugging, so you would omit this. On \Windows\ you may want to change
the forward slash path separators to backslashes. Another
\Windows-specific modification involves the first component of this
command, \fn{app/mmdisp/windows-x86\_64/mmdispsh.exe}. This is an
executable containing an embedded \Tcl\ interpreter that processes the
\Tcl\ script specified as the second command component. If you examine
the \fn{app/mmdisp/windows-x86\_64/} directory you'll find two
executables, \fn{mmdispsh.exe} and \fn{condispsh.exe}. On \Unix\ and
\MacOSX\ these two programs are the same, but on \Windows\ the first is
linked as a native \Windows\ application and the second as a console
application. The importance of this is that only the second provides the
usual \Cplusplus\ standard channels \cd{stdin}, \cd{stdout}, and
\cd{stderr}. In case of abnormal operation programs will sometimes write
error messages to \cd{stdout} or \cd{stderr}, which will be lost if the
program is not running as a console application. The upshot is that for
debugging purposes you would probably want to run \app{mmDisp} (for
example) from a \Windows\ command console as
\begin{alltt}
> \shellcmd{app{\bs}mmdisp{\bs}windows-x86_64{\bs}condispsh.exe app/mmdisp/scripts/mmdisp.tcl}
\end{alltt}

It is worth noting that on the bootstrap command line, arguments
starting with `\cd{+}' (for example, ``\cd{+command}'') are options to
\cd{oommf.tcl} itself. Run ``\cd{tclsh oommf.tcl +h}'' to see the
bootstrap help message. Options to the \OOMMF\ application follow the
application name and start with `\cd{-}'.  For example, to see the help
message for a particular application, run
``\cd{tclsh oommf.tcl <appName> -h}''.


\section{Segfaults and other asynchronous termination}\label{sec:debug:segfaults}
If an \OOMMF\ application suddenly aborts without displaying an error
message, the most likely culprit is a segfault caused by attempted
access to memory outside the program's purview. If this occurs while
running \app{oxsii} or \app{boxsi}, the first thing to check is the
\fn{oxsii.log} and \fn{boxsi.log} log files in the \OOMMF\ root
directory. If there are no hints there, and the error is repeatable,
then you can enable core dumps and re-run the program until the crash
repeats. You can then obtain a stack trace from the core dump to
determine the origin of the failure.

On \Linux, enable core dumps with the shell command \cd{ulimit -Sc
  unlimited}, and then run \cd{ulimit -Sc} to check that the
request was honored. If not, then ask your sysadmin about enabling core
dumps. (Core dumps can be rather large, so after analysis is complete
you should disable core dumps by running \cd{ulimit -Sc 0} in the
affected shell, or else exit that shell altogether.) Once core dumps are
enabled, run the offending application from the core-dumped enabled
shell prompt. When the application aborts an image of the program state
at the time of termination is written to disk. The name and location of
the core dump varies between \Linux\ distributions. On older systems the
core file will be written to the current working directory with a name
of the form \fn{core.<pid>}, where \cd{<pid>} is the pid of the
process. (If the process is \app{oxsii} or \app{boxsi} then the working
directory will be the directory containing the \fn{.mif} file.)
Otherwise, use the command \cd{sysctl kernel.core\_pattern} to determine
the pattern used to create core files. If the pattern begins with a
\cd{|} ``pipe'' symbol, then the core is piped through the indicated
program, and you will have to check the system documentation for that
program to figure out where the core went!

If the core was piped through \app{systemd-coredump}, then you can use
the \app{coredumpctl} utility to gain information about the
process. (More on this below.) Some \Linux\ variants, for example Ubuntu, use
\app{apport}, but may configure it to effectively disable core dumps for
executables outside the system package management system. In this case
you might want to install the \cd{systemd-coredump} package to replace
\app{apport}, or else use \cd{sysctl} to change
\cd{kernel.core\_pattern} to a simple file pattern (e.g.,
\cd{/tmp/core-\%e.\%p.\%h.\%t}).

If you have a core dump, you can run the GNU debugger \app{gdb} on the
executable and core dump to determine where the fault occurred:
% The \shellcmd and \pgmcmd commands color text a predefined color.
% See oommfhead.tex for specifics.
\begin{alltt}
$ \shellcmd{cd app/oxs}
$ \shellcmd{gdb linux-x86_64/oxs /tmp/core.12345}
Program terminated with signal 11, Segmentation fault.
#0  0x00000000005a40da in Oxs_UniaxialAnisotropy::RectIntegEnergy
  (Oxs_SimState const&, Oxs_ComputeEnergyDataThreaded&,
  Oxs_ComputeEnergyDataThreadedAux&, long, long) const ()
(gdb) \pgmcmd{bt}
#0  0x00000000005a40da in Oxs_UniaxialAnisotropy::RectIntegEnergy
  (Oxs_SimState const&, Oxs_ComputeEnergyDataThreaded&,
  Oxs_ComputeEnergyDataThreadedAux&, long, long) const ()
#1  0x00000000005a6fed in Oxs_UniaxialAnisotropy::ComputeEnergyChunk
  (Oxs_SimState const&, Oxs_ComputeEnergyDataThreaded&,
  Oxs_ComputeEnergyDataThreadedAux&, long, long, int) const ()
#2  0x000000000040ce44 in Oxs_ComputeEnergiesChunkThread::Cmd(int,
   void*) ()
#3  0x00000000004697bd in _Oxs_Thread_threadmain(Oxs_Thread*) ()
#4  0x00007f90ea7fb330 in ?? () from /lib64/libstdc++.so.6
#5  0x00007f90ea019ea5 in start_thread () from /lib64/libpthread.so.0
#6  0x00007f90e9d42b0d in clone () from /lib64/libc.so.6
(gdb) \pgmcmd{quit}
\end{alltt}\html{\newline}
(For visibility, shell commands are colored
\shellcmd{\shellcmdcolorname}, and \app{gdb} commands are
\pgmcmd{\pgmcmdcolorname}. The \app{gdb} commands are also prefixed with
the \cd{(gdb)} prompt. For example, ``bt'' above invokes the \app{gdb}
``backtrace'' command.) We see that the segmentation fault occurred in
the member routine \cd{RectIntegEnergy} of class
\cd{Oxs\_UniaxialAnisotropy}, called by \cd{ComputeEnergyChunk}, and so
on. If \app{oxs} had been built with debugging symbols
(\hyperrefhtml{cf. configuration files}{cf. configuration files,
  Sec.~}{}{sec:debug:configfiles}), then the stack trace would include
the corresponding source code files and line numbers.

If the core dump was journaled by \app{systemd-coredump}, then the
command \cd{coredumpctl list} will list all available core dumps,
including a timestamp, the pid, and the name of the executable. You can
get a stack trace with \cd{coredumpctl info <pid>}, or load the core
dump directly into \app{gdb} with \cd{coredumpctl gdb <pid>}. (Some
versions of \app{coredumpctl} want ``debug'' in place of ``gdb'' in that
command; check your system documentation for details.)

On \MacOSX, crash reports are automatically generated and can be viewed
from the \MacOSX\ \app{Console} app. Select ``User Reports'' or ``Crash
Reports'' from the left hand sidebar, and select the crashed
process. The report provides details about the run, including a stack
trace.

You can also create core files on \MacOSX\ in a very similar way as on
\Linux. Set \cd{ulimit -Sc unlimited} and run the application. Core
files are written to the directory \fn{/cores/}, with naming convention
\fn{core.<pid>}. If you built \OOMMF\ with \app{g++}, then you can
obtain a stack trace with \app{gdb} as above. (Note that in MacPorts the
\app{gdb} executable is named \cd{ggdb}.) If you built with
\app{clang++} then you may want to use the LLVM \app{lldb} debugger,
which should be included with the \app{clang++} package. Here is an
example \app{lldb} session, for an \app{oxs} executable built with
debugging symbols:
\begin{alltt}
% \shellcmd{cd app/oxs}
% \shellcmd{lldb -c /cores/core.54416 darwin/oxs}
(lldb) target create "darwin/oxs" --core "/cores/core.54416"
Core file '/cores/core.54416' (x86_64) was loaded.
(lldb) \pgmcmd{bt}
* thread #1, stop reason = signal SIGSTOP
 * frame #0: 0x0000000103cfc188 oxs`Oxs_UniaxialAnisotropy::RectIntegEnergy
 (this=0x00007ff0f4801000, state=0x00007ff0f350e830, ocedt=0x00007ffeec35a9a8,
 ocedtaux=0x00007ff0f350e6a0, node_start=16384, node_stop=20000) const at
 uniaxialanisotropy.cc:246
   frame #1: 0x0000000103cfd864 oxs`Oxs_UniaxialAnisotropy::ComputeEnergyChunk
 (this=0x00007ff0f4801000, state=0x00007ff0f350e830, ocedt=0x00007ffeec35a9a8,
 ocedtaux=0x00007ff0f350e6a0, node_start=16384, node_stop=20000, (null)=0)
 const at uniaxialanisotropy.cc:454
   frame #2: 0x00000001038a1739 oxs`Oxs_ComputeEnergiesChunkThread::Cmd
 (this=0x00007ffeec35b440, threadnumber=0, (null)=0x0000000000000000) at
 chunkenergy.cc:199
   frame #3: 0x00000001039eabaf oxs`Oxs_ThreadTree::LaunchTree
 (this=0x0000000103ef3860, runobj=0x00007ffeec35b440, data=0x0000000000000000)
 at oxsthread.cc:856
[...]
(lldb) \pgmcmd{quit}
\end{alltt}\html{\newline}
Similar to the \app{gdb} example, the debugger prompt is ``(lldb)'', and
``bt'' requests a stack trace.

To create and examine core dumps on \Windows, download and install
\app{ProcDump} and either \app{WinDbg} or \app{Visual Studio}
applications from Microsoft. To get symbols in the process dump file you
will need to build OOMMF with symbols, i.e., include
\begin{verbatim}
Oc_Option Add * Platform cflags {-debug 1}
\end{verbatim}
in the \fn{config/local/options.tcl}. Also, since \cd{-def NDEBUG} is
not included on this line, the \C\ macro \cd{NDEBUG} will not be
defined, which enables code \cd{assert} statements and other consistency
checks, including in particular array bound checks for
\cd{Oxs\_MeshValue} arrays.

You can create an \cd{oxs} process dump by
\begin{alltt}
> \shellcmd{cd app{\bs}oxs}
> \shellcmd{procdump -ma -t -e -x . windows-x86_64{\bs}oxs.exe boxsi.tcl foo.mif}
\end{alltt}\html{\newline}
On program exit (termination, \cd{-t}) or unhandled exception (\cd{-e})
\cd{procdump} will write a full dump file (\cd{-ma}) to
\fn{oxs.exe\_YYMMDD\_HHMMSS.dmp} in the \fn{app/oxs} directory.

Follow this procedure to examine the dump file in \app{WinDbg}:
\begin{enumerate}
\item Launch \app{WinDbg}.
\item Use the menu item \cd{File|Open Crash Dump...} to load the
  \fn{.dmp} file.
\item Then \cd{View|Call Stack} will open a call stack window.
\item Double-clicking on a call stack frame will highlight the
  corresponding line of code in the \Cplusplus\ source. By default only
  the upper portion of the call stack is displayed, which may be just
  system exit handling code. You may need to click the ``More'' control
  in the toolbar one or more times and scroll down to reach
  \OOMMF\ routines. Enable the ``Source'' toolbar option to include
  filenames and line references in the stack list.
\item You can examine variable values at the time of the crash by
  opening the \cd{View|Locals} window. Referring to the the source code
  and local variable windows in
  \hyperrefhtml{the figure below}{Fig.~}{}{fig:windbgstack},
  we see that the index variable \cd{i} has value 40000, but the size of
  the \cd{Ms\_inverse} array only has size 40000. Thus the access into
  \cd{Ms\_inverse} on line 241 (highlighted) is one element beyond the
  end of the array.
\end{enumerate}
% NB: Make certain that the fourth argument to \includeimage
% does not include any newlines or extraneous whitespace.
\ofig{\includeimage{\sswidth}{!}{windbg-stacktrace}{\app{WinDbg}~screenshot}}{\app{WinDbg}
  screenshot displaying call stack, source code, and local variables
  read from a crash dump generated by \app{procdump}.}{fig:windbgstack}

An alternative to \app{WinDbg} is to use the debugger built into Visual
Microsoft's Visual Studio:
\begin{enumerate}
\item Launch Visual Studio.
\item Select the \cd{Continue without code} option (below the ``Get
  started'' column).
\item Select \cd{File|Open|File ...}, and load the \fn{*.dmp} file.
\item Under ``Actions'' in the ``Minidump File Summary'' window, select
    \cd{Debug with Native Only}.
\item If not automatically displayed, bring up \cd{Debug|Windows|Call Stack}.
\item Double-clicking in the call stack will bring up and highlight the
    corresponding line of code in the \Cplusplus\ source.
\item Use the \cd{Debug|Windows|Autos} and \cd{Debug|Windows|Locals} menu
  items to display variable values.
\end{enumerate}

\section{Out-of-bounds memory access}\label{sec:debug:outofbounds}
One of the more common coding errors is allowing array access outside
the allocated range of an array. This error can be insidious because the
program may continue to run past the point of invalid access, but plant
a seed that grows into a seemingly unrelated fatal error later on. There
are a number of tools designed to uncover this problem, but an
especially easy one to use that is common on \Linux\ systems is the
venerable Electric Fence, original written by Bruce Perens in 1987. If
the \fn{libefence.so} shared library is installed, then from the
\cd{bash} prompt in the \fn{oommf/app/oxs} directory you can run
\begin{alltt}
$ \shellcmd{LD_PRELOAD=libefence.so linux-x86_64/oxs boxsi.tcl foo.mif}
\end{alltt}\html{\newline}
(On some installations there may also be an equivalent shell wrapper
\cd{ef}.)  This will abort with a segfault if an invalid memory
reference (read or write) is detected. One nice feature is that you
don't have to rebuild \OOMMF\ to use this debugger---the \cd{efence}
shared library transparently replaces the standard system memory
allocator with the instrumented Electric Fence version at runtime. If
you enable core dumps as explained above, then on \Linux\ systems even
without debug symbols a stack trace on the core dump will provide the
function call list. If you build \OOMMF\ with debugging symbols
(\cd{Oc\_Option cflags} option \cd{-debug} in
\fn{config/local/options.tcl}), then the core stack trace will give the
source file and line number where the invalid memory access
occurred. Also, \OOMMF\ runs at normal speed with Electric Fence
enabled, so you can use it to check for errors in large simulations.

One caveat is that for performance reasons, \OOMMF\ sometimes allocates
larger memory blocks than needed. Electric Fence detects memory
accesses outside the requested memory range, so \OOMMF\ accesses of
memory outside its proper range but inside the requested range will not
be flagged. You can have \OOMMF\ request tight blocks by putting these
lines in your \fn{local/<platform>.tcl} file:
\begin{verbatim}
$config SetValue program_compiler_c++_property_cache_linesize 1
$config SetValue program_compiler_c++_property_pagesize 1
$config SetValue sse_no_aligned_access 1
\end{verbatim}
and rebuilding \OOMMF\ (\cd{pimake distclean} plus \cd{pimake}).

Normally Electric Fence detects accesses to memory locations above the
allocated range (index too high), but you can have it check
instead for memory accesses preceding the allocated range (index too
low) by setting the environment variable \cd{EF\_PROTECT\_BELOW} to 1.

The Electric Fence documentation warns that core dumps of Electric Fence
enabled runs can be significantly larger than core dumps without
Electric Fence, and so recommends running Electric Fence with the
selected executable (here \fn{oxs}) from inside a debugger rather than
creating a core dump. This does not appear to be a problem when used
with \OOMMF\ however, as the core dumps with Electric Fence tend to be
only modestly larger than those without.

A similar tool on \MacOSX\ is the gmalloc (Guard Malloc) package, which
is included with Xcode. Run it from the \fn{oommf/app/oxs} bash or zsh
command line with
\begin{alltt}
% \shellcmd{DYLD_INSERT_LIBRARIES=/usr/lib/libgmalloc.dylib darwin/oxs boxsi.tcl foo.mif}
\end{alltt}\html{\newline}
See the documentation from Apple for full details.

\section{\Cplusplus\ source code debuggers}\label{sec:debug:debuggers}
If you know roughly where a bug is occurring in the code, you can often
debug it by temporarily inserting \cd{printf} or \cd{std::cout <{}<}
statements in the code. But for more complex problems it can be more
informative and quicker in the long run to create a debugging build (i.e.,
one with debugging symbols and perhaps with compiler optimizations
disabled) and run the program in a debugger. This section provides
general information on running \OOMMF\ in a debugger, including short
examples in three common debuggers: \app{gdb}, \app{lldb}, and \app{Visual
  Studio Debugger}.

First edit the configuration files for debugging, as explained in
\hyperrefhtml{the \textbf{Configuration files}
  section.}{Sec.~}{.}{sec:debug:configfiles}
Then run
\begin{alltt}
$ \shellcmd{tclsh oommf.tcl pimake distclean}
$ \shellcmd{tclsh oommf.tcl pimake}
\end{alltt}\html{\newline}
to create a build of \OOMMF\ with debugging symbols. After this you can
load an \OOMMF\ executable into a debugger, run the
program, and examine its execution.  (Remember to \hyperrefhtml{bypass
the \cd{oommf.tcl} bootstrap}{bypass the \cd{oommf.tcl} bootstrap as
explained in Sec.~}{}{sec:debug:bootstrap}.)  There are many debuggers
available, some with multiple front-ends. But one overriding criterion
in selecting a debugger is to choose one that supports the debugging
symbol format output by your \Cplusplus\ compiler. To provide a brief
taste of this subject, we will look at three debuggers: GNU's venerable
\app{gdb} for use with \cd{g++}, the \app{lldb} debugger packaged with
Xcode/\app{clang++} on \MacOSX, and the debugger built into Microsoft's
\app{Visual Studio} for use with Visual \Cplusplus\ \cd{cl} binaries.

\subsection{Introduction to the GNU \app{gdb} debugger}\label{sec:debug:gdbintro}
This section provides a brief overview on using \app{gdb} for debugging
\OOMMF\ programs. For a more thorough background you can refer to the
extensive documentation available from the GNU Project or the many
online tutorials.

In the following examples, the (\app{bash}) shell prompt is indicated by
\cd{\$}, and the \app{gdb} prompt with \cd{(gdb)}. You launch \app{gdb}
from the command line with the name of the executable file. You can
provide arguments to the executable when you \cd{run} the program inside
\app{gdb}. For example, to debug a problem with an \cd{Oxs} extension,
we would run \app{Boxsi} with a sample troublesome \cd{.mif} file, say
\begin{alltt}
$ \shellcmd{cd oommf/app/oxs}
$ \shellcmd{gdb linux-x86_64/oxs}
(gdb) \pgmcmd{run boxsi.tcl local/foo/foo.mif -threads 1}
\end{alltt}\html{\newline}
Subsequent \cd{run} commands will reuse the same arguments unless you
specify new ones. In this example the \cd{-threads 1} option to
\app{Boxsi} is used to simplify the debugging process. If you need or
want to debug with multiple threads, then read up on the ``thread''
command in the \app{gdb} documentation.

The program run will automatically terminate and return to the
\cd{(gdb)} prompt if the program exits or aborts. Alternately you can
\cd{Ctrl-C} at any time to manually halt. To exit \app{gdb} type
\cd{quit} at the \cd{(gdb)} prompt.

\app{gdb} has a large collection of commands that you can use to control
program flow and inspect program data. An example we saw before is
\cd{backtrace}, which can be abbreviated as
\cd{bt}. Fig.~\ref{oommfgdbcheat} lists a few of the more common
commands, and Figs.~\ref{fig:oommfgdbsession1} and
\ref{fig:oommfgdbsession2} provide an example debugging session
illustrating their use.

%%%%%%%%%%%%%%%%%%%%%%%%%%%%%%%%%%%%%%%%%%%%%%%%%%%%%%%%%%%%%%%%%%%%%%%%
\begin{codelisting}{f}{oommfgdbcheat}{\app{gdb} Debugger
    Cheatsheet\HTMLoutput{\phantom{\rule{1pt}{1.5\baselineskip}}}}{sec:debug:gdbintro}{ref}
% For some reason latexml shoves the following table flush up against
% the figure caption; maybe it's not expecting the caption to be at the
% top of the figure? Also, if I make the \rule with 0pt wide then
% latexml drops the rule space. But 1pt wide inside \phantom works.
\begin{center}\begin{tabular}{|l|l|l|}\hline
  \multicolumn{3}{|l|}{\rule[-1ex]{0pt}{3ex}\textbf{Shell%
   command:}\texttt{ gdb linux-x86\_64/oxs [corefile (opt)]}}\\\hline
  \multicolumn{1}{|c}{\rule[-1ex]{0pt}{3.5ex}
    \textbf{Command}}
  & \multicolumn{1}{|c}{\textbf{Abbr.}}
  & \multicolumn{1}{|l|}{\textbf{Description}}\\\hline
  \multicolumn{3}{|l|}{
  \rule{0pt}{2.5ex}\textcolor[rgb]{0,0.7,0}{\textbf{Process control}}}\\\hline
  run [\textit{args}] & & run executable with \textit{args}\\
  run & &  run executable with last \textit{args}\\
  show args & & display current \textit{args}\\
  set env FOO bar  & & set envr.\ variable FOO to ``bar''\\
  unset env FOO & & unset environment variable FOO\\
  Ctrl-C & & stop and return to (gdb) prompt\\
  kill & & terminate current run\\
  quit & & exit gdb\\[0.5ex]\hline

  \multicolumn{3}{|l|}{
  \rule{0pt}{2.5ex}\textcolor{blue}{\textbf{Introspection}}}\\\hline
  backtrace & bt & stack trace\\
  frame 7 & f 7 & change to stack frame 7\\
  list 123 & l 123 & list source about line 123\\
  list foo.cc:50 & & list source about line 50 of foo.cc\\
  list - & l - & list preceding ten lines\\
  list foo::bar & & list first ten lines of function foo::bar()\\
  set listsize 20 & & change list output length to 20 lines\\
  info locals & i lo & print local variables\\
  info args  & & print function arguments\\
  print foo & p foo & write info on variable foo\\
  printf \verb+"+\%g\verb+"+, foo &
    & print foo with format \%g (note comma)\\[0.5ex]\hline

  \multicolumn{3}{|l|}{
    \rule{0pt}{2.5ex}\textcolor{red}{\textbf{Flow control}}}\\\hline
  break bar.cc:13 & b bar.cc:13
    & set breakpoint at line 13 of bar.cc\\
  break foo::bar
  & b foo::bar & break on entry to \Cplusplus\ routine foo::bar()\\
  info breakpoints & i b & list breakpoints\\
  delete 4 & d 4 & delete breakpoint 4\\
  delete & d & delete all breakpoints\\
  ignore 3 100 & & skip breakpoint 3 100 times\\
  watch -location foo & & break when foo changes value\\
  condition 2 foo\verb+>+10 & & break if foo\verb+>+10 at breakpoint 2\\
  continue & c & continue running\\
  step [\#] & s [\#] & take \# steps, follow into subroutines\\
  next [\#] & n [\#] & take \# steps, step over subroutines\\
  finish & & run to end of current subroutine (step out)\\[0.5ex]\hline

  \multicolumn{3}{|l|}{
    \rule{0pt}{2.5ex}\textcolor[rgb]{0.6,0,0.9}{\textbf{Threads}}}\\\hline
  info threads & i th & list threads\\
  thread 4 & t 4 & switch context to thread 4\\\hline
  \end{tabular}
  \end{center}\html{\newline}
\end{codelisting}
%%%%%%%%%%%%%%%%%%%%%%%%%%%%%%%%%%%%%%%%%%%%%%%%%%%%%%%%%%%%%%%%%%%%%%%%

\begin{codelisting}{p}{fig:oommfgdbsession1}{Sample \app{gdb} session,
    part 1: Locating the error}{sec:debug:gdbintro}{ref}
\NONHTMLoutput{\small}
\begin{alltt}
$ \shellcmd{cd app/oxs}
$ \shellcmd{gdb linux-x86_64/oxs}
(gdb) \pgmcmd{run boxsi.tcl examples/stdprob1.mif -threads 1}
Starting program: oommf/app/oxs/linux-x86_64/oxs boxsi.tcl examples/stdp...
oxs: oommf/app/oxs/base/meshvalue.h:319: const T& Oxs_MeshValue<T>::oper...
  Assertion `0<=index && index<size' failed.

Thread 1 "oxs" received signal SIGABRT, Aborted.
0x00007ffff65d837f in raise () from /lib64/libc.so.6
(gdb) \pgmcmd{bt}
#0  0x00007ffff65d837f in raise () from /lib64/libc.so.6
[...]
#4  0x000000000041012a in Oxs_MeshValue<double>::operator[]
  (this=0xcbeb58, index=40000) at oommf/app/oxs/base/meshvalue.h:319
#5  0x000000000061e88a in Oxs_UniaxialAnisotropy::RectIntegEnergy
  (this=0x1307d60, state=..., ocedt=..., ocedtaux=..., node_start=36864,
  node_stop=40000) at oommf/app/oxs/ext/uniaxialanisotropy.cc:241
[...]
(gdb) \pgmcmd{frame 5}
#5  0x000000000061e88a in Oxs_UniaxialAnisotropy::RectIntegEnergy...
241           field_mult = (2.0/MU0)*k*Ms_inverse[i];
(gdb) \pgmcmd{set listsize 5}
(gdb) \pgmcmd{list}
239         if(aniscoeftype == K1_TYPE) \{
240           if(!K1_is_uniform) k = K1[i];
241           field_mult = (2.0/MU0)*k*Ms_inverse[i];
242         \} else \{
243           if(!Ha_is_uniform) field_mult = Ha[i];
(gdb) \pgmcmd{print i}
$1 = 40000
(gdb) \pgmcmd{print Ms_inverse}
$2 = (const Oxs_MeshValue<double> &) @0xcbeb58: \{arr = 0x7ffff7ebf000,
  size = 40000, arrblock = \{datablock = 0x7ffff7ebe010 "",
  arr = 0x7ffff7ebf000, arr_size = 40000, strip_count = 1,
  strip_size = 320000, strip_pos = std::vector of length 2,
  capacity 2 = \{0, 320000\}\}, static MIN_THREADING_SIZE = 10000\}
(gdb) \pgmcmd{kill}
Kill the program being debugged? (y or n) y
[Inferior 1 (process 1309854) killed]
\end{alltt}\html{\newline}
\end{codelisting}

\begin{codelisting}{f}{fig:oommfgdbsession2}{Sample \app{gdb} session,
    part 2: Bug details}{sec:debug:gdbintro}{ref}
\NONHTMLoutput{\small}
\begin{alltt}
(gdb) \pgmcmd{break uniaxialanisotropy.cc:239}
Breakpoint 1 at 0x61e811: file ext/uniaxialanisotropy.cc, line 239.
(gdb) \pgmcmd{run}
Starting program: oommf/app/oxs/linux-x86_64/oxs boxsi.tcl examples/s...
[...]
Thread 1 "oxs" hit Breakpoint 1, Oxs_UniaxialAnisotropy::RectIntegEne...
239         if(aniscoeftype == K1_TYPE) \{
(gdb) \pgmcmd{info breakpoints}
Num     Type           Disp Enb Address            What
1       breakpoint     keep y   0x000000000061e811 in Oxs_UniaxialAni...
        breakpoint already hit 1 time
(gdb) \pgmcmd{ignore 1 39999}
Will ignore next 39999 crossings of breakpoint 1.
(gdb) \pgmcmd{continue}

Thread 1 "oxs" hit Breakpoint 1, Oxs_UniaxialAnisotropy::RectIntegEne...
239         if(aniscoeftype == K1_TYPE) \{
(gdb) \pgmcmd{print i}
$3 = 39991
(gdb) \pgmcmd{condition 1 i>=40000}
(gdb) \pgmcmd{c}

Thread 1 "oxs" hit Breakpoint 1, Oxs_UniaxialAnisotropy::RectIntegEne...
239         if(aniscoeftype == K1_TYPE) \{
(gdb) \pgmcmd{l}
237
238       for(OC_INDEX i=node_start;i<=node_stop;++i) \{
239         if(aniscoeftype == K1_TYPE) \{
240           if(!K1_is_uniform) k = K1[i];
241           field_mult = (2.0/MU0)*k*Ms_inverse[i];
(gdb) \pgmcmd{next}
240           if(!K1_is_uniform) k = K1[i];
(gdb) \pgmcmd{n}
241           field_mult = (2.0/MU0)*k*Ms_inverse[i];
(gdb) \pgmcmd{step}
Oxs_MeshValue<double>::operator[] (this=0xcbeb58, index=40000)
  at oommf/app/oxs/base/meshvalue.h:319
319       assert(0<=index && index<size);
(gdb) \pgmcmd{printf "%d,%d{\bs}n", index, size}
40000,40000
(gdb) \pgmcmd{quit}
\end{alltt}\normalsize\html{\newline}
\end{codelisting}

Two notes concerning \app{gdb} on \MacOSX: First, as mentioned earlier,
if you install \app{gdb} through MacPorts, the executable name is
\cd{ggdb}. Second, debuggers operate outside the normal end-user program
envelope and may run afoul of the OS security system. In particular to
use \app{gdb} you may need to set up a certificate in the
\MacOSX\ System Keychain for it; details on this process can be found
online. This issue might be resolved for \app{lldb} (next section) as
part of the installation process if it and \app{clang++} were installed
as part of the Xcode package.

This introduction only scratches the surface of \app{gdb} commands and
capabilities. You can find tutorials and additional information online,
or else refer to the \app{gdb} documentation from GNU for full details.

\subsection{Introduction to the LLVM \app{lldb}}\label{sec:debug:lldbintro}
If you are working on \MacOSX, you may be building \OOMMF\ with
\app{g++} or \app{clang++}. The native debugger for \app{clang++} is
\app{lldb}, which is included as part of the Xcode package. Both
\app{g++} and \app{clang++} use the same debugging symbol format, so in
principle you should be able to use either debugger with either
compiler, but if you have problems with one try the other.

The \app{lldb} debugger is a command-line debugger very similar in
concept to \app{gdb}, and although the command syntax is somewhat
different, \app{lldb} provides a fair number of aliases to ease the
transition for veteran \app{gdb} users.  Fig.~\ref{fig:oommflldbcheat}
lists a few of the more common \app{lldb} commands, and
Figs.~\ref{fig:oommflldbsession1} and \ref{fig:oommflldbsession2}
illustrate an \app{lldb} debugging session analogous to the \app{gdb}
session presented in Figs.~\ref{fig:oommfgdbsession1} and
\ref{fig:oommfgdbsession2}.

% NOTE: I had trouble getting latex2html to render ``--'' as two
%       separate dashes instead of an endash. \verb works, but is not
%       allowed inside \makecell. However, the following \dblhyp command
%       appears to do the trick. You can use \textsf in place of \texttt
%       if you with, but the html hyphen looks low up against \textrm in
%       my text browser. (And \textrm{-} puts a big space between the
%       hyphens in the html.)
\newcommand{\dblhyp}{\texttt{-}\texttt{-}}

\begin{codelisting}{f}{fig:oommflldbcheat}{\app{lldb} Debugger
    Cheatsheet\HTMLoutput{\phantom{\rule{1pt}{1.5\baselineskip}}}}{sec:debug:lldbintro}{ref}
  \begin{center}
    \begin{tabular}{|l|l|l|}\hline
  \multicolumn{3}{|l|}{\rule[-1ex]{0pt}{3ex}\textbf{Shell
      command:}\texttt{ lldb [-c corefile (opt)] darwin/oxs}}\\\hline
  \multicolumn{1}{|c}{\rule[-1ex]{0pt}{3.5ex}
    \textbf{Command}}
  & \multicolumn{1}{|c}{\textbf{Abbr.}}
  & \multicolumn{1}{|l|}{\textbf{Description}}\\\hline
  \multicolumn{3}{|l|}{
  \rule{0pt}{3ex}\textcolor[rgb]{0,0.7,0}{\textbf{Process control}}}\\\hline
  process launch \dblhyp\ [\textit{args}] & r [\textit{args}]
  & run executable with \textit{args}\\
  process launch & r &  run executable with last \textit{args}\\

  settings show target.run-args & & display current \textit{args}\\

  settings set target.env-vars & \multirow{2}{*}{env FOO=bar}
     & \multirow{2}{*}{set envr.\ variable FOO to ``bar''} \\
  ~~~FOO=bar && \\
  % AFAICT, LaTeX2HTML either draws line across every row
  % or nont at all, depending on whether or not any \hline
  % command appears in the tabular. This provides less than
  % ideal behavior in the non-multirowed column. OTOH, LaTeXML
  % appears to handle this OK.

  Ctrl-C & & stop and return to (lldb) prompt\\
  process kill & kill & terminate current run\\
  quit & & exit lldb\\\hline

  \multicolumn{3}{|l|}{
    \rule{0pt}{3ex}\textcolor{blue}{\textbf{Introspection}}}\\\hline
  thread backtrace & bt & stack trace of current thread\\

  frame select 5 & f 5 & change to stack frame 5\\
  frame variable &  & print args \& vars for current frame\\
  frame variable foo & p foo & print value of variable foo\\
  source list -f foo.cc -l 50 & l foo.cc:50 & list source after line 50 of foo.cc\\
  source list & l & list next ten lines\\
  source list -r & l - & list preceding ten lines\\
  source list -c 20 & & list 20 lines\\\hline

  \multicolumn{3}{|l|}{
    \rule{0pt}{3ex}\textcolor{red}{\textbf{Flow control}}}\\\hline

  breakpoint set && \multirow{2}{*}{set breakpoint at line 99 of foo.cc}\\
  ~~{\dblhyp}file foo.cc {\dblhyp}line 99 && \\

  breakpoint set && \multirow{2}{*}{break at \Cplusplus\ routine foo::bar()}\\
  ~~{\dblhyp}name foo::bar && \\

  breakpoint list & br l & list breakpoints\\
  breakpoint delete 4 & br del 4 & delete breakpoint 4\\
  breakpoint delete & br del & delete all breakpoints\\

  breakpoint modify -i 100 3 & & skip breakpoint 3 100 times\\
  breakpoint modify -c i\verb+>+7 3
   && break if i\verb+>+7 at breakpoint 3\\
  watchpoint set variable foo && break when foo changes value\\[1ex]

  thread continue & c & continue running\\
  thread step-in & s & take one step, into subroutines\\
  thread step-over & n & take one step, over subroutines\\
  thread step-out & finish & run to end of current subroutine\\[1ex]\hline

  \multicolumn{3}{|l|}{
  \rule{0pt}{3ex}\textcolor[rgb]{0.6,0,0.9}{\textbf{Threads}}}\\\hline
  thread list &  & list all threads\\
  thread select 2 & & switch context to thread 2\\\hline
    \end{tabular}
  \end{center}\html{\newline}
\end{codelisting}
%%%%%%%%%%%%%%%%%%%%%%%%%%%%%%%%%%%%%%%%%%%%%%%%%%%%%%%%%%%%%%%%%%%%%%%%

\begin{codelisting}{f}{fig:oommflldbsession1}{Sample \app{lldb} session,
    part 1: Locating the error}{sec:debug:lldbintro}{ref}
\NONHTMLoutput{\small}
\begin{alltt}
% \shellcmd{cd app/oxs}
% \shellcmd{lldb darwin/oxs}
(lldb) target create "darwin/oxs"
Current executable set to 'oommf/app/oxs/darwin/oxs' (x86_64).
(lldb) \pgmcmd{process launch -- boxsi.tcl examples/stdprob1.mif -threads 1}
Process 36662 launched: 'oommf/app/oxs/darwin/oxs' (x86_64)
Assertion failed: (0<=index && index<size) [...] file meshvalue.h, line 319.
Process 36662 stopped
* thread #1, queue = 'com.apple.main-thread', stop reason = hit program assert
    frame #4: 0x00000001000065cc oxs [...] at meshvalue.h:319:3
   316  template<class T>
   317  const T& Oxs_MeshValue<T>::operator[](OC_INDEX index) const
   318  \{
-> 319    assert(0<=index && index<size);
   320    return arr[index];
   321  \}
   322
Target 0: (oxs) stopped.
(lldb) \pgmcmd{bt}
* thread #1, queue = 'com.apple.main-thread', stop reason = hit program assert
    frame #0: 0x00007fff207ba91e libsystem_kernel.dylib`__pthread_kill + 10
[...]
  * frame #4: 0x00000001000065cc oxs`Oxs_MeshValue<double>::operator[](th...
    frame #5: 0x0000000100350fa8 oxs`Oxs_UniaxialAnisotropy::RectIntegEne...
[...]
(lldb) \pgmcmd{frame select 5}
frame #5: 0x0000000100350fa8 oxs`Oxs_UniaxialAnisotropy::RectIntegEnergy(...
   238    for(OC_INDEX i=node_start;i<=node_stop;++i) \{
   239      if(aniscoeftype == K1_TYPE) \{
   240        if(!K1_is_uniform) k = K1[i];
-> 241        field_mult = (2.0/MU0)*k*Ms_inverse[i];
   242      \} else \{
   243        if(!Ha_is_uniform) field_mult = Ha[i];
   244        k = 0.5*MU0*field_mult*Ms[i];
(lldb) \pgmcmd{frame variable i}
(OC_INDEX) i = 40000
(lldb) \pgmcmd{frame variable Ms_inverse}
(const Oxs_MeshValue<double> &) Ms_inverse = 0x0000000102b77928: \{
  arr = 0x0000000101da4000
  size = 40000
[...]
(lldb) \pgmcmd{process kill}
Process 36662 exited with status = 9 (0x00000009)
\end{alltt}\html{\newline}
\end{codelisting}

\begin{codelisting}{f}{fig:oommflldbsession2}{Sample \app{lldb} session,
   part 2: Bug details (lldb output edited for
   space)}{sec:debug:lldbintro}{ref}
\NONHTMLoutput{\small}
\begin{alltt}
(lldb) \pgmcmd{breakpoint set --file uniaxialanisotropy.cc --line 239}
Breakpoint 1: where = oxs`Oxs_UniaxialAnisotropy::RectIntegEnergy(Oxs_Sim...
(lldb) \pgmcmd{process launch}
Process 36718 launched: 'oommf/app/oxs/darwin/oxs' (x86_64)
[...]
* thread #1, queue = 'com.apple.main-thread', stop reason = breakpoint 1.1
   238    for(OC_INDEX i=node_start;i<=node_stop;++i) \{
-> 239      if(aniscoeftype == K1_TYPE) \{
   240        if(!K1_is_uniform) k = K1[i];
   241        field_mult = (2.0/MU0)*k*Ms_inverse[i];
(lldb) \pgmcmd{breakpoint list}
Current breakpoints:
1: file = 'uniaxialanisotropy.cc', line = 239, exact_match = 0, locations...
  1.1: where = oxs`Oxs_UniaxialAnisotropy::RectIntegEnergy(Oxs_SimState c...
(lldb) \pgmcmd{breakpoint modify -i 39999 1}
(lldb) \pgmcmd{thread continue}
* thread #1, queue = 'com.apple.main-thread', stop reason = breakpoint 1.1
-> 239      if(aniscoeftype == K1_TYPE) \{
(lldb) \pgmcmd{p i}
(OC_INDEX) $0 = 39991
(lldb) \pgmcmd{breakpoint modify -c i>=40000}
(lldb) \pgmcmd{c}
* thread #1, queue = 'com.apple.main-thread', stop reason = breakpoint 1.1
-> 239      if(aniscoeftype == K1_TYPE) \{
(lldb) \pgmcmd{thread step-over}
* thread #1, queue = 'com.apple.main-thread', stop reason = step over
-> 240        if(!K1_is_uniform) k = K1[i];
(lldb) \pgmcmd{n}
* thread #1, queue = 'com.apple.main-thread', stop reason = step over
-> 241        field_mult = (2.0/MU0)*k*Ms_inverse[i];
(lldb) \pgmcmd{thread step-in}
* thread #1, queue = 'com.apple.main-thread', stop reason = step in
   317  const T& Oxs_MeshValue<T>::operator[](OC_INDEX index) const
   318  \{
-> 319    assert(0<=index && index<size);
(lldb) \pgmcmd{print (void) printf("%d,%d{\bs}n", index, size)}
40000,40000
(lldb) \pgmcmd{quit}
\end{alltt}\html{\newline}
\end{codelisting}

\subsection{Debugging \OOMMF\ in Visual Studio}\label{sec:debug:visualstudiodebugger}
The debugger built into Microsoft's Visual Studio provides largely
similar functionality to \app{gdb} and \app{lldb}, but with a GUI
interface. It understands the debugging symbol files produced by the
Visual \Cplusplus\ \cd{cl} compiler, namely ``Program DataBase'' files
having the \fn{.pdb} extension. Other debugger options for this symbol
file format include the GUI \app{WinDbg} mentioned earlier, and the
related command line tool \app{CDB}.

Visual Studio is an integrated development environment, and normal usage
involves building ``projects'' that specify all the source code files
and rules for building them into an executable program. \OOMMF\ does not
follow this paradigm, but rather maintains similar information in a
collection of \Tcl\ \fn{makerules.tcl} files distributed across the
development tree. Thus there is no \OOMMF\ project file to load into
Visual Studio. Instead, to debug an \OOMMF\ application in Visual Studio
you need to load the application executable directly, along with some
supplemental run information. The following details the process for
Visual Studio 2022; specifics may differ somewhat for other releases.
\begin{enumerate}
\item Launch Visual Studio
\item Select \cd{Open a project or solution} from the \cd{Getting
  started} pane and then navigate to and select the executable.
\item In the \cd{Solution Explorer} pane, right click on the executable
  and select \cd{Properties}.
\item Under \cd{Parameters}, fill in the \cd{Arguments} and \cd{Working
  Directory} fields as appropriate. You may also have to modify the
  \cd{Environment} setting, in particular if the \Tcl\ and
  \Tk\ \fn{.dll}'s are not on the default path used by Visual
  Studio. In this case click on the ellipsis at the right of the
  \cd{Environment} row, and then click the \cd{Fetch} button at the
  bottom of the \cd{Environment} pop-up to load the current
  environment. Scroll down to variable \cd{path} and edit as necessary.
  Close when complete.
\item Select \cd{Start} from the toolbar or \cd{Debug|Start Debugging}
  from the top-level menu bar.
\item Debug! You can use the drop-down menus to perform actions
  analogous to those described above for the \app{gdb} and \app{lldb}
  debuggers. If you get a message that no symbols were loaded, then most
  likely either the \cd{/Zi} switch was missing from the compile command
  or else the \cd{/DEBUG} option was missing from the link command. In
  this case review the \OOMMF\ \hyperrefhtml{configuration file
  settings}{configuration file settings
    (Sec.~}{)}{sec:debug:configfiles}) and rebuild \OOMMF.  The symbols 
  for the executable should be stored in a \fn{*.pdb} file next to the
  executable file.
\item The call stack should automatically appear when you start
  debugging. If not, you can manually call it up through the menu option
  \cd{Debug|Windows|Call Stack}. A curious feature of Visual Studio
  is that the call stack window disappears when execution exits. This
  happens even when the exit is caused by an abnormal event, for example
  via an assertion failure. In default \OOMMF\ builds many types
  of fatal errors are routed through the
  \cd{Oc\_AsyncError::CatchSignal(int)} routine in
  \fn{pkg/oc/ocexcept.cc}. If you set a breakpoint in this function then
  the debugger will stop if it hits this function, but will not exit the
  debugger, so you can still examine the call stack. Do this before you
  start the debugging run by pulling up the \cd{Debug|New
    Breakpoint|Function Breakpoint...} dialog, enter
  \cd{Oc\_AsyncError::CatchSignal(int)} in the ``Function Name'' box,
  and click ``OK''.
  \item Double-clicking on a row in the Call Stack window will bring up
    the relevant line of source code. Menu option
    \cd{Debug|Windows|Locals} will open a window showing the variable
    values accessible at this point in the code. An example is shown in
    \hyperrefhtml{the figure below}{Fig.~}{}{fig:vsdbgassert}, where we
    see that the index variable \cd{i} at line 241 of
    \fn{uniaxialanisotropy.cc} has value 40000, but the size of
    \cd{Ms\_inverse} is 40000, meaning the maximum valid index into
    \cd{Ms\_inverse} is only 39999.
\item When you exit the debugger you will be asked if you want to save the
  \fn{.sln} (solution) file. If you do, it will be written in the same
  directory as the executable and \fn{.pdb} files. In later debugging
  sessions you can load the solution file in step 2 above and bypass
  steps 3 and 4.
\end{enumerate}
% NB: Make certain that the fourth argument to \includeimage
% does not include any newlines or extraneous whitespace.
\ofig{\includeimage{\sswidth}{!}{vsdbg-assert}{\app{Visual~Studio~Debugger}~screenshot}}{\app{Visual
    Studio Debugger} screenshot displaying call stack, source code, and
  local variables from a debugging session.}{fig:vsdbgassert}

%end{latexonly}
\html{\clearpage\chapter{Debugging \OOMMF}\label{sec:debug}
\newlength{\sswidth} % Width for screen shot figures
\setlength{\sswidth}{\textwidth}
\addtolength{\sswidth}{-1em}
This chapter provides an introduction to debugging \OOMMF\ and
\OOMMF\ extension source code, providing background to the \OOMMF\ build
architecture and detailing some tools and techniques for uncovering
programming errors. It begins with a look at the \OOMMF\ \app{pimake}
application used for compiling and linking \OOMMF\ programs, followed by
some considerations involving the \app{oommf.tcl} bootstrap
wrapper. Then configuration files governing build and runtime behavior
are detailed.  After this methods for identifying and locating runtime
errors are presented, including a brief introduction on using debugger
applications with \OOMMF. Although the primary focus of this chapter is
on errors in \C++\ code, the interface and glue code linking the various
\OOMMF\ applications rely on \Tcl\ script.  An example of working with
\Tcl\ in \OOMMF\ is provided in Fig.~\ref{fig:oommftclintrospection}

Throughout this chapter, unless otherwise stated, commands are
implicitly assumed to be run from the \OOMMF\ root directory (i.e. the
directory containing the file \fn{oommf.tcl}), and directory paths are
taken relative to this directory (e.g., \fn{app/oxs/} refers
to the directory \fn{<oommf\_root>/app/oxs/}).

In text blocks containing command statements and program output,
command statements are indicated with a leading character
representing the shell command prompt. On \Windows\ this character is
typically ``\verb+>+'', whereas the \Unix\ and \MacOSX\ shells more commonly
use ``\verb+$+'' with \cd{bash} shells or ``\verb+%+'' with
\cd{zsh}. All three are used below, but ``\verb+%+'' is limited to
\MacOSX\ specific examples to minimize confusion with the \Tcl\ command
prompt, which is also ``\verb+%+''. For additional visibility shell
commands are colored \shellcmd{\shellcmdcolorname}\ and program commands
(\Tcl\ and debugger) are colored \pgmcmd{\pgmcmdcolorname}. (Computer
responses remain in black text.)

Some details in what follows may vary depending on the particular
operating system and application version, but hopefully the differences
are sufficiently small that this description remains a useful guide.

\section{Configuration Files}\label{sec:debug:configfiles}
There are several \OOMMF\ configuration settings that impact debug
operations. The controlling files are \fn{config/options.tcl} and
\fn{config/platforms/<platform>.tcl}, where the \texttt{<platform>} is
\texttt{windows-x86\_64}, \texttt{linux-x86\_64}, or \texttt{darwin} for
\Windows, \Linux, or \MacOSX\ operating systems respectively. In practice,
rather than modifying the default distribution files directly, you should
place your modifications in local files
\fn{config/local/options.tcl} and
\fn{config/platforms/local/<platform>.tcl}.
The \fn{local/} directories and files are not part of the
\OOMMF\ distribution; you will need to create them manually. The files
can be empty initially, and then populated as desired.

The \fn{options.tcl} file contains platform-agnostic settings that are
stored in the \cd{Oc\_Option} database. Some of these settings affect
the build process, while others control post-build runtime behavior.
All are set using the \cd{Oc\_Option} command, which takes
name\,+\,value pairs.  The \cd{cflags} and \cd{optlevel} settings
control compiler options. The default setting for \cd{cflags} is
\begin{verbatim}
Oc_Option Add * Platform cflags {-def NDEBUG}
\end{verbatim}
which causes the C macro ``\texttt{NDEBUG}'' to be defined. If this is
not set then various run-time checks such as \cd{assert} statements and
some array index checks are activated. These checks slow execution but
may be helpful in diagnosing errors. Other \cd{cflag} options include
\cd{-warn}, which enables compiler warning messages, and \cd{-debug},
which tells the compiler to generate debugging symbols. A good
\cd{cflags} setting for debugging is
\begin{verbatim}
Oc_Option Add * Platform cflags {-warn 1 -debug 1}
\end{verbatim}
There is also an \cd{lflags} option, similar to \cd{cflags}, that
controls options to the linker. The default is an empty string (no
options), and you generally don't need to change this.

The \cd{optlevel} option sets the compiler optimization level, with an
integer value between 0 and 3. The default value is 2, which selects for
a high but reliable level of optimizations. Some optimizations may
reorder and combine source code statements, making it harder to debug
code, so you may want to use
\begin{verbatim}
Oc_Option Add * Platform optlevel 0
\end{verbatim}
to disable all optimizations.

The \fn{config/platforms/<platform>.tcl} files set default platform and
compiler specific options. For example,
\fn{config/platforms/windows-x86\_64.tcl} is the base platform file for
64-bit \Windows. There are separate sections inside this file for the
various supported compilers. You can make local changes to the default
settings by creating a subdirectory of \fn{config/platforms/} named
\cd{local/}, and creating there an initially empty file with the
same name as the base platform file. Inside the base platform file is a
code block labeled \cd{LOCAL CONFIGURATION}, which lists all the
available local modifications. You can copy some or all of this
\Tcl\ code block to your new \cd{config/platforms/local/} file, and then
uncomment and modify options as desired. For example, if you are using
the Visual \Cplusplus\ compiler on \Windows, you may want to include the
\cd{/RTCus} compiler flag to enable some run-time error checks. You can
do that with these lines in your
\fn{local/windows-x86\_64.tcl} file:
\begin{verbatim}
$config SetValue program_compiler_c++_remove_flags {/O2}
$config SetValue program_compiler_c++_remove_valuesafeflags {/O2}
$config SetValue program_compiler_c++_add_flags {/RTCus}
$config SetValue program_compiler_c++_add_valuesafeflags {/RTCus}
\end{verbatim}
The \cd{*\_valuesafeflags} options are for code with sensitive
floating-point operations that must be evaluated exactly as
specified. This pertains primarily to the double-double routines in
\fn{pkg/xp/}. The \cd{*\_flags} options are for everything else. The
\cd{*\_remove\_*} controls remove options from the default compile
command. This can be a (\Tcl) list, with each element matching as a
regular expression. (Refer to the
\htmladdnormallinkfoot{\Tcl\ documentation}{https://www.tcl-lang.org/man/}
on the \cd{regexp} command for details.) The \cd{*\_add\_*} controls
append options. \OOMMF\ sets \cd{/O2} optimization by default, but
\cd{/O2} is incompatible with \cd{/RTCus}, so in this example \cd{/O2}
is removed to allow \cd{/RTCus} to be added. (Setting \cd{optlevel 0} in
the \fn{config/local/options.tcl} file, as explained above, replaces
\cd{/O2} with \cd{/Od}. So strictly speaking it is not necessary to
remove \cd{/O2} in that case, but it doesn't hurt either.)

You can run the command ``\cd{oommf.tcl +platform +v}'' to see the
effects of your current \fn{options.tcl} and \fn{<platform>.tcl}
settings. For example,
\begin{alltt}
$ \shellcmd{tclsh oommf.tcl +platform +v}
[...]
--- Local config options ---
[...]
   Oc_Option Add * Platform cflags {-debug 1 -warn 1}
   Oc_Option Add * Platform optlevel 0
[...]
--- Local platform options ---
   $config SetValue program_compiler_c++_remove_flags {/O2}
   $config SetValue program_compiler_c++_remove_valuesafeflags {/O2}
   $config SetValue program_compiler_c++_add_flags {/RTCus}
   $config SetValue program_compiler_c++_add_valuesafeflags {/RTCus}

--- Compiler options ---
     Standard options: /Od /D_CRT_SECURE_NO_DEPRECATE /RTCus
   Value-safe options: /Od /fp:precise /D_CRT_SECURE_NO_DEPRECATE /RTCus
\end{alltt}

To see the exact, full platform-specific compile and link commands, you
can delete and rebuild individual executables in the
\OOMMF\ package. Two examples, one using the standard build options
(\fn{pkg/oc/<platform>/varinfo}) and one using the value-safe options
(\fn{pkg/xp/<platform>/build\_port}) are presented below. (The response
lines have been edited for clarity.)
\begin{alltt}
% \shellcmd{cd pkg/oc}
% \shellcmd{tclsh ../../oommf.tcl pimake clean}
% \shellcmd{tclsh ../../oommf.tcl pimake darwin/varinfo}
clang++ -c -DNDEBUG -m64 -std=c++11 -Ofast -o darwin/varinfo.o varinfo.cc
clang++ -m64 darwin/varinfo.o -o darwin/varinfo

% \shellcmd{cd ../..}
% \shellcmd{cd pkg/xp}
% \shellcmd{tclsh ../../oommf.tcl pimake clean}
% \shellcmd{tclsh ../../oommf.tcl pimake darwin/build_port}
clang++ -c -DNDEBUG -m64 -std=c++11 -O3 -DXP_USE_MPFR=0
   -o darwin/build_port.o build_port.cc
clang++ -m64 darwin/build_port.o -o darwin/build_port
\end{alltt}\html{\newline}
The above is for \MacOSX. Adjust the \cd{<platform>} field as appropriate,
and on \Windows\ append \fn{.exe} to the executable targets (\fn{varinfo}
and \fn{build\_port}).

You can also use this method to manually compile and/or link individual
files: (1) Change to the relevant build directory (always one level below
either \cd{pkg} or \cd{app}), (2) delete the file you want to rebuild from
the \cd{<platform>} directory, (3) run \cd{pimake} as above to build the
file, (4) copy and paste the compile/link command to the shell prompt,
edit as desired, and rerun.

The \fn{varinfo} and \fn{build\_port} executables are
used to construct the platform-specific header files
\fn{pkg/oc/<platform>/ocport.h} and
\fn{pkg/xp/<platform>/xpport.h}. These files contain
\Cplusplus\ macro definitions, typedefs, and function wrappers,
and are an important adjunct when reading the \OOMMF\ source code.

\pttarget{PTtclintrospection}
For in-depth investigations \Tcl\ can be used to directly query and debug
\OOMMF\ initialization scripts. Start a \Tcl\ shell, and from inside the
shell append the \OOMMF\ \fn{pkg/oc} directory to the \Tcl\ global
\cd{auto\_path} variable. Next run \cd{package require Oc} to load the
\Tcl-only portion of the \OOMMF\ \cd{Oc} library into the shell. Then
you can check any and all \cd{Oc\_Option} values from
\fn{config/options.tcl}, platform configuration settings from
\fn{config/platforms/<platform>.tcl}, and perform various other types of
introspection from the \Tcl\ shell. See
Fig.~\ref{fig:oommftclintrospection} for a sample session.

\begin{codelisting}{f}{fig:oommftclintrospection}{Sample \Tcl-level
    \OOMMF\ introspection session. Shell commands are colored
    \shellcmd{\shellcmdcolorname}\ (with \texttt{\$} prompt) and
    \Tcl\ commands are colored \pgmcmd{\pgmcmdcolorname}\ (with
    \texttt{\%} prompt).}{PTtclintrospection}{hyperlink}
\begin{alltt}
$ \shellcmd{pwd}
/Users/barney/oommf
$ \shellcmd{tclsh}
% \pgmcmd{set env(OOMMF_BUILD_ENVIRONMENT_NEEDED) 1}
% \pgmcmd{lappend auto_path [file join [pwd] pkg oc]}
% \pgmcmd{package require Oc}

% # Miscellaneous utilities from Oc_Main (oommf/pkg/oc/main.tcl)
% \pgmcmd{Oc_Main GetOOMMFRootDir}    ;# OOMMF root directory
/Users/barney/oommf
% \pgmcmd{Oc_Main GetPid}             ;# Process id
17423

% # Oc_Option database (oommf/config/options.tcl)
% # Code details in oommf/pkg/oc/option.tcl
% \pgmcmd{Oc_Option Get *}            ;# Registered Option classes (glob-match)
Net_Link Oc_Url Platform Menu Nb_InputFilter Net_Server Oc_Class Color
Net_Host MIFinterp OxsLogs {}
% \pgmcmd{Oc_Option Get Platform *}   ;# All options for class Platform (glob-match)
cflags lflags optlevel
% \pgmcmd{Oc_Option GetValue Platform cflags}  ;# Platform,cflags value
-def NDEBUG

% # Configuration values (oommf/config/platforms/<platform>.tcl)
% # Code details in oommf/pkg/oc/config.tcl
% \pgmcmd{set config [Oc_Config RunPlatform]}
% \pgmcmd{$config GetValue platform_name}                          ;# Platform name
darwin
% \pgmcmd{$config GetValue program_compiler_c++_name}              ;# C++ compiler
clang++
% \pgmcmd{$config GetValue program_compiler_c++_typedef_realwide}  ;# realwide typedef
long double
% \pgmcmd{$config Features program_linker*}             ;# GetValue names (glob-match)
program_linker_option_lib program_linker program_linker_rpath
program_linker_uses_-L-l program_linker_option_out program_linker_option_obj

% \pgmcmd{exit}                                ;# Exit Tcl shell
\end{alltt}\html{\newline}
\end{codelisting}


\section{Understanding \app{pimake}}\label{sec:debug:pimake}
The \OOMMF\ \app{pimake} application controls the compiling and linking
of \OOMMF's \Cplusplus\ components. Based broadly on the \Unix\ make
utility, \app{pimake} is a platform independent tool written in
\Tcl. Each of the source code directories in the \OOMMF\ distribution
tree has a \fn{makerules.tcl} file that specifies build targets and
dependencies. A dependency tree is build from this information augmented
with recursive tracking of \cd{\#include} statements inside the
referenced source code files.  Each time \app{pimake} is run it compares
file timestamps against the dependency tree, and compiles and links any
object and executable files that are older than any of their
dependencies.

After editing \fn{*.h} or \fn{*.cc} files in \OOMMF, you should run
\app{pimake} to propagate your changes to the associated
\OOMMF\ executable(s).  If you run \cd{tclsh oommf.tcl pimake} in a
directory below the \OOMMF\ root directory, then only changes at that
directory and lower are affected. You can use the \cd{-cwd} option to
\app{pimake} to change the effective starting directory. Changes to the
\OOMMF\ \hyperrefhtml{configuration files}{configuration files
  (Sec.~}{)}{sec:debug:configfiles} do \textbf{not} trigger dependency
updates, so if you make changes affecting the build process in these
files you should manually run
\begin{alltt}
$ \shellcmd{tclsh oommf.tcl pimake distclean}
$ \shellcmd{tclsh oommf.tcl pimake}
\end{alltt}\html{\newline}
from the \OOMMF\ root directory to delete and then rebuild the full
\OOMMF\ project.

\section{Bypassing the \cd{oommf.tcl} bootstrap}\label{sec:debug:bootstrap}
When an application is launched by clicking a button in \app{mmLaunch} or from
the command shell like
\begin{alltt}
> \shellcmd{tclsh oommf.tcl mmdisp}
\end{alltt}\html{\newline}
the application (here \app{mmDisp}) is not executed directly but rather
through the ``bootstrap'' program \cd{oommf.tcl}. The bootstrap
constructs a list linking application names to commands
using the \fn{appindex.tcl} files in the various application (\fn{oommf/app/})
directories, and then runs the command associated with the given
name. This is convenient for normal use, but the additional execution
layer can obfuscate the debugging process. You can obtain the
direct command from the bootstrap program itself with the \cd{+command}
option
\begin{alltt}
> \shellcmd{tclsh oommf.tcl mmdisp +command}
app/mmdisp/windows-x86_64/mmdispsh.exe app/mmdisp/scripts/mmdisp.tcl &
\end{alltt}\html{\newline}
The response is the command as used inside a \Tcl\ shell to launch the
application. You may need to make minor edits to run the application at
your shell command prompt. For example, the trailing ampersand runs the
program in the background, which is not what one usually wants when
debugging, so you would omit this. On \Windows\ you may want to change
the forward slash path separators to backslashes. Another
\Windows-specific modification involves the first component of this
command, \fn{app/mmdisp/windows-x86\_64/mmdispsh.exe}. This is an
executable containing an embedded \Tcl\ interpreter that processes the
\Tcl\ script specified as the second command component. If you examine
the \fn{app/mmdisp/windows-x86\_64/} directory you'll find two
executables, \fn{mmdispsh.exe} and \fn{condispsh.exe}. On \Unix\ and
\MacOSX\ these two programs are the same, but on \Windows\ the first is
linked as a native \Windows\ application and the second as a console
application. The importance of this is that only the second provides the
usual \Cplusplus\ standard channels \cd{stdin}, \cd{stdout}, and
\cd{stderr}. In case of abnormal operation programs will sometimes write
error messages to \cd{stdout} or \cd{stderr}, which will be lost if the
program is not running as a console application. The upshot is that for
debugging purposes you would probably want to run \app{mmDisp} (for
example) from a \Windows\ command console as
\begin{alltt}
> \shellcmd{app{\bs}mmdisp{\bs}windows-x86_64{\bs}condispsh.exe app/mmdisp/scripts/mmdisp.tcl}
\end{alltt}

It is worth noting that on the bootstrap command line, arguments
starting with `\cd{+}' (for example, ``\cd{+command}'') are options to
\cd{oommf.tcl} itself. Run ``\cd{tclsh oommf.tcl +h}'' to see the
bootstrap help message. Options to the \OOMMF\ application follow the
application name and start with `\cd{-}'.  For example, to see the help
message for a particular application, run
``\cd{tclsh oommf.tcl <appName> -h}''.


\section{Segfaults and other asynchronous termination}\label{sec:debug:segfaults}
If an \OOMMF\ application suddenly aborts without displaying an error
message, the most likely culprit is a segfault caused by attempted
access to memory outside the program's purview. If this occurs while
running \app{oxsii} or \app{boxsi}, the first thing to check is the
\fn{oxsii.log} and \fn{boxsi.log} log files in the \OOMMF\ root
directory. If there are no hints there, and the error is repeatable,
then you can enable core dumps and re-run the program until the crash
repeats. You can then obtain a stack trace from the core dump to
determine the origin of the failure.

On \Linux, enable core dumps with the shell command \cd{ulimit -Sc
  unlimited}, and then run \cd{ulimit -Sc} to check that the
request was honored. If not, then ask your sysadmin about enabling core
dumps. (Core dumps can be rather large, so after analysis is complete
you should disable core dumps by running \cd{ulimit -Sc 0} in the
affected shell, or else exit that shell altogether.) Once core dumps are
enabled, run the offending application from the core-dumped enabled
shell prompt. When the application aborts an image of the program state
at the time of termination is written to disk. The name and location of
the core dump varies between \Linux\ distributions. On older systems the
core file will be written to the current working directory with a name
of the form \fn{core.<pid>}, where \cd{<pid>} is the pid of the
process. (If the process is \app{oxsii} or \app{boxsi} then the working
directory will be the directory containing the \fn{.mif} file.)
Otherwise, use the command \cd{sysctl kernel.core\_pattern} to determine
the pattern used to create core files. If the pattern begins with a
\cd{|} ``pipe'' symbol, then the core is piped through the indicated
program, and you will have to check the system documentation for that
program to figure out where the core went!

If the core was piped through \app{systemd-coredump}, then you can use
the \app{coredumpctl} utility to gain information about the
process. (More on this below.) Some \Linux\ variants, for example Ubuntu, use
\app{apport}, but may configure it to effectively disable core dumps for
executables outside the system package management system. In this case
you might want to install the \cd{systemd-coredump} package to replace
\app{apport}, or else use \cd{sysctl} to change
\cd{kernel.core\_pattern} to a simple file pattern (e.g.,
\cd{/tmp/core-\%e.\%p.\%h.\%t}).

If you have a core dump, you can run the GNU debugger \app{gdb} on the
executable and core dump to determine where the fault occurred:
% The \shellcmd and \pgmcmd commands color text a predefined color.
% See oommfhead.tex for specifics.
\begin{alltt}
$ \shellcmd{cd app/oxs}
$ \shellcmd{gdb linux-x86_64/oxs /tmp/core.12345}
Program terminated with signal 11, Segmentation fault.
#0  0x00000000005a40da in Oxs_UniaxialAnisotropy::RectIntegEnergy
  (Oxs_SimState const&, Oxs_ComputeEnergyDataThreaded&,
  Oxs_ComputeEnergyDataThreadedAux&, long, long) const ()
(gdb) \pgmcmd{bt}
#0  0x00000000005a40da in Oxs_UniaxialAnisotropy::RectIntegEnergy
  (Oxs_SimState const&, Oxs_ComputeEnergyDataThreaded&,
  Oxs_ComputeEnergyDataThreadedAux&, long, long) const ()
#1  0x00000000005a6fed in Oxs_UniaxialAnisotropy::ComputeEnergyChunk
  (Oxs_SimState const&, Oxs_ComputeEnergyDataThreaded&,
  Oxs_ComputeEnergyDataThreadedAux&, long, long, int) const ()
#2  0x000000000040ce44 in Oxs_ComputeEnergiesChunkThread::Cmd(int,
   void*) ()
#3  0x00000000004697bd in _Oxs_Thread_threadmain(Oxs_Thread*) ()
#4  0x00007f90ea7fb330 in ?? () from /lib64/libstdc++.so.6
#5  0x00007f90ea019ea5 in start_thread () from /lib64/libpthread.so.0
#6  0x00007f90e9d42b0d in clone () from /lib64/libc.so.6
(gdb) \pgmcmd{quit}
\end{alltt}\html{\newline}
(For visibility, shell commands are colored
\shellcmd{\shellcmdcolorname}, and \app{gdb} commands are
\pgmcmd{\pgmcmdcolorname}. The \app{gdb} commands are also prefixed with
the \cd{(gdb)} prompt. For example, ``bt'' above invokes the \app{gdb}
``backtrace'' command.) We see that the segmentation fault occurred in
the member routine \cd{RectIntegEnergy} of class
\cd{Oxs\_UniaxialAnisotropy}, called by \cd{ComputeEnergyChunk}, and so
on. If \app{oxs} had been built with debugging symbols
(\hyperrefhtml{cf. configuration files}{cf. configuration files,
  Sec.~}{}{sec:debug:configfiles}), then the stack trace would include
the corresponding source code files and line numbers.

If the core dump was journaled by \app{systemd-coredump}, then the
command \cd{coredumpctl list} will list all available core dumps,
including a timestamp, the pid, and the name of the executable. You can
get a stack trace with \cd{coredumpctl info <pid>}, or load the core
dump directly into \app{gdb} with \cd{coredumpctl gdb <pid>}. (Some
versions of \app{coredumpctl} want ``debug'' in place of ``gdb'' in that
command; check your system documentation for details.)

On \MacOSX, crash reports are automatically generated and can be viewed
from the \MacOSX\ \app{Console} app. Select ``User Reports'' or ``Crash
Reports'' from the left hand sidebar, and select the crashed
process. The report provides details about the run, including a stack
trace.

You can also create core files on \MacOSX\ in a very similar way as on
\Linux. Set \cd{ulimit -Sc unlimited} and run the application. Core
files are written to the directory \fn{/cores/}, with naming convention
\fn{core.<pid>}. If you built \OOMMF\ with \app{g++}, then you can
obtain a stack trace with \app{gdb} as above. (Note that in MacPorts the
\app{gdb} executable is named \cd{ggdb}.) If you built with
\app{clang++} then you may want to use the LLVM \app{lldb} debugger,
which should be included with the \app{clang++} package. Here is an
example \app{lldb} session, for an \app{oxs} executable built with
debugging symbols:
\begin{alltt}
% \shellcmd{cd app/oxs}
% \shellcmd{lldb -c /cores/core.54416 darwin/oxs}
(lldb) target create "darwin/oxs" --core "/cores/core.54416"
Core file '/cores/core.54416' (x86_64) was loaded.
(lldb) \pgmcmd{bt}
* thread #1, stop reason = signal SIGSTOP
 * frame #0: 0x0000000103cfc188 oxs`Oxs_UniaxialAnisotropy::RectIntegEnergy
 (this=0x00007ff0f4801000, state=0x00007ff0f350e830, ocedt=0x00007ffeec35a9a8,
 ocedtaux=0x00007ff0f350e6a0, node_start=16384, node_stop=20000) const at
 uniaxialanisotropy.cc:246
   frame #1: 0x0000000103cfd864 oxs`Oxs_UniaxialAnisotropy::ComputeEnergyChunk
 (this=0x00007ff0f4801000, state=0x00007ff0f350e830, ocedt=0x00007ffeec35a9a8,
 ocedtaux=0x00007ff0f350e6a0, node_start=16384, node_stop=20000, (null)=0)
 const at uniaxialanisotropy.cc:454
   frame #2: 0x00000001038a1739 oxs`Oxs_ComputeEnergiesChunkThread::Cmd
 (this=0x00007ffeec35b440, threadnumber=0, (null)=0x0000000000000000) at
 chunkenergy.cc:199
   frame #3: 0x00000001039eabaf oxs`Oxs_ThreadTree::LaunchTree
 (this=0x0000000103ef3860, runobj=0x00007ffeec35b440, data=0x0000000000000000)
 at oxsthread.cc:856
[...]
(lldb) \pgmcmd{quit}
\end{alltt}\html{\newline}
Similar to the \app{gdb} example, the debugger prompt is ``(lldb)'', and
``bt'' requests a stack trace.

To create and examine core dumps on \Windows, download and install
\app{ProcDump} and either \app{WinDbg} or \app{Visual Studio}
applications from Microsoft. To get symbols in the process dump file you
will need to build OOMMF with symbols, i.e., include
\begin{verbatim}
Oc_Option Add * Platform cflags {-debug 1}
\end{verbatim}
in the \fn{config/local/options.tcl}. Also, since \cd{-def NDEBUG} is
not included on this line, the \C\ macro \cd{NDEBUG} will not be
defined, which enables code \cd{assert} statements and other consistency
checks, including in particular array bound checks for
\cd{Oxs\_MeshValue} arrays.

You can create an \cd{oxs} process dump by
\begin{alltt}
> \shellcmd{cd app{\bs}oxs}
> \shellcmd{procdump -ma -t -e -x . windows-x86_64{\bs}oxs.exe boxsi.tcl foo.mif}
\end{alltt}\html{\newline}
On program exit (termination, \cd{-t}) or unhandled exception (\cd{-e})
\cd{procdump} will write a full dump file (\cd{-ma}) to
\fn{oxs.exe\_YYMMDD\_HHMMSS.dmp} in the \fn{app/oxs} directory.

Follow this procedure to examine the dump file in \app{WinDbg}:
\begin{enumerate}
\item Launch \app{WinDbg}.
\item Use the menu item \cd{File|Open Crash Dump...} to load the
  \fn{.dmp} file.
\item Then \cd{View|Call Stack} will open a call stack window.
\item Double-clicking on a call stack frame will highlight the
  corresponding line of code in the \Cplusplus\ source. By default only
  the upper portion of the call stack is displayed, which may be just
  system exit handling code. You may need to click the ``More'' control
  in the toolbar one or more times and scroll down to reach
  \OOMMF\ routines. Enable the ``Source'' toolbar option to include
  filenames and line references in the stack list.
\item You can examine variable values at the time of the crash by
  opening the \cd{View|Locals} window. Referring to the the source code
  and local variable windows in
  \hyperrefhtml{the figure below}{Fig.~}{}{fig:windbgstack},
  we see that the index variable \cd{i} has value 40000, but the size of
  the \cd{Ms\_inverse} array only has size 40000. Thus the access into
  \cd{Ms\_inverse} on line 241 (highlighted) is one element beyond the
  end of the array.
\end{enumerate}
% NB: Make certain that the fourth argument to \includeimage
% does not include any newlines or extraneous whitespace.
\ofig{\includeimage{\sswidth}{!}{windbg-stacktrace}{\app{WinDbg}~screenshot}}{\app{WinDbg}
  screenshot displaying call stack, source code, and local variables
  read from a crash dump generated by \app{procdump}.}{fig:windbgstack}

An alternative to \app{WinDbg} is to use the debugger built into Visual
Microsoft's Visual Studio:
\begin{enumerate}
\item Launch Visual Studio.
\item Select the \cd{Continue without code} option (below the ``Get
  started'' column).
\item Select \cd{File|Open|File ...}, and load the \fn{*.dmp} file.
\item Under ``Actions'' in the ``Minidump File Summary'' window, select
    \cd{Debug with Native Only}.
\item If not automatically displayed, bring up \cd{Debug|Windows|Call Stack}.
\item Double-clicking in the call stack will bring up and highlight the
    corresponding line of code in the \Cplusplus\ source.
\item Use the \cd{Debug|Windows|Autos} and \cd{Debug|Windows|Locals} menu
  items to display variable values.
\end{enumerate}

\section{Out-of-bounds memory access}\label{sec:debug:outofbounds}
One of the more common coding errors is allowing array access outside
the allocated range of an array. This error can be insidious because the
program may continue to run past the point of invalid access, but plant
a seed that grows into a seemingly unrelated fatal error later on. There
are a number of tools designed to uncover this problem, but an
especially easy one to use that is common on \Linux\ systems is the
venerable Electric Fence, original written by Bruce Perens in 1987. If
the \fn{libefence.so} shared library is installed, then from the
\cd{bash} prompt in the \fn{oommf/app/oxs} directory you can run
\begin{alltt}
$ \shellcmd{LD_PRELOAD=libefence.so linux-x86_64/oxs boxsi.tcl foo.mif}
\end{alltt}\html{\newline}
(On some installations there may also be an equivalent shell wrapper
\cd{ef}.)  This will abort with a segfault if an invalid memory
reference (read or write) is detected. One nice feature is that you
don't have to rebuild \OOMMF\ to use this debugger---the \cd{efence}
shared library transparently replaces the standard system memory
allocator with the instrumented Electric Fence version at runtime. If
you enable core dumps as explained above, then on \Linux\ systems even
without debug symbols a stack trace on the core dump will provide the
function call list. If you build \OOMMF\ with debugging symbols
(\cd{Oc\_Option cflags} option \cd{-debug} in
\fn{config/local/options.tcl}), then the core stack trace will give the
source file and line number where the invalid memory access
occurred. Also, \OOMMF\ runs at normal speed with Electric Fence
enabled, so you can use it to check for errors in large simulations.

One caveat is that for performance reasons, \OOMMF\ sometimes allocates
larger memory blocks than needed. Electric Fence detects memory
accesses outside the requested memory range, so \OOMMF\ accesses of
memory outside its proper range but inside the requested range will not
be flagged. You can have \OOMMF\ request tight blocks by putting these
lines in your \fn{local/<platform>.tcl} file:
\begin{verbatim}
$config SetValue program_compiler_c++_property_cache_linesize 1
$config SetValue program_compiler_c++_property_pagesize 1
$config SetValue sse_no_aligned_access 1
\end{verbatim}
and rebuilding \OOMMF\ (\cd{pimake distclean} plus \cd{pimake}).

Normally Electric Fence detects accesses to memory locations above the
allocated range (index too high), but you can have it check
instead for memory accesses preceding the allocated range (index too
low) by setting the environment variable \cd{EF\_PROTECT\_BELOW} to 1.

The Electric Fence documentation warns that core dumps of Electric Fence
enabled runs can be significantly larger than core dumps without
Electric Fence, and so recommends running Electric Fence with the
selected executable (here \fn{oxs}) from inside a debugger rather than
creating a core dump. This does not appear to be a problem when used
with \OOMMF\ however, as the core dumps with Electric Fence tend to be
only modestly larger than those without.

A similar tool on \MacOSX\ is the gmalloc (Guard Malloc) package, which
is included with Xcode. Run it from the \fn{oommf/app/oxs} bash or zsh
command line with
\begin{alltt}
% \shellcmd{DYLD_INSERT_LIBRARIES=/usr/lib/libgmalloc.dylib darwin/oxs boxsi.tcl foo.mif}
\end{alltt}\html{\newline}
See the documentation from Apple for full details.

\section{\Cplusplus\ source code debuggers}\label{sec:debug:debuggers}
If you know roughly where a bug is occurring in the code, you can often
debug it by temporarily inserting \cd{printf} or \cd{std::cout <{}<}
statements in the code. But for more complex problems it can be more
informative and quicker in the long run to create a debugging build (i.e.,
one with debugging symbols and perhaps with compiler optimizations
disabled) and run the program in a debugger. This section provides
general information on running \OOMMF\ in a debugger, including short
examples in three common debuggers: \app{gdb}, \app{lldb}, and \app{Visual
  Studio Debugger}.

First edit the configuration files for debugging, as explained in
\hyperrefhtml{the \textbf{Configuration files}
  section.}{Sec.~}{.}{sec:debug:configfiles}
Then run
\begin{alltt}
$ \shellcmd{tclsh oommf.tcl pimake distclean}
$ \shellcmd{tclsh oommf.tcl pimake}
\end{alltt}\html{\newline}
to create a build of \OOMMF\ with debugging symbols. After this you can
load an \OOMMF\ executable into a debugger, run the
program, and examine its execution.  (Remember to \hyperrefhtml{bypass
the \cd{oommf.tcl} bootstrap}{bypass the \cd{oommf.tcl} bootstrap as
explained in Sec.~}{}{sec:debug:bootstrap}.)  There are many debuggers
available, some with multiple front-ends. But one overriding criterion
in selecting a debugger is to choose one that supports the debugging
symbol format output by your \Cplusplus\ compiler. To provide a brief
taste of this subject, we will look at three debuggers: GNU's venerable
\app{gdb} for use with \cd{g++}, the \app{lldb} debugger packaged with
Xcode/\app{clang++} on \MacOSX, and the debugger built into Microsoft's
\app{Visual Studio} for use with Visual \Cplusplus\ \cd{cl} binaries.

\subsection{Introduction to the GNU \app{gdb} debugger}\label{sec:debug:gdbintro}
This section provides a brief overview on using \app{gdb} for debugging
\OOMMF\ programs. For a more thorough background you can refer to the
extensive documentation available from the GNU Project or the many
online tutorials.

In the following examples, the (\app{bash}) shell prompt is indicated by
\cd{\$}, and the \app{gdb} prompt with \cd{(gdb)}. You launch \app{gdb}
from the command line with the name of the executable file. You can
provide arguments to the executable when you \cd{run} the program inside
\app{gdb}. For example, to debug a problem with an \cd{Oxs} extension,
we would run \app{Boxsi} with a sample troublesome \cd{.mif} file, say
\begin{alltt}
$ \shellcmd{cd oommf/app/oxs}
$ \shellcmd{gdb linux-x86_64/oxs}
(gdb) \pgmcmd{run boxsi.tcl local/foo/foo.mif -threads 1}
\end{alltt}\html{\newline}
Subsequent \cd{run} commands will reuse the same arguments unless you
specify new ones. In this example the \cd{-threads 1} option to
\app{Boxsi} is used to simplify the debugging process. If you need or
want to debug with multiple threads, then read up on the ``thread''
command in the \app{gdb} documentation.

The program run will automatically terminate and return to the
\cd{(gdb)} prompt if the program exits or aborts. Alternately you can
\cd{Ctrl-C} at any time to manually halt. To exit \app{gdb} type
\cd{quit} at the \cd{(gdb)} prompt.

\app{gdb} has a large collection of commands that you can use to control
program flow and inspect program data. An example we saw before is
\cd{backtrace}, which can be abbreviated as
\cd{bt}. Fig.~\ref{oommfgdbcheat} lists a few of the more common
commands, and Figs.~\ref{fig:oommfgdbsession1} and
\ref{fig:oommfgdbsession2} provide an example debugging session
illustrating their use.

%%%%%%%%%%%%%%%%%%%%%%%%%%%%%%%%%%%%%%%%%%%%%%%%%%%%%%%%%%%%%%%%%%%%%%%%
\begin{codelisting}{f}{oommfgdbcheat}{\app{gdb} Debugger
    Cheatsheet\HTMLoutput{\phantom{\rule{1pt}{1.5\baselineskip}}}}{sec:debug:gdbintro}{ref}
% For some reason latexml shoves the following table flush up against
% the figure caption; maybe it's not expecting the caption to be at the
% top of the figure? Also, if I make the \rule with 0pt wide then
% latexml drops the rule space. But 1pt wide inside \phantom works.
\begin{center}\begin{tabular}{|l|l|l|}\hline
  \multicolumn{3}{|l|}{\rule[-1ex]{0pt}{3ex}\textbf{Shell%
   command:}\texttt{ gdb linux-x86\_64/oxs [corefile (opt)]}}\\\hline
  \multicolumn{1}{|c}{\rule[-1ex]{0pt}{3.5ex}
    \textbf{Command}}
  & \multicolumn{1}{|c}{\textbf{Abbr.}}
  & \multicolumn{1}{|l|}{\textbf{Description}}\\\hline
  \multicolumn{3}{|l|}{
  \rule{0pt}{2.5ex}\textcolor[rgb]{0,0.7,0}{\textbf{Process control}}}\\\hline
  run [\textit{args}] & & run executable with \textit{args}\\
  run & &  run executable with last \textit{args}\\
  show args & & display current \textit{args}\\
  set env FOO bar  & & set envr.\ variable FOO to ``bar''\\
  unset env FOO & & unset environment variable FOO\\
  Ctrl-C & & stop and return to (gdb) prompt\\
  kill & & terminate current run\\
  quit & & exit gdb\\[0.5ex]\hline

  \multicolumn{3}{|l|}{
  \rule{0pt}{2.5ex}\textcolor{blue}{\textbf{Introspection}}}\\\hline
  backtrace & bt & stack trace\\
  frame 7 & f 7 & change to stack frame 7\\
  list 123 & l 123 & list source about line 123\\
  list foo.cc:50 & & list source about line 50 of foo.cc\\
  list - & l - & list preceding ten lines\\
  list foo::bar & & list first ten lines of function foo::bar()\\
  set listsize 20 & & change list output length to 20 lines\\
  info locals & i lo & print local variables\\
  info args  & & print function arguments\\
  print foo & p foo & write info on variable foo\\
  printf \verb+"+\%g\verb+"+, foo &
    & print foo with format \%g (note comma)\\[0.5ex]\hline

  \multicolumn{3}{|l|}{
    \rule{0pt}{2.5ex}\textcolor{red}{\textbf{Flow control}}}\\\hline
  break bar.cc:13 & b bar.cc:13
    & set breakpoint at line 13 of bar.cc\\
  break foo::bar
  & b foo::bar & break on entry to \Cplusplus\ routine foo::bar()\\
  info breakpoints & i b & list breakpoints\\
  delete 4 & d 4 & delete breakpoint 4\\
  delete & d & delete all breakpoints\\
  ignore 3 100 & & skip breakpoint 3 100 times\\
  watch -location foo & & break when foo changes value\\
  condition 2 foo\verb+>+10 & & break if foo\verb+>+10 at breakpoint 2\\
  continue & c & continue running\\
  step [\#] & s [\#] & take \# steps, follow into subroutines\\
  next [\#] & n [\#] & take \# steps, step over subroutines\\
  finish & & run to end of current subroutine (step out)\\[0.5ex]\hline

  \multicolumn{3}{|l|}{
    \rule{0pt}{2.5ex}\textcolor[rgb]{0.6,0,0.9}{\textbf{Threads}}}\\\hline
  info threads & i th & list threads\\
  thread 4 & t 4 & switch context to thread 4\\\hline
  \end{tabular}
  \end{center}\html{\newline}
\end{codelisting}
%%%%%%%%%%%%%%%%%%%%%%%%%%%%%%%%%%%%%%%%%%%%%%%%%%%%%%%%%%%%%%%%%%%%%%%%

\begin{codelisting}{p}{fig:oommfgdbsession1}{Sample \app{gdb} session,
    part 1: Locating the error}{sec:debug:gdbintro}{ref}
\NONHTMLoutput{\small}
\begin{alltt}
$ \shellcmd{cd app/oxs}
$ \shellcmd{gdb linux-x86_64/oxs}
(gdb) \pgmcmd{run boxsi.tcl examples/stdprob1.mif -threads 1}
Starting program: oommf/app/oxs/linux-x86_64/oxs boxsi.tcl examples/stdp...
oxs: oommf/app/oxs/base/meshvalue.h:319: const T& Oxs_MeshValue<T>::oper...
  Assertion `0<=index && index<size' failed.

Thread 1 "oxs" received signal SIGABRT, Aborted.
0x00007ffff65d837f in raise () from /lib64/libc.so.6
(gdb) \pgmcmd{bt}
#0  0x00007ffff65d837f in raise () from /lib64/libc.so.6
[...]
#4  0x000000000041012a in Oxs_MeshValue<double>::operator[]
  (this=0xcbeb58, index=40000) at oommf/app/oxs/base/meshvalue.h:319
#5  0x000000000061e88a in Oxs_UniaxialAnisotropy::RectIntegEnergy
  (this=0x1307d60, state=..., ocedt=..., ocedtaux=..., node_start=36864,
  node_stop=40000) at oommf/app/oxs/ext/uniaxialanisotropy.cc:241
[...]
(gdb) \pgmcmd{frame 5}
#5  0x000000000061e88a in Oxs_UniaxialAnisotropy::RectIntegEnergy...
241           field_mult = (2.0/MU0)*k*Ms_inverse[i];
(gdb) \pgmcmd{set listsize 5}
(gdb) \pgmcmd{list}
239         if(aniscoeftype == K1_TYPE) \{
240           if(!K1_is_uniform) k = K1[i];
241           field_mult = (2.0/MU0)*k*Ms_inverse[i];
242         \} else \{
243           if(!Ha_is_uniform) field_mult = Ha[i];
(gdb) \pgmcmd{print i}
$1 = 40000
(gdb) \pgmcmd{print Ms_inverse}
$2 = (const Oxs_MeshValue<double> &) @0xcbeb58: \{arr = 0x7ffff7ebf000,
  size = 40000, arrblock = \{datablock = 0x7ffff7ebe010 "",
  arr = 0x7ffff7ebf000, arr_size = 40000, strip_count = 1,
  strip_size = 320000, strip_pos = std::vector of length 2,
  capacity 2 = \{0, 320000\}\}, static MIN_THREADING_SIZE = 10000\}
(gdb) \pgmcmd{kill}
Kill the program being debugged? (y or n) y
[Inferior 1 (process 1309854) killed]
\end{alltt}\html{\newline}
\end{codelisting}

\begin{codelisting}{f}{fig:oommfgdbsession2}{Sample \app{gdb} session,
    part 2: Bug details}{sec:debug:gdbintro}{ref}
\NONHTMLoutput{\small}
\begin{alltt}
(gdb) \pgmcmd{break uniaxialanisotropy.cc:239}
Breakpoint 1 at 0x61e811: file ext/uniaxialanisotropy.cc, line 239.
(gdb) \pgmcmd{run}
Starting program: oommf/app/oxs/linux-x86_64/oxs boxsi.tcl examples/s...
[...]
Thread 1 "oxs" hit Breakpoint 1, Oxs_UniaxialAnisotropy::RectIntegEne...
239         if(aniscoeftype == K1_TYPE) \{
(gdb) \pgmcmd{info breakpoints}
Num     Type           Disp Enb Address            What
1       breakpoint     keep y   0x000000000061e811 in Oxs_UniaxialAni...
        breakpoint already hit 1 time
(gdb) \pgmcmd{ignore 1 39999}
Will ignore next 39999 crossings of breakpoint 1.
(gdb) \pgmcmd{continue}

Thread 1 "oxs" hit Breakpoint 1, Oxs_UniaxialAnisotropy::RectIntegEne...
239         if(aniscoeftype == K1_TYPE) \{
(gdb) \pgmcmd{print i}
$3 = 39991
(gdb) \pgmcmd{condition 1 i>=40000}
(gdb) \pgmcmd{c}

Thread 1 "oxs" hit Breakpoint 1, Oxs_UniaxialAnisotropy::RectIntegEne...
239         if(aniscoeftype == K1_TYPE) \{
(gdb) \pgmcmd{l}
237
238       for(OC_INDEX i=node_start;i<=node_stop;++i) \{
239         if(aniscoeftype == K1_TYPE) \{
240           if(!K1_is_uniform) k = K1[i];
241           field_mult = (2.0/MU0)*k*Ms_inverse[i];
(gdb) \pgmcmd{next}
240           if(!K1_is_uniform) k = K1[i];
(gdb) \pgmcmd{n}
241           field_mult = (2.0/MU0)*k*Ms_inverse[i];
(gdb) \pgmcmd{step}
Oxs_MeshValue<double>::operator[] (this=0xcbeb58, index=40000)
  at oommf/app/oxs/base/meshvalue.h:319
319       assert(0<=index && index<size);
(gdb) \pgmcmd{printf "%d,%d{\bs}n", index, size}
40000,40000
(gdb) \pgmcmd{quit}
\end{alltt}\normalsize\html{\newline}
\end{codelisting}

Two notes concerning \app{gdb} on \MacOSX: First, as mentioned earlier,
if you install \app{gdb} through MacPorts, the executable name is
\cd{ggdb}. Second, debuggers operate outside the normal end-user program
envelope and may run afoul of the OS security system. In particular to
use \app{gdb} you may need to set up a certificate in the
\MacOSX\ System Keychain for it; details on this process can be found
online. This issue might be resolved for \app{lldb} (next section) as
part of the installation process if it and \app{clang++} were installed
as part of the Xcode package.

This introduction only scratches the surface of \app{gdb} commands and
capabilities. You can find tutorials and additional information online,
or else refer to the \app{gdb} documentation from GNU for full details.

\subsection{Introduction to the LLVM \app{lldb}}\label{sec:debug:lldbintro}
If you are working on \MacOSX, you may be building \OOMMF\ with
\app{g++} or \app{clang++}. The native debugger for \app{clang++} is
\app{lldb}, which is included as part of the Xcode package. Both
\app{g++} and \app{clang++} use the same debugging symbol format, so in
principle you should be able to use either debugger with either
compiler, but if you have problems with one try the other.

The \app{lldb} debugger is a command-line debugger very similar in
concept to \app{gdb}, and although the command syntax is somewhat
different, \app{lldb} provides a fair number of aliases to ease the
transition for veteran \app{gdb} users.  Fig.~\ref{fig:oommflldbcheat}
lists a few of the more common \app{lldb} commands, and
Figs.~\ref{fig:oommflldbsession1} and \ref{fig:oommflldbsession2}
illustrate an \app{lldb} debugging session analogous to the \app{gdb}
session presented in Figs.~\ref{fig:oommfgdbsession1} and
\ref{fig:oommfgdbsession2}.

% NOTE: I had trouble getting latex2html to render ``--'' as two
%       separate dashes instead of an endash. \verb works, but is not
%       allowed inside \makecell. However, the following \dblhyp command
%       appears to do the trick. You can use \textsf in place of \texttt
%       if you with, but the html hyphen looks low up against \textrm in
%       my text browser. (And \textrm{-} puts a big space between the
%       hyphens in the html.)
\newcommand{\dblhyp}{\texttt{-}\texttt{-}}

\begin{codelisting}{f}{fig:oommflldbcheat}{\app{lldb} Debugger
    Cheatsheet\HTMLoutput{\phantom{\rule{1pt}{1.5\baselineskip}}}}{sec:debug:lldbintro}{ref}
  \begin{center}
    \begin{tabular}{|l|l|l|}\hline
  \multicolumn{3}{|l|}{\rule[-1ex]{0pt}{3ex}\textbf{Shell
      command:}\texttt{ lldb [-c corefile (opt)] darwin/oxs}}\\\hline
  \multicolumn{1}{|c}{\rule[-1ex]{0pt}{3.5ex}
    \textbf{Command}}
  & \multicolumn{1}{|c}{\textbf{Abbr.}}
  & \multicolumn{1}{|l|}{\textbf{Description}}\\\hline
  \multicolumn{3}{|l|}{
  \rule{0pt}{3ex}\textcolor[rgb]{0,0.7,0}{\textbf{Process control}}}\\\hline
  process launch \dblhyp\ [\textit{args}] & r [\textit{args}]
  & run executable with \textit{args}\\
  process launch & r &  run executable with last \textit{args}\\

  settings show target.run-args & & display current \textit{args}\\

  settings set target.env-vars & \multirow{2}{*}{env FOO=bar}
     & \multirow{2}{*}{set envr.\ variable FOO to ``bar''} \\
  ~~~FOO=bar && \\
  % AFAICT, LaTeX2HTML either draws line across every row
  % or nont at all, depending on whether or not any \hline
  % command appears in the tabular. This provides less than
  % ideal behavior in the non-multirowed column. OTOH, LaTeXML
  % appears to handle this OK.

  Ctrl-C & & stop and return to (lldb) prompt\\
  process kill & kill & terminate current run\\
  quit & & exit lldb\\\hline

  \multicolumn{3}{|l|}{
    \rule{0pt}{3ex}\textcolor{blue}{\textbf{Introspection}}}\\\hline
  thread backtrace & bt & stack trace of current thread\\

  frame select 5 & f 5 & change to stack frame 5\\
  frame variable &  & print args \& vars for current frame\\
  frame variable foo & p foo & print value of variable foo\\
  source list -f foo.cc -l 50 & l foo.cc:50 & list source after line 50 of foo.cc\\
  source list & l & list next ten lines\\
  source list -r & l - & list preceding ten lines\\
  source list -c 20 & & list 20 lines\\\hline

  \multicolumn{3}{|l|}{
    \rule{0pt}{3ex}\textcolor{red}{\textbf{Flow control}}}\\\hline

  breakpoint set && \multirow{2}{*}{set breakpoint at line 99 of foo.cc}\\
  ~~{\dblhyp}file foo.cc {\dblhyp}line 99 && \\

  breakpoint set && \multirow{2}{*}{break at \Cplusplus\ routine foo::bar()}\\
  ~~{\dblhyp}name foo::bar && \\

  breakpoint list & br l & list breakpoints\\
  breakpoint delete 4 & br del 4 & delete breakpoint 4\\
  breakpoint delete & br del & delete all breakpoints\\

  breakpoint modify -i 100 3 & & skip breakpoint 3 100 times\\
  breakpoint modify -c i\verb+>+7 3
   && break if i\verb+>+7 at breakpoint 3\\
  watchpoint set variable foo && break when foo changes value\\[1ex]

  thread continue & c & continue running\\
  thread step-in & s & take one step, into subroutines\\
  thread step-over & n & take one step, over subroutines\\
  thread step-out & finish & run to end of current subroutine\\[1ex]\hline

  \multicolumn{3}{|l|}{
  \rule{0pt}{3ex}\textcolor[rgb]{0.6,0,0.9}{\textbf{Threads}}}\\\hline
  thread list &  & list all threads\\
  thread select 2 & & switch context to thread 2\\\hline
    \end{tabular}
  \end{center}\html{\newline}
\end{codelisting}
%%%%%%%%%%%%%%%%%%%%%%%%%%%%%%%%%%%%%%%%%%%%%%%%%%%%%%%%%%%%%%%%%%%%%%%%

\begin{codelisting}{f}{fig:oommflldbsession1}{Sample \app{lldb} session,
    part 1: Locating the error}{sec:debug:lldbintro}{ref}
\NONHTMLoutput{\small}
\begin{alltt}
% \shellcmd{cd app/oxs}
% \shellcmd{lldb darwin/oxs}
(lldb) target create "darwin/oxs"
Current executable set to 'oommf/app/oxs/darwin/oxs' (x86_64).
(lldb) \pgmcmd{process launch -- boxsi.tcl examples/stdprob1.mif -threads 1}
Process 36662 launched: 'oommf/app/oxs/darwin/oxs' (x86_64)
Assertion failed: (0<=index && index<size) [...] file meshvalue.h, line 319.
Process 36662 stopped
* thread #1, queue = 'com.apple.main-thread', stop reason = hit program assert
    frame #4: 0x00000001000065cc oxs [...] at meshvalue.h:319:3
   316  template<class T>
   317  const T& Oxs_MeshValue<T>::operator[](OC_INDEX index) const
   318  \{
-> 319    assert(0<=index && index<size);
   320    return arr[index];
   321  \}
   322
Target 0: (oxs) stopped.
(lldb) \pgmcmd{bt}
* thread #1, queue = 'com.apple.main-thread', stop reason = hit program assert
    frame #0: 0x00007fff207ba91e libsystem_kernel.dylib`__pthread_kill + 10
[...]
  * frame #4: 0x00000001000065cc oxs`Oxs_MeshValue<double>::operator[](th...
    frame #5: 0x0000000100350fa8 oxs`Oxs_UniaxialAnisotropy::RectIntegEne...
[...]
(lldb) \pgmcmd{frame select 5}
frame #5: 0x0000000100350fa8 oxs`Oxs_UniaxialAnisotropy::RectIntegEnergy(...
   238    for(OC_INDEX i=node_start;i<=node_stop;++i) \{
   239      if(aniscoeftype == K1_TYPE) \{
   240        if(!K1_is_uniform) k = K1[i];
-> 241        field_mult = (2.0/MU0)*k*Ms_inverse[i];
   242      \} else \{
   243        if(!Ha_is_uniform) field_mult = Ha[i];
   244        k = 0.5*MU0*field_mult*Ms[i];
(lldb) \pgmcmd{frame variable i}
(OC_INDEX) i = 40000
(lldb) \pgmcmd{frame variable Ms_inverse}
(const Oxs_MeshValue<double> &) Ms_inverse = 0x0000000102b77928: \{
  arr = 0x0000000101da4000
  size = 40000
[...]
(lldb) \pgmcmd{process kill}
Process 36662 exited with status = 9 (0x00000009)
\end{alltt}\html{\newline}
\end{codelisting}

\begin{codelisting}{f}{fig:oommflldbsession2}{Sample \app{lldb} session,
   part 2: Bug details (lldb output edited for
   space)}{sec:debug:lldbintro}{ref}
\NONHTMLoutput{\small}
\begin{alltt}
(lldb) \pgmcmd{breakpoint set --file uniaxialanisotropy.cc --line 239}
Breakpoint 1: where = oxs`Oxs_UniaxialAnisotropy::RectIntegEnergy(Oxs_Sim...
(lldb) \pgmcmd{process launch}
Process 36718 launched: 'oommf/app/oxs/darwin/oxs' (x86_64)
[...]
* thread #1, queue = 'com.apple.main-thread', stop reason = breakpoint 1.1
   238    for(OC_INDEX i=node_start;i<=node_stop;++i) \{
-> 239      if(aniscoeftype == K1_TYPE) \{
   240        if(!K1_is_uniform) k = K1[i];
   241        field_mult = (2.0/MU0)*k*Ms_inverse[i];
(lldb) \pgmcmd{breakpoint list}
Current breakpoints:
1: file = 'uniaxialanisotropy.cc', line = 239, exact_match = 0, locations...
  1.1: where = oxs`Oxs_UniaxialAnisotropy::RectIntegEnergy(Oxs_SimState c...
(lldb) \pgmcmd{breakpoint modify -i 39999 1}
(lldb) \pgmcmd{thread continue}
* thread #1, queue = 'com.apple.main-thread', stop reason = breakpoint 1.1
-> 239      if(aniscoeftype == K1_TYPE) \{
(lldb) \pgmcmd{p i}
(OC_INDEX) $0 = 39991
(lldb) \pgmcmd{breakpoint modify -c i>=40000}
(lldb) \pgmcmd{c}
* thread #1, queue = 'com.apple.main-thread', stop reason = breakpoint 1.1
-> 239      if(aniscoeftype == K1_TYPE) \{
(lldb) \pgmcmd{thread step-over}
* thread #1, queue = 'com.apple.main-thread', stop reason = step over
-> 240        if(!K1_is_uniform) k = K1[i];
(lldb) \pgmcmd{n}
* thread #1, queue = 'com.apple.main-thread', stop reason = step over
-> 241        field_mult = (2.0/MU0)*k*Ms_inverse[i];
(lldb) \pgmcmd{thread step-in}
* thread #1, queue = 'com.apple.main-thread', stop reason = step in
   317  const T& Oxs_MeshValue<T>::operator[](OC_INDEX index) const
   318  \{
-> 319    assert(0<=index && index<size);
(lldb) \pgmcmd{print (void) printf("%d,%d{\bs}n", index, size)}
40000,40000
(lldb) \pgmcmd{quit}
\end{alltt}\html{\newline}
\end{codelisting}

\subsection{Debugging \OOMMF\ in Visual Studio}\label{sec:debug:visualstudiodebugger}
The debugger built into Microsoft's Visual Studio provides largely
similar functionality to \app{gdb} and \app{lldb}, but with a GUI
interface. It understands the debugging symbol files produced by the
Visual \Cplusplus\ \cd{cl} compiler, namely ``Program DataBase'' files
having the \fn{.pdb} extension. Other debugger options for this symbol
file format include the GUI \app{WinDbg} mentioned earlier, and the
related command line tool \app{CDB}.

Visual Studio is an integrated development environment, and normal usage
involves building ``projects'' that specify all the source code files
and rules for building them into an executable program. \OOMMF\ does not
follow this paradigm, but rather maintains similar information in a
collection of \Tcl\ \fn{makerules.tcl} files distributed across the
development tree. Thus there is no \OOMMF\ project file to load into
Visual Studio. Instead, to debug an \OOMMF\ application in Visual Studio
you need to load the application executable directly, along with some
supplemental run information. The following details the process for
Visual Studio 2022; specifics may differ somewhat for other releases.
\begin{enumerate}
\item Launch Visual Studio
\item Select \cd{Open a project or solution} from the \cd{Getting
  started} pane and then navigate to and select the executable.
\item In the \cd{Solution Explorer} pane, right click on the executable
  and select \cd{Properties}.
\item Under \cd{Parameters}, fill in the \cd{Arguments} and \cd{Working
  Directory} fields as appropriate. You may also have to modify the
  \cd{Environment} setting, in particular if the \Tcl\ and
  \Tk\ \fn{.dll}'s are not on the default path used by Visual
  Studio. In this case click on the ellipsis at the right of the
  \cd{Environment} row, and then click the \cd{Fetch} button at the
  bottom of the \cd{Environment} pop-up to load the current
  environment. Scroll down to variable \cd{path} and edit as necessary.
  Close when complete.
\item Select \cd{Start} from the toolbar or \cd{Debug|Start Debugging}
  from the top-level menu bar.
\item Debug! You can use the drop-down menus to perform actions
  analogous to those described above for the \app{gdb} and \app{lldb}
  debuggers. If you get a message that no symbols were loaded, then most
  likely either the \cd{/Zi} switch was missing from the compile command
  or else the \cd{/DEBUG} option was missing from the link command. In
  this case review the \OOMMF\ \hyperrefhtml{configuration file
  settings}{configuration file settings
    (Sec.~}{)}{sec:debug:configfiles}) and rebuild \OOMMF.  The symbols 
  for the executable should be stored in a \fn{*.pdb} file next to the
  executable file.
\item The call stack should automatically appear when you start
  debugging. If not, you can manually call it up through the menu option
  \cd{Debug|Windows|Call Stack}. A curious feature of Visual Studio
  is that the call stack window disappears when execution exits. This
  happens even when the exit is caused by an abnormal event, for example
  via an assertion failure. In default \OOMMF\ builds many types
  of fatal errors are routed through the
  \cd{Oc\_AsyncError::CatchSignal(int)} routine in
  \fn{pkg/oc/ocexcept.cc}. If you set a breakpoint in this function then
  the debugger will stop if it hits this function, but will not exit the
  debugger, so you can still examine the call stack. Do this before you
  start the debugging run by pulling up the \cd{Debug|New
    Breakpoint|Function Breakpoint...} dialog, enter
  \cd{Oc\_AsyncError::CatchSignal(int)} in the ``Function Name'' box,
  and click ``OK''.
  \item Double-clicking on a row in the Call Stack window will bring up
    the relevant line of source code. Menu option
    \cd{Debug|Windows|Locals} will open a window showing the variable
    values accessible at this point in the code. An example is shown in
    \hyperrefhtml{the figure below}{Fig.~}{}{fig:vsdbgassert}, where we
    see that the index variable \cd{i} at line 241 of
    \fn{uniaxialanisotropy.cc} has value 40000, but the size of
    \cd{Ms\_inverse} is 40000, meaning the maximum valid index into
    \cd{Ms\_inverse} is only 39999.
\item When you exit the debugger you will be asked if you want to save the
  \fn{.sln} (solution) file. If you do, it will be written in the same
  directory as the executable and \fn{.pdb} files. In later debugging
  sessions you can load the solution file in step 2 above and bypass
  steps 3 and 4.
\end{enumerate}
% NB: Make certain that the fourth argument to \includeimage
% does not include any newlines or extraneous whitespace.
\ofig{\includeimage{\sswidth}{!}{vsdbg-assert}{\app{Visual~Studio~Debugger}~screenshot}}{\app{Visual
    Studio Debugger} screenshot displaying call stack, source code, and
  local variables from a debugging session.}{fig:vsdbgassert}
\clearpage}

\chapter*{Credits}\label{sec:credits}
\addcontentsline{toc}{chapter}{Credits}


\newcommand{\myCredit}[3]{#1 (#3)}
\HTMLoutput{\renewcommand{\myCredit}[3]{\htmladdnormallink{#1}{#2}}}

\newcommand{\CreditMJD}{%
\myCredit{Michael J. Donahue}{mailto:michael.donahue@nist.gov}{michael.donahue@nist.gov}}

\newcommand{\CreditDGP}{%
\myCredit{Donald G. Porter}{mailto:donald.porter@nist.gov}{donald.porter@nist.gov}}

\newcommand{\CreditRDM}{%
\myCredit{Robert D. McMichael}{mailto:rmcmichael@nist.gov}{rmcmichael@nist.gov}}

\newcommand{\CreditJE}{%
\myCredit{Jason Eicke}{mailto:jeicke@seas.gwu.edu}{jeicke@seas.gwu.edu}}

The main contributors to this document are \CreditMJD\ and \CreditDGP,
both of
\htmladdnormallink{ITL}{https://www.nist.gov/itl/}/\htmladdnormallink{NIST}{https://www.nist.gov/}.

If you have bug reports\index{reporting~bugs}, contributed code, feature
requests, or other comments for the \OOMMF\ developers, please send them
in an e-mail\index{e-mail} message to
{\htmladdnormallink{\texttt{<michael.donahue@nist.gov>}}{mailto:michael.donahue@nist.gov}}
or
{\htmladdnormallink{\texttt{<donald.porter@nist.gov>}}{mailto:donald.porter@nist.gov}}%
\index{contact~information}.

\begin{latexonly}
  \iflatexml
  This document was generated by \LaTeXML.
  \fi
\end{latexonly}
\begin{htmlonly}
% Give Latex2Html version and reference, as specified by the
% Perl $INFO variable set in .latex2html-init.
\htmlinfo*
\end{htmlonly}
 % LaTeXML seems to be unhappy if credits comes
            % after bibliography, at least with report document class.

%%%%%%%%%%%%%%%%%%%%%%%% Bibliography and Index %%%%%%%%%%%%%%%%%%%%%%%%
% Manually add bibliography and index to the table of contents.  This
% could be done automatically using the tocbibind package, but the
% LaTeXML 0.8.6 binding for tocbibind is incomplete. We may revisit this
% if we decide we want Bibliography to be a numbered section, but
% otherwise the simplest solution probably creates the fewest
% complications across all output modalities (latex, pdflatex,
% latex2html, latexml).

% Use \phantomsection so hyperref index link in TOC point properly (otherwise
% it points to the previous section).
\clearpage\phantomsection
\addcontentsline{toc}{chapter}{Bibliography}
%\include{biblio}
%\bibliographystyle{abbr}
%begin{latexonly}
\iflatexml
\bibliographystyle{plain}
\else
\bibliographystyle{../common/oommf}
\fi
%end{latexonly}
\html{\bibliographystyle{../common/oommf}}
\bibliography{pmbiblio}\label{bibliography}
% Note: LaTeXML uses \label value to set file names.

% Use \phantomsection so hyperref index link in TOC point properly
% (otherwise it points to the bibliography).
\clearpage\phantomsection
\addcontentsline{toc}{chapter}{Index}
\printindex\label{index}
% Note: LaTeXML uses \label value to set file names.

\end{document}
